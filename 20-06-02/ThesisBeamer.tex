%!TEX program = xelatex
\documentclass[xcolor=svgnames,serif,table,10pt]{beamer}
%\includeonlyframes{Representation}%Acknowledgement
\mode<presentation>{
% Setup appearance:
\setbeamercovered{transparent}
\usecolortheme[named=FireBrick]{structure}
\setbeamertemplate{caption}[numbered]
\setbeamertemplate{navigation symbols}{}

\useoutertheme{infolines}
\usetheme{Darmstadt}

\setbeamertemplate{blocks}[rounded][shadow=true]
\setbeamercovered{transparent}

% 修改样式
%\setbeamertemplate{blocks}[rounded][shadow=true]

\setbeamercolor{box}{bg=black!20!orange,fg=white}

\setbeamercolor{block title}{use=sidebar,fg=sidebar.fg!10!white,bg=orange!70!black}
%\setbeamercolor{block body}{use=sidebar,fg=black,bg=sidebar.bg!90!blue}

\setbeamercolor{block title example}{use=sidebar,fg=sidebar.fg!10!white,bg=black!60!green}
%\setbeamercolor{block body example}{use=sidebar,fg=black,bg=sidebar.bg!90!green}

\setbeamercolor{block title alerted}{use=sidebar,fg=sidebar.fg!10!white,bg=black!50!red}
%\setbeamercolor{block body alerted}{use=sidebar,fg=black,bg=sidebar.bg!90!red}

%\setbeamerfont{frametitle}{size=\small,series=\CJKfamily{FZDH}\Arial\boldmath}
%\setbeamerfont{frametitle}{series=\bfseries}


\setbeamertemplate{headline}
{%
  \begin{beamercolorbox}[shadow=true]{section in head/foot}
  \vskip2pt\insertnavigation{\paperwidth}\vskip2pt
  \end{beamercolorbox}%
}

\iffalse
\AtBeginSection[]
{
  \frame{
    \footnotesize
    \frametitle{主要内容}
    \tableofcontents[currentsection]
  }
}

\AtBeginSubsection[]
{
  \begin{frame}
    \footnotesize
    \frametitle{主要内容}
    \tableofcontents[currentsection,currentsubsection]
  \end{frame}
}
\fi

\renewcommand{\raggedright}{\leftskip=0pt \rightskip=0pt plus 0cm}
\raggedright
}

\usepackage{tabularx,multirow,multicol,longtable}
\usepackage{tabu}
\usepackage{graphics}
\usepackage{xcolor}
\usepackage[no-math]{fontspec}%--------------------------------------------------提供字体选择命令
\usepackage{xunicode}%-----------------------------------------------------------提供Unicode字符宏
\usepackage{xltxtra}%------------------------------------------------------------提供了针对XeTeX的改进并且加入了XeTeX的LOGO
\usepackage[BoldFont,SlantFont,CJKchecksingle]{xeCJK}%---------------------------使用xeCJK宏包
%================================== 设置中文字体 ================================%
%\setCJKmainfont{Adobe Heiti Std}%------------------------------------------------设置正文为黑体
%\setCJKmonofont{Adobe Song Std}%-------------------------------------------------设置等距字体
%\setCJKsansfont{Adobe Kaiti Std}%------------------------------------------------设置无衬线字体
%\setCJKfamilyfont{zxzt}{FZShouJinShu-S10S}
%\setCJKfamilyfont{FZDH}{FZDaHei-B02S}
%================================== 设置中文字体 ================================%

%================================== 设置英文字体 ================================%
%\setmainfont[Mapping=tex-text]{TeX Gyre Pagella}%--------------------------------英文衬线字体
%\setsansfont[Mapping=tex-text]{Trebuchet MS}%------------------------------------英文无衬线字体
%\setmonofont[Mapping=tex-text]{Courier New}%-------------------------------------英文等宽字体
%\newfontfamily\Arial{Arial}
%================================== 设置英文字体 ================================%

%================================== 设置数学字体 ================================%
%\setmathsfont(Digits,Latin,Greek)[Numbers={Lining,Proportional}]{Minion Pro}
%================================== 设置数学字体 ================================%
\punctstyle{kaiming}%------------------------------------------------------------开明式标点格式
\usepackage{graphicx}
\usepackage{tikz}
\usetikzlibrary{positioning,backgrounds}
\usetikzlibrary{fadings}
\usetikzlibrary{patterns}
\usetikzlibrary{calc}
\usetikzlibrary{shadings}
\pgfdeclarelayer{background}
\pgfdeclarelayer{foreground}
\pgfsetlayers{background,main,foreground}
\usepackage{xifthen}
\usepackage{colortbl,dcolumn}
\usepackage{enumerate}
\usepackage{pifont}
\usepackage{tabularx}
\usepackage{booktabs}

%=================================== 数学符号 =================================%
\newcommand{\rtn}{\mathrm{\mathbf{R}}}
\newcommand{\N}{\mathrm{\mathbf{N}}}
\newcommand{\As}{\mathrm{a.s.}}
\newcommand{\Ae}{\mathrm{a.e.}}
\newcommand*{\PR}{\mathrm{\mathbf{P}}}
\newcommand*{\EX}{\mathrm{\mathbf{E}}}
\newcommand{\EXlr}[1]{\mathrm{\mathbf{E}}\left[#1\right]}
\newcommand*{\dif}{\,\mathrm{d}}
\newcommand*{\F}{\mathcal{F}}
\newcommand*{\h}{\mathcal{H}}
\newcommand*{\vp}{\varepsilon}
\newcommand*{\prs}{\dif\PR-\As}
\newcommand*{\dte}{\dif t-\Ae}
\newcommand*{\pts}{\dif\PR\times\dif t-\Ae}
\newcommand{\Ito}{It\^{o}}
\newcommand{\tT}[1][0]{[#1,T]}
\newcommand{\intT}[2][T]{\int^{#1}_{#2}}
\newcommand{\intTe}[1][t]{\intT[t+\varepsilon]{#1}}
\newcommand{\s}{\mathcal{S}}
\newcommand{\me}{\mathrm{e}}
\newcommand{\one}[1]{{\bf 1}_{#1}}
\renewcommand{\M}{{\rm M}}
\newcommand{\Me}[1][t]{M^{\varepsilon}_{#1}}
\newcommand{\Ne}[1][t]{N^{\varepsilon}_{#1}}
\newcommand{\Pe}[1][t]{P^{\varepsilon}_{#1}}
\DeclareMathOperator*{\sgn}{sgn}
%=================================== 数学符号 =================================%

\graphicspath{{figures/}}

\title[2010 级硕士学位答辩 (BSDE)]{一类倒向随机微分方程 \boldmath$L^p$ $(p\geq 1)$ 解的\\ 存在惟一性及生成元的表示定理}

\author[肖立顺]
{姓名: \makebox[4em][s]{肖立顺}\\
  导师: \makebox[4em][s]{范胜君}\\
  专业: \makebox[4em][s]{应用数学}\\
  方向: \makebox[4em][s]{随机分析}}

\institute[CUMT]{\includegraphics[width=1cm]{cumt.pdf}\\ 2010 级硕士学位论文答辩}

\date{\tiny 2013-05-21}

\setlength{\baselineskip}{22pt}
\renewcommand{\baselinestretch}{1.4}

\begin{document}

\setlength{\abovedisplayskip}{1ex}%------------------------------------------ 公式前的距离
\setlength{\belowdisplayskip}{1ex}%------------------------------------------ 公式后的距离

%\includeonlyframes{Brown}

\begin{frame}
  \titlepage
\end{frame}

%\begin{frame}{内容提纲}
%\begin{figure}
%\vskip-2.5em
%  \includegraphics[page=1,scale=0.5]{mainmap}
%\end{figure}
%\end{frame}

\section{BSDE 简介}
\begin{frame}{倒向随机微分方程 (BSDE) 简介}
\begin{figure}
\vskip-2.5em
  \includegraphics[page=2,scale=0.5]{mainmap}
\end{figure}
\end{frame}

\begin{frame}[t]{倒向随机微分方程 (BSDE) 简介}

\qquad 线性的倒向随机微分方程 (BSDE) 由 \cite{Bismut1973JMAA} 提出;

\qquad \cite{PardouxPeng1990SCL} 提出非线性 BSDE 如下:

\medskip
  \begin{beamercolorbox}[shadow=true,sep=0pt,rounded=true]{box}
    \begin{equation}\label{eq:BSDEs}
      {\color<6>[rgb]{0,0,0}y_t}={\color<3,5>[rgb]{0,0,0}\xi}+
          \int_t^{{\color<2,5>[rgb]{0,0,0}T}} {\color<4,5>[rgb]{0,0,0}g}(s,{\color<6>[rgb]{0,0,0}y_s},{\color<6>[rgb]{0,0,0}z_s})\dif s
          -\int_t^{{\color<2,5>[rgb]{0,0,0}T}}{{\color<6>[rgb]{0,0,0}z_s}\dif B_s},\quad t\in[0,{\color<2,5>[rgb]{0,0,0}T}].
    \end{equation}
  \end{beamercolorbox}

\pause

 \begin{center}
  \setlength{\extrarowheight}{1.5mm}
  %\addtolength{\tabcolsep}{1mm}
  \rowcolors[]{1}{orange!70}{white!90!gray}
   \begin{tabular}{cl}
      \onslide<2->{$T$}         & \onslide<2->{终端时间, $0\leq T\leq +\infty$} \\
      \onslide<3->{$\xi$}       & \onslide<3->{终端条件, 可测随机变量}\\
      \onslide<4->{$g$}         & \onslide<4->{生成元, $g(\omega,t,y,z):\Omega\times[0,T]\times\rtn^k\times\rtn^{k\times d}\to\rtn^k$}\\
      \onslide<5->{$(\xi,T,g)$} & \onslide<5->{BSDE 的参数}\\
      \onslide<6->{$(y_t,z_t)_{t\in\tT}$} & \onslide<6->{BSDE 的适应解}\\
    \end{tabular}
 \end{center}
\end{frame}

\begin{frame}{倒向随机微分方程 (BSDE) 简介}
  \begin{itemize}
    \item \cite{DuffieEpstein1992Econometrica}, \alert{效用函数理论};
    \item \mbox{[Peng(1991)]}, \alert{反应扩散方程}和 \alert{Navier-Stokes 方程};
    \item \cite{ElKarouiPengQuenez1997MF}, \alert{派生证券} (如期权期货等);
    \item \mbox{[Peng(1997)]}, $g$-期望和条件 $g$-期望, \alert{金融风险度量};
    \item 反射倒向随机微分方程 (RBSDE), 正倒向随机微分方程 (FBSDE), 倒向重随机微分
          方程 (BDSDE), 以及带跳的、超前的 BSDE;
    \item 解的性质: [Peng(1992)] 提出 BSDE 解的比较定理;
    \item \cite{BriandCoquetHuMeminPeng2000ECIP} 提出解的逆比较定理, 生成元表示定理;
    \item \ldots\ldots \ \ldots\ldots
  \end{itemize}
\end{frame}

\begin{frame}{倒向随机微分方程 (BSDE) 简介}
\scriptsize
  \begin{block}{$(Y_t)_{t\in\tT}$ 所在的空间}
    ${\s}^p(0,T;\rtn^k)$  表示 $\rtn^k$-值, 适应且满足
    $$\|Y\|_{{\s}^p}:=
    \left(\EX
      \left[
        \sup_{t\in\tT}|Y_t|^p
      \right]
    \right)^{1\wedge 1/p}<+\infty$$
    的连续过程 $(Y_t)_{t\in\tT}$ 全体.
  \end{block}

  \begin{block}{$(Z_t)_{t\in\tT}$ 所在的空间}
    $\M^p(0,T;\rtn^{k\times d})$ 表示
    ${\rtn}^{k\times d}$-值, $(\F_t)$-循序可测且满足
    $$\|Z\|_{{\M}^p}:=
      \left\{\EX
      \left[
        \left(
          \int_0^T |Z_t|^2\dif t
        \right)^{p\over 2}
      \right]
      \right\}^{1\wedge 1/p}<+\infty$$
    的连续过程 $(Z_t)_{t\in\tT}$ 全体.
  \end{block}

  \qquad 连续过程 $(Y_t)_{t\in\tT}$ 属于 (D) 类是指, 随机过程族
$\{Y_\tau:\tau\in\Sigma_T\}$
是一致可积的, 其中 $\Sigma_T$ 表示所有满足 $\tau\leq T$ 的停时全体.
\end{frame}

\section{研究的出发点}

\begin{frame}{研究的出发点}
\begin{figure}
\vskip-2.5em
  \includegraphics[page=3,scale=0.5]{mainmap}
\end{figure}
\end{frame}

\subsection{$L^p$ $(p\geq 1)$ 解的存在惟一性}


\subsection{生成元表示定理}


\section{$L^p$ $(p>1)$ 解存在惟一性}

\begin{frame}{有限或无限时间终端多维 BSDE 的 $L^p$ $(p>1)$ 解的存在惟一性}
\begin{figure}
\vskip-2.5em
  \includegraphics[page=4,scale=0.5]{mainmap}
\end{figure}
\end{frame}

\subsection{$L^p$ $(p>1)$ 解存在惟一性的主要结果}

\begin{frame}[t]{有限或无限时间终端多维 BSDE 的 $L^p$ $(p>1)$ 解 --- 主要结果}
\vspace{-1ex}
令 $p>1$, $u(t)$, $v(t):\tT\mapsto \rtn^+$ 满足
\alert<2>{$\intT{0}\big(u(t)+v^2(t)\big)\dif t<+\infty$}.
\vspace{-1ex}
  \begin{block}{生成元 $g$ 的主要假设, $0\leq T\leq +\infty$}
    \begin{enumerate}[(H1)]
      \item $\EX
              \big[
                \big(
                  \intT{0}|g(t,0,0)|\dif t
                \big)^p
              \big]<+\infty$;
      \item $\pts$, $\forall z\in\rtn^{k\times d}$, $y\mapsto g(t,y,z)$ 连续;

      \item $g$ 关于 $y$ 满足广义一般增长条件\only<1-2>{:}\only<3->{;}
            \only<1-2>{$$\forall r'\in\rtn^+,\
            \psi_{r'}(t):=\sup_{|y|\leq r'}|g(t,y,0)-g(t,0,0)|\in L^1(\tT\times\Omega);$$}
      \item $g$ 关于 $y$ 满足对 $t$ 不一致的单调条件\only<1-2>{:}\only<3->{;}
            \only<1-2>{
            \[
               \langle y_1-y_2,g(t,y_1,z)-g(t,y_2,z)\rangle\leq \alert<2>{u(t)}|y_1-y_2|^2;
            \]}
      \item $g$ 关于 $z$ 满足对 $t$ 不一致的 Lipschitz 连续条件\only<1-2>{:}\only<3->{.}
            \only<1-2>{
            \[
              |g(t,y,z_1)-g(t,y,z_2)|\leq \alert<2>{v(t)}|z_1-z_2|.
            \]}
    \end{enumerate}
  \end{block}
\only<4>{
  \begin{alertblock}{定理 2.3. [P8]: $L^p$ $(p>1)$ 解的存在惟一性}
    \qquad 如果 $0\leq T\leq +\infty$, $p>1$ 且 $g$ 满足 (H1)--(H5), 则
    $\forall\xi\in L^p(\Omega,\F_T,\PR;\rtn^k)$, BSDE $(\xi,T,g)$
    存在惟一解 $(y_t,z_t)_{t\in\tT}\in\s^p\times \M^p$ .
  \end{alertblock}
}

  \only<5>{
  \begin{alertblock}{定理 2.4.[P9]: $L^p$ $(p>1)$ 解的比较定理, $0\leq T\leq+\infty$}
    \qquad BSDE $(\xi^i,T,g_i)$ 存在惟一解
    $(y^i_\cdot,z^i_\cdot)\in\s^p\times\M^p$. 如果 $\xi^1\leq\xi^2$,
    $g$ 满足 (H4) -- (H5), 且 $g_1(t,y^2_t,z^2_t)\leq g_2(t,y^2_t,z^2_t)$,
    则 $\forall t\in\tT$, 有 $y^1_t\leq y^2_t$.
  \end{alertblock}}
\end{frame}

\subsection{证明简述}

\begin{frame}{有限或无限时间终端多维 BSDE 的 $L^p$ $(p>1)$ 解 --- 惟一性证明}
\vspace{-2ex}
  \begin{block}{先验估计的假设: $0\leq T\leq+\infty$}
    \begin{enumerate}[(A)]
      \item $\forall(y,z)\in\rtn^k\times\rtn^{k\times d}$, $\EX\big[\big(\intT{0}f_t\dif t\big)^p\big]<+\infty$
            \[
              \langle y,g(t,y,z)\rangle\leq u(t)|y|^2+v(t)|y||z|+f_t|y|.
            \]
    \end{enumerate}
  \end{block}

  \begin{exampleblock}{命题 2.9. [P14]: $L^p$ $(p>1)$ 解的先验估计}
    \qquad 令 $0\leq T\leq+\infty$, $g$ 满足 (A), $\forall p>1$,
    则 $\exists C_p>0$ 使得 $\forall\ 0\leq r\leq t\leq T$,
    \begin{align*}
     \EX
       \left[\left.
         \sup_{s\in\tT[t]}|y_s|^p\right|\F_r
       \right]+
       \EX
       \left[\left.
        \left(
          \intT{t}|z_s|^2\dif s
        \right)^{p\over 2}\right|\F_r
       \right]
     \leq C_p\EX
        \left[\left.|\xi|^p+
          \left(
            \intT{t}f_s\dif s
          \right)^p\right|\F_r
        \right].
    \end{align*}
  \end{exampleblock}
\end{frame}



\begin{frame}[t]{有限或无限时间终端多维 BSDE 的 $L^p$ $(p>1)$ 解 --- 存在性证明}
\vspace{-2ex}
  \begin{block}{第一步: \hspace{1.5cm} \only<3->{P16--P22, 共 6 页}}
    \qquad 假设 $g$ 满足 (H2), (H3'), (H4) 和 (H5), $V\in\M^p$,
    \begin{equation}\label{eq:AssumptionFirstStepLpMultiInfinite}
      |\xi|\leq K,\prs, \qquad |g(t,0,V_t)|\leq K\me^{-t},\pts.
    \end{equation}
    证明 BSDE \eqref{eq:BSDE-g-yt-VtLpMultiInfinite} 在空间 $\s^2\times\M^2$
    中存在解,
    \begin{equation}\label{eq:BSDE-g-yt-VtLpMultiInfinite}
      y_t=\xi+\intT{t}g(s,y_s,V_s)\dif s-\intT{t}z_s\dif B_s,\quad t\in\tT;
    \end{equation}
  \end{block}
  \pause
  \begin{itemize}
    \item 对 $g$ 与 $\rho_n$ 关于 $y$ 做\alert{卷积}得到 $g_n$, 关于 $y$ 局部 Lipschitz 连续;
    \item 对 $g_n$ \alert{截断}得到 $g_{n,q}$, 关于 $y$ 是 Lipschitz 连续;
    \item BSDE $(\xi,T,g_{n,q})$ 有解 $(y^{n,q},z^{n,q})\in\s^2\times\M^2$, $(y^{n},z^{n})\in\s^2\times\M^2$;
    \item 对 BSDE $(\xi,T,g_n)$ 两侧取\alert{弱极限}.
  \end{itemize}
\end{frame}

\begin{frame}[t]{有限或无限时间终端多维 BSDE 的 $L^p$ $(p>1)$ 解 --- 存在性证明}

  \begin{block}{第二步}
    \qquad 通过改进的特殊\alert{截断}技术, 将假设 (H3') 弱化为 (H3);
  \end{block}

  \only<2>{\qquad 困难之处: 此时需在 $g$ 不依赖于 $z$ 时进行截断, 且为第三步做铺垫.

  \begin{equation*}
    \pi_{u}(x):=\frac{u x}{u\vee |x|},\quad \forall u\in\rtn;
  \end{equation*}
  \begin{align*}
    h_n(t,y,V_t):=&\
    \theta_{r'}(y)\big(g(t,y,\pi_{n\me^{-t}}(V_t))-g(t,0,\pi_{n\me^{-t}}(V_t))\big)
    \frac{n\me^{-t}}{\psi_{r'+1}(t)\vee(n\me^{-t})}\\
    &+g(t,0,V_t).
  \end{align*}
  }

  \only<3->{
  \begin{block}{第三步}
     \qquad 再次使用\alert{截断}技术, 去掉假设 \eqref{eq:AssumptionFirstStepLpMultiInfinite},
     同时证明 $\forall V\in\M^p$,
     BSDE \eqref{eq:BSDE-g-yt-VtLpMultiInfinite} 在假设 (H1) -- (H5) 下有
     $\s^p\times\M^p$ 解;
  \end{block}}
  \only<4>{
  \begin{equation*}
    \xi^n:=\pi_n(\xi), \quad g^n(t,y,V_t):=g(t,y,V_t)-g(t,0,V_t)+\pi_{n\me^{-t}}(g(t,0,V_t)).
  \end{equation*}}

  \only<5->{
  \begin{block}{第四步}
     \qquad 类似于 \cite{FanJiang2011Stochastics} 通过\alert{先验估计, 分区间构建压缩映射},
     证明 BSDE $(\xi,T,g)$ 在假设 (H1) -- (H5) 下有 $\s^p\times\M^p$ 解.
  \end{block}}
\end{frame}


\begin{frame}{有限或无限时间终端多维 BSDE 的 $L^p$ $(p>1)$ 解 --- 举例}

  \begin{exampleblock}{例 2.16. [P29]: 终端时间有限 $0\leq T<+\infty$, 一维情况}
    $$g(t,y,z)=\alert{|\ln t\,|}(-\me^{y}+|y|)+\alert{\frac{1}{\sqrt[4]{t\,}}}|z|+|B_t|.$$
  \end{exampleblock}\pause

  \begin{exampleblock}{例 2.17. [P30]: 终端时间无限 $0\leq T\leq +\infty$, 二维情况}
    \begin{equation*}
      g(t,y,z)=\alert{t^2\me^{-t}}
      \begin{bmatrix}
        \displaystyle-y^3_1+y_2\\
        \displaystyle-y^5_2-y_1
      \end{bmatrix}
      +\alert{\frac{1}{\sqrt{1+t^2}}}
      \begin{bmatrix}
        \displaystyle|z_1|\\
        \displaystyle|z_2|
      \end{bmatrix}
      +\frac{t^2}{t^4+1}
      \begin{bmatrix}
        \displaystyle 1\\
        \displaystyle 1
      \end{bmatrix},
    \end{equation*}
  \end{exampleblock}

  \qquad $g$ 满足 (H1) -- (H5). BSDE $(\xi,T,g)$ 在空间 $\s^p\times\M^p$
  中有惟一解.
\end{frame}

\section{$L^1$ 解存在惟一性}

\begin{frame}{有限或无限时间终端多维 BSDE 的 $L^1$ 解}
\begin{figure}
\vskip-2.5em
  \includegraphics[page=5,scale=0.5]{mainmap}
\end{figure}
\end{frame}

\subsection{有限或无限时间终端多维 BSDE 的 $L^1$ 解的存在惟一性}

\begin{frame}[t]{有限或无限时间终端多维 BSDE 的 $L^1$ 解 --- 主要结果}
  \begin{block}{生成元 $g$ 的假设: $0\leq T\leq +\infty$}
    \begin{enumerate}[(H1')]
      \item $\EXlr{\intT{0}|g(t,0,0)|\dif t}<+\infty$;
      \item[(H6)] $\exists\alpha\!\in\!(0,1)$,
        \alert<3>{$\intT{0}\!\!\big(\gamma(t)\!+\!\gamma^{1/(1-\alpha)}(t)\!+\!\gamma^{2/(2-\alpha)}(t)\big)\!\dif t\!<\!+\infty$},
        $\EX[\int^T_0\!g_t\!\dif t]\!<\!+\infty$,
        $$|g(t,y,z)-g(t,y,0)|\leq \alert<3>{\gamma(t)}(g_t+|y|+|z|)^\alpha.$$
    \end{enumerate}
  \end{block}

  \only<2-3>{
  \begin{alertblock}{定理 3.1. [P31]: $L^1$ 解的存在惟一性}
    \qquad 令 $0\leq T\leq +\infty$ 且 $g$ 满足 (H1'), (H2) -- (H6),
    则 $\forall\xi\in L^1$ 及 $\beta\in(0,1)$,
    BSDE $(\xi,T,g)$ 存在解 $(y_\cdot,z_\cdot)\in\s^\beta\times \M^\beta$, 且
    $(y_\cdot)$ 属于 (D) 类; $\forall\beta\in(\alpha,1)$, 解惟一.
  \end{alertblock}}

  \only<4>{\qquad 受 \cite{FanJiang2012JTP} 及 \cite{FanLiu2010SPL} 启发.
  \begin{alertblock}{定理 3.6. [38]: $L^1$ 解的比较定理, $0\leq T\leq+\infty$}
    \qquad $\xi^i\in L^1$, $\beta\in(\alpha,1)$, BSDE $(\xi^i,T,g_i)$
    存在惟一解 $(y^i_\cdot,z^i_\cdot)\in\s^\beta\times\M^\beta$,
    且 $(y^i_\cdot)$ 属于 (D) 类.
    若 $\xi^1\leq\xi^2$, $g$ 满足
    (H4) -- (H6), 且
    $g_1(t,y^2_t,z^2_t)\leq g_2(t,y^2_t,z^2_t)$,
    则 $\forall t\in\tT$, 有
    $y^1_t\leq y^2_t$.
  \end{alertblock}}
\end{frame}

\subsection{证明简述}

\begin{frame}{有限或无限时间终端多维 BSDE 的 $L^1$ 解 --- 证明思路}

  借鉴 [Briand et al.(2003)] 的证明方法.
  \begin{block}{惟一性}
    \begin{enumerate}
      \item 假设存在两对 $L^1$ 解 $(y_t,z_t)_{t\in\tT}$ 和 $(y'_t,z'_t)_{t\in\tT}$;
      \item 作差得 $\hat y_\cdot=y_\cdot-y'_\cdot$, $\hat z_\cdot=z_\cdot-z'_\cdot$,
            $(\hat y_\cdot,\hat z_\cdot)\in\s^p\times\M^p$;
      \item 利用 $L^p$ 解的先验估计得 $\hat y_\cdot=0$, $\hat z_\cdot=0$.
    \end{enumerate}
  \end{block}
  \pause
  \begin{block}{存在性}
    \begin{enumerate}
      \item $g$ 不依赖于 $z$ 时, 利用\alert{截断}技术得到 $L^1$ 解的存在性;
      \item $g$ 依赖于 $z$ 时, 使用 \alert{Picard 迭代}分区间构造压缩映射证明 $L^1$ 解的存在性.
    \end{enumerate}
  \end{block}
\end{frame}

\begin{frame}{有限或无限时间终端多维 BSDE 的 $L^1$ 解 --- 举例}
  \begin{exampleblock}{例 3.4. [P37]: 终端时间有限 $0\leq T<+\infty$, 一维情况}
    $$g(t,y,z)=\alert{\frac{1}{\sqrt[3]{t\,}}}
    \big(\me^{-y}\one{y\leq 0}+(1-y^2)\one{y>0}\big)
    +\alert{\frac{t+1}{\sqrt[4]{t\,}}}\big(|z|^2\wedge\sqrt{|z|}\,\big)+\frac{1}{1+t^4}.$$
  \end{exampleblock}

  \begin{exampleblock}{例 3.5. [P38]: 终端时间无限 $0\leq T\leq +\infty$, 二维情况}
    \begin{equation*}
      g(t,y,z)=\alert{\frac{1}{1+t^2}}
      \begin{bmatrix}
        \displaystyle\me^{-y_1}+3y_2\\
        \displaystyle-\me^{y_2}-3y_1
      \end{bmatrix}+\alert{\me^{-t}}
      \begin{bmatrix}
        \displaystyle\sin|z_1|\\[3pt]
        \displaystyle\sin|z_2|
      \end{bmatrix}+
      \begin{bmatrix}
        \displaystyle\me^{-t}\sin t\\[3pt]
        \displaystyle t\me^{-t}
      \end{bmatrix}.
    \end{equation*}
  \end{exampleblock}

  \qquad $g$ 满足 (H1'), (H2) -- (H6). $\forall\beta\in(\alpha,1)$, BSDE $(\xi,T,g)$ 存
  在惟一解 $(y_\cdot,z_\cdot)\in\s^\beta\times\M^\beta$.
\end{frame}

\section{表示定理及应用}

\begin{frame}{$L^p$ $(p>1)$ 空间中的表示定理及应用}
\begin{figure}
\vskip-2.5em
  \includegraphics[page=6,scale=0.5]{mainmap}
\end{figure}
\end{frame}

\subsection{$L^p$ $(p>1)$ 空间中的表示定理}

\begin{frame}[t]{$L^p$ $(p>1)$ 空间中的表示定理 --- 主要结果}
\vskip-2ex
  $0\leq T\leq+\infty$, $1<p\leq 2$, $\alpha\geq 1$,
  $\intT{0}\big(\mu(t)+\nu^2(t)\big)\dif t<+\infty$.

  \begin{block}{生成元 $g$ 的主要假设}
    \begin{enumerate}[(B1)]
      \item $\EX\big[\big(\intT{0}|g(t,0,0)|\dif t\big)^p\big]<+\infty$;
      \item $\forall z\in\rtn^d$, $y\mapsto g(t,y,z)$ 连续;
      \item $g$ 关于 $y$ 满足对 $t$ 不一致的单调条件\only<1>{:}\only<2->{;}
            \only<1>{
            $$(y_1-y_2)\cdot\big(g(t,y_1,z)-g(t,y_2,z)\big)\leq \mu(t)|y_1-y_2|^2;$$}
      \item $g$ 关于 $y$ 满足对 $t$ 不一致的多项式增长条件\only<1>{:}\only<2->{;}
            \only<1>{
            $$|g(t,y,z)|\leq |g(t,0,z)|+\mu(t)|y|^\alpha;$$}
      \item $g$ 关于 $z$ 满足对 $t$ 不一致的 Lipschitz 条件\only<1>{:}\only<2->{;}
            \only<1>{
            $$|g(t,y,z_1)-g(t,y,z_2)|\leq \nu(t)|z_1-z_2|;$$}
      \only<2>{\item 对于 $\dte$ $t\in\tT$, 有 $\EX[|g(t,0,0)|^{p}]<+\infty$, 且
            $\exists k_t>0$, $\delta_t>0$ 使得
            $$\EXlr{\left(\frac{1}{\varepsilon}\intTe|g(u,0,0)|\dif u\right)^{\alpha p}}\leq k_t,\quad
             0<\varepsilon\leq\min\{\delta_t,T-t\}.$$}
    \end{enumerate}
  \end{block}

  \only<3>{\vskip-.8ex
  \begin{alertblock}{定理 4.2. [P42]: $L^p$ $(p>1)$ 空间中的表示定理, $0\leq T\leq+\infty$}
    \qquad 假设 $g$ 满足 (B1) -- (B6),
    则 $\forall y\in\rtn$, $z\in\rtn^d$, $1\leq q<p$,
    \begin{equation*}
      g(t,y,z)=L^q-\lim_{\varepsilon\to 0^+}\frac{1}{\varepsilon}
             \left[
               y_t(g,t+\varepsilon,y+z\cdot(B_{t+\varepsilon}-B_t))-y
             \right],\quad \dte\ t\in\tT.
    \end{equation*}
  \end{alertblock}}
\end{frame}

\begin{frame}[t]{$L^p$ $(p>1)$ 空间中的表示定理 --- 简要证明}
\vspace{-2ex}\small
  \begin{block}{引理 4.7. [P43]}
    \qquad 假设 $0\leq T\leq+\infty$, $y\in\rtn$, $z\in\rtn^d$ 且 $g$ 满足 (B2),
    (B4) -- (B6). 对于 $\dte$ $t\in\tT$,
    $\exists\{\psi^n(t)\}^\infty_{n=1}\in L^{p}(\Omega,\F_t,\PR)$ 满足
    $\lim_{n\to\infty}\EXlr{|\psi^n(t)|^{p}}=0$, 且
    $\forall n\in\N$, $\widetilde y\in\rtn$ 有
    \vspace{-.8ex}
    \begin{equation*}
      \psi^n(t)\leq (2^\alpha+2)(|g(t,0,0)+\nu(t)|z|+\mu(t)|y|^\alpha),\quad\prs,\vspace{-.5ex}
    \end{equation*}
    \begin{equation*}
      |g(t,\widetilde y,z)-g(t,y,z)|\leq2^\alpha n\mu(t)|\widetilde y-y|^\alpha+\psi^n(t), \quad\prs.
    \end{equation*}
  \end{block}
  \vspace{-.5ex}
  \begin{block}{引理 4.8. [P45]}
    \qquad $\forall p\in(1,2]$, 若 $(\widetilde{y}^\varepsilon_s,\widetilde{z}^\varepsilon_s)_{s\in[t,t+\varepsilon]}$
    为%\vspace{-1ex}
    \begin{equation*}
      \widetilde{y}^\varepsilon_s=
      \intTe[s]g(u,\widetilde y+y+z\cdot(B_u-B_t),\widetilde z+z)\dif u
      -\intTe[s]\widetilde{z}^\varepsilon_u\cdot\dif B_u, \quad s\in[t,t+\varepsilon],%\vspace{-1ex}
    \end{equation*}
    的解, 则对于 $\dte$ $t\in\tT$ 有%\vspace{-1ex}
    \begin{equation*}
      \lim_{\varepsilon\to0^+}\frac{1}{\varepsilon}
      \EXlr{\sup_{s\in[t,t+\varepsilon]}|\widetilde{y}^\varepsilon_s|^{\alpha p}
        +\left(
           \intTe|\widetilde z^\varepsilon_s|^2\dif s
         \right)^{p\over 2}}=0.
    \end{equation*}
  \end{block}
\end{frame}

\subsection{$L^p$ $(p>1)$ 的逆比较定理}

\begin{frame}[t]{有限或无限时间终端一维 BSDE 的 $L^p$ $(p>1)$ 解的逆比较定理}
\vspace{-2ex}
  \begin{alertblock}{定理 4.9.[P50]: $L^p$ $(p>1)$ 解的逆比较定理, $0\leq T\leq+\infty$}
    \qquad $g_i$ 满足 (B1) -- (B6).
    如果 $\forall \xi\in L^p$,
    BSDE $(g_i,T,\xi)$ 的解 $(y^i_\cdot,z^i_\cdot)$ 满足
    $\forall t\in\tT$,
    \begin{equation*}
      y^1_t\geq y^2_t, \quad \prs,
    \end{equation*}
    则 $\forall y\in\rtn$, $z\in\rtn^d$ 有,
    \begin{equation*}
      g_1(t,y,z)\geq g_2(t,y,z),\quad\pts.
    \end{equation*}
  \end{alertblock}\pause

  \begin{alertblock}{命题 4.10.[P50]: 等价命题}
    \qquad 假定 $0\leq T\leq+\infty$, $g_i$ 满足 (B1) -- (B6).
    则如下两条陈述等价:
    \begin{enumerate}
      \item $\forall\xi\in L^p$, $t\in\tT$,
            $y_t(g_1,T,\xi)\geq y_t(g_2,T,\xi)$, $\prs$;
      \item $\forall t\in\tT$, $y\in\rtn$, $z\in\rtn^d$,
            $g_1(t,y,z)\geq g_2(t,y,z)$, $\pts$.
    \end{enumerate}
  \end{alertblock}
\end{frame}

\section{其他}

\begin{frame}{其他}
\begin{figure}
\vskip-2.5em
  \includegraphics[page=7,scale=0.5]{mainmap}
\end{figure}
\end{frame}

\subsection{发表论文及获奖情况}

\begin{frame}{发表论文情况}
\scriptsize\sffamily
  \begin{enumerate}
    \item {\normalfont 肖立顺}, 李慧颖, 范胜君. One-dimensional BSDEs with monotonic,
          H\"older continuous and integrable parameters. 华东师范大学学报
          (自然科学版), 2012, 1: 130--137.\pause

          江苏省概率统计学会第十次学术年会 (苏州) 做 15 分钟报告. \pause

    \item {\normalfont 肖立顺}, 范胜君, 徐娜. 随机变量本性上 (下) 确界与范数的关系.
          烟台大学学报 (自然科学与工程版), 2012, 25(4):217--219.
    \item {\normalfont  肖立顺}, 徐娜, 山显雷. 房地产行业供需模型及调控模型的研究.
          数学的实践与认识, 2012, 42(15):148--154.
    \item 刘德群, {\normalfont 肖立顺}, 范胜君. An existence and uniqueness result of $L^l$
          solutions to multidimensional BSDEs with a finite and an infinite time interval.
          应用数学, 2012, 25(4):777--784.
    \item 侯杰, {\normalfont 肖立顺}, 范胜君. 终端时间可为无限的 BSDE 解的递归迭代序列的收敛
          性及解的存在唯一性. 云南大学学报 (自然科学版), 2012, 34(4):380--384.
    \item Lishun XIAO ({\normalfont 肖立顺}), Shengjun FAN (范胜君), Na XU. $L^p$
          $(p\geq 1)$ solutions
          of multidimensional BSDEs with monotone generators in general time intervals.
          {\itshape Stochastics and Dynamics} (SCI), 2012.04 投稿, 2013.04 回修.
  \end{enumerate}
\end{frame}

\begin{frame}[t]{获奖情况及参与科研项目}
\sffamily

\begin{block}{{\normalfont 获奖情况}}
  \begin{enumerate}\footnotesize
    \item 2012年, 研究生国家奖学金;
    \item 2010年 --- 2012 年, 研究生一等奖学金 (2 次), 研究生特等奖学金 (1 次);
    \item 2011 年, 第八届全国研究生数学建模竞赛一等奖;
    \item 2011 年, 第八届苏北数学建模联赛二等奖;
    \item 2011 年, 理学院第四届研究生科技创新论坛二等奖.
  \end{enumerate}
\end{block}
\pause
\begin{block}{{\normalfont 参与科研项目}}
  \begin{enumerate}\footnotesize
    \item 终端时间有限或无限的多维倒向随机微分方程 $L^p$ $(p\geq 1)$ 解的存在惟一性

          {\normalfont 中央高校基本科研业务费专项资金} (No. 2012LWB57), 2012 年 6 月至 2013 年 5 月, 主持人.
    \item 生成元 $g$ 关于 $y$ 弱单调且广义一般增长的倒向随机微分方程解的存在唯一性及
          相关问题研究

          {\normalfont 国家自然科学基金青年基金资助} (No. 11101422),
          2012 年 1 月至 2014 年 12 月, 排名第 4.
  \end{enumerate}
\end{block}
\end{frame}




\section*{参考文献}

\begin{frame}[allowframebreaks]{参考文献}
\tiny
\sffamily
\begin{thebibliography}{57}
%\bibitem[Bahlali(2001)]{Bahlali2001CRASPSI}
%Bahlali, K.
%\newblock Backward stochastic differential equations with locally {L}ipschitz
%  coefficient~[J].
%\newblock \emph{Comptes Rendus de l'Acad{\'e}mie des Sciences-Series
%  I-Mathematics}, 2001, 333\penalty0 (5):\penalty0 481--486.

\bibitem[Bismut(1973)]{Bismut1973JMAA}
Bismut, J.-M.
\newblock Conjugate convex functions in optimal stochastic control~[J].
\newblock \emph{Journal of Mathematical Analysis and Applications}, 1973,
  44\penalty0 (2):\penalty0 384--404.

\bibitem[Briand-Carmona(2000)]{BriandCarmona2000JAMSA}
Briand, P., Carmona, R.
\newblock {BSDE}s with polynomial growth generators~[J].
\newblock \emph{Journal of Applied Mathematics and Stochastic Analysis}, 2000,
  13\penalty0 (3):\penalty0 207--238.

\bibitem[Briand-Coquet-Hu-M\'{e}min-Peng(2000)]{BriandCoquetHuMeminPeng2000ECIP}
Briand, P., Coquet, F., Hu, Y., M\'{e}min, J., Peng, S.
\newblock A converse comparison theorem for {BSDE}s and related properties of
  $g$-expectation~[J].
\newblock \emph{Electronic Communications in Probability}, 2000, 5:\penalty0
  101--117.

\bibitem[Briand-Delyon-Hu-Pardoux-Stoica(2003)]{BriandDelyonHu2003SPA}
Briand, P., Delyon, B., Hu, Y., Pardoux, E., Stoica, L.
\newblock {$L^p$} solutions of backward stochastic differential equations~[J].
\newblock \emph{Stochastic Processes and Their Applications}, 2003,
  108\penalty0 (1):\penalty0 109--129.

\bibitem[Briand-Lepeltier-San~Martin(2007)]{BriandLepetierSanMrtin2007Bernoulli}
Briand, P., Lepeltier, J.-P., San~Martin, J.
\newblock One-dimensional backward stochastic differential equations whose
  coefficient is monotonic in $y$ and non-{L}ipschitz in $z$~[J].
\newblock \emph{Bernoulli}, 2007, 13\penalty0 (1):\penalty0 80--91.

\bibitem[Chen(2010)]{Chen2010SAA}
Chen, S.
\newblock {$L^p$} solutions of one-dimensional backward stochastic differential
  equations with continuous coefficients~[J].
\newblock \emph{Stochastic Analysis and Applications}, 2010, 28\penalty0
  (5):\penalty0 820--841.

\bibitem[Chen-Wang(2000)]{ChenWang2000JAMSA}
Chen, Z., Wang, B.
\newblock Infinite time interval {BSDE}s and the convergence of
  $g$-martingales~[J].
\newblock \emph{Journal of the Australian Mathematical Society (Series A)},
  2000, 69\penalty0 (2):\penalty0 187--211.

\bibitem[Duffie-Epstein(1992)]{DuffieEpstein1992Econometrica}
Duffie, D., Epstein, G.~L.
\newblock Stochastic differential utility~[J].
\newblock \emph{Econometrica: Journal of the Econometric Society}, 1992,
  60\penalty0 (2):\penalty0 353--394.

\bibitem[El~Karoui-Kapondjian-Pardoux-Peng-Quenez(1997a)]{ElKarouiKapondjianPardouxPengQuenez1997AP}
El~Karoui, N., Kapondjian, C., Pardoux, E., Peng, S., Quenez, M.~C.
\newblock Reflected solutions of {B}ackward {SDE} and related obstacle problems
  for {PDE}s~[J].
\newblock \emph{Annals of Probability}, 1997a, 25\penalty0
  (2):\penalty0 702--737.

\bibitem[El~Karoui-Peng-Quenez(1997b)]{ElKarouiPengQuenez1997MF}
El~Karoui, N., Peng, S., Quenez, M.~C.
\newblock Backward stochastic differential equations in finance~[J].
\newblock \emph{Mathematical Finance}, 1997b, 7\penalty0
  (1):\penalty0 1--71.

%\bibitem[Fan-Hu(2008)]{FanHu2008SPL}
%Fan, S., Hu, J.
%\newblock A limit theorem for solutions to {BSDE}s in the space of
%  processes~[J].
%\newblock \emph{Statistics and Probability Letters}, 2008, 78:\penalty0
%  1024--1033.

\bibitem[Fan-Jiang(2010a)]{FanJiang2010SPL}
Fan, S., Jiang, L.
\newblock Finite and infinite time interval {BSDE}s with non-{L}ipschitz
  coefficients~[J].
\newblock \emph{Statistics and Probability Letters}, 2010a, 80\penalty0
  (11-12):\penalty0 962--968.

\bibitem[Fan-Jiang(2010b)]{FanJiang2010JCAM}
Fan, S., Jiang, L.
\newblock A representation theorem for generators of {BSDE}s with continuous
  linear-growth generators in the space of processes~[J].
\newblock \emph{Journal of Computational and Applied Mathematics}, 2010b,
  235:\penalty0 686--695.

\bibitem[Fan-Jiang(2012a)]{FanJiang2011Stochastics}
Fan, S., Jiang, L.
\newblock ${L}^p$ solutions of finite and infinite time interval {BSDE}s with
  non-{L}ipschitz coefficients~[J].
\newblock \emph{Stochastics An International Journal of Probability and
  Stochastic Processes: formerly Stochastics and Stochastics Reports},
  2012a, 84\penalty0 (5):\penalty0 478--506.

\bibitem[Fan-Jiang(2012b)]{FanJiang2012JTP}
Fan, S., Jiang, L.
\newblock A generalized comparison theorem for {BSDE}s and its
  applications~[J].
\newblock \emph{Journal of Theoretical Probability}, 2012b,
  25\penalty0 (1):\penalty0 50--61.

\bibitem[Fan-Jiang(2012a)]{FanJiang2012JAMC}
Fan, S., Jiang, L.
\newblock ${L}^p$ $(p>1)$ solutions for one-dimensional {BSDE}s with
  linear-growth generators~[J].
\newblock \emph{Journal of Applied Mathematics and Computing}, 2012a,
  38\penalty0 (1--2):\penalty0 295--304.

\bibitem[Fan-Jiang(2012b)]{FanJiang2012SciSinMathChina}
Fan, S., Jiang, L.
\newblock {BSDE}s with uniformly continuous generators and integrable
  parameters ({I}n {C}hinese)~[J].
\newblock \emph{Sci Sin Math}, 2012b, 42\penalty0 (2):\penalty0 119--131.
\newblock doi:10.1360/012010-746.

\bibitem[Fan-Liu(2010)]{FanLiu2010SPL}
Fan, S., Liu, D.
\newblock A class of {BSDE}s with integrable parameters~[J].
\newblock \emph{Statistics and Probability Letters}, 2010, 80\penalty0
  (23):\penalty0 2024--2031.

\bibitem[Fan-Jiang-Tian(2011a)]{FanJiangTian2011SPA}
Fan, S., Jiang, L., Tian, D.
\newblock One-dimensional {BSDE}s with finite and infinite time horizons~[J].
\newblock \emph{Stochastic Processes and Their Applications},
  2011{a}, 121\penalty0 (3):\penalty0 427--440.

\bibitem[Fan-Jiang-Xu(2011b)]{FanJiangXu2011EJP}
Fan, S., Jiang, L., Xu, Y.
\newblock Representation theorem for generators of {BSDE}s with monotonic and
  polynomial-growth generators in the space of processes~[J].
\newblock \emph{Electronic Journal of Probability}, 2011b,
  16\penalty0 (27):\penalty0 830--844.

%\bibitem[Hamad\`{e}ne(2003)]{Hamadene2003Bernoulli}
%Hamad\`{e}ne, S.
%\newblock Multidimensional backward stochastic differential equations with
%  uniformly continuous coefficients~[J].
%\newblock \emph{Bernoulli}, 2003, 9\penalty0 (3):\penalty0 517--534.

%\bibitem[Hewitt-Stromberg(1978)]{HewittStromberg1978RAA}
%Hewitt, E., Stromberg, K.~R.
%\newblock \emph{Real and Abstract Analysis}~[M].
%\newblock New York: Spring-Verleg, 1978.

%\bibitem[Hu-Peng(1995)]{HuPeng1995PTRF}
%Hu, Y., Peng, S.
%\newblock Solution of forward-backward stochastic differential equations~[J].
%\newblock \emph{Probability Theory and Related Fields}, 1995, 103:\penalty0
%  273--283.

%\bibitem[Jia(2008)]{Jia2008SPL}
%Jia, G.
%\newblock A class of backward stochastic differential equations with
%  discontinuous coefficients~[J].
%\newblock \emph{Statistics and Probability Letters}, 2008, 78\penalty0
%  (3):\penalty0 231--237.

\bibitem[Jia(2010)]{Jia2010SPA}
Jia, G.
\newblock Backward stochastic differential equations with a uniformly
  continuous generator and related $g$-expectation~[J].
\newblock \emph{Stochastic Processes and Their Applications}, 2010,
  120\penalty0 (11):\penalty0 2241--2257.

\bibitem[Jiang(2005a)]{Jiang2005aSPL}
Jiang, L.
\newblock Converse comparison theorems for backward stochastic differential
  equations~[J].
\newblock \emph{Statistics and Probability Letters}, 2005a, 71:\penalty0
  173--183.

\bibitem[Jiang(2005b)]{Jiang2005bCRASPSI}
Jiang, L.
\newblock Representation theorems for generators of backward stochastic
  differential equations~[J].
\newblock \emph{Comptes Rendus de l'Acad{\'e}mie des Sciences-Series
  I-Mathematics}, 2005b, 340:\penalty0 161--166.

\bibitem[Jiang(2005c)]{Jiang2005cSPA}
Jiang, L.
\newblock Representation theorems for generators of backward stochastic
  differential equations and their applications~[J].
\newblock \emph{Stochastic Processes and Their Applications}, 2005c,
  115\penalty0 (12):\penalty0 1883--1903.

\bibitem[Jiang(2006)]{Jiang2006ScienceInChina}
Jiang, L.
\newblock Limit theorem and uniqueness theorem for backward stochastic
  differential equations~[J].
\newblock \emph{Science In China, Series A}, 2006, 49\penalty0 (10):\penalty0
  1353--1362.

\bibitem[Jiang(2008)]{Jiang2008TAAP}
Jiang, L.
\newblock Convexity, translation invariance and subadditivity for
  $g$-expectations and related risk measures~[J].
\newblock \emph{The Annals of Applied Probability}, 2008, 18\penalty0
  (1):\penalty0 245--258.

%\bibitem[Kobylanski(2000)]{Kobylanski2000AP}
%Kobylanski, M.
%\newblock Backward stochastic differential equations and partial differential
%  equations with quadratic growth~[J].
%\newblock \emph{Annals of Probability}, 2000, 28\penalty0 (2):\penalty0
%  558--602.

\bibitem[Lepeltier-San~Martin(1997)]{LepeltierSanMartin1997SPL}
Lepeltier, J.-P., San~Martin, J.
\newblock Backward stochastic differential equations with continuous
  coefficient [J].
\newblock \emph{Statistics and Probability Letters}, 1997, 32\penalty0
  (4):\penalty0 425--430.

\bibitem[Mao(1995)]{Mao1995SPA}
Mao, X.
\newblock Adapted solutions of backward stochastic differential equations with
  non-{L}ipschitz coefficients [J].
\newblock \emph{Stochastic Processes and their Applications}, 1995, 58\penalty0
  (2):\penalty0 281--292.

\bibitem[Pardoux(1999)]{Pardoux1999NADEC}
Pardoux, E.
\newblock {BSDE}s, weak convergence and homogenization of semilinear
  {PDE}s [C].
\newblock In Clarke, F., Stern, R., editors, \emph{Nonlinear Analysis,
  Differential Equations and Control}, pages 503--549. Kluwer Academic, New
  York, 1999.

\bibitem[Pardoux-Peng(1990)]{PardouxPeng1990SCL}
Pardoux, E., Peng, S.
\newblock Adapted solution of a backward stochastic differential equation [J].
\newblock \emph{Systems Control Letters}, 1990, 14\penalty0 (1):\penalty0
  55--61.


%\bibitem[Pardoux-Peng(1994)]{PardouxPeng1994PTRF}
%Pardoux, E., Peng, S.
%\newblock Backward doubly stochastic differential equations and systems of
%  quasilinear {SPDE}s [J].
%\newblock \emph{Probability Theory and Related Fields}, 1994, 98\penalty0
%  (2):\penalty0 209--227.

%\bibitem[Peng(1991)]{Peng1991LectureNotes}
%Peng, S.
%\newblock \emph{A Generalized {H}amihon-{J}acobi {B}ellman Equation} [M],
%  volume Lecture Notes in CIS 184 of \emph{126--134}.
%\newblock Berlin: Spring-Verleg, 1991.

%\bibitem[Peng(1992)]{Peng1992SSR}
%Peng, S.
%\newblock A generalized dynamic programming principle and
%  hamilton-jacobibellman equation [J].
%\newblock \emph{Stochastics and Stochastic Reports}, 1992, 38\penalty0
%  (2):\penalty0 119--134.

%\bibitem[Peng(1997)]{Peng1997BSDEP}
%Peng, S.
%\newblock Backward {SDE} and related $g$-expectation [C].
%\newblock In El~Karoui, N., Mazliak, L., editors, \emph{Backward stochastic
%  differential equations ({P}aris,1995--1996)}, volume 364 of \emph{Pitman
%  Research Notes Mathematical Series}, pages 141--159. Longman, Harlow, 1997.

%\bibitem[Peng(2004)]{Peng2004LectureNotesOfMathematics}
%Peng, S.
%\newblock Nonlinear expectation, nonlinear evaluation and risk measures [C].
%\newblock In Back, K., Bielecki, T.~R., Hipp, C., Peng, S., Schachermayer, W.,
%  editors, \emph{Stochastic Methods in Finance}, volume 1856 of \emph{Lectures,
%  C.I.M.E.-E.M.S. Lecture Notes of Mathematics}, pages 165--253. Springer,
%  2004.

%\bibitem[Peng-Yang(2009)]{PengYang2009AP}
%Peng, S., Yang, Z.
%\newblock Anticipated backward stochastic differential equations [J].
%\newblock \emph{Annals of Probability}, 2009, 37\penalty0 (3):\penalty0
%  877--902.

\bibitem[Song-Hu-Chen(2012)]{SongHuChen2012AMASES}
Song, L., Hu, F., Chen, Z.
\newblock Representation theorems for generators of {BSDE}s in ${L}^p$
  spaces [J].
\newblock \emph{Acta Mathematicae Applicatae Sinica, English Series}, 2012,
  28\penalty0 (2):\penalty0 255--264.

\bibitem[Touzi(2013)]{Touzi2013StochasticControlNote}
Touzi, N.
\newblock \emph{Optimal stochastic control, stochastic target problems, and
  backward {SDE}} [M], volume 29 of \emph{Fields Institute Monographs}.
\newblock New York: Springer, 2013.

\bibitem[Wang-Huang(2009)]{WangHuang2009SPL}
Wang, Y., Huang, Z.
\newblock Backward stochastic differential equations with non-{L}ipschitz
  coefficients [J].
\newblock \emph{Statistics and Probability Letters}, 2009, 79\penalty0
  (12):\penalty0 1438--1443.

{\beamertemplatebookbibitems
\bibitem[范胜君(2011)]{FanShenJun2011PhD}
范胜君.
\newblock
  \emph{一类随机动力系统 --- 倒向随机微分方程 --- 解的存在惟一性及生成元的表示定理} [D].
\newblock 博士毕业论文, 中国矿业大学, 2011.}

{\beamertemplatebookbibitems
\bibitem[贾广岩(2008)]{JiaGuangYan2008PhD}
贾广岩.
\newblock \emph{倒向随机微分方程、$g$-期望及其相关的半线性偏微分方程} [D].
\newblock 博士毕业论文, 山东大学, 2008.}

%{\beamertemplatebookbibitems
%\bibitem[刘德群(2012)]{LiuDeQun2012MD}
%刘德群.
%\newblock \emph{具有可积参数的倒向随机微分方程解的存在惟一性} [D].
%\newblock 硕士毕业论文, 中国矿业大学, 2012.}

%{\beamertemplatebookbibitems
%\bibitem[马明(2011)]{MaMing2011MD}
%马明.
%\newblock
%  \emph{关于 $y$ 单调且关于 $z$ 一致连续倒向随机微分方程 ${L}^p$ 解的存在唯一性及单调连续性} [D].
%\newblock 硕士毕业论文, 中国矿业大学, 2011.}

{\beamertemplatebookbibitems
\bibitem[汪嘉冈(2005)]{WangJiaGang2005XianDaiGaiLvLunJiChu}
汪嘉冈.
\newblock \emph{现代概率论基础} [M].
\newblock 复旦大学出版社, 第 2 版, 2005.}

\end{thebibliography}
\end{frame}

\begin{frame}{致谢评委}
  \begin{center}
    \begin{tikzpicture}
      \node[above,xscale=1.2,yscale=1.4]{\Huge\bfseries 欢迎各位老师批评指正!};
      \node[xscale=1.2,above,yscale=-1.4,scope fading=south,opacity=0.2]{\Huge\bfseries 欢迎各位老师批评指正!};
    \end{tikzpicture}
  \end{center}
\end{frame}

\end{document}


\begin{abstract}
	This note is an introduction of Logvinenko-Sereda Theorem. 
\end{abstract}
\tableofcontents
\section{Introduction}
Before we introduce the Logvinenko-Sereda Theorem, we first give some definitions.

\begin{definition}
	 Let $\Sigma$ be a measurable set and $\mathscr{E}$ be a subspace of $L^2(\R)$. A measurable set $S$ is called $\mathscr{E}$-determining if there is a positive number $c$ such that 
	 \begin{equation}\label{eqn:def}
	 	f \in \mathscr{E} \Rightarrow c\|f\|_2^2\le \int_S|f|^2\d x.
	 \end{equation}
	 Furthermore, if $S$ is $\mathscr{E}(\widehat{\Sigma})$-determining for any bounded $\Sigma \subset \R$, we say $S$ is determining. Here $\mathscr{E}(\widehat{\Sigma})$ denotes the subspace of all functions with spectrum contained in $\Sigma$.
\end{definition}


\begin{definition}\label{def:thick}
	Let $S$ be a measurable set on $\R$. We say $S$ is $\gamma$-thick if there exists an interval $K=[-L,L]$ and a constant $\gamma>0$ such that:
	\[
	\forall h\in \R \implies |S\cap (K+h)|\ge \gamma. 
	\] 
	We call $S$ thick if the exact value of  $\gamma$ is not concerned.
\end{definition}

Now we can state the theorem as following:
\begin{theorem} \label{thm:main}
	Let $S$ be a measurable set on $\R$. Then the following two assertions are equivalent:
	\begin{itemize}
		\item[(a)] $S$ is determining;
		\item[(b)] $S$ is thick.
	\end{itemize}
\end{theorem}

The proof of $(\mathrm{a})\Rightarrow (\mathrm{b})$ is much easier than the converse.

\noindent{\itshape Proof of $(\mathrm{a})\Rightarrow(\mathrm{b})$.} Given bounded set $\Sigma$, then $\mathscr{E}(\widehat{\Sigma})$ is a shift invariant subspace of $L^2(\R)$. Hence we only need to prove: If $\mathscr{E}$ is a shift invariant non-trivial subspace of $L^2(\R)$, then any $\mathscr{E}$-determining set is thick. Assume $f\in \mathscr{E}$ and define $\omega_f(\delta):=\sup \bigl\{ \int_e|f|^2\d x: |e|\le \delta \bigr\}$. Let $K=[-L,L]$ large enough so that 
$$\int_{K'}|f|^2\d x\le \frac{c}{2},$$ where $c$ is the constant from \cref{eqn:def}. Set $f_h(x):=f(x-h),\forall h\in \R$ and we have
\[
	\int_{(K+h)'}|f_h|^2\d x \le \frac{c}{2}.
\]
Then 
\begin{align*}
	c&= c\int |f_h|^2\d x\le \int_S|f_h|^2\d x=\bigl(\int_{S\cap (K+h)}|f_h|^2\d x+\int_{S\cap (K+h)'}|f_h|^2\d x\bigr)\\
	 &\le \omega_f\bigl(|S\cap(K+h)|\bigr)+\frac{c}{2}
.\end{align*}
Then we obtain $\omega_f\bigl(|S\cap(K+h)|\bigr)\ge \frac{c}{2}$, and this implies $|S\cap (K+h)|$ is bounded off zero for any  $h\in \R$. \hfill $\square$
\begin{definition}
	The Poisson measure is defined by
	\[
	\Pi:=\frac{1}{\pi (1+x^{2})}m
	\] 
	where $m$ is the Lebesgue measure on $\R$.

	Furthermore, we define the probability measure with respect to $x$ 
	\[
	\Pi_x(A) :=\Pi(x-A)=\frac{1}{\pi}\int_A \frac{1}{1+(x-t)^2}\d t,\quad \forall A\subset \R.
	\] 
\end{definition} 

Then we have the following property:
\begin{proposition} \label{prp:equiv}
	Let $S$ be a measurable set on $\R$. Then $S$ is thick if and only if $\inf_{x\in \R}\Pi_x(S)>0$.
\end{proposition}
\begin{proof}
	$\Rightarrow$: Assume $S$ is $\gamma$-thick for some $\gamma>0$ and $K=[-L,L]$ defined in \cref{def:thick}. Then for all $x\in\R$, we obtain
	\begin{align*}
		\Pi_x(S)&=\int_{S}\,\mathrm{d}\Pi_x\ge \int_{(K+x)\cap S}\,\mathrm{d}\Pi_x=\frac{1}{\pi}\int_{x-L}^{x+L}\chi_S(t) \frac{\mathrm{d} t}{1+(x-t)^2}\\
			&= \frac{1}{\pi}\int_L^L\chi_S(t+x) \frac{\mathrm{d} t}{1+t^2}\ge \frac{1}{\pi}\bigl( 1+L^2\bigr)^{-1}\left|S\cap(K+x)\right|\ge \pi^{-1}\bigl(1+L^2\bigr)^{-1}\gamma
	.\end{align*}

	$\Leftarrow$: Assume $\Pi_x(S)\ge \sigma>0$ for all $x\in\R$. Let $K=[-L,L]$ and $L$ is defined later. We have
	\begin{align*}
		\pi \sigma \le & \pi \Pi_x(S) = \int_{\R} \chi_S(t) \frac{1}{1+(x-t)^2}\d t\\
		= & \int_{x-L}^{x+L} \chi_S(t) \frac{1}{1+(x-t)^2}\d t+\int_{|x-t|>L}\chi_{S}(t) \frac{1}{1+(x-t)^2}\d t\\
		\le & \left|S\cap (K+x)\right|+2\int_{L}^{\infty} \frac{1}{1+t^2}\d t\\
		\le &\left| S\cap (K+x)\right|+2\left(\frac{\pi}{2}-\arctan L\right)
	.\end{align*}
	Choose $L$ large enough so that  $\gamma=\pi \sigma -2\left(\frac{\pi}{2}-\arctan L\right)>0 $, then we have $\left| S\cap (K+x)\right|\ge \gamma>0$.
\end{proof}

\section{Jensen's Inequalities and Poisson Operator} 

\begin{definition}
	We call a distribution $f \in S'(\R)$ a plus-function (resp. a minus-function) if $\widehat{f}\lvert_{(-\infty,0)}=0$ (resp. $\widehat{f}\lvert_{(0,\infty)}=0$). A distribution $f\in S'(\T)(=D'(\T)$ is called a plus-function (resp. a minus-function) if $\widehat{f}\lvert_{\Z\cap(-\infty,0)}=0$ (resp. $\widehat{f}\lvert_{\Z\cap [0,\infty)}=0$).
\end{definition}

\begin{definition}
	The set of all plus-functions $f\in L^{p}(X,m)$ where $p \in [1,\infty)$ is denoted by the symbol $H^{p}(X)$ and is called the Hardy class (with the index $p$).
\end{definition}

\subsection{Jensen's Inequality for Plus-Functions on $\T$} 

For Jensen's inequality, we are most familiar with
\[
\exp \int \log |f| \d \mu\le \int |f|\d \mu,
\] 
where $\mu$ is any probability measure and $f\in L^{1}(\mu)$.
Now we introduce another Jensen's inequality:
\begin{equation}\label{eqn:jt}
	f \in H^{1}(\T)\Rightarrow \log \bigl| \widehat{f}(0) \bigr| \le \int_{\T}\log |f|\d m.\tag{$J_{\T}$}
\end{equation}
Here $m$ is the normalized Lebesgue measure on $\T$. Compaired with the trivial estimate $\bigl| \int f \d m \bigr|\le \int |f|\d m$, \cref{eqn:jt} is an essential refinement.

Let $C_{+}:=H^{1}(\T)\cap C(\T)$. We first have the following lemma:
\begin{lemma}
	Any function in $C_{+}$ satisfies \cref{eqn:jt}.
\end{lemma}
\begin{proof}
	Take $f\in C_{+}$ and $\varepsilon >0$. By density of trigonometric polynomials in $C(\T)$, there is a real trigonometric polynomial $t$ such that 
	 \[
	t(\zeta )-\varepsilon <\log\left( |f(\zeta )|+\varepsilon  \right) <t(\zeta )+\varepsilon, \quad \forall \zeta  \in  \T.
	\] 
	Put $s:= t+i \widetilde{t}$, where $\widetilde{t}:=\frac{1}{i}\sum_{k\in \Z}^{} \mathrm{sgn} k\cdot \widehat{t}(k)z^{k}$ and $\widetilde{t}$ is also real. Then $s \in C_{+}$. Then we obtain
	\[
	\bigl| f(\zeta )e^{-s(\zeta )}\bigr|=|f(\zeta )|e^{-t(\zeta )}< e^{\log(|f(\zeta )|+\varepsilon )-t(\zeta )}<e^{\varepsilon },\quad \forall \zeta \in \T. 
	\]
	Hence
	\[
	\bigl|\widehat{f e^{-s}}(0)\bigr|=\Bigl| \int_{\T} fe^{-s}\d m \Bigr|\le e^{\varepsilon }.
	\] 
	On the other hand, $\bigl|\widehat{fe^{-s}}(0)\bigr|=|\widehat{f}(0)|\bigl|e^{-\widehat{s}(0)}\bigr|=|\widehat{f}(0)|e^{-\widehat{t}(0)}$, hence
	\[
		|\widehat{f}(0)|\le e^{\varepsilon +\widehat{t}(0)} =e^{\varepsilon +\int t\d m}\le e^{2\varepsilon +\int \log(|f|+\varepsilon )\d m}. 
	\] 
	The right hand side tends to $\exp \int \log |f|\d m$ as $\varepsilon \to 0$, then we obtain \cref{eqn:jt}.
\end{proof}

Now we can give the proof of \cref{eqn:jt} for any $f\in H^{1}(\T)$.

\noindent{\itshape Proof of \cref{eqn:jt}.} If $f \in H^{1}(\T)$ m then $f=\lim_{j \to \infty} f_j$ (in $L^{1}(\T)$) for a sequence $(f_j)$ of plus-polynomials. Thus, for any $\varepsilon >0$,
\[
\int_{\T}\log(|f|+\varepsilon )\d m =\lim_{j \to \infty} \int_{\T}\log(|f_j|+\varepsilon )\d m,
\] 
since $\bigl|\log(|f|+\varepsilon )-\log(|f_j|+\varepsilon )\bigr|\le \frac{1}{\varepsilon }\bigl| \mid f|-|f_j|\bigr|$. Since \cref{eqn:jt} is valid for $f\in C_{+}$, we obtain
\[
\log |\widehat{f_j}(0)|\le \int_{\T}\log (|f_j|+\varepsilon )\d m,
\] 
and then passing to the limit we obtain
\[
\log |\widehat{f}(0)|\le \int_{\T}\log(|f|+\varepsilon )\d m.
\] 
Now using the Monotone Convergence Theorem we obtain
\[
\lim_{\varepsilon  \to +0} \int_{\T}\log(|f|+\varepsilon )\d m=\int_{\T}\log|f|\d m.
\]\hfill $\square$ 

\subsection{Jensen's Inequality for Plus-Functions on $\R$} We have an analogue of \cref{eqn:jt} for plus-functions on $\R$:
\begin{equation}\label{eqn:jr}
	f\in H^{1}(\R,\Pi)\Rightarrow \log \Bigl|\int_{\R}f\d \Pi\Bigr|\le \int_{\R} \log|f|\d \Pi.\tag{$J_{\R}$}
\end{equation}
Here we define $H^{p}\left( \R,\Pi \right) :=\left\{f \in L^{p}\left( \R,\Pi \right) : f \text{ is a plus-function} \right\} $ for $p \in [1,\infty)$.


 
\begin{lemma}
	Let $f \in H^{1}(\R,m)$, $\displaystyle \omega:=\frac{x-i}{x+i}$. Then 
	\begin{equation}\label{eqn:wn}
		\int_{\R} f \omega^{n}\d \Pi=0
	\end{equation}
	holds for all $n\in \N_{+}$
\end{lemma}

\begin{lemma}\label{lma:h1}
	Let $f \in H^{1}\left( \R,\Pi \right) $. Then \cref{eqn:wn} still holds for all $n\in \N_{+}$. 
\end{lemma}



\noindent{\itshape Proof of \cref{eqn:jr}.} Suppose $f\in H^{1}(\R,\Pi)$. Define
\[
F(e^{i\theta }):=f\left( -\cot \frac{\theta }{2} \right), \quad \theta \in (0,2\pi).
\] 
The variables $\theta $ and $x=-\cot \frac{\theta }{2}$ are connected by the following equalities:
\[
x=-i \frac{e^{i\theta }+1}{e^{i\theta }-1}, \quad \mathrm{d}\theta =\frac{2\mathrm{d}x}{1+x^2}.
\] 
Hence 
\[
\int_0^{2\pi}\bigl|F(e^{i\theta })\bigr|\d \theta =2 \int_{\R}|f(t)| \frac{1}{1+t^2}\d t<\infty.
\] 
Then by \cref{lma:h1} we have
\[
\widehat{F}(-n)=\frac{1}{2\pi}\int_0^{2\pi}F(e^{i\theta })e^{i n \theta  }\d \theta  = \int_{\R}f\omega^{n}\d \Pi =0.
\]
Thus $F\in H^{1}(\T)$ and hence satisfies \cref{eqn:jt}, which is equivalent to \cref{eqn:jr}.
\hfill $\square$

\begin{remark}
	Indeed, the transformation $\displaystyle F(e^{i\theta })=f\left( -\cot \frac{\theta }{2} \right) $ comes from conformal maping between the upper half plane and the disk:
	\[
	\eta = -i \frac{z+1}{z-1}.
	\]
	So \cref{eqn:jr} and \cref{eqn:jt} are essentially the same inequality.
\end{remark}

\subsection{Poisson Operator}
	We define an operator on $L^1(\R,\Pi)$ by
	\[
	P(f)(x):=\int_{\R}f\d \Pi_x=\int_{\R}f(x+t)\d \Pi(t),\quad x\in \R.
	\] 
	The operator $P$ is called Poisson transformation.


If $f\in H^{1}\left( \R,\Pi \right) $, then the same is true for the function $t\mapsto f(x+t)$. Hence we can rewrite the inequality in \cref{eqn:jr} as
\begin{equation}\label{eqn:jr'}
	 \log |P(f)(x) | \le P\left( \log |f| \right) (x). \tag{$J_{\R}'$} 
\end{equation}

If $f\ge 0$, then 
\begin{equation}\label{eqn:1}
\int_{\R}P(f)(x) \d x=\int_{\R}\left[ \int_{\R}f(x+t)\d x \right] \d \Pi(t)=\int_{\R}f(x) \d x.
\end{equation}
In particular, for any measurable $S\subset \R$, we have
\begin{equation}\label{eqn:2}
\int_{\R}\left( \int_{S}f\d \Pi_x \right) \d x= \int_{\R}P\left( \chi_S f \right) (x)\d x=\int_{S}f(x)\d x.
\end{equation} 


\begin{lemma} \label{lma:key}
	Let $p \in [1,+\infty)$, $\gamma>0$. Let $S\subset \R$ be a measurable set and $\Pi_x(S)\ge \gamma$ for all $x\in \R$ and a constant $\gamma>0$. If $f\in H^{p}(\R)$, then
	\begin{equation}\label{eqn:key}
		\int_{\R}|P(f)|^{p}\d x\le 2\left(\int_S|f|^{p}\d x\right)^{\gamma}\|f\|_p^{p\left( 1-\gamma \right) }.
	\end{equation}
\end{lemma}

\begin{proof}
	Fix $x\in \R$ and define
	\[
	k:=\Pi_x(S), k':=\Pi_x(S'), \]
	\[\lambda(A):=k^{-1}\Pi_x(A\cap S), \lambda'(A):=(k')^{-1}\Pi_x(A\cap S'), \forall A\subset \R.
	\] 
	Then we obtain
	\begin{align*}
		& p \log |P(f)(x)|\le p P\left( \log |f| \right) (x) \\
		= & p \int_S  \log|f|\d \Pi_x + p \int_{S'} \log|f|\d \Pi_x\\
		\le & k \log \Bigl( \int_S |f|^{p}\d \lambda \Bigr)+  k' \log \Bigl( \int_{S'}|f|^{p}\d \lambda \Bigr)\\
		= & k \log \frac{1}{k}+ k' \log \frac{1}{k'}+k\log \int_{S}|f|^{p} \d \Pi_x +k' \log \int_{S'}|f|^{p}\d \Pi_x\\
		\le & \log 2 +\gamma \log \int_S |f|^{p}\d \Pi_x +(k-\gamma)\log\int_{S}|f|^{p}\d \Pi_x+k' \log \int_{S'}|f|^{p}\d \Pi_x\\
		\le & \log 2+ \gamma \log \int_S |f|^{p}\d \Pi_x+ (k-\gamma+k') \log \int_{\R}|f|^{p}\d \Pi_x \\
		= & \log 2 +\gamma \log \int_S|f|^{p}\d \Pi_x+ (1-\gamma) \log \int _{\R}|f|^{p}\d \Pi_x
	.\end{align*}
Hence we obtain
\[
|P(f)(x)|^{p}\le 2\left( \int_S|f|^{p}\d \Pi_x\right)^{\gamma}\left( \int_{\R} |f|^{p}\d \Pi_x \right)^{1-\gamma}. 
\] 
Integrating both sides together with \cref{eqn:1,eqn:2} we obtain \cref{eqn:key}. 
\end{proof}

\section{End of the Proof}

Let $f \in L^{p}\left( \R \right) $, Poissson transformation $P(f)$ can be rewritten as convolution
\[
P(f)=f * k,
\] 
where $\displaystyle k:= \frac{1}{\pi (1+x^2)}$. Since $\displaystyle \widehat{k}(\xi)=e^{-|\xi|} /2\pi, \xi \in \R$, we have $\widehat{P(f)}(\xi)=e^{-|\xi|}\widehat{f}(\xi)$.

We introduce a lemma below, which is used in $L^{p}$-version ($p \in [1,\infty)$) of the Logvinenko-Sereda Theorem:
\begin{lemma} \label{lma:p}
	Let $p \in [1,\infty], f \in L^{p}(\R)$, then $P(f) \in L^{p}(\R)$ and 
	\[
	\|P(f)\|_p\le \|f\|_p.
	\] 
\end{lemma}

\begin{proof}
	The case $p=1$ or $\infty$ is simple, hence we suppose $1<p<\infty$ and $q:=p /(p-1)$. Then
	 \begin{align*}
		&|P(f)(x)|^{p}\le \left(\int \left( k(x-t) \right)^{1 /p}|f(t)|\left( k(x-t) \right) ^{1 /q}\d t \right)^{p}\\
		 \le & \left( \int k(x-t)|f(t)|^{p}\d t \right) \cdot \left( \int k(x-t) \d t \right) ^{p /q}=P(|f|^{p})(x)
	.\end{align*}
	Integrating this estimate with respect to $x$ and use the equality $\int P(|f|^{p})=\int |f|^{p}$, we obtain the desired result.
\end{proof}
\begin{lemma}\label{lma:inverse-control}
	If $f\in H^2\left( \R \right) $ and $\mathrm{ess}\,\mathrm{spec} f\subset (0,l)$, then
	\begin{equation}
		\|f\|_2\le e^{l}\|P(f)\|_2.
	\end{equation}
\end{lemma} 

\begin{proof}
	Since $f\in H^2(\R)$, we have $\widehat{P(f)}(\xi)=e^{-\xi}\widehat{f}(\xi)$. Hence
	\begin{equation}
		|\widehat{f}(\xi)|=e^{\xi}|\widehat{P(f)}(\xi)|\le e^{l}|\widehat{P(f)}(\xi)|,\quad \xi \in \R.
	\end{equation}
	Then use the Plancherel Theorem we obtain the desired inequality.
\end{proof}



Now we finish the proof of \cref{thm:main}.

\noindent{\itshape Proof of $(\mathrm{b})\Rightarrow (\mathrm{a})$.} Suppose $f\in  L^2(\R)$ and $\mathrm{ess}\,\mathrm{spec} f \subset (a,b),l=b-a$. We transform $f$ into a plus-function by
\[
\varphi:=f e^{-i a x}.
\] 
Then $\varphi \in H^2(\R)$ and $\mathrm{ess}\,\mathrm{spec}\varphi \subset (0,l)$, $|\varphi|\equiv |f|$. By \cref{lma:inverse-control},
\begin{equation}\label{eqn:mid}
\int_{\R}|f|^2\d x=\int_{\R}|\varphi|^2\d x \le e^{2l}\int_{\R}|P(\varphi)|^2. 
\end{equation}
Then by \cref{prp:equiv}  we know $\Pi_x(S)\ge \sigma$ for any $x\in \R$, where $\sigma>0$ does not depend on $x$. Hence we can apply \cref{lma:key} and obtain
\[
\|P(\varphi)\|_2^2\le 2\left( \int_S|\varphi|^2 \right) ^{\sigma}\|\varphi\|_2^{2(1-\sigma)}=2\left( \int_S|f|^2 \right) ^{\sigma} \|f\|_2^{2(1-\sigma)}.
\]
This estimate combined with \cref{eqn:mid} gives 
\[
\|f\|_{2}^{2\sigma}\le 2 e^{2l}\left( \int_S |f|^2 \right) ^{\sigma} \,\Rightarrow \, \|f\|_2^2\le (2e^{2l})^{1 /\sigma}\int_S|f|^2 \d x.
\] \hfill $\square$





















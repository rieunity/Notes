%TEX program = xelatex
\documentclass[12pt]{amsart}
\usepackage[dvipsnames]{xcolor}
\usepackage[colorlinks,
	linkcolor=Red,
	anchorcolor=blue,
	citecolor=ForestGreen
]
{hyperref}
\usepackage[utf8]{inputenc}
\usepackage[T1]{fontenc}
\usepackage{textcomp}
\usepackage{amsmath, amssymb,mathrsfs}
\usepackage{mathtools}

% A symbol for the quotient of two objects
%\usepackage{xfrac}
%\usepackage{faktor}
\newcommand{\sfrac}[2]{{\left.\raisebox{.2em}{$#1$}\middle/\raisebox{-.2em}{$#2$}\right.}}
\newcommand{\bsfrac}[2]{{\left.\raisebox{-.2em}{$#1$}\middle\backslash\raisebox{.2em}{$#2$}\right.}}
\usepackage{tikz-cd}
\usetikzlibrary{decorations.pathreplacing}
\usepackage{caption}
\usepackage{subcaption}
\usepackage{tkz-berge}

\usepackage[capitalize,nameinlink]{cleveref}[0.19]
 
\crefname{section}{section}{sections}
\crefname{subsection}{subsection}{subsections}
\Crefname{section}{Section}{Sections}
\Crefname{subsection}{Subsection}{Subsections}
 
\Crefname{figure}{Figure}{Figures}

 
\crefformat{equation}{\textup{#2(#1)#3}}
\crefrangeformat{equation}{\textup{#3(#1)#4--#5(#2)#6}}
\crefmultiformat{equation}{\textup{#2(#1)#3}}{ and \textup{#2(#1)#3}}
{, \textup{#2(#1)#3}}{, and \textup{#2(#1)#3}}
\crefrangemultiformat{equation}{\textup{#3(#1)#4--#5(#2)#6}}%
{ and \textup{#3(#1)#4--#5(#2)#6}}{, \textup{#3(#1)#4--#5(#2)#6}}{, and \textup{#3(#1)#4--#5(#2)#6}}

 
\Crefformat{equation}{#2Equation~\textup{(#1)}#3}
\Crefrangeformat{equation}{Equations~\textup{#3(#1)#4--#5(#2)#6}}
\Crefmultiformat{equation}{Equations~\textup{#2(#1)#3}}{ and \textup{#2(#1)#3}}
{, \textup{#2(#1)#3}}{, and \textup{#2(#1)#3}}
\Crefrangemultiformat{equation}{Equations~\textup{#3(#1)#4--#5(#2)#6}}%
{ and \textup{#3(#1)#4--#5(#2)#6}}{, \textup{#3(#1)#4--#5(#2)#6}}{, and \textup{#3(#1)#4--#5(#2)#6}}

%\oddsidemargin=-.0cm
%\evensidemargin=-.0cm
\textwidth=15cm
%\textheight=22cm
%\topmargin=0cm
\calclayout

% integral vraiable
\renewcommand{\d}{\,\mathrm{d}}

% figure support
\usepackage{import}
\usepackage{xifthen}
\newcommand{\incfig}[1]{%
  \def\svgwidth{\columnwidth}
  \import{./figures/}{#1.pdf_tex}
}
% Some shortcuts
\newcommand\N{\ensuremath{\mathbb{N}}}
\newcommand\R{\ensuremath{\mathbb{R}}}
\newcommand\Z{\ensuremath{\mathbb{Z}}}
\newcommand\T{\ensuremath{\mathbb{T}}}
\renewcommand\O{\ensuremath{\emptyset}}
\newcommand\Q{\ensuremath{\mathbb{Q}}}
\newcommand\C{\ensuremath{\mathbb{C}}}
\newcommand\CP{\ensuremath{\mathbb{CP}}}
\newcommand\CR{\ensuremath{\mathbb{CR}}}
\newcommand\Sph{\ensuremath{\mathbb{S}}}

% Some theorem environment settings
\newtheorem{theorem}{Theorem}
\newtheorem{proposition}[theorem]{Proposition}
\newtheorem{corollary}[theorem]{Corollary}
\newtheorem{lemma}[theorem]{Lemma}
\theoremstyle{definition}
\newtheorem{remark}{Remark}
\newtheorem*{question}{Question}
\newtheorem{definition}{Definition}
\newtheorem{exercise}{Exercise}
\newtheorem*{solution}{Solution}
\newtheorem*{claim}{Claim}
\newtheorem{example}{Example}
% Enumerate Style
\renewcommand{\labelenumi}{{\normalfont(\roman{enumi})}}

% Tensor
\usepackage{tensor}

\DeclareMathOperator{\order}{ord}
% bracket notations 
\DeclarePairedDelimiter\bra{\langle}{\rvert}
\DeclarePairedDelimiter\ket{\lvert}{\rangle}
\DeclarePairedDelimiterX\braket[2]{\langle}{\rangle}{#1 \delimsize\vert #2}

% Notations in differential geometry
% inner product
\DeclarePairedDelimiterX\ipd[2]{\langle}{\rangle}{#1\delimsize , #2}

% Notations in quantum field theory
% normal ordering
\newcommand{\normord}[1]{:\mathrel{#1}:}

% Real and imaginary parts
\DeclareMathOperator{\Realpart}{Re}
\renewcommand{\Re}{\Realpart}

\DeclareMathOperator{\Impart}{Im}
\renewcommand{\Im}{\Impart}


\DeclareMathOperator{\ddet}{det}
\renewcommand{\det}{\ddet}

% Groups
\DeclareMathOperator{\GL}{GL}
\DeclareMathOperator{\SL}{SL}

% Make a box
\newcommand{\makeabox}[2]{\textcolor{#1}{\fbox{\parbox{\textwidth}{#2}}}}
\newcommand{\makeredbox}[1]{\textcolor{Red}{\boxed{#1}}}

\usepackage[foot]{amsaddr}
%\numberwithin{equation}{section}
%\usepackage[notref,notcite]{showkeys}
\begin{document}
\title[Spectral inequality with explicit dependence on the scale]{Spectral inequality with explicit dependence on the scale}
\author[Y. Wang]{Yunlei Wang}
\email{yunlei.wang@math.u-bordeaux.fr}

%\address{Institut de Mathématiques de Bordeaux, Université de Bordeaux 351, cours de la Libération, F 33405 TALENCE cedex}

\maketitle



\begin{abstract}
	In this note, under more restrictive assumption of the potential, we give the spectral inequality with explicit dependence on the scale $l>0$ in the region $\Omega$ who satisfies some density conditions.
\end{abstract}
\tableofcontents
\section{Assumption and notation}
In this note, we always assume the potential  
\begin{equation}\label{assump}
V(x)=c|x|^{\beta },\quad x \in \R^{n}
\end{equation}
and $\beta >1$ for simplicity.\\
\textcolor{Gray}{ \fbox{\parbox{\textwidth}{In \cite{zhu2023spectral}, it has the following general assumption: Assume $V\in L^{\infty}_{\mathrm{loc}}\left( \R^{n} \right) $ satisfies the following two conditions:
\begin{enumerate}
	\item There exist positive constant $c_1$ and $\beta_1$ such that for all $x \in \R^{n}$,
		\begin{equation}
			c_1\left( |x|-1 \right) ^{\beta_1}\le V(x),
		\end{equation}
	\item We can write $V=V_1+V_2$ such that there exist positive constants $c_2$ and $\beta_2$ such that 
		\begin{equation}
			|V_1(x)|+|DV_1(x)|+|V_2(x)|^{\frac{4}{3}}\le c_2\left( |x|+1 \right) ^{\beta_2}.
		\end{equation}
\end{enumerate}}
}
}

Given $L>0$, we denote
 \[
\Lambda_L:= \left[ -\frac{L}{2},\frac{L}{2} \right] ^{n}.
\] 
Denote $\mathcal{B}_{r}(x) \subset \R^{n}$ be a ball with radius $r$ and center $x$. Denote  $\mathbb{B}_r(x)$ be a ball in $\R^{n+1}$.

There two kinds of sensor sets $\Omega$ which we should be careful about:
\begin{enumerate}
	\item let $l>0,\gamma \in (0,1)$ and  $\sigma >0$, the set $\Omega$ is said to be $(l,\gamma,\sigma )$-distributed if there exists a set of points $\left\{z_k: k\in \Z^{n}\right\} $ such that
		\begin{equation}\label{density-1}
			\Omega \cap (lk+\Lambda_l) \supset \mathcal{B}_{\gamma^{1+|k|^{\sigma }}l}(z _k)\tag{Type 1}
		\end{equation}
	\item sometimes the density condition \eqref{density-1} can be modified to the following form
		\begin{equation}
			\Omega \cap (lk+\Lambda_l) \supset \mathcal{B}_{\gamma^{1+|lk|^{\sigma }}l}(z_k).\label{density-2}\tag{Type 2}
		\end{equation} 
\end{enumerate}

In this note, we always use the type 1 density condition, i.e., the sensor set $\Omega$ who satisfies \eqref{density-1}. 


\section{Scaling}


We consider the solutions of
\begin{equation}
	-\Delta v+V v=0, \text{ in }\R^{n+1}.\label{eqn-4}
\end{equation}
Define the scaling function 
\[
       g(x)=lx, \quad x \in \R^{n+1}.
\] 
Then for any solution $v$ of \eqref{eqn-4} we define 
\begin{equation}
\widetilde{v}(x):=(v\circ g_{n+1})(x)=v(lx).
\end{equation}
Then
\begin{gather*}
	-\left( \Delta v \right) \circ g + \left( Vv \right) \circ g =0,\\
	-\frac{1}{l^2} \Delta \left( v\circ g \right) +\left( V\circ g \right) \left( v\circ g \right) =0,
\end{gather*}
\begin{equation}
	-\Delta \widetilde{v}+\widetilde{V}\widetilde{v}=0.\label{eqn-5}
\end{equation}
here we denote $\widetilde{V}(x)= l^2 V(lx)$. Then
\begin{equation}
	\widetilde{V}(x)= cl^{2+\beta }|x|^{\beta }
\end{equation}
by assumption.


Let $\delta \in (0,\frac{1}{2})$, $b=(0,\cdots ,0,-b_{n+1})$ and $b_{n+1}= \frac{\delta}{100}$. Define
\begin{equation*}
	\begin{aligned}
		W_1=&\left\{y \in \R^{n+1}_+\lvert |y-b|\le \frac{1}{4}\delta\right\}, \\
		W_2=& \left\{y \in \R^{n+1}_+\lvert |y-b|\le \frac{1}{2}\delta\right\}, \\
		W_3=& \left\{y \in \R^{n+1}_+\lvert |y-b|\le \frac{2}{3}\delta\right\}.
	\end{aligned} 
\end{equation*}
Then $W_1\subset W_2\subset W_3\subset \mathbb{B}_{\delta}$. Define
\[
W_j(z_i):=(z_i,0)+W_j, \quad  j=1,2,3,
\]
with $Q_L:= \Lambda_L \cap \Z^{n}$,
and
\[
P_j(L)= \bigcup_{i\in Q_{L}} W_j\left( z_i \right) \text{ and } D_{\delta}(L)=\bigcup_{i\in Q_L} \mathcal{B}_{\delta}(z_i).
\] 

\begin{lemma}\label{lma-1}
	Let  $\delta \in (0,\frac{1}{2})$. Let $v$ be the solution of \eqref{eqn-4} with $v(y)=0$ on the hyperplane $\left\{y\left| y_{n+1}=0\right.\right\} $. There exist $0<\alpha <1$ and $C>0$, depending only on $n$ such that
	\begin{equation}\label{eqn-7}
		\|v\|_{H^{1}\left( P_1(L) \right) }\le \delta^{-\alpha }\exp \left( C\left( 1+\mathcal{G}(V_1,V_2,9\sqrt{n} L \right)  \right) \|v\|^{\alpha }_{H^{1}\left( P_3(L) \right) }\|\frac{\partial v}{\partial y_{n+1}}\|^{1-\alpha }_{L^2\left( D_\delta(L) \right) }, 
	\end{equation}
	where 
	\begin{equation}\label{eqn-8}
		\mathcal{G}\left( V_1,V_2,L \right) =\|V_1\|^{\frac{1}{2}}_{W^{1,\infty}(\Lambda_L)}+\|V_2\|^{\frac{2}{3}}_{L^{\infty}(\Lambda_L)}.
	\end{equation}
\end{lemma}
Since $\widetilde{v}$ satisfy \eqref{eqn-8}, we substitute $\widetilde{v}$ into \eqref{eqn-7} and get
\begin{equation}\label{inq-1}	
	\|\widetilde{v}\|_{H^{1}\left( P_1(L) \right) }\le \delta^{-\alpha }\exp \left( C\left( 1+\mathcal{G}(\widetilde{V}_1,\widetilde{V}_2,9\sqrt{n} L \right)  \right) \|\widetilde{v}\|^{\alpha }_{H^{1}\left( P_3(L) \right) }\|\frac{\partial \widetilde{v}}{\partial y_{n+1}}\|^{1-\alpha }_{L^2\left( D_\delta(L) \right) } 
\end{equation}
with
\begin{equation}\label{eqn-10}
	\mathcal{G}(\widetilde{V}_1,\widetilde{V}_2,9\sqrt{n} L)=\|\widetilde{V}_1\|^{\frac{1}{2}}_{W^{1,\infty}(\Lambda_{9\sqrt{n} L})}+ \|\widetilde{V}_2\|^{\frac{2}{3}}_{\Lambda_{9\sqrt{n} L}}.
\end{equation}
Now we compute each term in \eqref{inq-1}:
\begin{equation}\label{term-1}
	\begin{aligned}
		\|\widetilde{v}\|_{H^{1}(P_1(L))}^2&= \int_{P_1(L)}|\widetilde{v}|^2\d x+ \int_{P_1(L)} \left| D_x \widetilde{v} \right| ^2\d x \\
						   &= \int_{P_1(L)}|v(lx)|^2\d x+ \int_{P_1(L)}\left| l (D_{x} v)(lx) \right|^2\d x\\
						   &= \frac{1}{l^{n+1}}\int_{lP_1(L)}|v(x)|^2\d x+ \frac{1}{l^{n-1}}\int_{lP_1(L)} \left| D_x v \right| ^2\d x\\
						   &\textcolor{Red}{\boxed{= \frac{1}{l^{n+1}}\|v\|^2_{L^2\left( lP_1(L) \right) }+\frac{1}{l^{n-1}}\|Dv\|^2_{L^2\left( lP_1(L) \right) }},} 
	\end{aligned} 
\end{equation}
\begin{equation}\label{term-2}
	\begin{aligned}
		\|\widetilde{v}\|^2_{H^{1}(P_3(L))}&= \frac{1}{l^{n+1}} \int_{lP_1(L)} |v(x)|^2\d x+ \frac{1}{l^{n-1}}\int_{lP_3(L)} |D_xv|^2\d x\\
						   &\textcolor{Red}{\boxed{= \frac{1}{l^{n+1}}\|v\|^2_{L^2(l P_1(L))}+\frac{1}{l^{n-1}}\|Dv\|^2_{L^2(lP_3(L))}},}
	\end{aligned}
\end{equation}
and
\begin{equation}\label{term-3}
	\begin{aligned}
		\|\frac{\partial \widetilde{v}}{\partial y_{n+1}} \|^2_{L^2\left( D_{\delta}(L) \right) }&= \int_{D_\delta(L)}\left| \frac{\partial \widetilde{v}}{\partial y_{n+1}}  \right| ^2\d x\\
													 &= \int_{D_{\delta}(L)}l^2\left| \frac{\partial v}{\partial y_{n+1}} (lx) \right|^2\d x\\
													 &= \frac{1}{l^{n-1}}\int_{lD_\delta(L)}\left| \frac{\partial v}{\partial y_{n+1}} (x) \right| ^2\d x\\
													 &\textcolor{Red}{\boxed{= \frac{1}{l^{n-1}}\|\frac{\partial v}{\partial y_{n+1}} \|^2_{L^2(lD_\delta(L)}}}.
	\end{aligned} 
\end{equation}
Similarly, we have
\begin{equation}\label{potential-1}
	\|\widetilde{V}_1\|_{W^{1,\infty}\left( \Lambda_{9 \sqrt{n} L} \right) }= \|V_1\|_{L^{\infty}(\Lambda_{9\sqrt{n} lL})}+l \|DV_1\|_{L^{\infty}\left( \Lambda_{9\sqrt{n} lL} \right) }
\end{equation}
and
\begin{equation}\label{potential-2}
	\|\widetilde{V}_2\|_{L^{\infty}(\Lambda_{9\sqrt{n} L})}= \|V_2\|_{L^{\infty}\left( \Lambda_{9\sqrt{n} lL} \right) }.
\end{equation}

Let $\delta\in (0,\frac{1}{2})$ and
\begin{equation*}
	\begin{aligned}
		R_1=\frac{1}{16}\delta & \text{ and }r_1=\frac{1}{32}\delta, \\
		R_2=3\sqrt{n} & \text{ and } r_2=\frac{1}{2},\\
		R_3=9\sqrt{n} & \text{ and } r_3=6\sqrt{n}.
	\end{aligned} 
\end{equation*}
Choose $R=2R_3=18\sqrt{n} $. Define
\begin{equation}
	X_1=\Lambda_L\times [-1,1] \text{ and }\widetilde{X}_{R_3}=\Lambda_{L+R_3}\times [-R_3,R_3].
\end{equation}
\begin{lemma}\label{lma-2}
	Let $\delta \in (0,\frac{1}{2})$. Let $v$ be the solution of \eqref{eqn-4} which is odd with repect to $y_{n+1}$. There exist $C>0$ depending only on $n$, $0<\alpha <1$ depending on $\delta$ and $n$ such that
	\begin{equation}\label{eqn-17}
		\|v\|_{H^{1}(X_1)}\le \delta^{-2\alpha _1}\exp\left( C\left( 1+\mathcal{G}\left( V_1,V_2,9\sqrt{n} L \right)  \right)  \right) \|v\|^{1-\alpha _1}_{H^{1}\left( \widetilde{X}_{R_3} \right) }\|v\|^{\alpha_1}_{H^{1}\left( P_1(L) \right) },
	\end{equation}
	where $\mathcal{G}(V_1,V_2,L)$ is given by \eqref{eqn-10}. 
\end{lemma}
To explicitly denote $\alpha _1$, we define $\psi(\widehat{r})=e^{-s\widehat{r}}$ and then
\begin{equation}
	\kappa_1= \log \frac{96 \sqrt{n} }{\delta} \text{  and }\kappa_2= \psi\left( \frac{r_0}{3} \right) -\psi\left( \frac{r_0}{6} \right).
\end{equation}
Now we have
\begin{equation}
	0<\alpha_1= \frac{|\kappa_2|}{|\log \delta|+\log (96\sqrt{n})+|\kappa_2| }<1.
\end{equation}
The same as in Lemma \ref{lma-1}, substitute $\widetilde{v}$ into \eqref{eqn-17} and get
\begin{equation}
	\|\widetilde{v}\|_{H^{1}(X_1)}\le \delta^{-2\alpha _1}\exp\left( C\left( 1+\mathcal{G}\left( \widetilde{V}_1,\widetilde{V}_2,9\sqrt{n} L \right)  \right)  \right) \|\widetilde{v}\|^{1-\alpha _1}_{H^{1}\left( \widetilde{X}_{R_3} \right) }\|\widetilde{v}\|^{\alpha_1}_{H^{1}\left( P_1(L) \right) }.\label{inq-2}
\end{equation}
Now we compute each term in \eqref{inq-2}:
\begin{equation}\label{term-4}
	\makeredbox{\|\widetilde{v}\|_{H^{1}(X_1)}^2=\frac{1}{l^{n+1}}\|v\|^2_{L^2(lX_1)}+\frac{1}{l^{n-1}}\|Dv\|^2_{L^2(lX_1)},}
\end{equation}
\begin{equation}
	\makeredbox{\|\widetilde{v}\|^2_{H^{1}\left( \widetilde{X}_{R_3} \right) }= \frac{1}{l^{n+1}}\|v\|^2_{L^2(l\widetilde{X}_{R_3})}+\frac{1}{l^{n-1}}\|v\|^2_{L^2\left( l\widetilde{X}_{R_3} \right) },}
\end{equation}
and
\begin{equation}
	\makeredbox{\|\widetilde{v}\|^2_{H^{1}\left( P_1(L) \right) }=\frac{1}{l^{n+1}}\|v\|^2_{L^2(lP_1(L))}+\frac{1}{l^{n-1}}\|v\|^2_{L^2\left( lP_1(L) \right) }.}
\end{equation}

\section{Ghost dimension}
We denote $H=-\Delta+V$ and let $\phi \in \mathcal{E}_\lambda(H) $ satisfying
\begin{equation}
	\phi=\sum_{\lambda_k\le \lambda} \alpha _k\phi_k.
\end{equation}
Define
\begin{equation}\label{eqn-28}
	\Phi(x,x_{n+1})=\sum_{0<\alpha _k\le \lambda} \alpha _k \phi_k(x) \frac{\sinh \left( \sqrt{\lambda_k} x_{n+1} \right) }{\sqrt{\lambda_k} }.
\end{equation}
Then $\Phi(x,x_{n+1})$ satisfies \eqref{eqn-4}.

\begin{proposition}\label{prop-3}
	There exists a constant $\widehat{C}$ depending on $\beta_1,c_1,\beta_2,c_2$ such that for all $\lambda\ge \lambda_1$ and $\phi \in \mathcal{E}_\lambda(H) $, we have
	\begin{equation}
		\|\phi\|^2_{H^{1}\left( \R^{n}\backslash \mathcal{B}_{\widehat{C}\lambda^{1 /\beta_1}} \right) }\le \frac{1}{2}\|\phi\|_{L^2(\R^{n})}.
	\end{equation}
\end{proposition}
According to this proposition, we have
\begin{equation}\label{eqn-30}
	\|\Phi\|^2_{H^{1}\left( \R^{n}\times (-l,l) \right) }\le 2 \|\Phi\|^2_{H^{1}\left( \mathcal{B}_{\widehat{C}\lambda^{1 /\beta_1}} \times (-l,l)\right) }.
\end{equation}

\begin{lemma}\label{lma-4}
	Let $\phi \in \mathcal{E}_\lambda(H) $ and $\Phi$ be given in \eqref{eqn-28}. For any $\rho >0$, we have 
	\begin{equation}
		2\rho \|\phi\|^2_{L^2\left( \R^{n} \right) }\le \|\Phi\|^2_{H^{1}\left( \R^{n}\times (-\rho,\rho ) \right) }\le 2\rho \left( 1+\frac{\rho ^2}{3}(1+\lambda) \right) e^{2\rho \sqrt{\lambda} }\|\phi\|^2_{L^2(\R^{n})}.
	\end{equation}
\end{lemma}


\section{Large scale}
In this section, we try to prove the spectral inequality in the large scale, i.e., $l\ge1$.

From \eqref{term-1} and \eqref{term-2}, we obtain
\begin{equation}\label{term-1'}
	\|\widetilde{v}\|^2_{H^{1}\left( P_1(L) \right) }\ge \frac{1}{l^{n+1}} \|v\|^2_{H^{1}\left( lP_1(L) \right) }
\end{equation}
and
\begin{equation}\label{term-2'}
	\|\widetilde{v}\|^2_{H^{1}(P_3(L))} \le  \frac{1}{l^{n-1}}\|v\|^2_{H^{1}(P_3(L))}.
\end{equation}
Substituting \eqref{term-1'}, \eqref{term-2'} and \eqref{term-3} into \eqref{inq-1}, we obtain
\begin{equation}\label{inq-1'}
	\|v\|_{H^{1}\left( lP_1(L) \right) }\le l  \delta^{-\alpha }\exp \left( C\left( 1+\mathcal{G}\left( \widetilde{V}_1,\widetilde{V}_2,9\sqrt{n} L \right)  \right)  \right) \|v\|^{\alpha }_{H^{1}\left( lP_3(L) \right) }\|\frac{\partial v}{\partial y_{n+1}}\|^{1-\alpha }_{L^2\left( lD_\delta(L) \right) }. 
\end{equation}
Here we need to esitmate $\mathcal{G}\left( \widetilde{V}_1,\widetilde{V}_2,9\sqrt{n} L \right) $, by \eqref{potential-1} and \eqref{potential-2} we obtain
\begin{equation}\label{potential'}
	\mathcal{G}\left( \widetilde{V}_1,\widetilde{V}_2,9\sqrt{n} L \right) \le l^{\frac{1}{2}} \left( \|V_1\|^{\frac{1}{2}}_{W^{1,\infty}\left( \Lambda_{9\sqrt{n} lL} \right) }+\|V_2\|^{\frac{2}{3}}_{L^{\infty}\left( \Lambda_{9\sqrt{n} lL} \right) } \right) . 
\end{equation}
Similarly, we can derive the following from \eqref{inq-2}
\begin{equation}\label{inq-2'}
	\|v\|_{H^{1}(lX_1)}\le l \delta^{-2\alpha_1}\exp\left( C\left( 1+\mathcal{G}\left( \widetilde{V}_1,\widetilde{V}_2,9\sqrt{n}L  \right)  \right)  \right) \|v\|^{1-\alpha_1}_{H^{1}\left( l\widetilde{X}_{R_3} \right) } \|v\|^{\alpha _1}_{H^{1}(lP_1(L)}.
\end{equation}

We assume the potential $V$ satisfy \eqref{assump},  $V_1=V$ and $V_2=0$.

\noindent \textit{Proof of the case $l>1$.}
Let $lL=2 \left\lceil \widehat{C}\lambda^{1 /\beta } \right\rceil +1$, then it is easy to see that $\mathcal{B}_{\widehat{C}\lambda ^{1 /\beta _1}}\subset \Lambda_{lL}=l\Lambda_{L}$. Now we decompose $\Lambda_{lL}$ as
\begin{equation}
	\Lambda_{lL}=\bigcup_{k\in \Lambda_{L}\cap \Z^{n}} \left( lk+\left[ -\frac{l}{2},\frac{l}{2} \right]  \right).
\end{equation}
Observe that $l|k|\le \sqrt{n} \left\lceil \widehat{C} \lambda ^{1 /\beta } \right\rceil $ for each $k\in \Lambda_{L}\cap \Z^{n}$. Let $\gamma \in (0,\frac{1}{2})$ be as in \eqref{density-1} and
\begin{equation}
	\delta:= \gamma ^{1+\left( \frac{1}{l}\sqrt{n} \left\lceil \widehat{C}\lambda^{\frac{1}{\beta }} \right\rceil  \right) ^{\sigma }}\le \gamma^{1+|k|^{\sigma }} \text{ for all }k \in \Lambda_L \cap \Z^{n}.
\end{equation}

Now we show an interpolation inequality using (\ref{inq-1'},~\ref{potential'},~\ref{inq-2'})
\begin{equation}
	\begin{aligned}
		\|\Phi\|_{H^{1}(lX_1)}\le & \delta^{-2\alpha_1} \exp \left( C\left( 1+l^{\frac{1}{2}}\|V\|_{W^{1,\infty}\left( \Lambda_{9\sqrt{n} lL} \right) }^{\frac{1}{2}} \right)  \right)\|\Phi\|^{1-\alpha _1}_{H^{1}\left( l\widetilde{X}_{R_3} \right) } \|\Phi\|^{\alpha_1}_{H^{1}\left( lP_1(L) \right) }\\
		\le & \delta^{-2 \alpha_1-\alpha \alpha_1}\exp\left( C\left( 1+l^{\frac{1}{2}}\|V\|^{\frac{1}{2}}_{W^{1,\infty}(\Lambda_{9\sqrt{n} lL})} \right)  \right)\|\Phi\|^{\alpha \alpha_1}_{H^{1}\left( lP_3(L) \right) }\|\frac{\partial \Phi}{\partial y_{n+1}} \|^{\alpha_1(1-\alpha)}_{L^2\left( lD_{\delta}(L) \right) }\|\Phi\|^{1-\alpha_1}_{H^{1}\left( l \widetilde{X}_{R_3} \right) } \\
		\le& \delta^{-3\alpha_1}\exp \left( C\left( 1+ l^{\frac{1}{2}}\|V\|^{\frac{1}{2}}_{W^{1,\infty}\left( \Lambda_{9\sqrt{n} lL} \right) } \right)    \right) \|\phi\|^{\widehat{\alpha}}_{L^2\left( lD_\delta(L) \right) }\|\Phi\|^{1-\widehat{\alpha}}_{H^{1}\left( l\widetilde{X}_{R_3} \right) }
	\end{aligned} 
\end{equation}
where $\widehat{\alpha}=\alpha_1(1-\alpha)$ and we have used the facts $lP_3(L)\subset l \widetilde{X}_{R_3}$ and $\frac{\partial \Phi}{\partial y_{n+1}}(\cdot ,0)=\phi$, and absorbed the polynomial factor $l$ by taking $C$ large enough (which is not dependent on $l$). Since $\alpha_1\approx \widehat{\alpha}\approx \frac{1}{|\log \delta|}$ for any $\delta \in (0,\frac{1}{2})$ and then $\delta ^{-3\alpha_1}\le C$, we obtain
\begin{equation}
	\|\Phi\|_{H^{1}(lX_1)}\le \exp \left( C\left( 1+l^{\frac{1}{2}}\|V\|^{\frac{1}{2}}_{W^{1,\infty}\left( \Lambda_{9\sqrt{n} lL} \right) } \right)  \right) \|\phi\|^{\widehat{\alpha }}_{L^2\left( \Omega \cap \Lambda_{lL} \right) }\|\Phi\|^{1-\widehat{\alpha }}_{H^{1}\left( l \widetilde{X}_{R_3} \right) }
\end{equation}
where we have also used the fact $lD_{\delta}(L)\subset \Omega \cap \Lambda_{lL}$.

Remember we assume $V=c|x|^{\beta}$, hence we have
\begin{equation}
	C\left( 1+l^{\frac{1}{2}}\|V\|^{\frac{1}{2}}_{W^{1,\infty}\left( \Lambda_{9\sqrt{n} lL} \right) } \right) \le C\left( 1+l^{\frac{1}{2}} \left( 9\sqrt{n} lL \right) ^{\frac{\beta }{2}} \right)\le C_* l^{\frac{1}{2}} \lambda ^{\frac{1}{2}}. 
\end{equation}
\makeabox{Gray}{
		If we do not assume $V=c|x|^{\beta }$ and just use the general assumption, then the above estimate should be derived from \eqref{potential'} and becomes
\begin{equation}
	\mathcal{G}\left( \widetilde{V}_1,\widetilde{V}_2,9\sqrt{n} L \right) \le C_*l^{\frac{1}{2}} \lambda ^{\frac{\beta_2}{\beta_1}}.
\end{equation}
But the constant $C_{*}$ now depends on  $l$ also. The reason is we do not give the explicit dependence relation of $\widehat{C}$ with respect to $c_1,c_2,\beta_1,\beta_2$.
}
Then we have
\begin{equation}\label{eqn-42}
	\|\Phi\|_{H^{1}(lX_1)}\le \exp\left( C_* l^{\frac{1}{2}}\lambda^{\frac{1}{2}} \right) \|\phi\|^{\widehat{\alpha }}_{L^2\left( \Omega \cap \Lambda_{lL} \right) }\|\Phi\|^{1-\widehat{\alpha }}_{H^{1}\left( l \widetilde{X}_{R_3} \right) }.
\end{equation}

Applying $\rho =lR_3$ and $\rho =l$ in Lemma \ref{lma-4} for upper and lower bounds, respectively, we obtain
\begin{equation}
	\frac{\|\Phi\|^2_{H^{1}\left( \R^{n}\times (-lR_3,lR_3) \right) }}{\|\Phi\|^2_{H^{1}\left( \R^{n}\times (-l,l) \right) }}\le \frac{2l R_3 \left( 1+ \frac{(lR_3)^2}{3}(1+\lambda) \right) \exp(2lR_3\sqrt{\lambda} )}{2l}\le \exp\left(C_2 l \sqrt{\lambda}   \right).  
\end{equation}
With the aid of \eqref{eqn-30} and the fact $\mathcal{B}_{\widehat{C}\lambda ^{1 /\beta }}\subset \Lambda_{lL}$, we get
\begin{equation}
	\|\Phi\|_{H^{1}\left( \R^{n}\times (-lR_3,lR_3) \right) }\le \exp\left( \frac{1}{2}C_2 \sqrt{\lambda}  \right) \|\Phi\|_{H^{1}\left( \R^{n}\times (-l,l) \right) }\le \sqrt{2}  \exp \left( \frac{1}{2}C_2 \sqrt{\lambda}  \right) \|\Phi\|_{H^{1}\left( \Lambda_{lL}\times (-l,l) \right) }.
\end{equation}
Note that $lX_1=\Lambda_{lL}\times (-l,l)$, by \eqref{eqn-42} we obtain
\begin{equation}	
	\|\Phi\|_{H^{1}(\R^{n}\times (-lR_3,lR_3))}\le \exp\left( C_3 l^{\frac{1}{2}}\lambda^{\frac{1}{2}} \right) \|\phi\|^{\widehat{\alpha }}_{L^2\left( \Omega \cap \Lambda_{lL} \right) }\|\Phi\|^{1-\widehat{\alpha }}_{H^{1}\left( l \widetilde{X}_{R_3} \right) }.
\end{equation}
Since $l \widetilde{X}_{R_3}\subset \R^{n}\times \left( -l R_3,lR_3 \right) $, it follows that
\begin{equation}
	\|\Phi\|_{\R^{n}\times (-lR_3,lR_3)}\le \exp \left( \widehat{\alpha }^{-1}C_3 l^{\frac{1}{2}}\lambda^{\frac{1}{2}} \right). 
\end{equation}
Recall $\widehat{\alpha }^{-1}\approx \alpha ^{-1}_1\approx |\log \delta |\approx \frac{1}{l^{\sigma }}|\log \gamma|\lambda ^{\frac{\sigma }{\beta }}$. It follows that
\begin{equation}
	\|\Phi\|_{H^{1}\left( \R^{n}\times (-lR_3,lR_3) \right) }\le \left( \frac{1}{\gamma} \right) ^{C l^{\frac{1}{2}}\lambda^{\frac{\sigma }{\beta }+\frac{1}{2}}}\|\phi\|_{L^2\left( \Omega \cap \Lambda_{lL} \right) }.
\end{equation}
Finally, using the lower bound in Lemma~\ref{lma-4} with $\rho =lR_3$, we obtain
\begin{equation}
	\|\phi\|_{L^2\left( \R^{n} \right) }\le \left( \frac{1}{2lR_3} \right) ^{\frac{1}{2}}\|\Phi\|_{H^{1}\left( \R^{n}\times (-lR_3,lR_3) \right) }\le \left( \frac{1}{\gamma} \right) ^{C l^{\frac{1}{2}}\lambda^{\frac{\sigma }{\beta }+\frac{1}{2}}}\|\phi\|_{L^2\left( \Omega \cap \Lambda_{lL} \right) }.
\end{equation}
This complets the proof of case $l\ge 1$.\hfill  $\square$

\section{Small scale}
In this section, we try to prove the spectral inequality in the small scale, i.e., $l<1$.

From \eqref{term-1} and \eqref{term-2}, we obtain
\begin{equation}\label{eqn-32}
	\|\widetilde{v}\|^2_{H^{1}\left( P_1(L) \right) }\ge \frac{1}{l^{n-1}}\|v\|^2_{H^{1}\left( lP_1(L) \right) }
\end{equation}
and
\begin{equation}\label{eqn-33}
	\|\widetilde{v}\|^2_{H^{1}\left( P_3(L) \right) }\le \frac{1}{l^{n+1}}\|v\|^2_{H^{1}\left( lP_3(L) \right) }.
\end{equation}
Substituting \eqref{eqn-32}, \eqref{eqn-33} and \eqref{term-3} into \eqref{inq-1}, we obtain
\begin{equation}
	\|v\|_{H^{1}\left( lP_1(L) \right) } \le\frac{1}{l^{1-\alpha}} \delta^{-\alpha }\exp \left( C\left(     1+\mathcal{G}\left( \widetilde{V}_1,\widetilde{V}_2,9\sqrt{n} L \right)  \right)  \right    ) \|v\|^{\alpha }_{H^{1}\left( lP_3(L) \right) }\|\frac{\partial v}{\partial y_{n+1}}\|^    {1-\alpha }_{L^2\left( lD_\delta(L) \right) }.   
\end{equation}
Similarly, we obtain the variance of \eqref{inq-2}
\begin{equation}
	\|\widetilde{v}\|_{H_1\left( lX_1 \right) }\le \frac{1}{l}\cdots .
\end{equation}
\begin{remark}
These two inequalities with the polynomial factors of $\frac{1}{l}$ in the cost constants are bad for our final purpose. The reason is that we choose the density condition \eqref{density-1}. This implies that we can only consider the Generalized Grushin operator under for $y \in \T^{m}$ case. 

The same situation appears if we use the density condition \eqref{density-2}. In my opinion, it is due to the cutoff property which Carleman estimates used. More precisely, if $l\to 0$, then the cutoff function must be sharper and leads to a large deviation, hence the carleman estimates fails.

Hence for $y\in \R^{m}$ case, we must change the method of the proof. 
\end{remark}
\bibliographystyle{plain}
\bibliography{mybib}
\end{document}



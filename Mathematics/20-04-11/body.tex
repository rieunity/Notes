\section{Lecture Notes 1 for 247A}
\subsection{什么是调和分析}
第一份讲义介绍了这份讲义(总共6个note)讲述的内容:
\begin{framed}
\begin{center}
  \textit{实变量调和分析理论(theory of real-variable harmonic analysis)以及研究该理论的基本工具}
\end{center}
\end{framed}
\begin{enumerate}
  \item 什么是调和分析?
  \item [] 关于某个定义域或者类似对象(测度,分布,定义域的子集,或者从一个定义域到另一个定义域的映射)上的函数的定量(quantitative)研究.
  \item 典型的例子?
  \item [] 例如一个从某个Banach函数空间$V$到另一个空间$W$的算子 $T$,这个算子并非定义在全空间 $V$ 上而是定义在它的一个稠密子集上.考虑这样一个定性(qualitative)问题:能否将$T$连续拓展到整个空间 $V$ 上?实际上若$T$是线性的,如果我们能够建立
    \[
    \|Tf\|_{W}\le C\|f\|_{V}
    \] 
    这样一种定量关系,那么就可以作一个唯一连续延拓.这里就是把一个定性问题转化成了一个定量问题来研究,这个定量问题就是调和分析中的一个典型问题.
\end{enumerate}

从这个典型的例子可以知道:不等式在调和分析中有着非常重要的地位,甚至是核心的地位(就我目前读这个讲义的感觉).在介绍一些不等式的建立时,Tao在脚注中说了一句非常经典的话,来自于他自身的经验:
\begin{framed}
\begin{center}
  \textit{Algebra draws its power from modularity, abstraction and identities; analysis draws its power from robustness, physical intuition and estimates.}
\end{center}
\end{framed}

\subsection{建立不等式的方法}

对于$1\le p\le \infty$,有下述三角不等式成立
\begin{framed}
  \noindent\textbf{Triangle inequality}
  \begin{equation}
  \|f+g\|_{p}\le \|f\|_{p}+\|g\|_{p}.
\end{equation}
\end{framed}
证明这个不等式的时候,我们会用到一些处理不等式的一般技巧.
\begin{enumerate}
  \item 根据$f$ 和$g$在不等式两边的齐次性(对称性的一种),可以设$\|f\|_{p}=1-\theta$,$\|g\|_{p}=\theta$, $0< \theta< 1$($\theta=$0和1的情形是平凡的).
  \item  再将$f$和 $g$ 的范数归一化,设$F=\frac{f}{1-\theta}$,$G=\frac{g}{\theta}$.则不等式转化为
    \[
      \|(1-\theta)F+\theta G\|_{p}\le 1,
    \] 
    其中$\|F\|_p=1, \|G\|_{p}=1$.
  \item 根据 $z\mapsto |z|^{p}$ 在$p\ge 1$时的凸性可知
    \[
      \left| (1-\theta)F(x)+\theta G(x) \right| ^{p}\le (1-\theta)|F(x)|^{p}+\theta |G(x)|^{p}.
    \] 
    对上式积分即可.
\end{enumerate}
概括起来就是,根据不等式的对称性,把不等式的选取范围缩小,限定在尽可能小的子集内,实际就是空间中的商作用.如有必要,对函数进行归一化.简化后的不等式往往更容易处理,在几何上有一些直观的特性(在这里就是凸性).
\begin{framed}
  \noindent \textbf{H\"{o}lder's inequality}
  \begin{equation}
    \|fg\|_{r}\le \|f\|_p \|g\|_q,
  \end{equation}
  其中$0<p,q,r\le \infty$,$\frac{1}{r}=\frac{1}{p}+\frac{1}{q}$.
\end{framed}
这里除了总的齐次性之外,每个函数各自单独也有齐次性(separate homogeneity symmetry),所以可以直接设$\|f\|_{p}=\|g\|_{q}=1$.令$F:=|f|^{p},G:=|g|^{q},\theta = \frac{r}{q}$,则原不等式化为
\[
\int_{X}F^{1-\theta}G^{\theta}\le 1, \text{ 其中 } \int_{X}F=\int_{X}G=1.
\] 
这里再次利用 $\ln x$ 的凸性可得逐点估计:
\[
  F^{1-\theta}G^{\theta}\le (1-\theta)F(x)+\theta G(x).
\] 
对上式进行积分即可.
\begin{framed}
  \noindent\textbf{Log-convexity of $L^{p}$ norms}
  \begin{equation}
    \|f\|_{r}\le \|f\|_{p}^{1-\theta}\|f\|_{q}^{\theta},
  \end{equation}
  其中$0<p<q<\infty,0<\theta<1,\frac{1}{r}=\frac{1-\theta}{p}+\frac{\theta}{q}$.
\end{framed}
用类似的方法证明这个不等式的时候,需要注意到该不等式有关于$f$ 和$\mu$ 的齐次对称性,从而可以设
$\|f\|_p=\|f\|_q=1$.当然这个不等式可以由H\"{o}lder不等式简单推导得到(实际上这两个不等式是等价的).

讲义还给出了log-convexity的一个不常见的证明,这个证明没有用到任何的逐点凸性估计(pointwise convexity estimate),而是用到了``分治-合并''策略(``divide and conquer'' strategy)以及非常优雅(并且相当无赖)的``张量幂技巧''(``tensor power trick'').仍然假设$\|f\|_p=\|f\|_q=1$,我们将函数$f$ 分成宽广平坦函数和窄高函数
\[
f=f 1_{|f|\le 1}+f 1_{|f|>1}.
\] 
进而
\[
\|f\|_r^{r}=\int_{|f|\le 1}|f|^{r}+\int_{|f|>1}|f|^{r}.
\] 
当$|f|\le 1$ 的时候,由$r>p$ 可得$|f|^{r}\le |f|^{p}$;当$|f|>1$ 的时候,由$r<q$可得 $|f|^{r}\le |f|^{q}$.从而
\[
\|f\|_r^{r}\le \int_X |f|^{p}+\int_{X}|f|^{q}=2.
\] 
下面就是张量幂技巧的使用:设$M$ 为一个正整数,用测度空间$(X^{M},\mathcal{B}^{\oplus M},\mu^{\oplus M})$ 代替$(X,\mathcal{B},\mu)$,用函数$f^{\oplus M}:X^{M}\to \C$代替函数$f$,其中
 \[
   f^{\oplus M}(x_1,x_2,\cdots,x_M):=f(x_1)f(x_2)\cdots f(x_M).
\] 
易知
\begin{equation*}
  \begin{aligned}
    \|f^{\oplus M}\|_{L^{p}(X^{M})}&=\|f\|^{M}_{L^{p}(X)}=1;\\
    \|f^{\oplus M}\|_{L^{q}(X^{M})}&=\|f\|^{M}_{L^{q}(X)}=1;\\
    \|f^{\oplus M}\|_{L^{r}(X^{M})}&=\|f\|_{L^{r}(X)}^{M}.
  \end{aligned}
\end{equation*}
再次利用前面的论证,这次是应用到函数$f^{\oplus M}$ 上,得到
\[
  \|f^{\oplus M}\|^{r}_{L^{r}(X^{M}}\le 2.
\] 
进而
\[
  \|f\|^{r}_{L^{r}(X)}\le 2^{1 /M}.
\] 
令$M\to \infty$,我们就得到了$\|f\|_r\le 1$.
\subsection{一些简单但是重要的结论}
 \begin{framed}
      \noindent 1. 设$\mu\left( X \right)<\infty $,$f$ 是$X$上的函数,则更高的$L^{p}$ 范数控制更低的范数:
      \begin{equation}
	\|f\|_p\le \|f\|_q\mu(X)^{\frac{1}{p}-\frac{1}{q}},
      \end{equation}
      其中$0<p\le q\le \infty$.

     \noindent 2. $l^{p}$ 和与$L^{p}$ 和的可交换性:
     \begin{equation}
       \|\left( \sum_{n}^{} |f_n|^{p} \right) ^{1 /p}\|=\left( \sum_{n}^{} \|f\|_{L^{p}}^{p} \right) ^{1 /p}.
     \end{equation}

     \noindent 3. 对任意$0<p,q<\infty$:
     \begin{equation}
       \| |f|^{p}\|_{L^{q}}=\| f\|^{p}_{L^{pq}}.
     \end{equation}
\end{framed}

\subsection{Lorentz Spaces}
考虑weak $L^{p}$和$\|\cdot \|_{L^{p}}$ :
\begin{equation*}
  \begin{aligned}
    \|f\|_{L^{p,\infty}} & = \|\lambda \mu\left( \left\{ |f|\ge \lambda \right\}  \right) ^{1 /p}\|_{L^{\infty}(\R^{+},\frac{\mathrm{d}\lambda}{\lambda})}\\
    \|f\|_{L^{p}} & = p^{1 /p}\|\lambda\mu\left( \left\{ |f|\ge \lambda \right\}  \right) ^{1 /p}\|_{L^{p}(\R^{+}, \frac{\mathrm{d}\lambda}{\lambda})}.
  \end{aligned}
\end{equation*}
由这两个式子可以让我们推广出一个新的拟范数 Lorentz norm $L^{p,q}(X,\mu)$,$0<p<\infty,0<q\le \infty$:
\[
  \|f\|_{L^{p,q}(X,\mu)}:=p^{1 /q}\|\lambda \mu\left( \left\{ |f|\ge \lambda \right\}  \right) ^{1 /p}\|_{L^{q}(\R^{+}, \frac{\mathrm{d}\lambda}{\lambda})}.
\]

除了$L^{\infty,\infty}=L^{\infty}$ 这一特殊情况, Lorentz norms中$p$ 都不会取$\infty$.对于$q$ 来说,按照重要程度依次递减,只会用到$q=p,q=\infty,q=1,q=2$.
可以通过简单的计算知对于一个高$H$ 宽$W$ 的函数,它的$L^{p,q}$ norm是 $(p /q)^{\frac{1}{q}}HW^{\frac{1}{p}}$,其中$0<p<\infty,0<q\le \infty$.

\begin{definition}
  \begin{enumerate}
    \item []
    \item  如果一个支撑为$E$的函数$f$,几乎处处满足$|f(x)|\le H$并且$\mu(E)\le W$,则称$f$ 是一个高$H$ 宽$W$ 的\textit{ sub-step function}.(因此 $|f|\le H 1_{E}$.)
    \item 如果函数$f$ 满足几乎处处$|f(x)|\sim H$,以及$\mu(E)\sim W$,则称 $f$ 是一个高$H$ 宽$W$ 的 \textit{quasi-step function} .(因此 $|f|\sim H 1_E$.)
  \end{enumerate}
\end{definition}

\begin{remark}\label{rmk1-1}
一个高$1$宽 $W$的 sub-step function 按照二进制展开,总能分解成真正的 step functions 的和:
\[
f=\sum_{k=1}^{\infty} 2^{-k}f_k,
\] 
其中$f_k$是高 $1$宽 $H$的 step functions.
因此, 上述定义的两个函数类型都可以由 step function 逼近.
\end{remark}

\begin{theorem}[Characterisation of $L^{p,q}$]\label{thm1-1}
  设$f$ 为一个函数,$0<p<\infty,1\le q\le \infty$,设$0<A<\infty$.那么下述5个在相差一个常数的情况下是等价的:
  \begin{enumerate}
    \item  $\|f\|_{L^{p,q}}\lesssim_{p,q}A$.
    \item  存在一个分解 $f=\sum_{m\in \Z}^{} f_m$,其中$f_m$ 是支撑互不相交,高$2^{m}$ 宽$0<W_m<\infty$ 的 quasi-step function,并且
      \begin{equation}
	\|2^{m}W^{1 /p}_m\|_{l^{q}_m(\Z)}\lesssim_{p,q}A.
      \end{equation}
      这里$l^{q}_m$ 中的下标$m$ 是表示$l^{q}$ norm 针对的变量.
    \item 存在一个逐点界 $|f|\le \sum_{m\in \Z}^{} 2^{m} 1_{E_m}$, $E_m$ 满足
      \begin{equation}
	\|2^{m}\mu(E_m)^{1 /p}\|_{l^{q}_m(\Z)}\lesssim_{p,q}A.
      \end{equation}
    \item 存在一个分解$f=\sum_{n \in \Z}^{} f_{n}$,其中$f_n$ 是支撑互不相交,高$0<H_n<\infty$ 宽 $2^{n}$ 的 quasi-step function,$H_n$ 关于$n$单调不增, 在$f_n$ 的支撑上有$H_{n+1}\le |f_n|\le H_n$,以及
      \begin{equation}\label{1-2}
	\|H_n 2^{n /p}\|_{l^{q}_n(\Z)}\lesssim_{p,q}A.
      \end{equation}
  \item 一个逐点界 $|f|\le \sum_{n\in \Z}^{} H_n 1_{E_n}$, $E_n$满足 $\mu(E_n)\lesssim_{p,q}2^{n}$ 并且{\normalfont (\ref{1-2})}式成立.
  \end{enumerate}
\end{theorem}
\begin{proof}
 由齐次对称性可设$A=1$.$(b)\Rightarrow (c)$和 $(d)\Rightarrow (e)$ 式显然的.下面说明$(a)\Rightarrow(b)$.

 设
 \[
 f_m:=f 1_{2^{m-1}<|f|\le 2^{m}},
 \] 
  \[
    W_m:=\mu\left( \left\{ 2^{m-1}<|f|\le 2^{m} \right\}  \right) .
  \]
  这种分解方式被称为``vertically dyadic layer cake decomposition''.可以验证
  \[
    2^{m}W_m^{1 /p}\lesssim_{p,q} \|\lambda \mu\left( \left\{ |f|>\lambda \right\}  \right) ^{1 /p}\|_{L^{q}\left( [2^{m-2},2^{m-1}], \frac{\mathrm{d}\lambda}{\lambda} \right) }
  \]
  然后再对上式进行$l^{q}$ 求和即可.

  类似地,为了得到$(a)\Rightarrow (d)$,定义
  \[
    H_n:=\inf \left\{ \lambda:\mu\left( \left\{ |f|>\lambda \right\}  \right) \le 2^{n-1} \right\} ,
  \] 
  注意到这是一个关于$n$ 的单调不增序列,当$n\to \infty$ 时会趋于$0$.再定义
  \[
  f_n:=f 1_{H_n\ge |f|>H_{n+1}}.
  \] 
  该分解被称为``horizontally dyadic layer cake decomposition''.唯一需要验证的就是(\ref{1-2})式.下述的估计方式被称为``telescoping estimate'':
  \begin{equation*}
    \begin{aligned}
      H_n 2^{n /p} = & (H_n^{q}2^{nq /p})^{1 /q}\\
      = & \left( \sum_{k=0}^{\infty} \left( H^{q}_{n+k}-H^{q}_{n+k+1} \right) 2^{ nq /p} \right)^{1 /q}\\
      \lesssim_{p,q} & \left( \sum_{k=0}^{\infty} 2^{-kq /p}\|\lambda 2^{(n+k) /p}\|^{q}_{L^{q}([H_{n+k+1},H_{n+k}], \frac{\mathrm{d}\lambda}{\lambda})} \right)^{1 /q}\\
      \lesssim_{p,q} & \left( \sum_{k=0}^{\infty} 2^{-kq /p}\|\lambda \mu\left( \left\{ |f|\ge \lambda \right\}  \right) ^{1 /p}\|^{q}_{L^{q}([H_{n+k+1},H_{n+k}])} \right) ^{1 /q}.
    \end{aligned}
  \end{equation*}
  同样对上式进行$l^{q}$ 求和即可(这里处理的是$q<\infty$ 的情形,对于$q=\infty$ 的情形可以用类似的处理方法).

  为了完成等价性的证明,还要验证$(c)\Rightarrow (a)$ 和$(d)\Rightarrow (a)$.首先假设$(c)$ 成立,易知
  \[
    \mu\left( \left\{ |f|>2^{m} \right\}  \right) \le \sum_{k=0}^{\infty} \mu\left( E_{m+k} \right) 
  \] 从而有
  \[
    \|\lambda \mu \left( \left\{ |f|\ge \lambda \right\}  \right) ^{1 /p}\|_{L^{q}((2^{m},2^{m+1}], \frac{\mathrm{d}\lambda}{\lambda}}\lesssim_{p,q} 2^{m}\left( \sum_{k=0}^{\infty} \mu\left( E_{m+k} \right)  \right) ^{1 /p}.
  \] 
  对上式进行$l^{q}$ 求和之后,我们只需要证明
  \[
    \|2^{m}\left( \sum_{k=0}^{\infty} \mu\left( E_{m+k} \right)  \right) ^{1 /p}\|_{l^{q /p}_m}\lesssim_{p,q}1.
  \] 
  上式可改写为
  \[
    \|\sum_{k=0}^{\infty} 2^{pm}\mu\left( E_{m+k}  \right) \|_{l^{q /p}_m}\lesssim_{p,q}1.
  \]
  但是依据假设我们有
  \[
    \|2^{pm}\mu\left( E_m \right) \|_{l^{q /p}_m}\lesssim_{p,q}1
  \] 
  所以通过对$m$ 平移$k$ 可得
  \[
    \|2^{pm}\mu\left( E_{m+k} \right) \|_{l^{q /p}_{m}}\lesssim_{p,q}2^{-kp}.
  \] 
  再进行求和以及三角不等式即可.

  现在假设$(d)$ 成立.对于任意的$\lambda >0$ 我们有
  \[
  \mu\left( \left\{ |f|>\lambda \right\}  \right) \lesssim_{p,q}\sup \left\{ 2^{n}:H_n'\ge \lambda \right\} 
  \] 
  其中$H_n'$ 是
  \[
  H'_n:=\sum_{k=0}^{\infty} H_{n+k}.
  \]
  实际上,如果对某个$n$有$\mu\left( \left\{ |f|>\lambda \right\}  \right) >2^{n-1}$,那么易知$H_n\ge \lambda$ 从而$H'_n\ge \lambda$.
  通过对下标$n$ 的平移和三角不等式可得$H'_n$ 有着和(\ref{12})式中$H_n$ 一样的限制,因此
  \[
    \| H'_n 2^{n /p}\|_{l^{q}_{n}(\Z)}\lesssim_{p,q} 1.
  \] 
  这里以$q<\infty$ 的情形为例:
  \begin{equation*}
    \begin{aligned}
      \|\lambda \mu\left( \left\{ |f|\ge \lambda \right\}  \right) ^{1 /p}\|^{q}_{L^{q}(\R^{+}, \frac{\mathrm{d}\lambda}{\lambda}} & \lesssim_{p,q} \int_{0}^{\infty}\lambda^{q-1}\sup \left\{ 2^{nq /p}:H'_n\ge \lambda \right\} \mathrm{d}\lambda\\
     & \lesssim_{p,q} \sum_{n}^{} \int_0^{\infty}\lambda^{q-1}2^{nq /p} 1_{H'_n\ge \lambda}\mathrm{d}\lambda\\
     & \sim_{p,q}  \sum_{n}^{} 2^{nq /p}(H'_n)^{q}\\
     & \lesssim  1.
    \end{aligned}
  \end{equation*}
\end{proof}
这个定理说明,如果一个函数$f$ 可以写成$\sum_{n}^{} f_n$,其中$f_n$ 为高 $H_n$ 宽$W_n$ 的quasi-step function,并且只要其中一个变化得足够块,那么就有
\[
\|\sum_{n}^{} f_n\|_{L^{p,q}}\sim_{p,q}\|H_n W_n^{1 /p}\|_{l^{q}_n}.
\]

该定理的一个简单推论就是 Lorentz spaces上的H\"{o}lder不等式.
\begin{theorem}[H\"{o}lder inequality in Lorentz spaces]
 设 $0<p_1,p_2,p<\infty$, $0<q_1,q_2,q\le \infty$,并且满足 $\frac{1}{p}=\frac{1}{p_1}+\frac{1}{p_2}$ 和$\frac{1}{q}=\frac{1}{q_1}+\frac{1}{q_2}$,则有
 \[
 \|fg\|_{L^{p,q}}\lesssim_{p_1,p_2,q_1,q_2}\|f\|_{L^{p_1,q_1}}\|g\|_{L^{p_2,q_2}},
 \] 
 前提是不等式右边有意义.
\end{theorem}

还有两个类似于Riesz表示定理的结论如下,在证明Marcinkiewicz插值定理的时候会被用到.
\begin{theorem}[Dual formulation of weak $L^{p}$ ]
  设$1<p\le \infty $,对任意$L^{p,\infty}(X,\mathrm{d}\mu)$ 中的函数$f$都有
   \begin{equation}
     \|f\|_{L^{p,\infty}(X,\mathrm{d}\mu)}\sim_p \sup\left\{ \mu(E)^{-1 /p'} |\int_Xf 1_E \mathrm{d}\mu|:0<\mu(E)<\infty \right\}.\label{1-4} 
  \end{equation}
\end{theorem}
\begin{theorem}[Dual characterisation of $L^{p,q}$]\label{thm1-1}
  设$1<p<\infty$,$1\le q\le \infty$,对任意$f\in L^{p,q}$ 都有
  \begin{equation}
    \|f\|_{L^{p,q}}\sim_{p,q}\sup\left\{ |\int_{X}f\overline{g}\mathrm{d}\mu|:\|g\|_{L^{p',q'}}\le 1 \right\}. 
  \end{equation}
\end{theorem}

\subsection{Real Interpolation}
这里开始讨论线性算子.首先我们需要定义一些概念.
\begin{definition}
  设$0<p,q\le \infty$, $T$是sublinear operator,则有如下定义
   \begin{enumerate}
     \item $T$ 是\textit{strong-type} $(p,q)$ (或者就简单地称为 \textit{type} $(p,q)$),如果其满足
       \[
	 \|Tf\|_{L^{q}(Y)}\lesssim_{T,p,q}\|f\|_{L^{p}(X)},
       \] 
       其中$f$ 是$L^{p}$ 中的任意函数,或者是$L^{p}$ 的一个稠密子集中的任意函数.
     \item 若$q<\infty$,称$T$ 是 \textit{weak-type} $(p,q)$,如果其满足
       \begin{equation}
	 \|Tf\|_{L^{q,\infty}(Y)}\lesssim_{T,p,q}\|f\|_{L^{p}(X)}.
       \end{equation}
     \item 设$f$ 是任意的高$H$ 宽$W$ 的sub-step function,则称$T$是 \textit{restricted strong-type}  $(p,q)$,如果其满足
       \begin{equation}
	 \|Tf\|_{L^{q}(Y)}\lesssim_{T,p,q} HW^{1 /p}.
       \end{equation}
       特别地,我们有
       \begin{equation}
	 \|T 1_{E}\|_{L^{q}(Y)}\lesssim_{T,p,q} \mu(E)^{1 /p}.
       \end{equation}
     \item 若$q<\infty$,$f$ 是任意的高$H$ 宽$W$ 的sub-step function,则称$T$是\textit{restricted weak-type } $(p,q)$,如果其满足
       \begin{equation}
	 \|Tf\|_{L^{q,\infty}(Y)}\lesssim_{T,p,q}HW^{1 /p}.
       \end{equation}
       特别地,我们有
       \begin{equation}
	 \|T 1_E\|_{L^{q,\infty}(Y)}\lesssim_{T,p,q}\mu(E)^{1 /p}.
       \end{equation}
  \end{enumerate}
\end{definition}

在许多应用中,我们需要的往往是strong-type bounds.在这里我们会用\textit{real interpolation method} 来从weak-type甚至是restricted weak-type得到strong-type bounds.

首先我们假设
\begin{equation}\label{1-3}
  \left<|Tf|,|g| \right>:=\int_{Y}|Tf| |g|\mathrm{d}\nu
\end{equation}
是well-defined的,其中$f,g$ 是拥有有限测度支撑的简单函数(simple functions).现在我们来看一个indicator function的例子$\left<|T 1_E|,1_F \right>$,其中$E\subset X,F\subset Y$ 都具有有限测度.假设$T$ 有strong-type $(p,q)$ bound,$0<p<\infty$,$1\le q\le \infty$,也就是
\[
  \|Tf\|_{L^{q}(Y)}\lesssim_{p,q}A\|f\|_{L^{p}(X)}.
\] 
那么就有
\[
  \|T 1_E\|_{L^{q}(Y)}\lesssim_{p,q}A\mu(E)^{1 /p},
\] 
从而由H\"{o}lder不等式可得
\begin{equation}
  \left<|T 1_E|, 1_F \right>\lesssim_{p,q}\mu(E)^{1 /p}\nu(F)^{1 /q'}.
\end{equation}
实际上, strong-type这个限制过于强了,只要是restricted strong-type就能得出上面的结论.

如果$q>1$,我们还可以把条件放宽到restricted weak-type:
\begin{proposition}\label{prp1-1}
  设$0<p\le \infty$,$1<q\le \infty$,$A>0$.设$T$ 是使得(\ref{1-3})式良定义的sublinear operator.则下面两个在相差一个隐含常数的情况下是等价的:
  \begin{itemize}
    \item $T$ 是restricted weak-type $(p,q)$,$A>0$,并满足
       \begin{equation}
        \|Tf\|_{L^{q,\infty}}\lesssim_{p,q}AHW^{1 /p},
      \end{equation}
      其中$f$ 是任意高$H$ 宽$W$ 的sub-step function.
    \item 对于任意的$E\subset X,F\subset Y$ 且测度有限的可测集,我们有限制
      \[
	\left<|T 1_E|,1_F \right>\lesssim_{p,q}A\mu(E)^{1 /p}\nu(F)^{1 /q'}.
      \] 
    \end{itemize}
\end{proposition}
这个性质的证明需要用到(\ref{1-4})式,以及注 \ref{rmk1-1}.

\begin{corollary}[Baby real interpolation]
  设$T$ 是使得(\ref{1-3})式有意义的sublinear operator,$0<p_0,p_1\le \infty$,$1<q_0,q_1\le \infty$,$A_0,A_1>0$. $T$ 是具有常数$A_i$ 的restricted weak-type $(p_i,q_i)$,$i=0,1$.则$T$ 也是具有常数$A_\theta$ 的restricted weak-type $(p_\theta,q_\theta$,其中$0\le \theta\le 1$,
   \[
   \frac{1}{p_\theta}:=\frac{1-\theta}{p_0}+\frac{\theta}{p_1};\quad \frac{1}{q_\theta}:=\frac{1-\theta}{q_0}+\frac{\theta}{q_1};\quad A_\theta:=A_0^{1-\theta}A_1^{\theta}.
   \]  
\end{corollary}
上面推论的证明实际上就是由$X\lesssim Y_0$ 和$X\lesssim Y_1$ 可以推出$X\lesssim Y_\theta=Y_0^{1-\theta}Y_1^{\theta}$,$0\le \theta\le 1$.

\begin{theorem}[Marcinkiewicz interpolation theorem]
  设$T$ 是一个sublinear operator而且(\ref{1-3})式是well-defined的.设$0<p_0,p_1\le \infty$,$0<q_0,q_1\le \infty$ 以及$A_0,A_1>0$.定义
  \[
    \frac{1}{p_\theta}:=\frac{1-\theta}{p_0}+\frac{\theta}{p_1};\quad \frac{1}{q_\theta}:=\frac{1-\theta}{q_0}+\frac{\theta}{q_1};\quad A_\theta:=A_0^{1-\theta}A_1^{\theta}.
  \] 
  设$T$ 是一个bound为$A_i$的restricted weak-type $(p_i,q_i)$,$i=0,1$.假设 $p_0\neq p_1,q_0\neq p_1$.那么对于$0<\theta<1$ 以及$1\le r\le \infty$,我们有
  \[
    \|Tf\|_{L^{q_\theta,r}(Y)}\lesssim_{p_0,p_1,q_0,q_1,r} A_{\theta}\|f\|_{L^{p_\theta,r}(X)},
  \] 
  其中$f$ 是任意拥有有限支撑的简单函数.特别地,如果$q_\theta\ge p_\theta$,那么$T$ 是一个具有常数bound $O_{p_0,p_1,q_1,\theta}(A_\theta)$ 的strong-type $(p_\theta,q_\theta)$.
\end{theorem}

Marcinkiewicz Interpolation Theorem的证明所用的方法就叫作real interpolation method.这个方法的本质就是利用dual characterisation,用定理\ref{thm1-1}分解函数,对每一部分尽可能最优估计,然后求和.有时间我会另外写一个笔记专门写详细证明.

\section{Lecture Notes 2 for 247A}
这一个讲义主要讲解了conplex interpolation method以及如何利用该方法证明Riesz-Thorin Interpolation Theorem. 
\subsection{Complex Interpolation}
\begin{lemma}
  设$1\le p,q\le \infty$ 是一对共轭指数(即$1=\frac{1}{p}+\frac{1}{q}$).如果$f$ 在所有有限测度集合(如果$q=\infty$ 则需要要求$\mu$ 是 semifinite)上都可积并且
  \[
  \sup_{\|g\|_{p}\le 1,g \text{ \upshape simple}}\left| \int fg \right| =M<\infty
  \] 
  则有$f\in L^{q}$ 并且$\|f\|_q=M$.
\end{lemma}
这个引理的证明需要用H\"{o}lder不等式以及选取一个特殊的函数来证明$\|f\|_q\le M$.

\begin{proof}
  首先我们考虑$p>1$ 以及$q<\infty$ 的情形.利用H\"{o}lder不等式可得
   \[
  M\le \|f\|_q \|g\|_p\le \|f\|_q
  .\]
  剩下需要证明不等式的$\|f\|_q\le M$.我们可以找到$L^{q} $ 中的一列简单函数$\{f_n\}_{n\in \N} $,它们从下方逐点收敛到$f$.我们定义
   \[
     g_n(x)= \frac{|f_n(x)|^{q-1}\cdot \sgn f}{\|f_n\|_q^{q-1}}.
  \]
  计算可得
  \[
    \|g_n\|_p^{p}=\frac{1}{\|f_n\|_q^{p(q-1)}}\int |f_n(x)|^{p(q-1)}= \frac{\|f\|_q^{q}}{\|f\|_q^{q}}=1.
  \]
  另一方面,
  \[
  \int f_n g_n= \frac{\int |f_n|^{q}}{\|f_n\|_q^{q-1}}=\|f_n\|_q.
  \] 
  所以
  \[
  \|f_n\|_q=\int f_n g_n \le  M.
  \] 
  利用Fatu引理可得
  \[
  \int |f|^{q}\le \lim \inf \int |f|^{q}\le M^{q},
  \] 
  即$\|f\|_q\le M$.
  
  接下来我门考虑$p=1,q=\infty$ 的情况. 固定$\epsilon >0$ 设$E=\{x| |f(x)|\ge  M+\epsilon \} $.因为$\mu$ 是semifinite,如果$\mu(E)$ 是正的,那么就存在$F\subset E$ 使得$0<\mu(F)<\infty$.设$g=\mu(F)^{-1} 1_{F}\sgn f$. 则$\|g\|_1=1$ 并且
  \[
    M\ge \int fg=\frac{1}{\mu(F)}\int_F |f|\ge M+\epsilon .
  \] 
  这显然是不可能的,所以$\mu(E)=0$,则$f\in L^{\infty}$ 并且$M\ge \|f\|_{\infty}$.反向不等式仍然是由H\"{o}lder不等式得到.
\end{proof}
上一节中的dual性质证明方法和上述这个几乎一样.
\begin{lemma}[Three Lines Lemma]
  设$f$ 是一个在带状区域$\{0\le\Re(z)\le 1 \} $ 上的复解析函数,并且是至多double-exponential增长的,即对于某个$\delta>0$,有$|f(z)|\lesssim_f e^{O_f\left( e^{(\pi-\delta)|z|} \right) }$.若在$\Re(z)=0$处有$|f(z)|\le A$,以及在$\Re(z)=1$ 处有$|f(z)|\le B$,则对于带状区域中的任意的$z$有
   \[
     |f(z)|\le A^{1-\Re(z)}B^{\Re(z)}.
  \] 
\end{lemma}
有了这两个引理就可以得到Riesz-Thorin Interpolation Theorem:
\begin{theorem}[Riesz-Thorin Interpolation Theorem]
设$T$ 是一个线性算子,并且
\[
\int_{X}Tfg \mathrm{d}\mu
\] 对所有具有有限测度支撑的简单函数都是良定义的.设$0<p_0,p_1\le \infty$,$1\le q_0,q_1\le \infty$.$A_0,A_1>0$,我们有
\[
  \|Tf\|_{L^{q_i}}(Y)\le A_i \|f\|_{L^{p_i}}(X),
\] 
其中$f$ 是任意的具有有限测度支撑的简单函数,$i=0,1$.则
 \[
   \|Tf\|_{L^{q_\theta}(Y)}\le A_\theta\|f\|_{L^{p_\theta}(X)},
\] 
其中$f$ 同上,$0\le \theta\le 1$,$p_\theta,q_\theta,A_\theta$ 同第一节的定义一样.
\end{theorem}
适当定义
\[
  \Phi(z)=\int Tf_z g_z \mathrm{d}\nu
\] 
再用Three Lines Lemma即可得到.
\subsection{Schur's Test}
这里我们考虑积分算子(integral operator) $T=T_K$,其定义为
\begin{equation}\label{2-20}
   Tf(y)=T_kf(y)=\int_{X}K(x,y)f(x)\mathrm{d}\mu_X(x)
 \end{equation} 
 其中$K:X\times Y\to \C$ 是某个可测函数,被称为算子$T$的 积分核(integral kernel)或者核(kernel).
 
 调和分析中一个基本的问题就是找到使得$T$ 是strong-type $(p,q)$ (或者weak-type或者restricted-type)的 $K$ 需要满足的条件,以及估计相应的常数.一般来说,这个问题是很困难的.但是如果是从$L^{1}$映射到Banach空间,或者从Banach空间映射到 $L^{\infty}$,问题会变得简单得多.我们用$L^{p}\left( 1\le p\le \infty \right) $ 空间来阐述这个事实:
 \begin{proposition}\label{prp2-5}
   (Mapping from $L^{1}$). 设$1\le q\le \infty$, $K:X\times Y\to \C$.假设$\|K(x,\cdot )\|_{L^{q}(Y)}$ 一致有界,那么算子{\normalfont (\ref{2-20})}是 strong-type $(1,q)$(并且对于所有的$L^{1}(X)$ 绝对收敛),并且
   \begin{equation}
     \|T_K\|_{L^{1}(X)\to L^{q}(Y)}=\sup_{x\in X}\|K(x,\cdot )\|_{L^{q}(Y)}.
   \end{equation}
 \end{proposition}
 \begin{proof}
   对任意的$f\in L^{1}(X)$,利用Minkowski不等式得
   \begin{align*}
     \|\int_{X}|K(x,y)| |f(x)|\mathrm{d}\mu_X(x)\|_{L^{q}_y(Y)}\le & \int_{X}\| |K(x,\cdot )| |f(x)|\|_{L^{q}_y(Y)}\mathrm{d}\mu_X(x)\\
     \le  & \int_{X}|f(x)| \mathrm{d}\mu_X(x)\sup_{x\in X}\| |K(x,\cdot )|\|_{L^{q}_y(Y)}\\
     \le  & \infty
   .\end{align*}
   所以$Tf(y)$ 对几乎所有的$y$ 都是绝对收敛的,从三角不等式可得
   \[
     \|T_K\|_{L^{1}(X)\to L^{q}(Y)}\le \sup_{x\in X}\|K(x,\cdot )\|_{L^{q}(Y)}.
   \] 
   为了得到反向不等式,我们可以取$K$ 为一个simple function;事实上,我们可以取$K$ 为一个product simple function,也就是说一个关于分别在$X$ 和$Y$ 上的indicator functions的张量积的线性组合.在这种情况下我们可以将$X$分成有限多个正测度集,对其中的每一个 $K$ 都与$x$ 无关(这里已经排除了测度为零的集).对其中的一个集合,记作 $A$,  $\|K(x,\cdot )\|_{L^{q}(Y)}$ 可以达到.如果我们用$1_{A'},A' \subset A$ 来测试$T_K$,我们就完成了证明.
 \end{proof}

 还有一个对偶性质:
 \begin{proposition}
   (Mapping into $L^{\infty}$).设$1\le p\le \infty$,$K:X\times Y\to \C$.假设$\|K(\cdot ,y)\|_{L^{p'}(X)}$ 一致有界,那么算子{\normalfont (\ref{2-20})}是strong-type $(p,\infty)$ (并且对所有的$L^{p}(X)$ 绝对收敛),并且
   \begin{equation}
     \|T_K\|_{L^{p}(X)\to L^{\infty}(Y)}=\sup_{y\in Y}\|K(\cdot ,y)\|_{L^{p'}(X)}.
   \end{equation}
 \end{proposition}
\begin{proof}
  对任意$f\in L^{p}(X)$,
  \begin{align*}
    \sup_{y\in Y} | \int_{X}| K(x,y)| |f(x)| \mathrm{d}\mu_X(x)  |\le & \sup_{y\in Y} | \|K(\cdot ,y)\|_{L^{p'}(X)}\|f\|_{L^{p}(X)} |\\
    \le & \sup_{y\in Y}| \|K(\cdot ,y)\|_{L^{p'}(X)} | \|f\|_{L^{p}(X)}
  .\end{align*}
  反向不等式的证明也与性质\ref{prp2-5}类似.
\end{proof}
利用上面两个性质以及Riesz-Thorin Theorem我们可以得到:
\begin{theorem}[Schur's Test]
  设$K:X\times Y\to \C$ 满足
  \[
    \int_{X}|K(x,y)| \mathrm{d}\mu_X(x) \le A  \quad \text{ a.e. }y \in Y
  \]以及
  \[
    \int_{Y}|K(x,y)| \mathrm{d}\mu_{Y}(y)\le B \quad \text{ a.e. }x \in X.
  \]
  那么对任意$1\le p\le \infty$,{\normalfont (\ref{2-20})}中的算子$T_K$ 对所有的$f\in L^{p}(X)$都是良定义的(积分对几乎所有的$y$都是绝对可积的)并且
  \[
    \|T_Kf\|_{L^{p}}(Y)\le A^{1 /p'}B^{1 /p}\|f\|_{L^{p}(X)}.
  \] 
\end{theorem}
该定理也可以由不等式
\[
ab\le \frac{a^{p}}{p}+ \frac{b^{p'}}{p'}
\] 
来证明(i.e., 利用real convexity).

\begin{frame}[t]
  函数$f$ 的傅立叶变换为
  \begin{equation}
    \widehat{f}(\xi):=\int_{\R} f(x)e^{-ix\cdot \xi}\,\mathrm{d}\xi.
  \end{equation}
  调和分析中的不确定性原理告诉我们:如果一个函数及其傅立叶变换在无穷远处等于$0$,则这个函数恒为$0$.


  Logvinenko-Sereda[Teor. Funkc. Funkc. Anal., 1974]证明了
 \begin{equation}\label{unc}
   \int_{\R^{n}}|f(x)|^2\,\mathrm{d}x\le C e^{Cr_1r_2}\left( \int_{|x|\ge r_1}|f(x)|^2\,\mathrm{d}x+\int_{|x|\ge r_2}|\widehat{f}(\xi)|^2\,\mathrm{d}\xi \right).
 \end{equation}
 不确定性原理有许多版本,更详细的不确定性原理的介绍可以参考Havin-J\"{o}ricke的专著\textit{The uncertainty principle in harmonic analysis}. 该原理在偏微分方程及控制理论中有着广泛应用,最近它被应用到Schr\"{o}dinger方程的控制理论中. 
\end{frame}

\begin{frame}[t]
  考虑薛定谔方程 
  \begin{equation}\label{sch}
    i\partial_t u -\Delta u=0,\quad u(0,x)=u_0(x)\in L^2(\R^{n}).
  \end{equation}
 能观测不等式是指下述形式的不等式
 \begin{equation}
   \int_{\Omega}|u(T,x)|^2\,\mathrm{d}x\le C(T,\Omega)\int_{0}^{T}\int_{\omega}|u(t,x)|^2\,\mathrm{d}x\mathrm{d}t,
 \end{equation}
 这里$\omega\subset \Omega$.
 \begin{itemize}
   \item Rosier-Zhang[JDE,2009]证明了全空间情形下有界区域外的可观测性, 即$\omega$ 可取为任意半径的球外.
   \item Jaffard[PM,1990], Komornik-Loreti[2005]证明周期边界条件下任意开集上的可观测性,即$\omega$ 可取为任意开集.
 \end{itemize}
\end{frame}
\begin{frame}[t]
  在前面的估计式 
 \begin{equation*}
   \int_{\Omega}|u(T,x)|^2\,\mathrm{d}x\le C(T,\Omega)\int_{0}^{T}\int_{\omega}|u(t,x)|^2\,\mathrm{d}x\mathrm{d}t
 \end{equation*}
 中,观测时间是一个开区间,至少在时间轴上具有正测度.
 
 \textbf{问题}: 那么观测时间可以取更小的集合吗?


  \begin{alertblock}{薛定谔方程两点时刻能观测不等式[Wang Ming及合作者, JEMS, 2019]} 
    对所有满足方程(\ref{sch})的$u(t,x)$都有
  \begin{equation}\label{obsch}
    \int_{\R^{n}}|u_0(x)|^2\,\mathrm{d}x\le Ce^{\frac{Cr_1 r_2}{t}}\left( \int_{|x|\ge r_1}|u_0(x)|^2\,\mathrm{d}x+\int_{|x|\ge r_2}|u(t,x)|^2 \,\mathrm{d}x \right).
  \end{equation}
  \end{alertblock}
  \begin{itemize}
    \item 该不等式把一段时间改进到两点时刻.
    \item 建立了能观测不等式和不确定性原理的等价性.
    \item 常数估计关于$r_1,r_2$是最优估计,两点时刻也是最优.
  \end{itemize}
%  在此结果出现之前,绝大多数关于薛定谔方程的能观测不等式都是建立在某一时间区间上的,而(\ref{obsch})中只取两个时刻的球外进行观测便能控制住整体.所以这是一个非常新颖的结果.
\end{frame}


\begin{frame}[t]{KdV方程的定性唯一延拓性}
  考虑下述线性 Korteweg-de Vries (KdV) 方程
  \begin{equation}\label{kdv}
    \partial_t u +\partial_x^3u=0,\quad u(0,x)=u_0(x)\in L^2(\R).
  \end{equation}

     当在 $u_0(x),u(t,x)$分别在球外为零时,$u(t,x)\equiv 0$.
  \begin{itemize}
    \item {}[B.Y. Zhang,SIAM J. Math. Anal., 1992] 若$u$在 $(-\infty,c)\times {t_0,t_1}$上等于零,则$u\equiv 0$.
    \item {}[J. Bourgain, IMRN, 1997] 若$u$在球外的一段时间等于零,则 $u\equiv 0$,这对非线性情形也成立.
    \item {}[Kenig et.al., JFA, 2007; Bull. AMS, 2012] 当$u$在 $t_0,t_1$处指数衰减,则$u\equiv 0$.
  \end{itemize}
\end{frame}
\begin{frame}[t]{主要结果}
  \begin{alertblock}{KdV上两点时刻可测集外能观测不等式}
    若$|S|,|\Sigma|<\infty$,对于任意的$t>0$,存在常数 $C=C(t,S,\Sigma)>0$使得
    \[ 
    \partial_t u +\partial_x^3u=0,\quad u(0,x)=u_0(x)\in L^2(\R).
    \] 
    所有解$u(t,x)$ 满足
  \begin{equation}\label{obkdv}
    \int_{\R}|u_0(x)|^2\,\mathrm{d}x\le C \left( \int_{\R\setminus S}|u_0(x)|^2\,\mathrm{d}x + \int_{\R\setminus \Sigma} |u(t,x)|^2\,\mathrm{d}x  \right). 
  \end{equation}
  \end{alertblock}
  对比薛定谔方程的能观测不等式:
  \begin{itemize}
    \item 集合$|S|,|\Sigma|$比薛定谔情形更广泛.后者$S=B_{r_1}(0),\Sigma=B_{r_2}(0)$.
    \item 薛定谔方程的解与傅立叶变换有紧密的联系,而KdV不具备这种性质.
  \end{itemize}
\end{frame}


\iffalse
\begin{frame}[t]
  Consider the following linear Korteweg-de Vries (KdV) equation
  \begin{equation}\label{3}
    \partial_t u +\partial_x^3u=0,\quad u(0,x)=u_0(x)\in L^2(\R).
  \end{equation}
  Our aim is to prove the observability inequality at two points in time as (\ref{obsch}) for (\ref{3}),

%  The solution of (\ref{3}) is expressed by 
%  \begin{equation}\label{4}
%    u(t,x)=\int_{-\infty}^{\infty}E(t,x-y)u_0(y)\,\mathrm{d}y,
%  \end{equation}
%  where $K(\cdot ,\cdot )$ is given by
%  \begin{equation}
%    E(t,x)=\begin{cases}
%      \frac{1}{3t ^{\frac{1}{3}}}\mathrm{Ai}\left( \frac{x}{\left( 3t \right) ^{\frac{1}{3}}} \right), & t>0 \\
%      \delta(x), & t=0.
%    \end{cases}
%  \end{equation}
%  Here, $\mathrm{Ai}(w)$ is the Airy function defined via 
%  \begin{equation}
%    \mathrm{Ai}(w)=\frac{1}{2\pi}\int_{-\infty}^{\infty}e^{iwz+\frac{1}{3}z^3}\, \mathrm{d}z.
%  \end{equation}
  i.e.,
  \begin{equation}\label{7}
    \int_{\R}|u_0(x)|^2\,\mathrm{d}x\le C \left( \int_{\R\setminus S}|u_0(x)|^2\,\mathrm{d}x + \int_{\R\setminus \Sigma} |u(t,x)|\,\mathrm{d}x  \right), 
  \end{equation}
  where $C$ is a constant depending on $t$ and finite sets $S$ and $\Sigma$. 
\end{frame}
\fi

\begin{frame}[t]{主要难点}
  对于薛定谔方程的解$u(t,x)$,可以写成
  \[
    u(x,t)=\int_{\R^{n}}\frac{e^{i|x-y|^2 /4t} }{(4\pi it)^{n /2}}u_0(y)\,\mathrm{d}y.
  \] 
  由此可以推出以下关系式
  \begin{equation}\label{rel}
    (2it)^{\frac{n}{2}}e^{-i|x|^2 /4t}u(x,t)=\widehat{e^{i|\cdot |^2 /4t}u_0}(x /2t), \text{ for all }t>0.
  \end{equation}
  该式说明$t$ 时刻的解在去掉一个模为1的因子外,可以看成是初值的傅立叶变换.因此可以借助于不确定性原理的思想来得到相应的能观测不等式.

如果将KdV方程的解写成上述的积分表示,积分核没有显式表示,因此失去了与不确定性原理的联系.
\end{frame}


\begin{frame}[t]{证明思路}

\begin{enumerate}
  \item [(1)] 设$S,\Sigma$为给定的有限测度集合,构造算子
    \begin{equation}
      Tf:=\chi_{S} S(-t)\left( \chi_{\Sigma}S(t)f \right),\quad f\in L^2(\R). 
    \end{equation}
    其中$\chi_S,\chi_\Sigma$为示性函数,$S(t)$定义为
   \[
     u(t,x)=S(t)u_0(x).
  \] 
\item [(2)] 证明$T$ 是紧算子.利用$S(t)$的积分核逐点估计尝试证明 $T$是Hilbert-Schimdt算子,从而是紧算子.
\item [(3)] 证明$\|T\|<1$.容易证明$\|T\|\le 1$,再由反证法及(2)证明其范数不可能等于$1$.
\item [(4)]  证明两点能观测不等式,由(3)以及KdV的守恒律得到.
\end{enumerate} 
\end{frame}
\iffalse
\begin{frame}[t]
  The strategy to prove (\ref{7}) is to construct the operator
  \begin{equation}\label{8}
    Tf := \chi_{S}S(-t)\left( \chi_{\Sigma}S(t)f \right).  
  \end{equation}
  and claim that $T$ is a compact operator and $\|T\|<1$.

 \begin{equation}
   Tf(x)=\int \chi_{S}(x)\left( \int E(-t,x-w)\chi_\Sigma(w) E(t,w-y)\,\mathrm{d}w \right) f(y)\,\mathrm{d}y.
 \end{equation}
 Define $K(x,y)=\chi_S(x)\int E(-t,x-w)\chi_{\Sigma}(w)E(t,w-y)\,\mathrm{d}w $. Then 
\begin{equation}
  Tf(x)= \int K(x,y)f(y)\,\mathrm{d}y.
\end{equation}

\end{frame}
\fi
%\begin{frame}[t]
%  By the conservation law, we have $\|S(t)\|_{L^2}=1$, Hence $\|T\|\le 1$. Hilbert-Schmidt norm of $T$ is finite. It follows that $T$ is a compact operator.
%\end{frame}



\iffalse
\begin{frame}[t]
  \begin{lemma}
    Let $C$ be a measurable set in $\R^{n}$ with $0<\mu(C)<\infty$, let $C_0$ be a measurable subset of $C$ with $\mu(C_0)>0$, and let $\varepsilon >0$. Then there exists a translation $a\in \R^{n}$ such that 
    \[
      \mu(C)<\mu\left( C\cup aC_0 \right) <\mu(C)+\varepsilon .
    \] 
  \end{lemma}
\end{frame}
\begin{frame}[t]
  Suppose $\|T\|=1$, i.e. that there exists a $f\in L^2(\R)$ with $\mathrm{supp}(f)\subset S$ and $\mathrm{supp}(S(t)f)\subset \Sigma$. Applying Lemma we can translating $f$ by sufficiently small and obtain an infinite collection of linearly independent functions compactly supported in some set $S'$ of finite measure and the corresponding $S(t)f$ compactly supported in some set $\Sigma'$. But since  $T'$ is a compact operator, like $T$, its eigenspaces with nonzero eigenvalue are all finite-dimensonal, so we obtained a contradiction.
\end{frame}
\fi

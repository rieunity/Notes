%%%%%%%%%%%%%%%%%%%%%%%%%%%%%%%%%%%%%%%%%%%%
%
%  KdV-DIMENSION
%
%%%%%%%%%%%%%%%%%%%%%%%%%%%%%%%%%%%%%%%%%%%%
\documentclass[12pt]{amsart}
\usepackage{amssymb}
\usepackage{amsmath}
\usepackage{mathrsfs}
\usepackage{mathbbol}
\usepackage{amsfonts}
\usepackage{amssymb,amsmath}
% \usepackage[pagewise]{lineno}\linenumbers
\oddsidemargin=-.0cm
\evensidemargin=-.0cm
\textwidth=16cm
\textheight=22cm
\topmargin=0cm
%%%%%%%%%%%%%%%%%%%%%%%%%%%%%%%%%%%%%%%%%%%%
% DEFS
\def\C {{\mathcal C}}

\def\R {\mathbb{R}}


\def\d{{\,\rm d}}

\def\i{{\rm i}}

%%%%%%%%%%%%%%%%%%%%%%%%%%%%%%%%%%%%%%%%%%%%

%%%%%%%%%%%%%%%%%%%%%%%%%%%%%%%%%%%%%%%%%%%%
\newtheorem{proposition}{Proposition}[section]
\newtheorem{theorem}[proposition]{Theorem}
\newtheorem{corollary}[proposition]{Corollary}
\newtheorem{lemma}[proposition]{Lemma}
\theoremstyle{definition}
\newtheorem{definition}[proposition]{Definition}
\newtheorem{remark}[proposition]{Remark}
\numberwithin{equation}{section}
%%%%%%%%%%%%%%%%%%%%%%%%%%%%%%%%%%%%%%%%%%%%
% BIBLIOGRAPHY
\def \au {\rm}
\def \ti {\it}
\def \jou {\rm}
\def \bk {\it}
\def \no#1#2#3 {{\bf #1} (#3), #2.}
%\no{Vol}{Pag}{Year}
\def \eds#1#2#3 {#1, #2, #3.}
%\eds{Pub}{City}{Year}
%%%%%%%%%%%%%%%%%%%%%%%%%%%%%%%%%%%%%%%%%%%%%%%%%

\renewcommand\baselinestretch{1.3}

%%%%%%%%%%%%%%%%%%%%%%%%%%%%%%%%%%%%%%%%%%%%%%%%%
\title[Observability inequality ]
{ \bf
 Observability inequality at two time points for KdV equations from measurable sets}

\author[ ]
{    }


%\address{Ming Wang
%\newline\indent
%School of Mathematics and Physics, China University of Geosciences
%\newline\indent
%Wuhan, 430074,  China
%}
%\email{mwang@cug.edu.cn}


 

\subjclass[2010]{35Q55, 39A12}
\keywords{KdV equation,   unique continuation}

%%%%%%%%%%%%%%%%%%%%%%%%%%%%%%%%%%%%%%%%%%%%%%%%%



%%%%%%%%%%%%%%%%%%%%%%%%%%%%%%%%%%%%%%%%%%%%%%%%%
\begin{document}



\begin{abstract}


%\keywords{KdV equation \and Global attractor \and Fractal dimension \and Bourgain space }
% \PACS{PACS code1 \and PACS code2 \and more}
%\subclass{35Q53  \and  35B41}
\end{abstract}

\maketitle

%%%%%%%%%%%%%%%%%%%%%%%%%%%%%%%%%%%%%%%%%%%%%%%%%

\section{Introduction}
Consider the linear KdV equation
$$
u_t+u_{xxx}=0, \quad u(x,0)=u_0\in L^2(\R).
$$
Our result reads as follows.
\begin{theorem}\label{thm-1}
Let $A,B$ be two measurable sets in $\R$ with finite measure. Then for every $t>0$, there exists $C=C(t,A,B)>0$ so that when $u(t,x)$ solves the KdV equation,
$$
\int_\R |u_0|^2\d x \leq C\left( \int_{A^c}|u_0|^2\d x + \int_{B^c}|u(t,x)|^2\d x \right).
$$
\end{theorem}

\section{The proof}

Let $S(t)$ be the solution group of linear KdV equation, namely the solution of KdV is given by
$$
u(t)=S(t)u_0=G(t,x)*u_0,
$$
where $G$ is the fundamental solution of linear KdV equation, given by
\begin{align*}
G(t,x)= \left\{
               \begin{array}{ll}
               \frac{1}{(3t)^{\frac{1}{3}}}\operatorname{Ai}(\frac{x}{(3t)^{\frac{1}{3}}}), & \hbox{ }t>0 \\
                                                   \delta(x), & \hbox{ }t=0.
                                                 \end{array}
                                               \right.
\end{align*}
Here, $\operatorname{Ai}(x)$ is the Airy function defined via
\begin{align*}
\operatorname{Ai}(x)= \frac{1}{2\pi}\int^{\infty}_{-\infty}e^{i(xz+\frac{1}{3}z^3)}\d z.
\end{align*}
According to [Stein, p.330],
$$|\operatorname{Ai}(x)| \lesssim
\begin{cases}
(1+|x|)^{-\frac{1}{4}}, \quad x<0, \\
 e^{-\frac{2}{3}|x|^{\frac{3}{2}}}, \quad x\geq 0.
\end{cases}
$$

Define an operator $T:L^2(\R) \mapsto L^2(\R)$
$$
(Tf)(x)=\chi_B(x)S(t)(\chi_Af), \quad f\in L^2(\R).
$$
Then we have the following
\begin{proposition}\label{prop-1}
Let $A,B$ be two measurable sets in $\R$ with $|A|,|B|<\infty$. Then the operator norm satisfies
$$
\|S(-t)T\|_{\mathcal {L}(L^2(\R))}<1.
$$
\end{proposition}

Before give the proof of Proposition \ref{prop-1}, we first show that Theorem \ref{thm-1} follows from Proposition \ref{prop-1}.
\begin{lemma}
Proposition \ref{prop-1} implies Theorem \ref{thm-1}.
\end{lemma}
\begin{proof}
Assume that Proposition \ref{prop-1} holds, we obtain
$$
\| T\|_{\mathcal{L}(L^2(\mathbb{R}))}=\|S(t)S(-t)T\|_{\mathcal{L}(L^2(\mathbb{R}))} \le \|T\|_{\mathcal{L}(L^2(\mathbb{R}))}<1. 
$$

Then for all $u_0\in L^2(\R)$,
$$
\|\chi_B(x)S(t)(\chi_Au_0)\|_{L^2(\R)}\leq c_1\|u_0\|_{L^2(\R)}
$$
with some $0\leq c_1<1$. This implies that
 $$
\|\chi_B(x)S(t) \chi_Au_0\|^2_{L^2(\R)}\leq c_1^2\|\chi_Au_0\|_{L^2(\R)}=c_1^2\|S(t) \chi_Au_0\|^2_{L^2(\R)}, \quad \forall u_0\in L^2(\R),
$$
where we used  the conservation law $\|u_0\|_{L^2(\R)}=\|S(t)u_0\|_{L^2(\R)}$ in the last step. From this, we find
 that with  $c_2=\sqrt{\frac{c_1^2}{1-c_1^2}+1}$
\begin{align}\label{equ-5}
 \|S(t) \chi_Au_0\|_{L^2(\R)}\leq c_2\|S(t) \chi_Au_0\|_{L^2(B^c)}, \quad \forall u_0\in L^2(\R).
\end{align}
Now we have using \eqref{equ-5} and conservation law again
\begin{align*}
\|u_0\|_{L^2(\R)} &= \|S(t)u_0\|_{L^2(\R)}\leq \|S(t) \chi_Au_0\|_{L^2(\R)}+\|S(t) \chi_{A^c}u_0\|_{L^2(\R)}\\
&\leq c_2\|S(t) \chi_Au_0\|_{L^2(B^c)}+\|S(t) \chi_{A^c}u_0\|_{L^2(\R)}\\
&\leq c_2\|S(t)u_0\|_{L^2(B^c)}+(1+c_2)\|S(t) \chi_{A^c}u_0\|_{L^2(\R)}\\
&=c_2\|u(t,\cdot)\|_{L^2(B^c)}+(1+c_2)\|u_0\|_{L^2(A^c)}.
\end{align*}
This proves Theorem \ref{thm-1}.
\end{proof}


Now we need to prove Proposition \ref{prop-1}. To this end, we first note that we always have
$$
\|S(-t)T\|_{\mathcal {L}(L^2(\R))}\leq 1.
$$
In fact, for all $f\in L^2(\R)$
$$
\|S(-t)Tf\|_{L^2(\R)}= \|S(-t)\chi_B(x)S(t)(\chi_Af)\|_{L^2(\R)}\leq \|S(t)(\chi_Af)\|_{L^2(\R)}=\|\chi_Af\|_{L^2(\R)}\leq \|f\|_{L^2(\R)}.
$$
Thus, it remains to show that
$$
\|S(-t)T\|_{\mathcal {L}(L^2(\R))}\neq 1.
$$
To show this, we need the following
\begin{lemma}\label{lma-3}
For every $t\neq 0$, $T$ is a compact operator on $L^2(\R)$.
\end{lemma}
\begin{proof}
We can rewrite the operator $T$ as an integral operator:
$$
(Tf)(x)=\int_{\R}\chi_A(x)G(t,x-y)\chi_B(y)f(y)\d y:=\int_\R K(t,x,y)f(y)\d y.
$$
We claim that for all $t \neq 0$,
\begin{align}\label{equ-10}
 \int_\R\int_\R K^2(t,x,y)\d x \d y<\infty.
\end{align}
Then $T$ is a Hilbert-Schmidt operator and thus a compact operator on $L^2(\R)$.

It remains to show \eqref{equ-10}. In fact, since $|G(t,x-y)|\leq C(t)$
\begin{align*}
 \int_\R\int_\R K^2(t,x,y)\d x \d y\leq  C^2(t) \int_\R\int_\R \chi_A(x)\chi_B(y)d x \d y=C^2(t)|A||B|<\infty.
\end{align*}
This proves \eqref{equ-10}.
\end{proof}

Define the translation operator $U_{\lambda}$:
$$
\mathcal{T}_{\lambda}f(x)=f(x-\lambda ).
$$
If $A$ is a measurable set in $\mathbb{R}$ and $\lambda\in \mathbb{R}$, we shall denote the set $\lambda+ A=\{\lambda+x\lvert x\in A\}$.
\begin{lemma}\label{lma-4}
  Let $C$ and $C'$ be  measurable sets in $\mathbb{R}^n$ with $0<|C|,|C'|<\infty$, let $A_0$ and $B_0$ be a measurable subset of $C$ and $C'$ with $|A_0|>0,|B_0|>0$, and let $\epsilon>0$. Then there exists a translation $\lambda\in \mathbb{R}$ such that 
  $$
  |C|\le|C\cup (\lambda+ A_0)|<|C|+\epsilon
  $$
  and
  $$
  |C'|\le |C'\cup (\lambda +B_0)|<|C'|+\epsilon.
  $$
  Moreover, $\lambda$ can be chosen such that at least one left inequality of the above is strict.

\end{lemma}
\begin{proof}
  Define $h(\lambda)=|C\cup( \lambda+ A_0)|$. We may express $h(\lambda)$ in terms of 
  $$
  h(\lambda)=\| \mathcal{T}_\lambda \chi_{A_0}-\chi_C \|^2_{L^(\mathbb{R})}+\langle \mathcal{T}_\lambda \chi_{A_0},\chi_{C}\rangle.
  $$
  Similarly, we can define $h'(\lambda)=|C'\cup(\lambda+ B_0)|$.
  The strong continuity of $U_\lambda$ implies that $h$ and $h'$ are continuous functions. Hence For every $\epsilon>0$, there exists $\delta>0$ such that 
   $$
   |C|\le |C\cup (\lambda+ A_0)|<|C|+\epsilon
   $$
   and
   $$
   |C'|\le |C'\cup (\lambda+ B_0)|<|C'|+\epsilon.
   $$
   for $0<\lambda<\delta$.

   Choose $\sigma$ such that $0<2\sigma <|A_0|$ and a ball $S_r=[-r,r]$ such that $|C\cap S_r^c|<\sigma$. Let $\lambda \in \mathbb{R}$ be such that $|\lambda|>2r$. Since $A_0\subset C$, we obtain $|\lambda C_0 \cap S_r|<\sigma$. Thus
   \begin{align*}
     h(\lambda)=&|C\cup \lambda A_0|\\
     \ge & |C\cap S_r|+|\lambda C_0\cap S_r'|\\
     \ge & |C|-\sigma +|\lambda A_0|-\sigma\\
     =& |C|+|A_0|-2\sigma\\
     >&|C|=h(0).
   \end{align*}
   This shows that $h$ is not a constant, and $h'$ is not a constant the same way. Hence the last claim of the Lemma \ref{lma-4} is true.
\end{proof}

Go back to the proof of Proposition \ref{prop-1}. Suppose by way of contradiction that
$$\|S(-t) T \|_{\mathcal{L}(L^2(\mathbb{R}))}=1,$$
then by Lemma \ref{lma-3} there exists a function $f\in L^2(\mathbb{R})$ such that $\|S(-t)Tf\|_{L^2(\R)}=\|f\|_{L^2(\R)}$. It is supported on $B$ and $S(t)f$ is supported on $A$.

Define $f_{\lambda}=\mathcal{T}_{\lambda}f$. Then $\mathrm{supp}f_\lambda=\lambda A$. Since
\begin{align*}
  S(t)f_\lambda=& S(t)\mathcal{T}_{\lambda}f\\
  =& \int_{\mathbb{R}}G(t,x-y)f(y-\lambda) \d y\\
  =&  \int_{\mathbb{R}} G(t,x-\lambda-y)f(y)\d y\\
  =& \mathcal{T}_{\lambda}(S(t)f),
\end{align*} 
we have $\mathrm{supp}S(t)f_\lambda=\lambda +B$.

Now we define a sequence $\{f_i\}_{i=1}^{\infty}$ recursively. By Lemma \ref{lma-4} with $\epsilon=\frac{1}{2^i}$, $C=A_{i-1},A_0=A$ and $C'=B_{i-1},B_0=B$, we choose a translation $\lambda_i$ such that
$$
|A_{i-1}|\le |A_{i-1}\cup (\lambda_i+A_0)|<|A_{i-1}|+\frac{1}{2^i},
$$ 
and
$$
|B_{i-1}|\le |B_{i-1}\cup (\lambda_i+B_0)|<|B_{i-1}|+\frac{1}{2^i},
$$ 
and we set $A_i=A_{i-1}\cup (\lambda_i+ A_0), B_i=B_{i-1}\cup (\lambda_i+ B_0)$. By the last sentence of Lemma \ref{lma-4}, $\lambda_i$ can be chosen such that at least one left inequality of the above two is strict. Using the above inequality recursively, we obtain
$$
|\bigcup_{i=0}^{\infty}A_i|<|A|+1,\quad |\bigcup_{i=0}^{\infty}B_i|<|B|+1.
$$
Define $f_i=\mathcal{T}_{\lambda_i}f$ and $f_0=f$, then $\mathrm{supp}f_i\subset A_i\subset \bigcup_{i=0}^{\infty}A_i$ and $\mathrm{supp}S(t)f_i\subset B_i\subset \bigcup_{i=0}^{\infty}B_i$. We shall prove that the sequence $\{ f_i\}_{i=0}^{\infty}$ are linearly independent. Denote the projection operator $E_{U}f=\chi_{U}f$. Since $A_m=A_0\cup (\lambda_1+ A_0)\cup\cdots \cup (\lambda_m +A_0)$ and $B_m=B_0\cup (\lambda_1+ B_0)\cup \cdots \cup (\lambda_m+ B_0)$, we have $E_{A_m}f_i=f_i$ and $E_{B_m}S(t)f_i=S(t)f_i$ for all $i=0,1,\cdots,m$. By the choice of $\lambda_i$, we have either $E_{A_m\setminus A_{m-1}}f_m\neq 0$ or $E_{B_m\setminus B_{m-1}}S(t)f\neq 0$, both can show that $f_m$ is not a linear combination of $f_0,f_1,\cdots,f_{m-1}$. This means that the sequence $\{f_i\}_{i=0}^{\infty}$ are linearly independent. Let $A'=\bigcup_{i=0}^{\infty}A_i$ and $B'=\bigcup_{i=0}^{\infty}B_i$, then the eigensppace of the new operator $S(-t)T=S(-t)\chi_{B'}S(t)(\chi_{A'}f)$  has infinitely many eigenfunctions of eigenvalue $1$, which contradicts to the compact proposition of $S(-t)T$ by Lemma \ref{lma-3}. Thus we complete the proof of Proposition \ref{prop-1}. 


\section{the case one of the sets has infinite measure}

\begin{figure}[ht]
    \centering
    \incfig{drawing}
    \caption{drawing}
    \label{fig:drawing}
\end{figure}
\section*{Acknowledgements}



 Wang was supported by the National Natural Science Foundation of China under grant No. 11701535.
\begin{thebibliography}{999}

{\scriptsize


 

\bibitem{L-Ponce14} F. Linares, G. Ponce, Introduction to nonlinear dispersive equations, 2nd edition. Springer, 2014.


 \bibitem{WWZ}  G. Wang, M. Wang, Y. Zhang, Observability and unique continuation inequalities for the Schr\"{o}dinger equation, J. Eur. Math. Soc.  21 (2019) 3513--3572.

\bibitem{12} B. Y. Zhang,   Unique continuation for the Korteweg-de Vries equation, SIAM J. Math. Anal. 32 (1992)  55--71.

  \bibitem{Zhang97} B. Y. Zhang,   Unique continuation for the nonlinear Schr\"odinger equation, Proc. Roy. Soc. Edinburgh Sect. A 127 (1997)  191--205.

  \bibitem{Amrein} W. O. Amerin, A. M. Berthier, On Support Properties of $L^p$-Functions and Their Fourier Transforms, J. Funct. Anal. 24 (1977) 258-267.
  }
\end{thebibliography}

\end{document}

\chapter{Varieties}
\section{Affine Varieties}
\begin{solution}
  \begin{enumerate}
    \item $A(Y)=k[x,y]/(y-x^2)=k[x,x^2]=k[x]$.
    \item Suppose there is an isomorphism $\phi:A(Z)\to k[x]$, then $\phi(x)$ and $\phi\left( y \right) $ are polynomials of positive degree in $k[x]$. But $\phi(x)\phi(y)=\phi(xy)=1$ implies $\mathrm{deg}(\phi(x))=\mathrm{deg}\left( \phi(y) \right) = 0$, hence a contradiction.
    \item Let 
      \[
	f(x,y)=ax^2+2bxy+cy^2+dx+ey+f.
      \] 
      It is easy to verify that for $a=b=0$, it can be rewritten as
       \[
	 f(x,y)=\widetilde{x}\widetilde{y}-1,
      \] 
      otherwise $f(x,y)=0$ would be a line or two lines not a conic. Hence we assume $a\neq 0$.
      \begin{enumerate}
	\item If $b^2-ac=0$, choose $b=\sqrt{ac} $($b=-\sqrt{ac} $ is similar), then $f(x,y)=\left( \sqrt{a} x+\sqrt{c} y \right) ^2+dx+ey+f$. Let $t=\sqrt{a} x+\sqrt{c} y $, then 
	  \[
	    f(x,y)=t^2+\widetilde{d}t+\widetilde{e}y+\widetilde{f}. 
	  \] 
	  Take $t$ as $t-\frac{\widetilde{d}}{2}$, we can drop $\widetilde{d}$, i.e., 
	  \[
	    f(x,y)=t^2+\widetilde{e}y+\widetilde{f}.
	  \] 
	  Here $\widetilde{e}$ and $\widetilde{f}$ is not the same as before. If $\widetilde{e}=0$, then $f(x,y)$ denotes a line or two lines, not a conic. Hence $\widetilde{e}\neq 0$, define $s=-(\widetilde{e}y+\widetilde{f})$, then 
	  \[
	    f(x,y)=g(t,s)=t^2-s.
	  \] 
	  It implies that $A(W)$ is isomorphic to $A(Y)$.
	\item If $b^2-ac\neq 0$, by some linear transformation (diagonalize matrix $\begin{bmatrix} a &b\\b&c \end{bmatrix} $ and translation) we can get 
	  \[
	    f(x,y)=u^2+v^2+d.
	  \] 
	  Here we notice that $d\neq 0$, without loss of generality let  $d=-1$.
Let $t=u+iv,s=u-iv$, we get
 \[
   f(x,y)=g(t,s)=ts-1.
\] 
This implies that $A(W)$ is isomorphic to  $A(Z)$.
      \end{enumerate}
  \end{enumerate}
\end{solution}
\begin{remark}
  In fact it is an exercise to diagonalize the quadric form. 
\end{remark}
\begin{solution}
  Let $x=t,y=t^2,z=t^3$, then we have the relation
  \[
  y=x^2,z=x^3.
  \]
  It is direct to check $Y=Z\left( y-x^2,z-x^3 \right)$, hence $A(Y) = k[x,y,z]/\left( y-x^2,z-x^3 \right) $. Define the map
  \begin{align*}
    \phi: A(Y) &\longrightarrow k[t] \\
    f(x,y,z) &\longmapsto \phi(f(x,y,z)) = f(t,t^2,t^3)
  .\end{align*}
  It is an isomorphism and $\mathrm{dim}A(Y)=\mathrm{dim}k[t]=1$.
\end{solution}
\begin{remark}
  From now on, we will write $A(Y)=k[x,y,z]/(y-x^2,z-x^3)=k[x,x^2,x^3]=k[x]$ directly.
\end{remark}
\begin{solution}
  \begin{enumerate}
    \item If $x\neq 0$, then $z-1=0$ and $x^2-y=0$, let $Y_1=I(x^2-y,z-1)$. $A(Y_1)=k[x,y,z]/(x^2-y,z-1)=k[x,x^2,1]=k[x]$, $k[x]$ is integral hence $A(Y_1)$ is prime, i.e., $Y_1$ is irreducible.
    \item If $x=0$, then $yz=0$, let $Y_2=I(x,y)$ and $Y_3=I(x,z)$. $A(Y_2)=k[x,y,z]/(x,y)=k[z]$ is integral hence $Y_2$ is irreducible, $Y_3$ the same as $Y_2$. Then 
      \[
      Y=Y_1\cup Y_2\cup Y_3
      \] 
      where $Y_1,Y_2,Y_3$ are irreducible.
  \end{enumerate}
\end{solution}
\begin{solution}
  Let $Z(xy-1)$ be a closed set in Zariski topology. Closed sets in $\mathbb{A}^{1}$ are finite points or the whole space, hence closed sets in product topology are union of finite points and finite lines or the whole space. But $Z(xy-1)$ is not like this, hence not closed in product topology.
\end{solution}

\begin{solution}
  The ``only if'' part is obvious. Suppose $B$ is a finitely generated $k$-algebra with no nilpotent lements and generated by $\left\{a_1,a_2,\cdots,a_n\right\} $, Define the map
  \begin{align*}
    \phi: k[x_1,x_2,\cdots,x_n] &\longrightarrow B \\
    f(x_1,x_2,\cdots,x_n) &\longmapsto f(a_1,a_2,\cdots,a_n)
  .\end{align*}
  Denote $I=\mathrm{ker}\phi$, then $k[x_1,x_2,\cdots,x_n]/I\cong B$. Since $B$ has no nilpotent elements, $I$ is a radical ideal and $k[x_1,x_2,\cdots,x_n]/I$ is an affine coordinate ring.
\end{solution}
\begin{solution}
  Suppose $U$ is an open subset of an irreducible topological space $X$. If $\overline{U}\neq X$, then $X=\overline{U}\cup \left( X\setminus U   \right) $, this maktes a contradiction, hnce $U$ is dense in $X$. 
  
  Let $Y$ be a subset of $X$ and irreducible in its induced topology. Let $\overline{Y}=Y_1\cup Y_2$ where $Y_1,Y_2$ are closed in $X$. We have  $Y=\left( Y_1\cap Y \right) \cup \left( Y_2\cap Y \right) $, by assumption $Y_1\cap Y=\varnothing$ or $Y_2\cap Y=\varnothing$. Without loss of generality, we choose $Y_2\cap Y=\varnothing$, then $Y=Y_1\cap Y$. This implies $Y\subset Y_1$ hence $\overline{Y}\subset Y_1$. Since we did not make further assumptions except closedness for $Y_1$ and $Y_2$, it simply says that $\overline{Y}$ can not be the union of two proper closed subsets, hence $\overline{Y}$ is irreducible.
\end{solution}
\begin{solution}
  \begin{enumerate}
    \item (i) $\Leftrightarrow$(iii) and (ii)$\Leftrightarrow$(iiii) are obvious. We only need to prove (i)$\Leftrightarrow$(ii).

    (i)$\Rightarrow$(ii) Let $\mathcal{U}$ be a nonempty family of closed subsets. Choose $X_1 \in \mathcal{U}$, if it is not minimal, then there exists $X_2$ such that $X_1\supset X_2$, again if $X_2$ is not minimal, there exists $X_3$ such that $X_1\supset X_2\supset X_3$. Doing it repeatedly we get a descending chain $X_1\supset X_2\supset X_2\supset \cdots$. Since $X$ is noetherian, there must be $X_n=X_{n+1}=\cdots$ for some positive interger $n$. Then $X_n$ is a minimal element of $\mathcal{U}$.
      (ii)$\Rightarrow$(i) Let $X_1\supset X_2\supset \cdots$ be a descending chain of closed subsets of $X$. By (ii), there exists a minimal element, say $X_n$, since  $X_m\subset X_n$ for $m\ge n$, we must have  $X_n=X_m$. Hence the chain is stationary.
    \item Let $\left\{U_\lambda\right\} _{\lambda\in \Lambda}$ be an open cover. Let $\mathcal{M}:= \left\{\bigcup_{\lambda \in \Lambda'} U_\lambda\lvert \Lambda' \text{ is a finite subset of }\Lambda  \right\} $. By (a)(iv) there exists a maximal element $U\in \mathcal{M}$. It is sufficient to illustrate that $U=X$. If not, there must be an open subset $U_{\lambda'}\not\subset U$. Then $U \cup U_{\lambda'}\in \mathcal{M}$ and $U\subsetneq U \cup U_{\lambda'}$. This contradicts to the choice of $U$.
    \item Let $Y$ be a subset of $X$. If not, there exists an infinite descending chain of closed subsets of $Y$
      \[
	Y_1\supsetneq Y_2\supsetneq Y_3\supsetneq \cdots
      \] 
      Since  $Y_i,i\in \N$ is a closed subset of $Y$ induced by the topology of $X$, there exists a closed subset  $X'_i$ of $X$ such that  $Y_i=X'_i\cap Y$. Let $X_i=\bigcap_{j=1} ^{i}X'_j$, we have $Y_i=X_i\cap Y$ and
     \[
     X_1\supsetneq X_2\supsetneq X_3\supsetneq \cdots.
     \] 
     Otherwise, say $X_n=X_{n+1}$, then $Y_n=Y_{n+1}$. This implies that $X$ is nonnoetherian.
   \item Suppose $X$ is noetherian and Hausdorff. Let $Y$ be a closed subset of $X$, then $X-Y$ is open. For all $x\in X-Y$ and $y\in Y$, there exists open neighbourhoods $U_{xy}$ and $V_{yx}$ of $x$ and $y$ such that $U_{xy}\cap V_{yx}=\varnothing$. Since $U_{xy}\cap (X-Y)$ is still open, we can assume $U_{xy}\subset X-Y$. Now fix $y\in Y$, $\left\{U_{xy}\right\} _{x\in X-Y}$ is an open cover of $X-Y$. By (c) and (d) we know that there exists a finite subcover, denoted by $\left\{U_{x_1 y},U_{x_2 y},\cdots,U_{x_n y}\right\} $. Then $V_y:= \bigcap_{i=1} ^{n}V_{yx_i}$ has no commen points with $X-Y$, hence  $V_y\subset Y$ and open. Then $Y=\bigcup_{y\in Y} V_y$ is open. Hence any closed subset of $X$ is also open. Consider a single point set $\left\{x\right\} $, it is closed by Hausdorff condition, hence open. Then the topology of $X$ is discrete.

     If $X$ is infinite, consider the following open ascending chain
     \[
     \left\{x_1\right\} \subsetneq \left\{x_1,x_2\right\} \subsetneq\cdots.
     \] 
     This chain is not stationary, this makes a contradiction.
  \end{enumerate}
\end{solution}
\begin{solution}
  Let $H=Z(f)$, $f$ by definition is an irreducible polynomial. Let $W$ be an irreducible component of $Y\cap H$. Then $I(W)$ is a minimal prime ideal of the principal ideal $(f)$ in $A(Y)$. By Krull's Hauptidealsatz, such $I(W)$ has height one, so by dimension theorem $A(Y)/I(W)$ has dimension $r-1$, by Proposition 1.7 $\mathrm{dim}W=r-1$. Hence $\mathrm{dim}Y\cap H=r-1$.
\end{solution}
\begin{solution}
  We need to prove that any minimal prime ideal $\mathfrak{p}$ of $\mathfrak{a}$ has height no more than $r$. We prove it by induction. For $r=1$ case, it is exactly Krull's Hauptidealsatz. Assume it is true for  $r\le n-1$, we show it is true for  $r=n$. If not, we have a chain of prime ideals 
  \[
    \mathfrak{p}=\mathfrak{p}_{n+1}\supsetneq \mathfrak{p}_{n}\supsetneq \cdots\supsetneq\mathfrak{p}_1 \supsetneq \mathfrak{p}_0.
  \] 
  Since $\mathfrak{p}_n$ is a minimal prime ideal of $\mathfrak{a}$. Let  $\mathfrak{a}=(x_1,x_2,\cdots,x_n)$. If $x_1\in \mathfrak{p}_1$, then $\mathfrak{p}$ is also a minimal prime ideal of $\mathfrak{p}_1+\mathfrak{a}$, and this implies that $\mathfrak{p} / \mathfrak{p}_1$ is a minimal prime ideal of the ideal generated by $x_2+\mathfrak{p}_1,x_3+\mathfrak{p}_1,\cdots,x_n+\mathfrak{p}_1$. The chain 
  \[
  \mathfrak{p}_{n+1}/\mathfrak{p}_1\supsetneq \mathfrak{p}_n / \mathfrak{p}_1 \cdots\supsetneq \mathfrak{p}_1 / \mathfrak{p}_1
  \] 
  then contradicts the induction hypothesis. Hence we only need to show that the chain
\[
    \mathfrak{p}=\mathfrak{p}_{n+1}\supsetneq \mathfrak{p}_{n}\supsetneq \cdots\supsetneq\mathfrak{p}_1 \supsetneq \mathfrak{p}_0.
\]
can be modified such that $x_1\in \mathfrak{p}_1$. Suppose that $x_1\in \mathfrak{p}_k$ but not in $\mathfrak{p}_{k-1}$ for $k\ge 2$. It will suffice to show that there exists a prime ideal strictly between $\mathfrak{p}_{k}$ and $\mathfrak{p}_{k-2}$ that contain $x_1$, then we may use this prime ideal instead of $\mathfrak{p}_{k-1}$. By doing this repeatedly we can get a chain such that $x_1\in \mathfrak{p}_1$.

To find such a prime ideal, we work in the local domain
\[
D=R_{\mathfrak{p}_k} / \mathfrak{p}_{k-2}R_{\mathfrak{p}_k}.
\]
The element $x_1$ has nonzero and nonunit image $\overline{x}_1$ in $D$. Let $\mathfrak{p}'$ be a minimal prime of $\overline{x}_1 D$. $\mathfrak{p}'$ cannot be $\mathfrak{p}_k D$, for that ideal has height at least $2$, and $\mathfrak{p}'$ has height at most one by Krull's Hauptidealsatz. Then the inverse image of $\mathfrak{p}'$ in $R$ gives the required prime. Hence we can modify the chain such that $x_1\in \mathfrak{p}_1$. This completes the proof.
\end{solution}
\begin{remark}
  It is the general version of Krull's Hauptidealsatz or called principal ideal theorem. In fact, the converse is also true: for a prime ideal $\mathfrak{p}$ of height $n$, we can choose $x_1,x_2,\cdots,x_n\in \mathfrak{p}$ such that $\mathfrak{p}$ is a minimal prime ideal of $(x_1,x_2,\cdots,x_n)$.
\end{remark}
\begin{solution}
  \begin{enumerate}
    \item Let 
  \[
  Y_n\supsetneq Y_{n-1}\supsetneq \cdots\supsetneq Y_1\supsetneq Y_0
  \] 
  be a chain of distinct irreducible closed subsets of $Y$. Let $\overline{Y}_i$ be closure of $Y_i$ in $X$ for $i=0,1,\cdots,n$, then $\overline{Y}_n \cap Y=Y_n$, $\overline{Y}_{i+1}\supsetneq \overline{Y}_i$, and $\overline{Y}_i$ is irreducible in $X$ by Exercise 1.6. Therefore
  \[
  \overline{Y}_n\supsetneq \overline{Y}_{n-1}\supsetneq \cdots\supsetneq \overline{Y}_1\supsetneq \overline{Y}_0
  \]
    \item  Let
      \[
  X_n\supsetneq X_{n-1}\supsetneq \cdots\supsetneq X_1\supsetneq X_0
    \]
    be a chain of distinct irreducible closed subsets of $X$. Since $\left\{U_i\right\} _{\in I}$ is an open cover of $X$, there exists an open set $U_j$ such that $U_j\cap X_0\neq 0$. By Exercise 1.6 $U_j$ is dense in for $i=0,1,\cdots,n$, and every $U_j\cap X_i$ is a closed irreducible subset of $U_j$. Therefore $U_j$ has a chain
    \[
    U_j\cap X_n\supsetneq U_j\cap X_{n-1}\supsetneq \cdots\supsetneq U_j\cap X_1\supsetneq U_j\cap  X_0.
    \]
    This implies $\mathrm{dim}U_j\ge n$. Since the chain we choose is arbitrary, we have $\sup \mathrm{dim}U_i\ge \mathrm{dim}X$. Combining (a) we complete the proof.
  \item Let $X=\Z$ equipped with the discrete topology and $U_i=\left\{i\right\} $.
  \item Let 
     \[
       Y\supset Y_n\supsetneq Y_{n-1}\supsetneq \cdots\supsetneq Y_0.
    \] 
    If $Y\subsetneq X$, then we can extend the above chain by adding $X$ into the leading term, which leads to $\mathrm{dim}X>\mathrm{dim}Y$, a contradiction.
  \item Let $X=\Z$ equipped with the topology such that every finite subset is closed.
  \end{enumerate}
\end{solution}
\begin{solution}
  Define a morphism
  \begin{align*}
    \phi: \mathbb{A}^{1} &\longrightarrow Y \\
    t &\longmapsto (t^3,t ^{4},t ^{5})
  .\end{align*} 
  Then we get a cooresponding homomorphism
  \begin{align*}
    \phi^{*}: A(Y) &\longrightarrow k[t] \\
    f(x,y,z) &\longmapsto f\circ \phi(t)= f(t^3,t ^{4},t ^{5})
  .\end{align*}
\end{solution}
It is an isomorphism. Therefore $\mathrm{dim}A(Y)=1$ and $\mathrm{dim}I(Y)$ is prime of height $2$. 

Now we will prove that $I(Y)$ cannot be generated by $2$ elements. Let $f=\sum_{i,j,k\ge 0}^{} a_{ijk}x^{i}y^{j}z^{k}$. Since $f(t^3,t ^{4},t ^{5})=0$ we have
\[
  \sum_{i,j,k\ge 0}^{} a_{ijk}t ^{3i+4j+5k}.
\]
Hence for any integer $s\ge 0$. 
\[
\sum_{3i+4j+5k=s} a_{ijk}=0.
\]
Choose $s=3$ we get $a_{100}=0$, hence $x\not\in  I(Y)$. By doing the similar calculation we get $y,z\notin I(Y)$. Let  $s=8$ we have 
\[
  y^2-xz \in I(Y).
\]
It cannot be generated by two elements of $I(Y)$ since  $x,y,z,y^2,xz\notin I(Y)$.
\begin{solution}
  Since $\varnothing$ is not considered to be irreducible, we can choose $f=x^2+y^2+1$. Or choose $f=(x^2-1)^2+y^2$, it is irreducible and has two points, hence the zero set is not irreducible.
\end{solution}
\section{Projective Varieties}

\begin{solution}
  We use $Z_{\mathrm{affine}}(\cdot )$ to denote the affine zero set of subset of $S$. Since $f(P)=0$ for all $P\in Z(\mathfrak{a})$, we get $f(p)=0$ for all $p\in Z_{\mathrm{affine}}(\mathfrak{a})$. By the usual Nullstellensatz, we get $f\in \sqrt{\mathfrak{a}}\Rightarrow f^{q}\in \mathfrak{a}$ for some $q>0$.
\end{solution}

\begin{solution}
  (i)$\Leftrightarrow$(ii) we notice that 
  \[
    Z(\mathfrak{a})=\varnothing \Longleftrightarrow Z_{\mathrm{affine}}(\mathfrak{a})=\left\{0\right\}  \text{ or }Z_{\mathrm{affine}}(\mathfrak{a})=\varnothing.
  \]
  If $Z_{\mathrm{affine}}(\mathfrak{a})=\left\{0\right\} $, then $\sqrt{\mathfrak{a}} =(x_0,x_1,\cdots,x_n)=\bigoplus_{d>0}S_d$. If $Z_{\mathrm{affine}}(\mathfrak{a})=\varnothing$, then $\sqrt{\mathfrak{a}}=S$.

  (ii)$\Rightarrow$(iii) If $\sqrt{\mathfrak{a}} =\bigoplus_{d>0}S_d$ or $S$, then for $x_i$ there exists $q_i>0$ such that $x_i^{q_i}\in \mathfrak{a},i=0,1,\cdots,n$. Let $d=\sum_{i=0}^{n} q_i$, then $S_d \subset \mathfrak{a}$. 
 
  (iii)$\Rightarrow$(i) is obvious.
\end{solution}

\begin{solution}
  (a), (b) and (c) are obvious. For (d), since $Z(\mathfrak{a})\neq 0$ and $\mathfrak{a}$ is homogeneous, we get $I(Z_{\mathrm{affine}}\left( \mathfrak{a} \right) )\supset I(Z\left( \mathfrak{a} \right) $. Since $I\left( Z_{\mathrm{affine}}(\mathfrak{a}) \right)= \sqrt{\mathfrak{a}}$, we only need to prove $I\left( Z_{\mathrm{affine}}(\mathfrak{a}) \right)\subset I\left( Z(\mathfrak{a}) \right)  $. Assume $f\in I\left( Z_{\mathrm{affine}}(\mathfrak{a}) \right) $,
  write $f=f_d+f_{d-1}+\cdots +f_1+f_0, f_i \in S_i,i=0,1,\cdots,d$. Then 
  $f(\lambda x)=f_d(x)\lambda^{d}+f_{d-1}(x)\lambda^{d-1}+\cdots+f_1(x)\lambda+f_0$. Let $x \in Z_{affine}(\mathfrak{a})$ and fix it, $g(\lambda):=f(\lambda x)=0$ for all $\lambda\neq 0$, this implies $f_0=f_1(x)=f_2(x)=\cdots=f_d(x)=0$. It is true for all $x\in Z_{\mathrm{affine}}(\mathfrak{a})$, hence $f_i(x) \in I(Z(\mathfrak{a}))$ for $i=0,1,\cdots,n$. Hence $f \in I(Z(\mathfrak{a}))$.

  For (e), it is obvious that $Z\left( I(Y) \right) \supset \overline{Y}$. Let $W$ be a closed subset and  $Y\subset W$. Then $W=Z(\mathfrak{a})$ for some ideal $\mathfrak{a}$. By (b) we get $I\left( Z(\mathfrak{a}) \right)\subset I(Y) $. But certainly $\mathfrak{a}\subset I\left( Z\left( \mathfrak{a} \right)  \right) $, so by (a) we have $W=Z\left( \mathfrak{a} \right) \supset Z\left( I\left( Y \right)  \right) $.  Thus we have $Z\left( I\left( Y \right)  \right) = \overline{Y}$. 
\end{solution}
\begin{remark}
  We did not differentiate $I(\cdot )$ and $I_{\mathrm{affine}}(\cdot )$ here, because they are identical with respect to the set $A$ who satisfies (i) $(x_0,x_1,\cdots,x_n) \in A\Leftrightarrow (\lambda x_0,\lambda x_1,\cdots,\lambda x_n) \in A,$ $ \forall \lambda \in k^{*} $ and (ii) $0\in A$.
\end{remark}
\begin{solution}
  \begin{enumerate}
    \item It is a direct result of Exercise 2.3(d).
    \item The proof is the same as Corollary 1.4.
    \item Same as Example 1.4.1.
  \end{enumerate}
\end{solution}
\begin{solution}
  \begin{enumerate}
    \item Because $S=k[x_0,x_1,\cdots,x_n]$ is noetherian.
    \item Same as Proposition 1.5.
  \end{enumerate}
\end{solution}
  \begin{solution}
    Choose an affine piece $U_i$ of $\mathbb{P}^{n}$ such that $U_i\cap Y\neq \varnothing$. Let $Y_1$ be the affine variety  $\varphi_i(Y\cap U_i)$, and let $A(Y_i)$ be its affine coordinate ring. There is a natural isomorphism between $A(Y_i)$ and subring of elements of degree  $0$ of the localized ring $S(Y)_{x_i}$ :
    \begin{align*}
      \psi: A(Y_i) &\longrightarrow S(Y)_{x_i} \\
      f(x_0,x_1,\cdots,\hat{x}_i,\cdots,x_n) &\longmapsto x_i^{e}f\left( \frac{x_0}{x_i}, \frac{x_1}{x_i},\cdots,\widehat{\frac{x_i}{x_i}},\cdots,\frac{x_n}{x_i} \right) 
    \end{align*}
   where $e=\mathrm{deg}f$.
    Then we have 
    \[
      S(Y)_{x_i}\cong A(Y_i)[x_i,x_i^{-1}].
    \] 
    Let $\mathrm{dim}A(Y_i)=r$, choose $f_1,f_2,\cdots,f_r$ such that $A(Y_i)$ is the algebraic extension of $k[f_1,f_2,\cdots,f_r]$. Then $A(Y_i)[x_i,x_i^{-1}]$ is the algebraic extension of $k[f_1,f_2,\cdots,f_r,x_i,x_i^{-1}]$. It is easy to verify that $x_i$ is transcendent over $A(Y_i)$ and $x_i^{-1}$ is transcendent over $A(Y_i)[x_i]$, hence in particular  $f_1,f_2,\cdots,f_r,x_{i},x_i^{-1}$ are algebraically independent over $k$. This implies 
    \[
      \mathrm{dim}S(Y)_{x_i}=r+2=\mathrm{dim}A(Y_i)+2.
    \] 
    On the other hand, we have $\mathrm{dim}S(Y)_{x_i}=\mathrm{dim}S(Y)[x_i^{-1}]$, by similar analysis we have
    \[
      \mathrm{dim}S(Y)_{x_i}=\mathrm{dim}S(Y)+1.
    \] 
    Combining the above two equalities we get 
    \[
      \mathrm{dim}S(Y)=\mathrm{dim}A(Y_i)+1=\mathrm{dim}Y_i +1.
    \] 
    Using Exercise 1.10(b) the desired equation can be established. We can also get the relation
    $\mathrm{dim}Y=\mathrm{dim}Y_i$ whenever $Y_i$ is nonempty.
  \end{solution}
 
  \begin{solution}
    \begin{enumerate}
      \item  
      By Exercise 2.6, we have
          \[
     \mathrm{dim}S = \mathrm{dim}\mathbb{P}^{n}+1.
          \] 
   Since $\mathrm{dim}S=n+1$, we obtain
          \[
     \mathrm{dim}\mathbb{P}^{n}=n.
        \]
      \item Before proving the projective version, we consider the affine versio: do we have $\mathrm{dim}Y=\mathrm{dim}\overline{Y}$? The answer is certainly yes. By definition $Y$ can be written as $Y=V\setminus W$ with $V,W$ closed. Then we have $\overline{Y}=V$. Let 
	\[
	X_n\supsetneq X_{n-1}\supsetneq \cdots\supsetneq X_0
	\] 
	be a chain of $V$, since  $Y$ is open in $V$, hence dense and irreducible, we get
	\[
	  X_n\cap Y\supsetneq X_{n-1}\cap Y\supsetneq \cdots\supsetneq X_0\cap Y\neq \varnothing.
	\] 
	It is exactly the proof or Exercise 1.10(b), hence $\mathrm{dim}Y=\mathrm{dim}\overline{Y}$.
	
	Now we go back to the projective version. Assume $Y$ be a quasi-projective variety, then we get some $Y_i=\varphi\left( Y\cap U_i \right) \neq\varnothing$ and the corresponding  $W_i=\varphi\left( \overline{Y}\cap U_i \right) $. Then $\overline{Y}_i= W_i$, this is an affine version, we have $\mathrm{dim}Y_i=\mathrm{dim}W_i$. Then combining the last sentence of the previous solution of Exercise 2.6 we complete the proof.
    \end{enumerate}
  \end{solution}
  \begin{remark}
    After completing the proof, I saw the proof of the affine version. It is Proposition 1.10 at page 6 of Hartshorne's book, feel happy for my excellent memory.
  \end{remark}
  \begin{solution}
    Same as Propositioon 1.13.
  \end{solution}
  \begin{solution}
    \begin{enumerate}
      \item  Let $f(x_0,x_1,\cdots,x_n)$ be a homogeneous element of $I(\overline{Y})$, then 
	\[
	  f(x_0,x_1,\cdots,x_n)=\beta\left( x_0^{\mathrm{deg}f}g(\frac{x_1}{x_0},\cdots,\frac{x_n}{x_0} \right) 
	\]
	where $g(x_1,x_2,\cdots,x_n)=f\left( 1,x_1,x_2,\cdots,x_n \right) $. We need to prove $f(1,x_1,x_2,\cdots,x_n)\in I(Y)$. If not, there exists a point $p=(p_1,p_2,\cdots,p_n)\in Y$ such that $f\left( 1,p_1,p_2,\cdots,p_n \right) \neq 0$, then $(1:p_1:p_2:\cdots p_n)\notin \overline{Y}$, which is a contradiction. Hence every homogeneous element of $I(\overline{Y})$ is generated by $\beta\left( I(Y) \right) $. $I(\overline{Y})$ is a homogeneous ideal, i.e., generated by homogeneous elements, hence generated by $\beta\left( I(Y) \right) $.
      \item Recall $I(Y)=(y-x^2,z-x^3)$, $y-x^2$ and $z-x^3$ are generators of $I(Y)$.
	Elements in $I(\overline{Y})$ are of the form 
	 \[
	   (w:x:y:z)=(s^{3}:s^2t:st^2:t^3).
	\] 
	Then we have generators $y^3=z^2w$, $x^3=w^2z$, $x^2=yw$, $y^2=xz$ for $I\left( \overline{Y} \right) $. But $y^3-z^2w$ cannot be generated by $yw-x^2=\beta\left( y-x^2 \right) $ and $zw^2-x^3=\beta\left( z-x^3 \right) $.
    \end{enumerate}
  \end{solution}

  \begin{solution}
    \begin{enumerate}
      \item $I(Y)\subset I\left( C(Y) \right) $ is obvious. Let $f\in I\left( C(Y) \right) $, write it as
	\[
	  f(x_0,x_1,\cdots,x_n)=f_d+f_{d-1}+\cdots+f_0
	\] where $f_i\in S_i,i=0,1,\cdots,n$. Fix $x\neq 0$, then 
	\[
	  f(\lambda x)=f_d \lambda^{d}+f_{d-1}\lambda^{d-1}+\cdots+f_0, \quad \lambda \in k^{*}.
	\]
	This implies $f_d=f_{d-1}=\cdots=f_0=0$. Since $x$ can be chosen arbitrarily in $Y$, we get $f_0,f_1,\cdots,f_d\in I(Y)$. Therefore $f\in I(Y)$. Hence We have proved $I\left( C(Y) \right) =I(Y)$.
\begin{align*}
 & f(x_0,x_1,\cdots,x_n)=0 \quad \forall f \in I(Y)\\
  \Longleftrightarrow & (x_0:x_1:\cdots:x_n)\in Y \text{ or }x_0=x_1=\cdots=x_n=0 \\
  \Longleftrightarrow & (x_0,x_1,\cdots,x_n)\in C(Y).
.\end{align*}
The second line uses the fact that $Y$ is a nonempty algebraic set. This implies $Z_{\mathrm{affine}}(C(Y))=C(Y)$, i.e., $C(Y)$ is an algebraic set in $\mathbb{A}^{n+1}$.
\item It is obvious since an algebraic affine or projective set is irreducible if and only if its corresponding ideal is prime.
\item It is enought to prove it under the irreducible case. By Exercise 2.6 we have $\mathrm{dim}S(Y)=\mathrm{dim}Y+1$. On the other hand, $\mathrm{dim}S(Y)=\mathrm{dim}A(C(Y))=\mathrm{dim}C(Y)$. Hence 
  \[
    \mathrm{dim}C\left( Y \right) =\mathrm{dim}Y+1.
  \] 
    \end{enumerate}
  \end{solution}

\begin{solution}
  \begin{enumerate}
    \item (i)$\Rightarrow$(ii) If $I(Y)=(f_1,f_2,\cdots,f_r)$ and $f_1,f_2,\cdots,f_r$ are linear polynomials, then 
      \[
	Y=Z\left( I(Y) \right) =Z\left( f_1,f_2,\cdots,f_r \right) =Z(f_1)\cap Z(f_2)\cap \cdots\cap Z(f_r).
      \] 
      
      (ii)$\Rightarrow$(i) If $Y=Y_1\cap Y_2\cap \cdots \cap  Y_r$ and $Y_1=Z(f_1),Y_2=Z(f_2),\cdots,Y_r=Z(f_r)$, then 
      \[
	I(Y)=I\left(Z(f_1)\cap Z(f_2)\cap \cdots Z(f_r)\right)=I\left(Z\left( f_1,f_2,\cdots,f_r \right)\right)=\sqrt{(f_1,f_2,\cdots,f_r)} .
      \]
      In fact $\sqrt{(f_1,f_2,\cdots,f_r)} =\left( f_1,f_2,\cdots,f_r \right) $ since $f_1,f_2,\cdots,f_r$ are linear polynomials.
    \item Since $Y$ is a linear variety, by definition $I(Y)=\left( f_1,f_2,\cdots,f_s \right) $. By principal ideal theorem every minimal prime ideal $\mathfrak{p}$ containing $I(Y)$ has height less than or equal to $s$, i.e., $\mathrm{height}\mathfrak{p}\le s$. This implies the dimension of the irreducible component of $Y$ corresponding to $\mathfrak{p}$ is at least $n-s$, hence  $r\ge n-s$. Thus we get $s\ge n-r$. The equality holds if $f_1,f_2,\cdots,f_r$ are linearly independent.
    \item Let $Y=Z\left( f_1,f_2,\cdots,f_{n-r} \right) $ and  $Z=Z\left( g_1,g_2,\cdots,g_{n-s} \right) $. Then $Y\cap Z=\left( f_1,f_2,\cdots,f_{n-r},g_1,g_2,\cdots,g_{n-s} \right) $. Then by (b) we obtain  $n-r+n-s\ge n-\mathrm{dim}Y\cap Z$ $\Rightarrow$ $\mathrm{dim}Y\cap Z\ge r+s-n\ge 0$. Consider linear polynomials all $2n-r-s$ polynomials above as polynomials defined in $\mathbb{A}^{n+1}$, then there must exists nonzero $x$ such that all $2n-r-s$ polynomials at this point is nonzero since $2n-r-s\le n<n+1$. Hece  $Y\cap Z\neq \varnothing$.
  \end{enumerate}
\end{solution}

\begin{solution}
  \begin{enumerate}
    \item $\mathfrak{a}$ is prime since $k[y_0,y_1,\cdots,y_{N}] / \mathfrak{a}\cong k[x_0,x_1,\cdots,x_n]$. For any $f\in \mathfrak{a}$, write it as 
      \[
      f=f_d+f_{d-1}+\cdots+f_0,\quad f_i \in S_i, i=0,1,\cdots,d.
      \] 
      Then we must have $\theta\left( f_i \right)=0 $ for $i=0,1,\cdots,d$. Hence $\mathfrak{a}$ is homogeneous.
    \item $\rho_d(\mathbb{P}^{n})\subset  Z\left( \mathfrak{a} \right) $ is obvious. We only prove the converse inclusion. Let $M_{ij}=M_{ji}=x_0^{d-2}x_i x_j$, then there exists an $y_k $ such that $\theta(y_k)=M_{ij}$. we rewrite $y_k$ as $y_{ij}:=y_k$. Then every monomial can be determined by $y_{00},y_{01},y_{02},\cdots ,y_{0n}$. Define $x_0=1,x_1=y_{01},\cdots ,x_n=y_{0n}$. Then $\rho_d\left( \left( x_0:x_1:\cdots :x_n \right)  \right) =(y_0:y_1:\cdots :y_N)$.
    \item Since $\rho_d$ is injective and continuous, by invariance of domain  $\rho_d$ is a homeomorphism between  $\mathbb{P}^{n}$ and its image.
    \item Twisted cubic curve in $\mathbb{P}^{3}$ is $\left( x_0^3:x_0^2 x_1:x_0x_1^2:x_1^3 \right) $, hence the $3$-uple embedding of $\mathbb{P}^{1}$ in $\mathbb{P}^3$.
  \end{enumerate}
\end{solution}

\begin{solution}
  Let $\varphi$ be the Veronese map
  \begin{align*}
    \varphi: \mathbb{P}^{2} &\longrightarrow \mathbb{P}^{5} \\
    (x_0:x_1:x_2) &\longmapsto \left( x_0^2:x_0x_1:x_0x_2:x_1^2:x_1x_2:x_2^2 \right) 
  .\end{align*}
  Let $M_{ij}=x_ix_j$. Then the veronese surface is  $Z\left(\{ M_{ij}M_{kl}=M_{ik}M_{jl}\lvert i,j=0,1,2\} \right) $. Since $\varphi$ is a homeomorphism, $\varphi^{-1}(Z)$ is a variety of dimension $1$. Hence there exists an irreducible homogeneous polynomial $f(x_0,x_1,x_2)$ such that $\varphi^{-1}(Z)=Z(f)$ and $g=f\circ \varphi^{-1}\in S=k[x_0,x_1,\cdots ,x_5]$. Therefore 
  \[
    Z=V(g)\cap Y.
  \] 
\end{solution}


\begin{solution}
  In the hint, it is easy to check that 
  \[
  \mathfrak{a}= \left\{ z_{ij}z_{kl}-z_{il}z_{kj}\lvert 0\le i,k \le r, 0\le j,l\le s\right\}.
  \] 
  We need to show $\Im \psi=Z(\mathfrak{a})$. It is obvious that $\Im \psi \subset Z(\mathfrak{a})$. For the converse inlusion, consider a point $z\in \mathbb{P}^{N}$ with homogeneous coordinates $z_{00},z_{01},\cdots ,z_{rs}$. At least one of these coordinates must be non-zero; we can assume without loss of generality that it is $z_{00}$. Let us pass to affine coordinates by setting $z_{00}=1$. Then we have $z_{ij}=z_{i_0}z_{0j}$ for all $i=0,\cdots ,r$ and $j=0,\cdots ,s$. Hence by setting $x_i =z_{i_0}$ and $y_{j}=z_{0j}$ we obtain a point of $\mathbb{P}^{r}\times \mathbb{P}^{s}$ that is mapped to $z$ by $\psi$.
\end{solution}


\begin{solution}
The Segre embedding of $\mathbb{P}^{1}\times \mathbb{P}^{1}$ in $\mathbb{P}^{3}$ is
\begin{align*}
  \psi: \mathbb{P}^{1}\times \mathbb{P}^{1} &\longrightarrow \mathbb{P}^{3} \\
  (x_0:x_1)\times (y_0:y_1) &\longmapsto (w:x:y:z)=(x_0y_0:x_0y_1:x_1y_0:x_1y_1)
.\end{align*}
\begin{enumerate}
  \item It is obvious by definition.
  \item $L_t=\psi\left( \mathbb{P}^{1}\times t \right)  $, $M_t=\psi\left( t\times \mathbb{P}^{1} \right) $.
  \item Denote the curve $x-y=0$ in $Q$ as $Y$, then $Y=\psi \left( Z\left( x_0y_1-x_1y_0 \right)  \right) $. $Z\left( x_0y_1-x_1y_0 \right) $ is not closed in product topology  of  $\mathbb{P}^{1}\times \mathbb{P}^{1}$.
\end{enumerate}
\end{solution}

\begin{solution}
  \begin{enumerate}
    \item If $p=(w:x:y:z)\in Q_1\cap Q_2$, then:
      \begin{itemize}
        \item if $w=0$, then  $x=0$;
	\item otherwise, let  $w=1$, then $y=x^2$, $z=xy=x^3$.
      \end{itemize}
    \item $C\cap L=\left\{\left(0:0:1\right) \right\} $. $I(C)+I(L)=(x^2,y)\neq I(P)$, where $P=(0:0:1)$.
  \end{enumerate}
\end{solution}

\begin{solution}
  \begin{enumerate}
    \item By Exercise 1.9, we have
      \[
	\mathrm{dim}C(Y)\ge n+1-q.
      \] 
      By Exercise 2.10(c), we obtain
      \[
      \mathrm{dim}Y\ge n-q.
      \] 
    \item Let $I(Y)=(f_1,f_2,\cdots ,f_{n-r})$, let $Y_i=Z(f_i),i=1,2,\cdots ,n-r$. Since $Y$ is a variety, $f_1,\cdots ,f_{n-r}$ are irreducible. Then
      \[
      Y=\bigcap_{i=1} ^{n-r}Y_i.
      \]
    \item By Exercise 2.9 we have $I(Y)= \left( xy-wz,x^3-w^2z,y^2-xz \right)  $. Let $H_1=Z(xy-wz)$ and $H_2=Z(x^3-w^2z)$, then we have $Y=H_1\cap H_2$.
  \end{enumerate}
\end{solution}

\section{Sobolev空间}
\begin{exercise}
  若$1<p,q<\infty$,$\frac{1}{p}+\frac{1}{q}=1$,证明H\"{o}lder不等式:
  \[
  \|fg\|_{L^{1}}\le \|f\|_{L^{p}}\|g\|_{L^{q}}.
  \] 
\end{exercise}
\begin{proof}
  由不等式的齐次对称性,不妨令$\|f\|_{L^{p}}=\|g\|_{L^{q}}=1$.
  设$\theta = \frac{1}{p}$, 则$1-\theta = \frac{1}{q}$. 令 $F=|f|^{p},G=|g|^{q}$. 则需要被证明的不等式转化为
  \begin{equation}
    \int_{X}F^{\theta}G^{1-\theta}\mathrm{d}\mu\le 1.\label{1}
\end{equation}
  由$\ln x$ 函数的凸性可得
  \[
    F^{\theta}(x)G^{1-\theta}(x)\le \theta F(x) +(1-\theta) G(x).
  \] 
  对上式积分便得到(\ref{1})式.
\end{proof}
\begin{exercise}
  若$f\in L^2(\R^{n})$, 证明$e^{-|x|^2}*f\in L^{p}(\R^{n}),2\le p\le \infty$.
\end{exercise}
\begin{proof}
  令 $q$满足
   \[
  1+\frac{1}{p}=\frac{1}{2}+\frac{1}{q},
  \]
  其中$1\le q\le 2$.
  则由Young不等式可得
  \begin{equation}
    \begin{aligned}
      \|e^{-|x|^2}*f\|_{L^{p}(\R^{n})}\le & \|e^{-|x|^2}\|_{L^{q}(\R^{n})}\|f\|_{L^{2}(\R^{n})}.
    \end{aligned}
  \end{equation}
  而
  \begin{equation*}
    \begin{aligned}
      \|e^{-|x|^2}\|_{L^{q}(\R^{n})}^{q} = & \int_{\R^{n}}e^{-q|x|^2}\mathrm{d}x<\infty,
    \end{aligned}
  \end{equation*}
  且$f\in L^2(\R^{n})$,所以 $e^{-|x|^2}*f \in L^{p}(\R^{n})$.
\end{proof}
\begin{exercise}
  设
  \begin{equation*}
    \phi(x)=\left\{ \begin{aligned}
      & e^{- \frac{1}{1-x^2}}, & -1<x<1,\\
      & 0, & |x|\ge 1.
    \end{aligned}\right.
  \end{equation*}
  求证: $\phi \in C_0^{\infty}(\R)$.
\end{exercise}
\begin{proof}
  设
  \begin{equation*}
    f(x)=\left\{
      \begin{aligned}
	& e^{-\frac{1}{x}}, & x>0,\\
	& 0, & x=0.
      \end{aligned}\right.
  \end{equation*}
当$x>0$ 时,求导得
\begin{align*}
  f'(x)= & \frac{1}{x^2}f(x)\\
  f''(x)= & \left(\frac{1}{x^4}-\frac{2}{x^3}\right)f(x)\\
  f'''(x) = & \left( \frac{1}{x^{6}}-\frac{6}{x^{5}}+\frac{6}{x^{4}} \right) f(x)\\
  & \ldots
.\end{align*}
从上述几个导数易知可以归纳证明得到当$x>0$ 时有
\[
  f^{(k)}(x)=P_{2k}(x^{-1})f(x),
\] 
其中$P_{2k}$ 是次数为$2k$ 的多项式.由多项式和$e$ 指数的增长速率关系可得
\[
  \lim_{x\to 0^{+}}f^{(k)}(x)=\lim_{x\to+\infty}P_{2k}(x)e^{-x}=0.
\] 
所以对任意的$n>0$,都有$\lim_{x\to 0^{+}}f^{(n)}(x)=0$.
再利用分析学中的一个定理:
\begin{framed}
  \small\itshape 
  \begin{theorem} 设$f$ 是在$x=a$ 处连续的函数,$f'(x)$ 在$x=a$ 的一个去心领域上处处存在且
  \[
    \lim_{x\to a}f'(x)
  \] 
  存在,则$f'(a)$ 存在且
  \[
    f'(a)=\lim_{x\to a}f'(x).
  \] 
\end{theorem}
  \begin{proof}
    根据导数定义 
    \[
      f'(a)=\lim_{x\to a}\frac{f(x)-f(a)}{x-a}.
    \] 
    由中值定理$\frac{f(x)-f(a)}{x-a}=f'(\xi_x)$,$a<\xi_x<x$.所以
    \[
      f'(a)=\lim_{x\to a}f'(\xi_x)=\lim_{x\to a}f'(x).
    \] 
  \end{proof}
\end{framed}
把该定理用到上面定义的函数上就得到函数$f$在$x=0$处任意阶右导数都存在且
\[
  f^{(n)}_{+}(0)=0,\quad n\in \N.
\] 
而当$|x|\le 1$时,由
\begin{align*}
  \phi(x)=&f(1-x^2)\\
  \phi'(x)=&-2xf'(1-x^2)\\
  \phi''(x)=& -2f'(1-x^2)+4x^2f''(1-x^2)\\
	    &\ldots
\end{align*}
知
\[
  \phi^{(n)}_{-}(1)=0, \quad n\in \N.
\] 
另一方面,由$|x|\ge 1$时$\phi(x)=0$知
\[
  \phi^{(n)}_{+}(1)=0, \quad n\in \N.
\] 
所以$x=1$ 时
\[
  \phi^{(n)}(1)=0,\quad n\in \N.
\] 
$x=-1$时同理.

\end{proof}
\begin{exercise}
  若$1\le q\le 2$,证明对任意$u\in C^{1}_0(\R^2)$ 
  \[
    \|u\|_{L^{q}(\R^2)}\le \|u\|_{W^{1,1}(\R^2)}.
  \] 
\end{exercise}
\begin{proof}
  设$x=(x_1,x_2)$.因为$u\in C^{1}_0(\R^2)$,所以$u$ 可以写成
  \[
    u(x)=\int_{-\infty}^{x_1}\partial_1 u(t,x_2)\mathrm{d}t=\int_{-\infty}^{x_2}u(x_1,t)\mathrm{d}t.
  \] 
  进而有
  \[
    |u(x)|\le \int_{-\infty}^{x_i}|\partial_{i}u|\mathrm{d}x_i\le \int_{\R}|\partial_i u|\mathrm{d}x_i, x\in \R^2
  .\]
  那么
  \begin{align*}
    \int_{\R^2}|u(x)|^2\mathrm{d}x\le& \int_{\R^2}\mathrm{d}x\int_{\R}|\partial_1 u(x)|\mathrm{d}x_1\int_\R |\partial_2u(x)|\mathrm{d}x_2\\
    =&\int_{\R^2}|\partial_1 u(x)|\mathrm{d}x \int_{\R^2}|\partial_2 u(x)|\mathrm{d}x
  .\end{align*}
  利用平均不等式(下式第二行到第三行)可得
  \begin{align*}
    \|u\|_{L^2(\R^2)}=&\left( \int_{\R^{n}}|u(x)|^2\mathrm{d}x \right) ^{\frac{1}{2}}\\
    \le& \left( \int_{\R^2}|\partial_1 u(x)|\mathrm{d}x \int_{\R^2}|\partial_2u(x)|\mathrm{d}x \right) ^{\frac{1}{2}}\\
    \le&\frac{1}{2}\left( \int_{\R^2}|\partial_1u(x)|\mathrm{d}x+\int_{\R^2}|\partial_2u(x)|\mathrm{d}x \right)\\
    \le &\frac{1}{2}\left( \|\partial_1u\|_{L^{1}(\R^2)}+\|\partial_2 u\|_{L^{1}(\R^2)}\right)\\
    \le  & \|\nabla  u\|_{L^{1}(\R^2)}\le \|u\|_{W^{1,1}(\R^2)}
  .\end{align*}
  对于$1<q<2$ 的情况,可以利用插值不等式
  \[
    \|u\|_{L^{q}(\R^2}\le \|u\|_{L^{1}(\R^2)}^{\theta}\|u\|^{1-\theta}_{L^{2}(\R^2)} \quad 0<\theta<1
  \] 
  得到.
\end{proof}
\begin{exercise}
  设$f(x)=x^{\frac{1}{4}}$ $x\in [0,1]$,则$f\in C^{0, \frac{1}{4}}$,但是$f\notin C^{0,\mu},\mu>\frac{1}{4}$. 
\end{exercise}
\begin{proof}
  根据$C^{0,\mu}$ 的定义,我们需要计算范数
  \[
    \|f\|_{C^{0,\mu}[0,1]}:=\sup_{x\in [0,1]}|f(x)|+[f]_{\mu,[0,1]},
  \] 
  其中半范数
  \[
    [f]_{\mu,[0,1]}:=\sup_{x,y\in [0,1],x\neq y} \frac{|f(x)-f(y)|}{|x-y|^{\mu}}.
  \]
  $\sup_{x\in [0,1]}|f(x)|=1$,所以只要计算半范数$[f]_{\mu,[0,1]}$.实际上我们可以证明一个一般结论:对于任意的$f(x)=x^{\alpha}$,$0<\alpha<1$,我们都有 $f\in C^{0,\alpha}$.为了证明这个一般结论,我们需要下述引理建立的不等式
  \begin{framed}
    \small\itshape
    \begin{lemma}
      设$x>0$,$0<\alpha<1$,则
      \[
	(x+1)^{\alpha}\le x^{\alpha}+1.
      \] 
    \end{lemma}
    \begin{proof}
      令$g(x)=(x+1)^{\alpha}-x^{\alpha}-1$,则
      \[
	g'(x)=\alpha \left( \frac{1}{(x+1)^{1-\alpha}}-\frac{1}{x^{1-\alpha}} \right) \le 0.
      \] 
      又因为$g(0)=0$,所以 $g(x)\le g(0)=0$,不等式得证.
    \end{proof}
  \end{framed}
  那么由引理中的不等式可得对$a,b>0$有
  \begin{align*}
  \left (\frac{a}{b}+1\right)^{\alpha}\le&\left(\frac{a}{b}\right)^{\alpha}+1\\
  (a+b)^{\alpha}\le &a^{\alpha}+b^{\alpha}.
  \end{align*}
  (上式对$a=0$ 或$b=0$ 时是显然的.)令$a=x-y,b=y,x>y$可得
   \[
     x^{\alpha}\le (x-y)^{\alpha}+y^{\alpha}.
   \]
   所以当$x>y$时,有
    \[
      x^{\alpha}-y^{\alpha}\le (x-y)^{\alpha}
   \] 
 特别地,当$\mu=\frac{1}{4}$且$x>y$ 时,
  \begin{align*}
    |f(x)-f(y)|=&|x^{\frac{1}{4}}-y^{\frac{1}{4}}|\\
    \le  & |x-y|^{\frac{1}{4}}
  .\end{align*}
  $x<y$ 的情况同理.上述不等式说明\[
    [f]_{\frac{1}{4},[0,1]}\le 1<\infty.
  \] 
  所以$f\in C^{0,\frac{1}{4}}$.
  对于$\mu>\frac{1}{4}$ 的情形,我们有
  \[
    \sup_{x\neq y} \frac{|f(x)-f(y)}{|x-y|^{\mu}}\overset{\text{令}y=0}{\ge }\sup_{x\in (0,1]} \frac{f(x)}{x^{\mu}}=\sup_{x\in (0,1]}x^{\frac{1}{4}-\mu}=\infty.
  \]  
  所以$f\notin C^{0,\mu},\mu>\frac{1}{4}$.
\end{proof}
\begin{exercise}
  设$\Omega =\left\{ x\in \R^{2}:|x|<1 \right\} $,说明$u\in W^{1,2}(\Omega)$ 但是$u \notin L^{\infty}(\Omega)$.其中
  \[
    u(x)=\ln \ln (1+\frac{1}{|x|}),\quad |x|<1,\quad,|x|=\sqrt{x_1^2+x^2_2}. 
  \]
\end{exercise}
\begin{proof}
  对$u(x)$ 求偏导可得
  \begin{align*}
    \partial_1u=&-\frac{1}{\ln(1+\frac{1}{|x|})}\frac{1}{1+\frac{1}{|x|}} \frac{x_1}{|x|^3},\\ 
    \partial_2u=&-\frac{1}{\ln(1+\frac{1}{|x|})}\frac{1}{1+\frac{1}{|x|}} \frac{x_2}{|x|^3}
  .\end{align*}
  从而
  \begin{align}\label{3}
    |\partial_i u|=\left| \frac{1}{|x| \ln(1+\frac{1}{|x|})} \frac{1}{|x|+1} \frac{x_i}{|x|}  \right| \le &  \frac{1}{|x|\ln(1+\frac{1}{|x|})}
  .\end{align}
   利用不等式$\ln(1+x)\le x$可得
   \[
     |u(x)|\le \ln \frac{1}{|x|}=-\ln|x|.
  \] 
  则利用该不等式以及极坐标来表示积分,$r=|x|,0<r<1$,可得
  \[
    \|u\|_{L^2(\Omega)}^2=\int_{\Omega}|u(x)|^2\mathrm{d}x\le \int_{\Omega}r\ln^2r \mathrm{d}r\mathrm{d}\theta=2\pi\int_0^{1}r\ln^2r\mathrm{d}r<\infty.
  \]
  另一方面,利用(\ref{3})式得
  \[
    \|\partial_i u\|_{L^2(\Omega)}^2=\int_{\Omega}|\partial_iu|^2\mathrm{d}x\le \int_{\Omega} \left(\frac{1}{r\ln( 1+\frac{1}{r})}\right)^2 r\mathrm{d}r\mathrm{d}\theta=2\pi\int_0^{1}\frac{1}{r \ln^2(1+\frac{1}{r})}\mathrm{d}r.
  \] 
  由$-\mathrm{d}\left( \ln(1+\frac{1}{r}) \right)=\frac{1}{r^2+r}\mathrm{d}r $,令$\ln (1+\frac{1}{r})=u$ 得
  \[
    \|\partial_iu\|_{L^2(\Omega)}\le 2\pi\int_{\ln 2}^{\infty}\frac{r+1}{u^2}\mathrm{d}u\le 4\pi \int_{\ln 2}^{\infty}\frac{1}{u^2}\mathrm{d}u<\infty.
  \] 
  综上可知,$u\in W^{1,2}(\Omega)$.

  由$u$在去心邻域 $\Omega \backslash 0$上的连续性以及$\lim_{|x|\to \infty}u(x)=\infty$知显然有$u\notin L^{\infty}(\Omega)$.
\end{proof}
\section{Fourier分析}
\begin{exercise}\label{exe2-7}
  设 $f\in L^{1}(\R)$,定义Fourier变换
  \[
    \widehat{f}(\xi)=\int_{\R}e^{-ix\xi}f(x)\mathrm{d}x,\quad \xi \in \R.
  \] 
  \begin{enumerate}
    \item [(1)]若$f(x)=(2\pi)^{-\frac{1}{2}}e^{- \frac{x^2}{2}}$,则$\widehat{f}(\xi)=e^{-\frac{\xi^2}{2}}$.
    \item [(2)]若$f,g\in L^{1}(\R)$,则
      \begin{align*}
	\int_{\R}f\overline{g}\mathrm{d}x=&(2\pi)^{-1}\int_{\R}\widehat{f}\overline{\widehat{g}}\mathrm{d}\xi;\\
	\widehat{fg}=&(2\pi)^{-1}\widehat{f}* \widehat{g}.
      \end{align*}
  \end{enumerate}
\end{exercise}
\begin{proof}\\
  (1)根据Fourier变换的定义可得
  \begin{equation}\label{4}
    \begin{aligned}
    \widehat{f}(\xi)=&\int_{\R}e^{-ix\xi}f(x)\mathrm{d}x\\
    =& (2\pi)^{-\frac{1}{2}} \int_{\R}e^{-ix\xi}e^{-\frac{x^2}{2}}\mathrm{d}x\\
    =&(2\pi)^{-\frac{1}{2}}e^{-\frac{\xi^2}{2}}\int_{\R} e^{-(x-i\xi)^2}\mathrm{d}x
    \end{aligned}
  .\end{equation}
  剩下的问题就是求$\int_{\R}e^{-(x-i\xi)^2}\mathrm{d}x$,可以通过计算$\int_{\R}e^{-x^2}\mathrm{d}x$ 的方法同样求得
  \[
    \int_{\R}e^{-(x-i\xi)^2}\mathrm{d}x=\int_{\R}e^{-x^2}\mathrm{d}x=(2\pi)^{\frac{1}{2}}.
  \] 
  代入(\ref{4})中可得
  \[
    \widehat{f}(\xi)=e^{-\frac{\xi^2}{2}}.
  \]\\
  (2)
  \begin{equation*}
    \begin{aligned} 
    &(2\pi)^{-1}\int_{\R}\widehat{f}\overline{\widehat{g}}\mathrm{d}\xi\\
    =& (2\pi)^{-1}\int_{\R}\left( \int_{\R}e^{-ix\xi}f(x)\mathrm{d}x \right) \overline{\widehat{g}}(\xi)\mathrm{d}\xi\\
    =& (2\pi)^{-1}\int_{\R}f(x)\left( \int_{\R}e^{-ix\xi}\overline{\widehat{g}}(\xi)\mathrm{d}\xi \right) \mathrm{d}x\\
    =& \int_{\R}f(x)\overline{\left( (2\pi)^{-1}\int_{\R}e^{ix\xi} \widehat{g}(\xi)\mathrm{d}\xi \right) }\mathrm{d}x\\
    =& \int_{\R}f(x)\overline{g}(x)\mathrm{d}x
    \end{aligned}
  \end{equation*}
\begin{equation*}
  \begin{aligned}
    &(2\pi)^{-1} \widehat{f}*\widehat{g}\\
    =&(2\pi)^{-1} \int_{\R}\widehat{f}(\xi-\zeta)\widehat{g}(\zeta)\mathrm{d}\zeta\\
    =& (2\pi)^{-1}\int_{\R}\left( \int_{\R}e^{-ix(\xi-\zeta)}f(x)\mathrm{d}x \right)g(\zeta)\mathrm{d}\zeta\\
    =& \int_{\R}e^{-ix\xi}f(x)\left( (2\pi)^{-1}\int_{\R}e^{ix\zeta}g(\zeta)\mathrm{d}\zeta \right) \mathrm{d}\xi\\
    = & \int_{\R}e^{-ix\xi}f(x)g(x)\mathrm{d}x\\
    =& \widehat{fg}.
  \end{aligned}
\end{equation*}
\end{proof}
\begin{exercise}
  \begin{enumerate}
    \item []
    \item [(1)]若$| \widehat{f}(\xi)| \le e^{-|\xi|},\xi \in \R $,求证$f\in C^{\infty}(\R)$.
    \item [(2)]若$1\le p\le 2$,且对于任意$f\in L^{p}$ 有不等式
      \[
	\|\widehat{f}\|_{L^{q}(\R)}\le C\|f\|_{L^{p}(\R)}
      \] 
      求证: $q$ 必须满足 $\frac{1}{p}+\frac{1}{q}=1$.
  \end{enumerate}
\end{exercise}
\begin{proof}(1)这是Paley-Wiener定理的特殊情况,只要在定理中取$a=\frac{1}{2}$,则 $e^{\frac{1}{2}|\xi|}|\widehat{f}|\le e^{-\frac{1}{2}|\xi|}$,所以$e^{\frac{1}{2}|\xi|}\widehat{f}\in L^{2}(\R)$,从而$f$ 在带形区域$\left\{(x+iy):x\in \R,|y|\le \frac{1}{2}\right\} $解析,当然有$f\in C^{\infty}$,命题得证.
 
  (2)假设对于所有的$f\in L^{p}$ 都有
  \begin{equation}\label{2-5}
   \|\widehat{f}\|_{L^{q}(\R)}\le C \|f\|_{L^{p}(\R)},
 \end{equation}
 用$\varphi(x)=f(\lambda x),\lambda\neq 0$代替 $f(x)$可得
 \begin{equation}\label{2-6}
    \|\widehat{\varphi}\|_{L^{q}(\R)}\le C\|\varphi\|_{L^{p}(\R)}.
 \end{equation}
 其中 
\[
  \widehat{\varphi}(\xi)=\frac{1}{\lambda}\widehat{f}\left( \frac{\xi}{\lambda} \right) .
\] 
对新的函数计算相应的范数:
\begin{align*}
  \|\widehat{\varphi}\|_{L^{q}(\R)} =& \left( \frac{1}{\lambda^{q}} \int_{\R}\left|\widehat{f}\left( \frac{\xi}{\lambda} \right)  \right|^{q}\mathrm{d}\xi \right)^{\frac{1}{q}}\\
  = & \left( \frac{1}{\lambda^{q-1}}\int_{\R}\left| \widehat{f}\left( \frac{\xi}{\lambda} \right)  \right| ^{q}\mathrm{d}\left( \frac{\xi}{\lambda} \right)  \right) ^{\frac{1}{q}}\\
  = & \left( \frac{1}{\lambda} \right) ^{\frac{q-1}{q}} \|\widehat{f}\|_{L^{q}(\R)}
,\end{align*}
\begin{align*}
  \|\varphi\|_{L^{p}(\R)} = & \left( \int_{\R}\left| f(\lambda x) \right| ^{p}\mathrm{d}x \right) ^{\frac{1}{p}}\\
  = &\frac{1}{\lambda^{\frac{1}{p}}} \left( \int_{\R}\left| f(\lambda x) \right| ^{p}\mathrm{d}\left( \lambda x \right)  \right)^{\frac{1}{p}}\\
  = & \frac{1}{\lambda^{\frac{1}{p}}} \|f\|_{L^{p}(\R)}
.\end{align*}
上述两个结果代入不等式(\ref{2-6})可得
\[
  \|\widehat{f}\|_{L^{q}(\R)}\le \left( \frac{1}{\lambda} \right) ^{\frac{1}{p}-\frac{q-1}{q}} C\|f\|_{L^{p}(\R)}.
\] 
如果指数$\frac{1}{p}-\frac{q-1}{q}$不等于$0$,例如当它 大于$0$ 的时候,取$\lambda\to \infty$,不等式右边便趋于$0$,这是不可能的(小于$0$时就令 $\lambda\to 0$). 所以指数只能等于$0$.所以
 \[
\frac{1}{p}-\frac{q-1}{q}=0,
\] 
整理可得
\[
\frac{1}{q}+\frac{1}{p}=1.
\]
\end{proof}

\begin{exercise}
  \begin{enumerate}
    \item []
    \item [(1)] 对于给定函数$w(x)\in C_0^{\infty}$,定义
      \[
	(Tf)(x)=\int_{\R}f(y)w(x-y)\mathrm{d}y, x\in \R,f\in C_0^{\infty}(\R).
      \] 
      证明$T$ 是平移不变算子.
    \item [(2)] 设$1<p<\infty$,利用Mihlin-Hormander乘子定理证明
      \[
	\|\partial_1\partial_2f\|_{L^{p}(\R^2)}\le C\|(\partial_1^2+\partial_2^2)f\|_{L^{p}(\R^2)},\quad \forall f \in C_0^{\infty}(\R^2).
      \] 
  \end{enumerate}
\end{exercise}
\begin{proof}
  \begin{enumerate}
    \item []
    \item [(1)] 
      \begin{align*}
	T(\tau _af) = & Tf(x-a)\\
	= & \int_{\R}f(y-a)w(x-y)\mathrm{d}y\\
	= & \int_{\R}f(y)w(x-y-a)\mathrm{d}y\\
	= & \int_{\R}f(y)w(x-a-y)\mathrm{d}y\\
	= & (Tf)(x-a)\\
	= & \tau _a Tf,\quad,\forall a\in \R,f\in C_0^{\infty}(\R)
      .\end{align*}
      其中第四个等号是利用$y+a$ 代替$a$ 并利用了函数具有紧支集的性质.
    \item [(2)] 令$g=(\partial_1^2+\partial_2^2)f$,则需要证明的不等式等价于
      \[
	\|\partial_1\partial_2(\partial_1^2+\partial_2^2)^{-1}g\|_{L^{p}(\R^2)}\le C\|g\|_{L^{p}(\R^2)}.
      \] 
      因为$\partial_1\partial_2(\partial_1^2+\partial_2^2)^{-1}g$ 的傅里叶变换为$\xi_1\xi_2(\xi_1^2+\xi_2^2)^{-1}$,因此只需说明$m=\xi_1\xi_2(\xi_1^2+\xi_2^2)^{-1}\in \mathcal{M}_{p}(\R)$.根据Mihlin-H\"{o}rmander乘子定理,只需要证明
      \[
	|\partial_{\xi}^{\alpha}m(\xi)|\le C_\alpha |\xi|^{-|\alpha|}
      \] 
      对所有的$|\alpha|\le 2$ 成立.
      \begin{itemize}
	\item [1.] $\alpha=0$.\\
	  因为
	  \[
	    \frac{\xi_1\xi_2}{\xi_1^2+\xi_2^2}\le \frac{1}{2}=\frac{1}{2}|\xi|^{0},
	  \] 
	  所以该情况下定理要求的条件成立.
	\item [2.] $\alpha=1$, $\partial_\xi^{\alpha}=\partial_1.$\\
	  $\partial_1m(\xi)=\frac{\xi^3_2-\xi_1^2\xi_2}{(\xi_1^2+\xi_2^2)^2},|\xi|^{-1}=(\xi_1^2+\xi_2^2)^{-1 /2},$ 我门要验证
	  \[
	    \frac{|\xi_2^3-\xi_1^2\xi_2|}{(\xi_1^2+\xi_2^2)^2}\le C_1 \frac{1}{(\xi_1^2+\xi_2^2)^{\frac{1}{2}}}
	 , \]
	  其中$C_1$ 是某个确定的常数.
	  也就是要证明
	  \begin{align}
	    |\xi_2| |\xi_2^2-\xi_1^2|\le C_1 (\xi_1^2+\xi_2^2)^{\frac{3}{2}}\label{2-7}
	  .\end{align}
	  注意到
	   \[
	     |\xi_2| \le (\xi_1^2+\xi_2^2)^{\frac{1}{2}} \text{ 以及 }|\xi_2^2-\xi_1^2|\le (\xi_2^2+\xi_1^2)
	 , \]
	 代入(\ref{2-7})式的左边并取$C_1=1$即可得到该不等式.
       \item [3.] $\alpha=1,\partial_\xi^{\alpha}=\partial_2$.\\
	 这与上一情况雷同,只是指标互换,验证同上.
       \item [4.]  $\alpha=2,\partial^{\alpha}_\xi=\partial_1^2.$\\
	 $\partial_1^2 m(\xi)= \frac{2\xi_1^3\xi_2-6\xi_1\xi_2^3}{(\xi_1^2+\xi_2^2)^3},|\xi|^{-2}=(\xi_1^2+\xi_2^2)^{-1}$,我们要验证
	 \[
	   \frac{|2\xi_1^3\xi_2-6\xi_1\xi_2^3|}{(\xi_1^2+\xi_2^2)^3}\le C_2 \frac{1}{\xi_1^2+\xi_2^2},
	 \] 
	 其中$C_2$ 是某个确定的常数.也就是要证明
	 \begin{equation}\label{2-8}
	   |2\xi_1\xi_2| |\xi_1^2-3\xi_2^2|\le C_2(\xi_1^2+\xi_2^2)^2.
	 \end{equation}
	 注意到
	 \[
	   |2\xi_1\xi_2|\le \xi_1^2+\xi_2^2\text{ 以及 }|\xi_1^2-3\xi_2^2|\le 3(\xi_1^2+\xi_2^2),
	 \] 
	 代入(\ref{2-8})式的左边并取$C_2=3$即可得到该不等式.
       \item [5.] $\alpha=2,\partial^{\alpha}=\partial_2^2$.\\
	 这与上一情况相同,只是指标互换,验证同上.
       \item [6.] $\alpha=2,\partial^{\alpha}=\partial_1\partial_2$.\\
	 $\partial_1\partial_2m(\xi)= \frac{6\xi_1^2\xi_2^2-\xi_1^{4}-\xi_2^{4}}{(\xi_1^2+\xi_2^2)^3},|\xi|^{-2}=(\xi_1^2+\xi_2^2)^{-1}$,我们要验证
	 \[
	   \frac{|6\xi_1^2\xi_2^2-\xi_1^{4}-\xi_2^{4}|}{(\xi_1^2+\xi_2^2)^{3}}\le C_3 \frac{1}{\xi_1^2+\xi_2^2}
	 ,\]
	 其中$C_3$ 是某个确定的常数.也就是要证明
	 \begin{equation}\label{2-9}
	   |6\xi_1^2\xi_2^2-\xi_1^{4}-\xi_2^{4}|\le (\xi_1^2+\xi_2^2)^2,
	 \end{equation}
	 注意到
	 \begin{align*}
	   | 6\xi_1^2\xi_2^2-\xi_1^{4}-\xi_2^{4}| = & |(\xi^2+\xi_2^2)^2-8\xi_1^2\xi_2^2|\\
	   \le & (\xi_1^2+\xi_2^2)^2+8\xi_1^2\xi_2^2\\
	   \le & 3(\xi_1^2+\xi_2^2)^2 
	 .\end{align*}
	 代入(\ref{2-9})式的左边并取$C_3=3$ 即可得到该不等式.
      \end{itemize}
      综上,Mihlin-H\"{o}rmander乘子定理的所有条件都满足.
  \end{enumerate}
\end{proof}
\begin{exercise}
  详细证明测不准原理
  \[
    \|xf\|_{L^2(\R)}\|\partial_x f\|_{L^{2}(\R)}\ge \frac{1}{2}\|f\|^2_{L^2(\R)}.
  \] 
\end{exercise}
\begin{proof}
  由$\widehat{\partial_x f}(\xi)=i\xi\widehat{f}(\xi)$ 以及Plancherel定理可将要证明的不等式转化为
  \[
    \|xf\|_{L^2(\R)}\|\xi \widehat{f}\|_{L^2(\R)}\ge \frac{1}{2}\|f\|^2_{L^2(\R)}.
  \] 
  定义$[A,B]=AB-BA$, $(\cdot ,\cdot )$表示$L^2$ 中的内积
  \[
    (f,g):=\int_{\R}f(x)\overline{g}(x)\mathrm{d}x.
  \] 
定义$D=i\partial_x$, 考虑
  \[
    I=\left( [x,D]f,f \right) .
  \] 
  下面用两种方式估计$I$.

  一方面,利用交换子的定义可得
   \begin{align*}
     I= & \left( xDf,f \right) -\left( Dxf,f \right) \\
     = & \left( Df,xf \right) -\left( xf,Df \right) \\
     = & 2 \mathrm{Im} \left( Df,xf \right) \\
     \le & 2 \|Df\|_{L^2(R)}\|xf\|_{L^2(\R)}.
  \end{align*}
最后一步用到了Cauchy-Schwartz不等式.

另一方面,直接计算可得
\begin{align*}
  [x,D]f = & xDf-Dxf\\
  = & xDf-Dxf=if.
\end{align*}
因此
\[
  I=\left( [x,D]f,f \right) =i(f,f)=i\|f\|^2_{L^2(\R)}.
\] 
联合第一步可知
\[
  \|f\|^2_{L^2(\R)}\le 2 \|\xi f\|_{L^2(\R)}\|xf\|_{L^2(\R)}.
\] 
\end{proof}
\section{椭圆方程}
\begin{exercise}
  设$f\in  L^2(\R^{n}),u\in W^{1,2}(\R^{n})$是方程
  \[
    (1+\Delta)u=f,\quad x\in \R^{n}
  \] 
  的弱解.证明:对于任意有界开集$\Omega \subset \R^{n}$,有$u\in W^{2,2}(\Omega)$.
\end{exercise}
\begin{proof}
  根据弱解的定义,我们有
  \begin{equation}\label{10}
    \int_{\R^{n}}\sum_{i=1}^{n} \partial_i u \partial_i v\mathrm{d}x=\int_{\R^{n}}(u-f)v\mathrm{d}x\quad\forall v\in W_0^{1}(\R^{n}).
  \end{equation}
  对任意的函数$f(x)$,定义在$x$ 点的$e^{i}$ 方向步长为$h$的差分为
   \[
     \Delta_i^{h}f(x):= \frac{f(x+he^{i})-f(x)}{h}.
  \]
  则对任意的$j=1,2,\cdots,n$,设$\Omega$ 是任意的有界开集,设$\Omega \subset  \subset  \Omega'$,$\Omega'$也是一个有界开集.设$v$ 的支撑在$\Omega'$中, 则有
  \begin{align*}
    \int_{\R^{n}}\sum_{i=1}^{n} \Delta_j^{h}\partial_i u\partial_iv\mathrm{d}x=& -\int_{\R^{n}}\sum_{i=1}^{n} \partial_i u \partial_i \Delta^{-h}_j v \mathrm{d}x\\
    =& -\int_{\R^{n}}(u-f)\Delta^{-h}_j v\mathrm{d}x\\
    \le & C(n) \left( \|u\|_{L^2(\R^{n})} + \|f\|_{L^2(\R^{n})} \right) \|\nabla v\|_{L^2(\Omega')}
  .\end{align*}
  其中第二个等号是将(\ref{10})式中的$v$ 替换成$\Delta^{-h}_j v$.为了得到$u$ 的二阶弱导数的存在性,设$\eta \in C_0^{1}(\Omega')$并且满足$0\le \eta\le 1$,令$v=\eta^2 \Delta_k^{h}u$,则
  \begin{align*}
    \int_{\R^{n}}| \eta \nabla \Delta_k^{h}u|^2\mathrm{d}x
      = & \int_{\R^{n}}\sum_{i=1}^{n} \eta^2 \Delta_k^{h} \partial_i u \Delta_k^{h}\partial_iu\mathrm{d}x\\
      = & \int_{\R^{n}}\sum_{i=1}^{n}  \Delta^{h}_k \partial_i u \left( \partial_i v-2 \Delta^{h}_k u \eta \partial_i \eta \right) \mathrm{d}x\\
     = & \int_{\R^{n}}\sum_{i=1}^{n} \Delta_k^{h}\partial_i u \partial_iv \mathrm{d}x-2\int_{\R^{n}}\sum_{i=1}^{n} \Delta_k^{h}\partial_i u \Delta_k^{h}u \eta \partial_i\eta \mathrm{d}x\\
     \le & C(n)\left( \|u\|_{L^2(\R^{n})}+\|f\|_{L^2(\R^{n})} \right) \|\nabla v\|_{L^2(\Omega')}\\
     + & 2\|\eta \nabla  \Delta^{h}_k u\|_{L^2(\Omega')}\|\Delta^{h}_k u \nabla u\|_{L^2(\Omega')}
  .\end{align*}
  注意到
  \[
    \|\nabla v\|_{L^2(\Omega')}= \|2\eta\nabla\eta \Delta^{h}_ku +\eta^2\nabla \Delta^{h}_ku \|_{L^2(\Omega')}\le \|2\eta \nabla \eta \Delta^{h}_k u\|_{L^2(\Omega')}+\|\eta \nabla \Delta^{h}_ku\|_{L^2(\Omega')}
  \]
  所以
  \begin{align*}
    \int_{\Omega'}|\eta \nabla  \Delta^{h}_k u|^2\mathrm{d}x\le & C(n)\left( \|u\|_{L^2(\R^{n})}+\|f\|_{L^2(\R^{n})} \right) \left( 2\|\eta  \Delta^{h}_ku \nabla \eta\|_{L^2(\Omega')}+\|\eta \nabla  \Delta^{h}_k u\|_{L^2(\Omega')} \right) \\
    + & 2 \|\eta \nabla \Delta^{h}_ku\|_{L^2(\Omega')}\|\Delta^{h}_ku \nabla \eta\|_{L^2(\Omega')}
  .\end{align*}
注意到$0\le \eta\le 1$,
  所以
  \begin{align*}
    \int_{\Omega'}|\eta\nabla \Delta^{h}_ku|^2\mathrm{d}x\le & C \left( \|u\|_{L^2(\R^{n})}+\|f\|_{L^2(\R^{n})} \right)\|\Delta_k^{h}u\nabla \eta\|_{L^2(\Omega')} \\
    + & C\left( \|u\|_{L^2(\R^{n})}+\|f\|_{L^2(\R^{n})} +\|\Delta_k^{h}u \nabla \eta\|_{L^2(\Omega')}\right) \|\eta \nabla \Delta^{h}_ku\|_{L^2(\Omega')}.
  \end{align*}
  第一项小于等于$C\left( \|u\|_{L^2(\R^{n})}+\|f\|_{L^2(\R^{n})} +\|\Delta^{h}_ku \nabla \eta\|_{L^2(\Omega')}\right)^2 $.对第二项利用Young不等式$ab\le\varepsilon a^{p}+\varepsilon ^{-q /p}b^{q}$ 可得
  \begin{align*}
    &C\left( \|u\|_{L^2(\R^{n})}+\|f\|_{L^2(\R^{n})}+\|\Delta^{h}_k u\nabla \eta\|_{L^2(\Omega')} \right) \|\eta \nabla  \Delta^{h}_k u\|_{L^2(\Omega')}\\
    \le & \varepsilon  \|\eta \nabla \Delta^{h}_k u \|_{L^2(\Omega')}^2\\
    + & \frac{1}{\varepsilon } C^2 \left( \|u\|_{L^2(\R^{n})}+\|f\|_{L^2(\R^{n})}+\|\Delta^{h}_k u\nabla \eta\|_{L^2(\Omega')} \right)^2 
  .\end{align*}
  代入上面的不等式可得($C$在不同的不等式中可以是不同的常数)
  \[
    \|\eta\nabla \Delta^{h}_k u\|_{L^2(\Omega')}\le C \left( \|u\|_{L^2(\R^{n})}+\|f\|_{L^2(\R^{n})}+\|\Delta^{h}_ku\nabla \eta\|_{L^2(\Omega')} \right) 
  \]
设$\eta$ 在$\Omega$ 上为$1$,并且在 $\Omega'$ 上有$\sup|\nabla \eta|\le \frac{2}{d}$,其中$d= \mathrm{dist}\left( \Omega,\partial \Omega' \right) $.则由上述不等式可得
\[
  \|\nabla  \Delta^{h}_k u\|_{L^2(\Omega)}\le C(1+\sup|\nabla \eta|)\left(\|u\|_{L^2(\R^{n})}+\|f\|_{L^2(\R^{n})}+\|\Delta^{h}_ku\|_{L^2(\Omega')}  \right) .
\] 
上述式子对于$|h|<d$ 的所有差分都成立,所以由差分和微分的关系可得$\|\Delta^{h}_k u\|_{L^2(\Omega')}\sim \|\nabla u\|_{L^2(\Omega')}\le \|\nabla u\|_{L^2(\R^{n})}$,所以
\[
  \|\nabla \Delta^{h}_ku\|_{L^2(\Omega)}\le C\left( 1+\sup|\nabla \eta| \right) \left( \|u\|_{W^{1,2}(\R^{n})}+\|f\|_{L^2(\R^{n})}\right). 
\] 
这说明对任意$|h|<d$ 的差分,$\|\nabla \Delta^{h}_ku\|_{L^2(\Omega)}$都有共同的上界,是有界集.因为Hilbert空间的有界集必有弱收敛的子列,所以根据弱导数的定义可知$\nabla u$的弱导数必定存在.
\end{proof}

\begin{exercise}
  设$I=(0,1)$.证明以下结论:
  \begin{enumerate}
    \item [(1)] 若$u\in H^{1}(I)$,则$u_{+}=\max(u(x),0)\in H^{1}(I)$.
    \item [(2)] 存在常数$c>0$,使得对任意$u\in H^{1}_0(I)$ 有
      \[
      \int_{I}|u|^2\mathrm{d}x\le c \int_{I}|\partial_xu|^2\mathrm{d}x.
      \] 
  \end{enumerate}
\end{exercise}
 
\begin{proof}
  \begin{enumerate}
    \item [(1)] 因为$u\in H^{1}(I)$,所以$u_{+}\in L^2(I)$,我们只需要证明$u_{+}$ 存在一阶弱导数.设函数序列$\left\{u_n\right\} $ 依$H^{1}(I)$ 中的范数收敛于$u$,并且 $u_n\in C^{1}(I)$.设$u_{n+}=\max(u_n(x),0)$.对任意的$v\in C_0^{\infty}(I)$,有
      \begin{align*}
	\int_{I}u_{+}Dv\mathrm{d}x=&\lim_{n\to \infty} \int_{I}u_{n+}Dv\mathrm{d}x\\
	=& -\lim_{n\to \infty}\int_{I}Du_{n+}v\mathrm{d}x\\
      .\end{align*}
      因为$\left\{Du_{n+}\right\} $ 在$L^2(I)$中有界,所以必定存在弱收敛的子列,这里不妨设该序列本身就弱收敛,所以一定存在函数$w$ 使得
      \[
      \lim_{n\to \infty}\int_{I}Du_{n+}v\mathrm{d}x=\lim_{n\to \infty}\int_I w v\mathrm{d}x,
      \] 
      进而
      \[
      \int_{I}u_+ Dv\mathrm{d}x=-\int_I w v\mathrm{d}x.
      \] 
      所以$u_+$的弱导数存在且 $u_{+}=w$.
    \item [(2)] 不妨令$u\in C_0^{1}(I)$,则
      \begin{align*}
	\int_{I}|u|^2\mathrm{d}x= & \int_{I} \left| \int_0^{x} Du(y)\mathrm{d}y\right|^2\mathrm{d}x\\
	\le & \int_{I}\left(\int_0^{x}1^2\mathrm{d}y\right)\left( \int_0^{x}|Du(y)|^2\mathrm{d}y \right) \mathrm{d}x\\
	\le & \int_I x\mathrm{d}x \int_I|Du(y)|^2\mathrm{d}y \\
	= & \frac{1}{2}\int_{I}|\partial_x u|^2\mathrm{d}x
      .\end{align*}
  \end{enumerate}
\end{proof}
\section{抛物方程}
\begin{exercise}
  简单叙述Galerkin方法的思路.
\end{exercise}
\begin{proof}
\begin{enumerate}
  \item 首先Galerkin方法是寻求抽象微分方程
  \[
    \frac{\partial u}{\partial t} +Au=f(t),\quad u(0)=u_0
  \] 
  的一个弱解,这是一个常微分方程,只是函数$u$的象的取值是无穷维的函数空间$H^{1}_0$,可以在该函数空间选择适当的基底,先把函数投影到有限维子空间上,把问题转化为常微分方程组来解决.选择基底最简单的方式是通过计算算子$A$的特征函数.$A$的特征函数能够组成一个基底需要满足一定的条件,这样的条件由Hilbert-Schimidt定理给出.
  \item 通过有限维投影,例如$n$ 维,我们把原问题转化为比较弱的求解
    \[
      \frac{\mathrm{d}\,}{\mathrm{d}t}(u,\phi_j)=\lambda_j(u,\phi_j)=(f(t),\phi_j)\quad j=1,\cdots,n
    \] 
    常微分方程组的问题.根据常微分方程组的解的存在唯一性定理可知,该方程组有唯一的解$u_n$.也就是存在唯一的解满足
    \[
      \frac{\mathrm{d}\,}{\mathrm{d}t}u_n+A u_n=P_n f(t).
    \] 
  \item 因为$\left\{u_n\right\} $ 在$L^2(0,T;H_0^{1})$ 上一致有界,因此存在弱收敛的子列,不妨仍然记为$\left\{u_n\right\} $,使得
    \[
      u_n \rightharpoonup u \text{ in } L^2(0,T;H^{1}_0).
    \] 
    类似可得
    \begin{equation*}
      \begin{aligned}
	\frac{\partial u_n}{\partial t} & \rightharpoonup \frac{\partial u}{\partial t}  \text{ in }L^2(0,T;H^{-1})\\
	\Delta u_n & \rightharpoonup \Delta u \text{ in }L^2(0,T;H^{-1})\\
	P_nf & \rightharpoonup f \text{ in }L^2(0,T;H^{-1}).
      \end{aligned}
    \end{equation*}
  \item 我们需要说明上述的弱极限就是我们要得到的弱解.通过简单的极限操作可得
    \[
      \langle \frac{\mathrm{d}\,}{\mathrm{d}t}u,v\rangle+\langle Au,v\rangle=\langle f,v\rangle.\quad \forall v \in L^2(0,T;H_0^{1}).
    \] 
    以上说明了极限函数$u$ 使得方程
    \[
      \frac{\partial u}{\partial t} +Au=f(t)
    \] 
    在$L^2(0,T;H^{-1})$ 中成立.
    由于弱极限$u$ 满足
    \[
      \|u\|_{L^2(0,T;H^{1}_0})+\|\frac{\partial u}{\partial t} \|_{L^2(0,T;H^{-1})}\le C.
    \] 
    由向量值嵌入定理得$u\in C([0,T],L^2(\Omega))$.因此初值$u(0,x)$ 在$L^2(\Omega)$ 上有意义. 边界条件已经包含在算子$A$的定义域中.所以弱极限就是我们要求的弱解.弱解的唯一性则利用到了 $A$的正定性质.

\end{enumerate}
\end{proof}
\section{半群理论}
\begin{exercise}
  设$A$是Banach空间 $X$上的压缩半群的生成元,求证
   \begin{enumerate}
     \item [(1)] $D(A)$ 在$X$中稠密;
     \item [(2)] $A$ 是闭算子;
     \item [(3)] 对任意 $\lambda>0$,有$\|(\lambda I-A)^{-1}\|\le \frac{1}{\lambda}$.
  \end{enumerate}
\end{exercise}
\begin{proof}
  \begin{enumerate}
    \item [(1)] 对任意的$t>0,u \in X$,定义
      \[
	u^{t}:=\int_0^{t}S(t)u(t)\,\mathrm{d}t.
      \] 
      因为在$X$ 上有$\frac{u(t)}{t}\to 0$,我们只要证明对任意的$t>0$,都有$u^{t}\in D(A)$ 即可.实际上,
      \begin{equation}
        \begin{aligned}
	  \frac{S(h)u^{t}-u^{t}}{h}&=\frac{S(h)\int_0^{t}S(\tau )u\,\mathrm{d}\tau -\int_0^{t}S(\tau )\,\mathrm{d}\tau }{h}\\
	  = &\frac{1}{h} \int_0^{t}S(\tau +h) u\,\mathrm{d}\tau - \frac{1}{h}\int_0^{t}S(\tau )u\,\mathrm{d}\tau\\
	  = & \frac{1}{h}\int_h^{h+t }S(\tau )u\,\mathrm{d}\tau - \frac{1}{h}\int_0^{t}S(\tau )u \,\mathrm{d}\tau \\
	  = & \frac{1}{h}\left( \int_h^{h+t}-\int_0^{t} \right) S(\tau )u\,\mathrm{d}\tau \\
	  = & \frac{1}{h}\left( \int_t ^{t+h}+\int_h^{t}-\int_0^{t} \right) S(\tau )u\,\mathrm{d}\tau\\
	  = & \frac{1}{h}\left( \int_t ^{t+h} -\int_0^{h} \right) S(\tau )u\,\mathrm{d}\tau \\
	  = & \frac{1}{h}\int_t ^{t+h}S(\tau )u\,\mathrm{d}\tau -\frac{1}{h}\int_0^{h}S(\tau )u\,\mathrm{d}\tau \\
	  \to & S(t)u-u\quad \left( h\to 0 \right) .
        \end{aligned}
      \end{equation}
      这说明$u^{t}\in D(A)$,所以$D(A)$ 在$X$ 中稠密.
    \item 设$u_k\in  D(A) (l=1,2,\cdots)$ 并假设在$X$中
      \begin{equation}\label{eqn-12}
         u_k\to u, A u_k\to v.
       \end{equation}
       因为
       \[
	 S(t)u_k -u_k=\int_0^{t}S(s)Au_k\,\mathrm{d}s.
       \] 
       令$k\to \infty$ 并利用(\ref{eqn-12})可得
       \[
	 S(t)u-u=\int_0^{t}S(s)v\,\mathrm{d}s.
       \] 
       因此
       \[
	 \lim_{t\to 0^{+}} \frac{S(t)u-u}{t}=\lim_{t\to 0^{+}}\frac{1}{t}\int_0^{t}S(s)v \,\mathrm{d}s=v.
       \] 
       这说明$u\in D(A)$ 并且$v=Au$. 
     \item 设$R_\lambda=\left( \lambda I-A \right) ^{-1}$我们首先证明
       \begin{equation}\label{eqn-13}
	 R_\lambda =\int_0^{\infty}e^{-\lambda t}S(t) u \, \mathrm{d}t \quad \left( u\in X \right) .
       \end{equation}
       因为$\lambda>0$ 以及$\|S(t)\|\le 1$,等式(\ref{eqn-13})右边有意义.我们用 $\widetilde{R}_\lambda u$ 来表示右边的积分.对任意的$h>0$ 以及$u \in X$,我们有
  \begin{align*}
    \frac{S(h)\widetilde{R}_\lambda u-\widetilde{R}_\lambda u}{h}=& \frac{1}{h}\left\{ \int_0^{\infty}e^{-\lambda t}[S(t+h)u-S(t)u]\,\mathrm{d}t\right\} \\
    =& -\frac{1}{h}\int_0^{h}e^{-\lambda(t-h)}S(t)u\,\mathrm{d}t\\
    &+ \frac{1}{h}\int_0^{\infty}(e^{-\lambda(t-h)}-e^{-\lambda t})S(t)u\,\mathrm{d}t\\
    = & -e^{\lambda h}\frac{1}{h}\int_0^{h}e^{-\lambda t }S(t)u\,\mathrm{d}t\\
    &+\left( \frac{e^{\lambda h}-1}{h} \right) \int_0^{\infty}e^{-\lambda t}S(t)u\,\mathrm{d}t
  .\end{align*}
  所以
  \[
    \lim_{h\to 0^{+}}\frac{S(h)\widetilde{R}_\lambda u-\widetilde{R}_\lambda u}{h}=-u +\lambda \widetilde{R}_\lambda u.
  \]
  根据$A$ 的定义可得$A \widetilde{R}_\lambda u=-u+\lambda \widetilde{R}_\lambda u$,也就是
  \begin{equation}
    \left( \lambda I-A \right) \widetilde{R}_\lambda u =u \quad (u\in X).\label{eqn-14}
  \end{equation}
  另一方面,若$u\in D(A)$,
  \begin{align*}
    A\widetilde{R}_\lambda u = & A\int_0^{\infty}e^{-\lambda t}S(t)u\,\mathrm{d}t=\int_0^{\infty}e^{-\lambda t}AS(t)u\,\mathrm{d}t\\
    = & \int_0^{\infty}e^{-\lambda t}S(t)Au\,\mathrm{d}t=\widetilde{R}_\lambda A u.
  \end{align*}
第二个等号利用了$A$是闭算子这一事实.因此
 \[
   \widetilde{R}_\lambda \left( \lambda I-A \right) u=u \left( u\in D(A) \right) .
 \] 因为$\lambda I-A$是一一映射且是满射,所以结合上面的等式以及(\ref{eqn-14})可得
  \begin{equation}
    \widetilde{R}_\lambda=(\lambda I-A)^{-1}=R_\lambda.
 \end{equation}
 因此
 \[
 \|R_\lambda u\|\le \int_0^{\infty}e^{-\lambda t}u\,\mathrm{d}t \|u\|\le \frac{1}{\lambda}\|u\|.
 \] 
  \end{enumerate}
\end{proof}

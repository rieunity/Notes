\section{Sobolev空间}
\begin{exercise}
  若$1<p,q<\infty$,$\frac{1}{p}+\frac{1}{q}=1$,证明H\"{o}lder不等式:
  \[
  \|fg\|_{L^{1}}\le \|f\|_{L^{p}}\|g\|_{L^{q}}.
  \] 
\end{exercise}
\begin{proof}
  由不等式的齐次对称性,不妨令$\|f\|_{L^{p}}=\|g\|_{L^{q}}=1$.
  设$\theta = \frac{1}{p}$, 则$1-\theta = \frac{1}{q}$. 令 $F=|f|^{p},G=|g|^{q}$. 则需要被证明的不等式转化为
  \begin{equation}
    \int_{X}F^{\theta}G^{1-\theta}\mathrm{d}\mu\le 1.\label{1}
\end{equation}
  由$\ln x$ 函数的凸性可得
  \[
    F^{\theta}(x)G^{1-\theta}(x)\le \theta F(x) +(1-\theta) G(x).
  \] 
  对上式积分便得到(\ref{1})式.
\end{proof}
\begin{exercise}
  若$f\in L^2(\R^{n})$, 证明$e^{-|x|^2}*f\in L^{p}(\R^{n}),2\le p\le \infty$.
\end{exercise}
\begin{proof}
  令 $q$满足
   \[
  1+\frac{1}{p}=\frac{1}{2}+\frac{1}{q},
  \]
  其中$1\le q\le 2$.
  则由Young不等式可得
  \begin{equation}
    \begin{aligned}
      \|e^{-|x|^2}*f\|_{L^{p}(\R^{n})}\le & \|e^{-|x|^2}\|_{L^{q}(\R^{n})}\|f\|_{L^{2}(\R^{n})}.
    \end{aligned}
  \end{equation}
  而
  \begin{equation*}
    \begin{aligned}
      \|e^{-|x|^2}\|_{L^{q}(\R^{n})}^{q} = & \int_{\R^{n}}e^{-q|x|^2}\mathrm{d}x<\infty,
    \end{aligned}
  \end{equation*}
  且$f\in L^2(\R^{n})$,所以 $e^{-|x|^2}*f \in L^{p}(\R^{n})$.
\end{proof}
\begin{exercise}
  设
  \begin{equation*}
    \phi(x)=\left\{ \begin{aligned}
      & e^{- \frac{1}{1-x^2}}, & -1<x<1,\\
      & 0, & |x|\ge 1.
    \end{aligned}\right.
  \end{equation*}
  求证: $\phi \in C_0^{\infty}(\R)$.
\end{exercise}
\begin{proof}
  设
  \begin{equation*}
    f(x)=\left\{
      \begin{aligned}
	& e^{-\frac{1}{x}}, & x>0,\\
	& 0, & x=0.
      \end{aligned}\right.
  \end{equation*}
当$x>0$ 时,求导得
\begin{align*}
  f'(x)= & \frac{1}{x^2}f(x)\\
  f''(x)= & \left(\frac{1}{x^4}-\frac{2}{x^3}\right)f(x)\\
  f'''(x) = & \left( \frac{1}{x^{6}}-\frac{6}{x^{5}}+\frac{6}{x^{4}} \right) f(x)\\
  & \ldots
.\end{align*}
从上述几个导数易知可以归纳证明得到当$x>0$ 时有
\[
  f^{(k)}(x)=P_{2k}(x^{-1})f(x),
\] 
其中$P_{2k}$ 是次数为$2k$ 的多项式.由多项式和$e$ 指数的增长速率关系可得
\[
  \lim_{x\to 0^{+}}f^{(k)}(x)=\lim_{x\to+\infty}P_{2k}(x)e^{-x}=0.
\] 
所以对任意的$n>0$,都有$\lim_{x\to 0^{+}}f^{(n)}(x)=0$.
再利用分析学中的一个定理:
\begin{framed}
  \noindent \textbf{定理. } 设$f$ 是在$x=a$ 处连续的函数,$f'(x)$ 在$x=a$ 的一个去心领域上处处存在且
  \[
    \lim_{x\to a}f'(x)
  \] 
  存在,则$f'(a)$ 存在且
  \[
    f'(a)=\lim_{x\to a}f'(x).
  \] 
  \begin{proof}
    根据导数定义 
    \[
      f'(a)=\lim_{x\to a}\frac{f(x)-f(a)}{x-a}.
    \] 
    由中值定理$\frac{f(x)-f(a)}{x-a}=f'(\xi_x)$,$a<\xi_x<x$.所以
    \[
      f'(a)=\lim_{x\to a}f'(\xi_x)=\lim_{x\to a}f'(x).
    \] 
  \end{proof}
\end{framed}
把该定理用到上面定义的函数上就得到函数$f$在$x=0$处任意阶右导数都存在且
\[
  f^{(n)}_{+}(0)=0,\quad n\in \N.
\] 
而当$|x|\le 1$时,由
\begin{align*}
  \phi(x)=&f(1-x^2)\\
  \phi'(x)=&-2xf'(1-x^2)\\
  \phi''(x)=& -2f'(1-x^2)+4x^2f''(1-x^2)\\
	    &\ldots
\end{align*}
知
\[
  \phi^{(n)}_{-}(1)=0, \quad n\in \N.
\] 
另一方面,由$|x|\ge 1$时$\phi(x)=0$知
\[
  \phi^{(n)}_{+}(1)=0, \quad n\in \N.
\] 
所以$x=1$ 时
\[
  \phi^{(n)}(1)=0,\quad n\in \N.
\] 
$x=-1$时同理.

\end{proof}
\begin{exercise}
  若$1\le q\le 2$,证明对任意$u\in C^{1}_0(\R^2)$ 
  \[
    \|u\|_{L^{q}(\R^2)}\le \|u\|_{W^{1,1}(\R^2)}.
  \] 
\end{exercise}
\begin{proof}
  设$x=(x_1,x_2)$.因为$u\in C^{1}_0(\R^2)$,所以$u$ 可以写成
  \[
    u(x)=\int_{-\infty}^{x_1}\partial_1 u(t,x_2)\mathrm{d}t=\int_{-\infty}^{x_2}u(x_1,t)\mathrm{d}t.
  \] 
  进而有
  \[
    |u(x)|\le \int_{-\infty}^{x_i}|\partial_{i}u|\mathrm{d}x_i\le \int_{\R}|\partial_i u|\mathrm{d}x_i, x\in \R^2
  .\]
  那么
  \begin{align*}
    \int_{\R^2}|u(x)|^2\mathrm{d}x\le& \int_{\R^2}\mathrm{d}x\int_{\R}|\partial_1 u(x)|\mathrm{d}x_1\int_\R |\partial_2u(x)|\mathrm{d}x_2\\
    =&\int_{\R^2}|\partial_1 u(x)|\mathrm{d}x \int_{\R^2}|\partial_2 u(x)|\mathrm{d}x
  .\end{align*}
  利用平均不等式(下式第二行到第三行)可得
  \begin{align*}
    \|u\|_{L^2(\R^2)}=&\left( \int_{\R^{n}}|u(x)|^2\mathrm{d}x \right) ^{\frac{1}{2}}\\
    \le& \left( \int_{\R^2}|\partial_1 u(x)|\mathrm{d}x \int_{\R^2}|\partial_2u(x)|\mathrm{d}x \right) ^{\frac{1}{2}}\\
    \le&\frac{1}{2}\left( \int_{\R^2}|\partial_1u(x)|\mathrm{d}x+\int_{\R^2}|\partial_2u(x)|\mathrm{d}x \right)\\
    \le &\frac{1}{2}\left( \|\partial_1u\|_{L^{1}(\R^2)}+\|\partial_2 u\|_{L^{1}(\R^2)}\right)\\
    \le  & \|\nabla  u\|_{L^{1}(\R^2)}\le \|u\|_{W^{1,1}(\R^2)}
  .\end{align*}
  对于$1<q<2$ 的情况,可以利用插值不等式
  \[
    \|u\|_{L^{q}(\R^2}\le \|u\|_{L^{1}(\R^2)}^{\theta}\|u\|^{1-\theta}_{L^{2}(\R^2)} \quad 0<\theta<1
  \] 
  得到.
\end{proof}
\begin{exercise}
  设$f(x)=x^{\frac{1}{4}}$ $x\in [0,1]$,则$f\in C^{0, \frac{1}{4}}$,但是$f\notin C^{0,\mu},\mu>\frac{1}{4}$. 
\end{exercise}
\begin{proof}
  根据$C^{0,\mu}$ 的定义,我们需要计算范数
  \[
    \|f\|_{C^{0,\mu}[0,1]}:=\sup_{x\in [0,1]}|f(x)|+[f]_{\mu,[0,1]},
  \] 
  其中半范数
  \[
    [f]_{\mu,[0,1]}:=\sup_{x,y\in [0,1],x\neq y} \frac{|f(x)-f(y)|}{|x-y|^{\mu}}.
  \]
  $\sup_{x\in [0,1]}|f(x)|=1$,所以只要计算半范数$[f]_{\mu,[0,1]}$.实际上我们可以证明一个一般结论:对于任意的$f(x)=x^{\alpha}$,$0<\alpha<1$,我们都有 $f\in C^{0,\alpha}$.为了证明这个一般结论,我们需要下述引理建立的不等式
  \begin{framed}
    \begin{lemma}
      设$x>0$,$0<\alpha<1$,则
      \[
	(x+1)^{\alpha}\le x^{\alpha}+1.
      \] 
    \end{lemma}
    \begin{proof}
      令$g(x)=(x+1)^{\alpha}-x^{\alpha}-1$,则
      \[
	g'(x)=\alpha \left( \frac{1}{(x+1)^{1-\alpha}}-\frac{1}{x^{1-\alpha}} \right) \le 0.
      \] 
      又因为$g(0)=0$,所以 $g(x)\le g(0)=0$,不等式得证.
    \end{proof}
  \end{framed}
  那么由引理中的不等式可得对$a,b>0$有
  \begin{align*}
  \left (\frac{a}{b}+1\right)^{\alpha}\le&\left(\frac{a}{b}\right)^{\alpha}+1\\
  (a+b)^{\alpha}\le &a^{\alpha}+b^{\alpha}.
  \end{align*}
  (上式对$a=0$ 或$b=0$ 时是显然的.)令$a=x-y,b=y,x>y$可得
   \[
     x^{\alpha}\le (x-y)^{\alpha}+y^{\alpha}.
   \]
   所以当$x>y$时,有
    \[
      x^{\alpha}-y^{\alpha}\le (x-y)^{\alpha}
   \] 
 特别地,当$\mu=\frac{1}{4}$且$x>y$ 时,
  \begin{align*}
    |f(x)-f(y)|=&|x^{\frac{1}{4}}-y^{\frac{1}{4}}|\\
    \le  & |x-y|^{\frac{1}{4}}
  .\end{align*}
  $x<y$ 的情况同理.上述不等式说明\[
    [f]_{\frac{1}{4},[0,1]}\le 1<\infty.
  \] 
  所以$f\in C^{0,\frac{1}{4}}$.
  对于$\mu>\frac{1}{4}$ 的情形,我们有
  \[
    \sup_{x\neq y} \frac{|f(x)-f(y)}{|x-y|^{\mu}}\overset{\text{令}y=0}{\ge }\sup_{x\in (0,1]} \frac{f(x)}{x^{\mu}}=\sup_{x\in (0,1]}x^{\frac{1}{4}-\mu}=\infty.
  \]  
  所以$f\notin C^{0,\mu},\mu>\frac{1}{4}$.
\end{proof}
\begin{exercise}
  设$\Omega =\left\{ x\in \R^{2}:|x|<1 \right\} $,说明$u\in W^{1,2}(\Omega)$ 但是$u \notin L^{\infty}(\Omega)$.其中
  \[
    u(x)=\ln \ln (1+\frac{1}{|x|}),\quad |x|<1,\quad,|x|=\sqrt{x_1^2+x^2_2}. 
  \]
\end{exercise}
\begin{proof}
  对$u(x)$ 求偏导可得
  \begin{align*}
    \partial_1u=&-\frac{1}{\ln(1+\frac{1}{|x|})}\frac{1}{1+\frac{1}{|x|}} \frac{x_1}{|x|^3},\\ 
    \partial_2u=&-\frac{1}{\ln(1+\frac{1}{|x|})}\frac{1}{1+\frac{1}{|x|}} \frac{x_2}{|x|^3}
  .\end{align*}
  从而
  \begin{align}\label{3}
    |\partial_i u|=\left| \frac{1}{|x| \ln(1+\frac{1}{|x|})} \frac{1}{|x|+1} \frac{x_i}{|x|}  \right| \le &  \frac{1}{|x|\ln(1+\frac{1}{|x|})}
  .\end{align}
   利用不等式$\ln(1+x)\le x$可得
   \[
     |u(x)|\le \ln \frac{1}{|x|}=-\ln|x|.
  \] 
  则利用该不等式以及极坐标来表示积分,$r=|x|,0<1<r$,可得
  \[
    \|u\|_{L^2(\Omega)}^2=\int_{\Omega}|u(x)|^2\mathrm{d}x\le \int_{\Omega}r\ln^2r \mathrm{d}r\mathrm{d}\theta=2\pi\int_0^{1}r\ln^2r\mathrm{d}r<\infty.
  \]
  另一方面,利用(\ref{3})式得
  \[
    \|\partial_i u\|_{L^2(\Omega)}^2=\int_{\Omega}|\partial_iu|^2\mathrm{d}x\le \int_{\Omega} \left(\frac{1}{r\ln( 1+\frac{1}{r})}\right)^2 r\mathrm{d}r\mathrm{d}\theta=2\pi\int_0^{1}\frac{1}{r \ln^2(1+\frac{1}{r})}\mathrm{d}r.
  \] 
  由$-\mathrm{d}\left( \ln(1+\frac{1}{r}) \right)=\frac{1}{r^2+r}\mathrm{d}r $,令$\ln (1+\frac{1}{r})=u$ 得
  \[
    \|\partial_iu\|_{L^2(\Omega)}\le 2\pi\int_{\ln 2}^{\infty}\frac{r+1}{u^2}\mathrm{d}u\le 4\pi \int_{\ln 2}^{\infty}\frac{1}{u^2}\mathrm{d}u<\infty.
  \] 
  综上可知,$u\in W^{1,2}(\Omega)$.

  由$u$在去心邻域 $\Omega \backslash 0$上的连续性以及$\lim_{|x|\to \infty}u(x)=\infty$知显然有$u\notin L^{\infty}(\Omega)$.
\end{proof}


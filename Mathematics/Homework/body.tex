\section{Sobolev空间}
\begin{exercise}
  若$1<p,q<\infty$,$\frac{1}{p}+\frac{1}{q}=1$,证明H\"{o}lder不等式:
  \[
  \|fg\|_{L^{1}}\le \|f\|_{L^{p}}\|g\|_{L^{q}}.
  \] 
\end{exercise}
\begin{proof}
  由不等式的齐次对称性,不妨令$\|f\|_{L^{p}}=\|g\|_{L^{q}}=1$.
  设$\theta = \frac{1}{p}$, 则$1-\theta = \frac{1}{q}$. 令 $F=|f|^{p},G=|g|^{q}$. 则需要被证明的不等式转化为
  \begin{equation}
    \int_{X}F^{\theta}G^{1-\theta}\mathrm{d}\mu\le 1.\label{1}
\end{equation}
  由$\ln x$ 函数的凸性可得
  \[
    F^{\theta}(x)G^{1-\theta}(x)\le \theta F(x) +(1-\theta) G(x).
  \] 
  对上式积分便得到(\ref{1})式.
\end{proof}
\begin{exercise}
  若$f\in L^2(\R^{n})$, 证明$e^{-|x|^2}*f\in L^{p}(\R^{n}),2\le p\le \infty$.
\end{exercise}
\begin{proof}
  令 $q$满足
   \[
  1+\frac{1}{p}=\frac{1}{2}+\frac{1}{q},
  \]
  其中$1\le q\le 2$.
  则由Young不等式可得
  \begin{equation}
    \begin{aligned}
      \|e^{-|x|^2}*f\|_{L^{p}(\R^{n})}\le & \|e^{-|x|^2}\|_{L^{q}(\R^{n})}\|f\|_{L^{2}(\R^{n})}.
    \end{aligned}
  \end{equation}
  而
  \begin{equation*}
    \begin{aligned}
      \|e^{-|x|^2}\|_{L^{q}(\R^{n})}^{q} = & \int_{\R^{n}}e^{-q|x|^2}\mathrm{d}x<\infty,
    \end{aligned}
  \end{equation*}
  且$f\in L^2(\R^{n})$,所以 $e^{-|x|^2}*f \in L^{p}(\R^{n})$.
\end{proof}
\begin{exercise}
  设
  \begin{equation*}
    \phi(x)=\left\{ \begin{aligned}
      & e^{- \frac{1}{1-x^2}}, & -1<x<1,\\
      & 0, & |x|\ge 1.
    \end{aligned}\right.
  \end{equation*}
  求证: $\phi \in C_0^{\infty}(\R)$.
\end{exercise}
\begin{proof}
  设
  \begin{equation*}
    f(x)=\left\{
      \begin{aligned}
	& e^{-\frac{1}{x}}, & x>0,\\
	& 0, & x=0.
      \end{aligned}\right.
  \end{equation*}
当$x>0$ 时,求导得
\begin{align*}
  f'(x)= & \frac{1}{x^2}f(x)\\
  f''(x)= & \left(\frac{1}{x^4}-\frac{2}{x^3}\right)f(x)\\
  f'''(x) = & \left( \frac{1}{x^{6}}-\frac{6}{x^{5}}+\frac{6}{x^{4}} \right) f(x)\\
  & \ldots
.\end{align*}
从上述几个导数易知可以归纳证明得到当$x>0$ 时有
\[
  f^{(k)}(x)=P_{2k}(x^{-1})f(x),
\] 
其中$P_{2k}$ 是次数为$2k$ 的多项式.由多项式和$e$ 指数的增长速率关系可得
\[
  \lim_{x\to 0^{+}}f^{(k)}(x)=\lim_{x\to+\infty}P_{2k}(x)e^{-x}=0.
\] 
所以对任意的$n>0$,都有$\lim_{x\to 0^{+}}f^{(n)}(x)=0$.
再利用分析学中的一个定理:
\begin{framed}
  \noindent \textbf{定理. } 设$f$ 是在$x=a$ 处连续的函数,$f'(x)$ 在$x=a$ 的一个去心领域上处处存在且
  \[
    \lim_{x\to a}f'(x)
  \] 
  存在,则$f'(a)$ 存在且
  \[
    f'(a)=\lim_{x\to a}f'(x).
  \] 
  \begin{proof}
    根据导数定义 
    \[
      f'(a)=\lim_{x\to a}\frac{f(x)-f(a)}{x-a}.
    \] 
    由中值定理$\frac{f(x)-f(a)}{x-a}=f'(\xi_x)$,$a<\xi_x<x$.所以
    \[
      f'(a)=\lim_{x\to a}f'(\xi_x)=\lim_{x\to a}f'(x).
    \] 
  \end{proof}
\end{framed}
把该定理用到上面定义的函数上就得到函数$f$在$x=0$处任意阶右导数都存在且
\[
  f^{(n)}_{+}(0)=0,\quad n\in \N.
\] 
而当$|x|\le 1$时,由
\begin{align*}
  \phi(x)=&f(1-x^2)\\
  \phi'(x)=&-2xf'(1-x^2)\\
  \phi''(x)=& -2f'(1-x^2)+4x^2f''(1-x^2)\\
	    &\ldots
\end{align*}
知
\[
  \phi^{(n)}_{-}(1)=0, \quad n\in \N.
\] 
另一方面,由$|x|\ge 1$时$\phi(x)=0$知
\[
  \phi^{(n)}_{+}(1)=0, \quad n\in \N.
\] 
所以$x=1$ 时
\[
  \phi^{(n)}(1)=0,\quad n\in \N.
\] 
$x=-1$时同理.

\end{proof}
\begin{exercise}
  若$1\le q\le 2$,证明对任意$u\in C^{1}_0(\R^2)$ 
  \[
    \|u\|_{L^{q}(\R^2)}\le \|u\|_{W^{1,1}(\R^2)}.
  \] 
\end{exercise}
\begin{proof}
  设$x=(x_1,x_2)$.因为$u\in C^{1}_0(\R^2)$,所以$u$ 可以写成
  \[
    u(x)=\int_{-\infty}^{x_1}\partial_1 u(t,x_2)\mathrm{d}t=\int_{-\infty}^{x_2}u(x_1,t)\mathrm{d}t.
  \] 
  进而有
  \[
    |u(x)|\le \int_{-\infty}^{x_i}|\partial_{i}u|\mathrm{d}x_i\le \int_{\R}|\partial_i u|\mathrm{d}x_i, x\in \R^2
  .\]
  那么
  \begin{align*}
    \int_{\R^2}|u(x)|^2\mathrm{d}x\le& \int_{\R^2}\mathrm{d}x\int_{\R}|\partial_1 u(x)|\mathrm{d}x_1\int_\R |\partial_2u(x)|\mathrm{d}x_2\\
    =&\int_{\R^2}|\partial_1 u(x)|\mathrm{d}x \int_{\R^2}|\partial_2 u(x)|\mathrm{d}x
  .\end{align*}
  利用平均不等式可得
  \begin{align*}
    \|u\|_{L^2(\R^2)}=&\left( \int_{\R^{n}}|u(x)|^2\mathrm{d}x \right) ^{\frac{1}{2}}\\
    \le& \left( \int_{\R^2}|\partial_1 u(x)|\mathrm{d}x \int_{\R^2}|\partial_2u(x)|\mathrm{d}x \right) ^{\frac{1}{2}}\\
    \le&\frac{1}{2}\left( \int_{\R^2}|\partial_1u(x)|\mathrm{d}x+\int_{\R^2}|\partial_2u(x)|\mathrm{d}x \right)\\
    \le &\frac{1}{2}\left( \|\partial_1u\|_{L^{1}(\R^2)}+\|\partial_2 u\|_{L^{1}(\R^2)}\right)\\
    \le  & \|\nabla  u\|_{L^{1}(\R^2)}\le \|u\|_{W^{1,1}(\R^2)}
  .\end{align*}
\end{proof}

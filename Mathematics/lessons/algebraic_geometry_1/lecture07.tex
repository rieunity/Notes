\documentclass{amsart}
\usepackage{amssymb,latexsym}
\usepackage{color}
 \definecolor{MyDarkBlue}{rgb}{0,0.08,0.45}\definecolor{yellow}{rgb}{0.99,0.99,0.70}\definecolor{white}{rgb}{1.0,1.0,1.0}\definecolor{black}{rgb}{0.00,0.00,0.00}
 %\pagecolor{yellow}

\theoremstyle{plain}
\newtheorem{theorem}{Theorem}
\newtheorem{corollary}{Corollary}
\newtheorem*{main}{Main~Theorem}
\newtheorem{lemma}{Lemma}
\newtheorem{proposition}{Proposition}
\theoremstyle{definition}
\newtheorem{definition}{Definition}
\newtheorem{example}{Example}
\theoremstyle{remark}
\newtheorem*{remark}{Remark}
\newtheorem*{notation}{Notation}
\newtheorem*{proofofnullstellensatz}{Proof of Nullstellensatz}
\newtheorem*{proofofproductsofaffinevarieties}{Proof of Theorem \ref{15}}
\numberwithin{equation}{section}
\begin{document}
\title[Complete-simple distributive lattices]
{Algebraic Geometry - Lothar G\"{o}ttsche \\
	Lecture 07}
\author{Wang Yunlei}
%\address{Harbin Institute of Technology\\
%	Harbin}
\email{wcghdpwyl@126.com}
%\urladdr{http://math.uwinnebago.edu/menuhin/}
%\thanks{Research supported by the NSF under grant number
%	23466.}
%\keywords{Complete lattice, distributive lattice,
%	complete congruence, congruence lattice}
%\subjclass[2010]{Primary: 06B10; Secondary: 06D05}
\date{June 19, 2017}
 
\maketitle


\begin{theorem}\label{12}
	Let $ X,Y $ be varieties, assume $ Y\subset \mathbb{A}^m $ be a closed affine variety. Then there is a bijection between morphisms $ X\to Y $ and $ k $-algebra homomorphisms $ A(Y)\to \mathcal{O}_X(X) $:
	$$\begin{array}{ccc}
	\{ \text{morphisms } X\to Y \} & \xrightarrow{bijection} & \{ \text{homomorphisms }A(Y)\to \mathcal{O}_X(X) \} \\
	\varphi & \xrightarrow{\qquad\quad} & \varphi^\ast
	\end{array}$$
\end{theorem}
\begin{proof}
	$ \Rightarrow $: Let $ \varphi :X\to Y $ be a morphism, then $ \varphi^\ast: A(Y)\to \mathcal{O}_X(X) $ is a $ k $-algebra homomorphism.
	
	$ \Leftarrow $: Let $\phi:A(Y)\to \mathcal{O}_X(X)  $ be a $ k $-algebraic homomorphism, let $ y_1,\dots,y_n\in A(Y) $ be the coordinate functions. We set
	$$
	f_i=\phi(y_i)\in \mathcal{O}_X(X).
	$$
	Let $ \varphi=(f_1,\dots,f_m):X\to \mathbb{A}^m $.
	This is a morphism from $ X $ to $ Y $. To see it is a morphism we have to show $ \varphi(X)\subset Y $. Let $ h\in I(Y) $, $ h\circ \varphi =h(f_1,\dots,f_m)=h(\phi(y_1),\dots,\phi(y_m))=\phi (h(y_1,\dots,y_m)) $. The second equality is based on the homomorphic property of $ \phi $, for example, if $ h(x_1,x_2)=x_1^2-x_2^3 $, then $ h(\phi(y_1),\phi(y_2))=\phi(y_1)^2-\phi(y_2)^3= \phi(y_1^2)-\phi(y_2^3)=\phi(y_1^2-y_2^3)=\phi(h(y_1,y_2)) $. So $ h(y_1,\dots,y_m)\in A(Y) $, we choose an arbitrary element $ p=(a_1,\dots,a_m)\in Y $, then $ h(y_1,\dots,y_m)(p)=h(a_1,\dots,a_m)=0 $ because $ h\in I(Y) $. So for arbitrary $ h\in I(Y) $, we get $ h\circ\varphi=0 $, it implies $ \varphi(X)\subset \mathop{\cap}_{h\in I(Y)} Z(h)= Y $.
\end{proof}
\begin{example}
	A bijective polynomial map need not to be an isomorphism. For example, let$ X=\mathbb{A}^1 $, $ Y=Z(x_2^2-x_1^3) \subset \mathbb{A}^2 $. Then
	$$
	\varphi=(t^2,t^3):X\to Y
	$$
	is a morphism and bijective and the inverse is
	$$
	\varphi^{-1}(a,b)=\left\lbrace \begin{matrix}
	\frac{b}{a} & \text{ if } a\neq 0\\
	0 & \text{ if } (a,b)=0
	\end{matrix}\right.
	$$
	$ \varphi $ is not an isomorphism($ \varphi^{-1} $ is not a morphism). To show this we see the pull back:
	$$
	\varphi^\ast : A(Y)\to \mathcal{O}_X(X)
	$$
	where $ A(Y)=k[x_1,x_2]/\langle x_2^2-x_1^3\rangle $ and $ A(X)=k[t] $. $ \varphi^\ast $ makes $ x_1\to t^2 $ and $ x_2\to t^3 $. Since $ \varphi^\ast $ is not surjective(there is no element maps into $ t $), $ \varphi^\ast $ is not an isomorphism. By theorem \ref{12} we know $ \varphi $ is not an isomorphism. So bijective morphism is not necessary to be an isomorphism.
\end{example}
\begin{definition}
	Let $ X\subset \mathbb{A}^n $ be a closed variety, $ F\in k[x_1,\dots,x_n]\backslash I(X) $. The principal open defined by $ F $ is $ X_F:=X\backslash Z(F) $.
\end{definition}
\begin{proposition}\label{14}
	$ X_F $ is an affine variety.
\end{proposition}
\begin{proof}
	Let $ Z:=Z(\langle I(X),F\cdot x_{n+1}-1\rangle )\subset \mathbb{A}^{n+1} $. We need to prove $ Z $ is a closed subvariety of $ \mathbb{A}^{n+1} $ isomorphic to $ X_F $. Let $ \varphi:(x_1,\dots,x_n,\frac{1}{F}):X_F\to \mathbb{A}^{n+1} $, it is a bijective morphism and $ \varphi(X_F)=Z $. As $ X_F $ is irreducible, $ Z $ is also irreducible. So $ Z $ is closed variety of $ \mathbb{A}^{n+1} $. On the other hand, the inverse of $ \varphi $ is
	$$
	\varphi^{-1}=(x_1,\dots,x_n):Z\to X_F
	$$
	is a morphism, so $ \varphi $ is an isomorphism.
\end{proof}

\begin{definition}
	Let $ X\subset \mathbb{P}^n,Y\subset \mathbb{P}^m $ be quasi-projective algebraic sets. A map $ \varphi:X\to Y $ is called a polynomial map if there exists homogeneous polynomials $ F_0,\dots,F_m\in k[x_0,\dots,x_n] $ of the same degree with no common zero on $ X $ s.t.
	$ \varphi(p)=[F_0(p),\dots,F_m(p)] $, $ \forall p\in X $, write $ \varphi=[F_0,\dots,F_m] $.
\end{definition}
\begin{definition}
	The homogenization of $ F\in k[x_0,\dots,x_n] $ is:
	$$
	F_a:=F(1,x_1,\dots,x_n).
	$$
\end{definition}
\begin{theorem}\label{13}
	$ \varphi_i=(\frac{x_0}{x_i},\dots,\hat{\frac{x_i}{x_i}},\dots,\frac{x_n}{x_i}):U_i\to \mathbb{A}^n $ is an isomorphism.
\end{theorem}
\begin{proof}
	We can assume $ i=0 $, $ \varphi:=\varphi_0 $, $ U:=U_0 $, then $ \varphi=(\frac{x_1}{x_0},\dots,\frac{x_n}{x_0}) $. $ \frac{x_i}{x_0} $ is a regular function in $ \mathcal{O}_{\mathbb{P}^n}(\mathbb{P}^n) $, so $ \varphi $ is a morphism. We need to show that $ u=\varphi^{-1}(x_1,\dots,x_n)=[1,x_1,\dots,x_n] $ is a morphism.
	
	(a) $ u=\varphi^{-1} $ is continuous. Let $ W=Z(F_1,\dots,F_m)\cap U $ be closed in $ U $, $ F_i\in k[x_0,\dots,x_n] $ are homogeneous, then
	$$\begin{array}{cc}
	u^{-1}(W)= & \{ (a_1,\dots,a_n)\in \mathbb{A}^n|[1,a_1,\dots,a_n]\in W \}\\
	= & \{ (a_1,\dots,a_n)\in \mathbb{A}^n|F_i(1,a_1\dots,a_n)=0, \forall i=1,\dots,m \} \\
	= & Z(F_{1a},\dots,F_{ma})
	\end{array}$$
	where $ F_{ia} $ is homogenization of $ F_i $, it shows that $ u^{-1}(W) $ is closed in $ \mathbb{A}^n  $.
	
	(b) Let $ V\subset U $ be open, $ h\in \mathcal{O}_U(V) $, we need to show $ u^\ast h\in \mathcal{O}_{\mathbb{A}^n}(u^{-1}(V)) $. Making $ V $ smaller necessary, we can assume $ h=\frac{F}{G} $, $ F,G \in k[x_0,\dots,x_n]$ are homogeneous polynomials of the same degree.
	$$
	u^\ast h = h\circ u=\frac{F\circ u}{G\circ u}=\frac{F(1,x_1,\dots,x_n)}{G(1,x_1,\dots,x_n)}.
	$$
	Thus $ u^\ast h\in \mathcal{O}_{\mathbb{A}^n}(u^{-1}(V)) $, $ phi:\mathbb{A}^n\to u $ is an isomorphism.
\end{proof}
\begin{remark}
	From theorem \ref{13} we find that if we identify $ \mathbb{A}^n $ with $ u_0\subset \mathbb{P}^n $, the Zariski topology on $ \mathbb{A}^n $ is equivalent to the induced topology of $ u_0 $ from $ \mathbb{P}^n $.
\end{remark}

 
% \begin{thebibliography}{9}
 %    \bibitem{a} bibitem
% \end{thebibliography}
\end{document}

\documentclass{amsart}
\usepackage{amssymb,latexsym}
\usepackage{graphicx}
\usepackage{color}
 \definecolor{MyDarkBlue}{rgb}{0,0.08,0.45}\definecolor{yellow}{rgb}{0.99,0.99,0.70}\definecolor{white}{rgb}{1.0,1.0,1.0}\definecolor{black}{rgb}{0.00,0.00,0.00}
 %\pagecolor{yellow}

\theoremstyle{plain}
\newtheorem{theorem}{Theorem}
\newtheorem{corollary}{Corollary}
\newtheorem*{main}{Main~Theorem}
\newtheorem{lemma}{Lemma}
\newtheorem{proposition}{Proposition}
\theoremstyle{definition}
\newtheorem{definition}{Definition}
\newtheorem{example}{Example}
\theoremstyle{remark}
\newtheorem*{remark}{Remark}
\newtheorem*{notation}{Notation}
\newtheorem*{proofofnullstellensatz}{Proof of Nullstellensatz}
\newtheorem*{proofofproductsofaffinevarieties}{Proof of Theorem \ref{15}}
\numberwithin{equation}{section}
\begin{document}
\title[Complete-simple distributive lattices]
{Algebraic Geometry - Lothar G\"{o}ttsche \\
	Lecture 18}
\author{Wang Yunlei}
%\address{Harbin Institute of Technology\\
%	Harbin}
\email{wcghdpwyl@126.com}
%\urladdr{http://math.uwinnebago.edu/menuhin/}
%\thanks{Research supported by the NSF under grant number
%	23466.}
%\keywords{Complete lattice, distributive lattice,
%	complete congruence, congruence lattice}
%\subjclass[2010]{Primary: 06B10; Secondary: 06D05}
\date{June 19, 2017}
 
\maketitle
In the previous lecture, we have defined tangent spaces for affine algebraic sets and for general cases. Now we want to prove check that two definitions are identical in affine cases. Recall two definitions 
\begin{definition}[Affine Cases]
	Let $ f\in k [x_1,\dots,x_n] $ and $ p\in \mathbb{A}^n $, the differential of $ f $ at $ p $ is 
	$$
	\mathrm{d}_pf=\sum\limits_{i=1}^n\frac{\partial f}{\partial x_i}(p)\cdot x_i.
	$$
	Let $ X\subset \mathbb{A}^n $ be an affine algebraic set. The tangent space to $ X $ at $ p\in X $ is 
	$$
	t_p(X)=Z(\mathrm{d}_pf|f\in I(X)).
	$$
\end{definition}
\begin{definition}[General Cases]
	Let $ X $ be a variety, $ p\in X $ be a point. The tangent space $ T_p(X) $ is 
	$$
	T_p(X):=(\mathfrak{m}_p/\mathfrak{m}_p^2)^{\ast}
	$$
	where $ \mathfrak{m}_p $ is the maximal ideal of the local ring $ \mathcal{O}_{X,p} $, the symbol $ \ast $ denotes the dual of vector space. In other words, 
	$$
	T_p(X)=\lbrace k \text{ linear maps }\nu: \mathfrak{m}_p/\mathfrak{m}_p^2\to k \rbrace
	$$
	or 
	$$
	T_p(X)=\lbrace k \text{ linear maps }\nu: \mathfrak{m}_p\to k \text{ with } \nu|_{\mathfrak{m}_p^2}=0  \rbrace.
	$$
\end{definition}
For the moment, let $ X\subset \mathbb{A}^n $ be an affine variety.
\begin{definition}
	If $ f\in A(X) $, $ a=(a_1,\dots,a_n)\in t_p(X) $, we define
	$$
	\mathrm{d}_pf(a):=\mathrm{d}_pF(a) 
	$$
	where $ [F]=f $, $ f\in k[x_1,\dots,x_n] $ and $ \mathrm{d}_pF(a)=\sum\limits_{i}^n \frac{\partial F}{\partial x_i}(p)\cdot a_i $.
	
	If $ h=\frac{f}{g}\in \mathrm{p} $, then $ f,g\in A(X) $, $ g(p)\neq 0 $ and $ f(p)=0 $. We define 
	$$
	\mathrm{d}_ph(a)=\frac{\mathrm{d}_pf(a)}{g(p)}.
	$$
	Thus for $ a\in t_p(X) $, we have defined a linear map
	$$
	\partial_a:\mathfrak{m}_p/\mathfrak{m}_p^2\to k.
	$$
	We define a linear map
	$$
	\delta: t_p(X)\to T_p(X).
	$$
\end{definition}
If we can prove $ \delta $ is an isomorphism, then we can identify two definitions.
\begin{theorem}
	\begin{enumerate}
			\item $ \delta $ is an isomorphism. 
			\item Usinng $ \delta $ to identify $ t_p(X) $ and $ T_p(X) $, the two definitions of $ \mathrm{d}_p\varphi $ for morphism $ \varphi :X\to Y $ are identified.
	\end{enumerate}
\end{theorem} 
\begin{proof}
	Let $ p\in X\subset\mathbb{A}^n $, $ t_i:=[x_i-p_i]\in \mathfrak{m}_p $
	
	Injectivity: For any $ a\in t_p(X) $, we have $ \delta (a)=\partial_a $, it is easy to check that $ \partial_a(t_i)=a_i $. If $ \partial_a=0 $, then $ a_i=0 $ for $ i=1,\dots,n $, then $ a=0 $. Hence $ \delta $ is injective.
	
	Surjectivity: To show surjectivity, it is enough to show $ t_1,\dots,t_n $ generate $ \mathfrak{m}_p/\mathfrak{m}_p^2 $ as a vector space over $ k $, If  it is true , then for any $ \nu:\mathfrak{m}_p/\mathfrak{m}_p^2\to k $ let $ a_i=\nu(t_i) $, we get $ \nu=\delta(a) $ where $ a=(a_1,\dots,a_n) $, and it is easy to check that $ a\in t_p(X) $. Now let's prove that $ t_1,\dots,t_n $ generate $ \mathfrak{m}_p/\mathfrak{m}_p^2 $. For $ f=\frac{g}{h}\in \mathfrak{m}_p $, $ f-\frac{g}{h(p)}=\frac{g\cdot(h(p)-h)}{h\cdot h(p)}\in \mathfrak{m}_p^2 $, thus $ f=\frac{g}{h(p)} $ in $ \mathfrak{m}_p/\mathfrak{m}_p^2 $. Since $ \frac{g}{h(p)}\in A(X) $, we know that $ \mathfrak{m}_p/\mathfrak{m}_p^2 $ is generated by elements in $ A(X) $.Then $ \mathfrak{m}_p/\mathfrak{m}_p^2 =k[t_1,\dots,t_n]$. For monomials of degree larger than 2 in $ t_i $, it lies in $ \mathfrak{m}_p^2 $.  Thus $ \mathfrak{m}_p/\mathfrak{m}_p^2 $ is a vector space generated by $ t_1,\dots,t_n $.
\end{proof}
\begin{theorem}
	Let $ X $ be a variety:
	\begin{enumerate}
		\item $ X_{\mathrm{reg}} $ is an open dense subset of $ X $;
		\item for all $ p\in X $, $ \mathrm{dim}T_pX\geq \mathrm{dim}X $.
	\end{enumerate}
\end{theorem}
\begin{proof}
	Any variety $ X $ has an open cover by affine varieties. The theorem is true if it is true for each open set in the cover. Thus we can assume $ X\subset\mathbb{A}^n $ is a closed subvariety. Let $ I(X)=\langle f_1,\dots,f_r\rangle $, $ f_i\in k[x_1,\dots,x_n] $. Then we get 
	$$
	\mathrm{dim}T_p(X)= n- \mathrm{rank}(J(f_1,\dots,f_r)(p)).
	$$
	this formula implies that $ \mathrm{dim}T_p(X)\geq d $ if and only if all the $ n-d+1 $ minors are equal to $ 0 $.
	Thus for all $ d $, $ X_d:=\lbrace p\in X|\mathrm{dim}T_pX\geq d\rbrace $ is closed in $ X $. Then we get a chain
	$$
	X_0\supset X_1\supset \dots\supset X_d\supset X_{d+1}\supset.
	$$
	Choose the largest $ d $ such that $ X_d=X $ and put $ X^0:=X\backslash X_{d+1} $. $ X^0 $ is open and dense in $ X $. Then we know $ \mathrm{dim}T_p(X)\geq d $ for all $ p\in X $ and $ \mathrm{dim}T_p(X)=d $ for all $ p\in X^0 $. Now we only have to show $ d=\mathrm{dim}(X) $. Since $ X $ is birational to a hypersurface $ Y $ in $ \mathbb{A}^{\mathrm{dim}(X)+1} $, there is a nonempty open subset $ U\subset X $ that is isomorphic to an open subset of $ Y_{\mathrm{reg}} $. Then $ \mathrm{dim}T_p(X)=\mathrm{dim}X $ for all $ p\in U $. Thus for all $ p\in X^0\cap U $, $ \mathrm{dim}X=\mathrm{dim}T_p(X)=d $. Thus $ \mathrm{dim}X=d $.
\end{proof}
\begin{corollary}
	\begin{enumerate}
		\item Let $ X\subset\mathbb{A}^n $ be an affine variety, $ I(X)=\langle f_1,\dots,f_r\rangle $.Then the following is equivalent:
		\begin{center}
			$ p\in X $ is nonsingular $ \Leftrightarrow $ $ \mathrm{rank}(J(f_1,\dots,f_r)(p))\geq n-\mathrm{dim}X $.
		\end{center}
		\item Let $ X\subset\mathbb{P}^n $ be a projective variety. Assume $ I(X)=\langle F_1,\dots,F_r\rangle $, where $ F_1,\dots,F_r $ are homogeneous. Then the following is equivalent:
		\begin{center}
			$ p\in X $ is nonsingular $ \Leftrightarrow $ $ \mathrm{rank}(J(F_1,\dots,F_r)(p))\geq n-\mathrm{dim}X $.
		\end{center}
	\end{enumerate}
\end{corollary}
\begin{proof}
	The first  term is obvious. To get the second term, assume $ p\in U_0\cap X $, i.e. $ p $ can be written as $ [1,a_1,\dots,a_n] $. Then $ p $ is nonsingular if and only if $ a=(a_1,\dots,a_n) $ is nonsingular in $ U_0\cap X $. Let $ f_i(x_1,\dots,x_n)=F(1,x_1,\dots,x_n) $ for $ i=1,\dots,n $. Via the first term we only need to show that the rank of $ J(F_1,\dots,F_r) $ is equal to the rank of $ J(f_1,\dots,f_r) $ at $ p $. By definition we know
	$$
	J(F_1,\dots,F_r)(p)=\left( \left.\begin{matrix}
	\frac{\partial F_1}{\partial x_0}(p) \\
	\vdots\\
	\frac{\partial F_r}{\partial x_0}(p)
	\end{matrix}\right| J(f_1,\dots,f_r)(a) \right).
	$$
	By Euler formula for homogeneous polynomial $ F_i $ of degree $ d_i $, we have 
	$$
	\sum\limits_{j=0}^n x_i\frac{\partial F_i}{\partial x_j}= d_iF_i.
	$$
	Then we get 
	$$
	\frac{\partial F_i}{\partial x_0}(p)=-\sum\limits_{j=1}^n a_i\frac{\partial f_i}{\partial x_j}(a).
	$$
	So the first column of $ J(F_1,\dots,F_r)(p) $ is the linear combination of other columns, i.e. $ J(F_1,\dots,F_r)(p)=J(f_1,\dots,f_r)(a) $.
\end{proof}
\begin{lemma}[Nakayama]
	Let $ A $ be a local ring and $ \mathfrak{m}\subset A $ be its maximal ideal. Let $ M $ be a finitely generated $ A $-module:
	\begin{enumerate}
		\item if $ M=\mathfrak{m}M $, then $ M=\lbrace 0 \rbrace $;
		\item write $ k=A/\mathfrak{m} $, let $ f_1,\dots,f_r\in M $ such that $ \bar{f_1},\dots,\bar{f_r} $ generate $ M/\mathfrak{m}M $ as $ k $-vector space. Then $ f_1,\dots,f_r $ generate $ M $ as an  $ A $-module.
	\end{enumerate}
\end{lemma}
\begin{proof}
	(1) Assume $ M\neq \lbrace 0 \rbrace $, let $ \lbrace u_1,\dots,u_r\rbrace $ be a minimal set of generators of $ M $ as an $ A $-module.  Note $ u_r\in M=\mathfrak{m}M $ i.e.
	$$
	u_r=\sum\limits_{i=1}^{r}m_iu_i
	$$
	where $ m_i\in \mathfrak{m} $. Then we get
	$$
	(1-m_r)u_r=\sum\limits_{i=1}^{r-1}m_iu_i.
	$$
	Since $ 1-m_r $ is a unit(if not, then $ 1-m_r \in \mathfrak{m} $, $ 1=1-m_r+m_r\in \mathfrak{m} $), we get
	$$
	u_r=\sum\limits_{i=1}^{r-1}m_i(1-m_r)^{-1}u_i.
	$$ 
 We get a contradiction, thus $ M=\lbrace 0 \rbrace $.
\end{proof}
% \begin{thebibliography}{9}
 %    \bibitem{a} bibitem
% \end{thebibliography}
\end{document}

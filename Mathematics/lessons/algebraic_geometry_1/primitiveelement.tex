\documentclass{amsart}
\usepackage{amssymb,latexsym}
\usepackage{graphicx}
\usepackage{color}
\definecolor{MyDarkBlue}{rgb}{0,0.08,0.45}\definecolor{yellow}{rgb}{0.99,0.99,0.70}\definecolor{white}{rgb}{1.0,1.0,1.0}\definecolor{black}{rgb}{0.00,0.00,0.00}
%\pagecolor{yellow}

\theoremstyle{plain}
\newtheorem{theorem}{Theorem}
\newtheorem{corollary}{Corollary}
\newtheorem*{main}{Main~Theorem}
\newtheorem{lemma}{Lemma}
\newtheorem{proposition}{Proposition}
\theoremstyle{definition}
\newtheorem{definition}{Definition}
\newtheorem{example}{Example}
\theoremstyle{remark}
\newtheorem*{remark}{Remark}
\newtheorem*{notation}{Notation}
\newtheorem*{proofofnullstellensatz}{Proof of Nullstellensatz}
\newtheorem*{proofofproductsofaffinevarieties}{Proof of Theorem \ref{15}}
\numberwithin{equation}{section}
\begin{document}
	\title[Complete-simple distributive lattices]
	{Existence of a Primitive Element}
	\author{Wang Yunlei}
	%\address{Harbin Institute of Technology\\
	%	Harbin}
	\email{wcghdpwyl@126.com}
	%\urladdr{http://math.uwinnebago.edu/menuhin/}
	%\thanks{Research supported by the NSF under grant number
	%	23466.}
	%\keywords{Complete lattice, distributive lattice,
	%	complete congruence, congruence lattice}
	%\subjclass[2010]{Primary: 06B10; Secondary: 06D05}
	\date{June 19, 2017}
	
	\maketitle
	\begin{theorem}[Existence of a Primitive Element]\label{17-1}
		Let $ k $ be a field of characteristic $ 0 $, $ L/k $ is a finite field extension. Then $ \exists b\in L $ such that $ L=k(b) $.
	\end{theorem}
	
	% \begin{thebibliography}{9}
	%    \bibitem{a} bibitem
	% \end{thebibliography}
\end{document}



\documentclass{amsart}
\usepackage{amssymb,latexsym}
\usepackage{graphicx}
\usepackage{color}
 \definecolor{MyDarkBlue}{rgb}{0,0.08,0.45}\definecolor{yellow}{rgb}{0.99,0.99,0.70}\definecolor{white}{rgb}{1.0,1.0,1.0}\definecolor{black}{rgb}{0.00,0.00,0.00}
 %\pagecolor{yellow}

\theoremstyle{plain}
\newtheorem{theorem}{Theorem}
\newtheorem{corollary}{Corollary}
\newtheorem*{main}{Main~Theorem}
\newtheorem{lemma}{Lemma}
\newtheorem{proposition}{Proposition}
\theoremstyle{definition}
\newtheorem{definition}{Definition}
\newtheorem{example}{Example}
\newtheorem{exercise}{Exercise}
\theoremstyle{remark}
\newtheorem*{remark}{Remark}
\newtheorem*{notation}{Notation}
\numberwithin{equation}{section}

\begin{document}
\title{Algebraic Geometry\\
	Affine Varieties}
\author{Wang Yunlei}
\email{wcghdpwyl@126.com}

\date{June 19, 2017}
\maketitle
\begin{exercise}
	     
\end{exercise}
\begin{proof}
	\begin{itemize}
		\item[ ]
		\item[(a)] $ A(Y)= k[x,y]/(y-x^2)\simeq k[x,x^2]=k[x] $.
		\item[(b)] $ A(Z)=k[x,y]/(xy-1)\simeq k[x,\frac{1}{x}]$. If $ A(Z) $ is isomorphic to some polynomial ring, then $ x $ is mapped to an element in $ k $ because $ x $ is invertible. Hence the map is not surjective, which contradicts to the isomorphism.
		\item[(c)] The general form of a quadratic polynomial is $ f= x^2+axy+by^2+cx+dy+e $. We can use the invertible linear transformation to simplify this into $ f=x^2+ay^2+bx+cy+d $. If $ a\neq 0 $, we can use the translation and linear transformation to get $ f=x^2-y^2-d $ and $ d\neq 0 $, it can be transformed into $ f=xy-d $ with $ d\neq 0 $ at last. If $ a=0 $, we can transform this into $ x^2-y $. 
	\end{itemize}
\end{proof}
 \begin{exercise}
 	
 \end{exercise}
 \begin{proof}
 	$ \mathrm{dim}Y=\mathrm{dim}A(Y)=1 $, since $ A(Y)=k[x,y,z]/(y-x^2,z-x^3)\simeq k[x,x^2,x^3]=k[x] $ has dimension $ 1 $.
 \end{proof}
 \begin{exercise}

\end{exercise}
\begin{proof}
	\begin{equation}
		\begin{array}{ccc}
		Y & = & Z(x^2-yz,xz-x)\\
		  & = & Z(x^2-yz)\cap (Z(x)\cup Z(z-1))\\
		  & = & Z(x^2-yz,x)\cup Z(x^2-yz,z-1)\\
		  & = & Z(yz,x)\cup Z(x^2-y,z-1)\\
		  & = & Z(y,x)\cup Z(z,x)\cup (x^2-y,z-1)  
		\end{array}
	\end{equation}
\end{proof}
\begin{exercise}
	
\end{exercise}
\begin{proof}
	$ Z(xy-1) $ is not closed in $ \mathbb{A}^1\times\mathbb{A}^1 $.
\end{proof}
\end{document}

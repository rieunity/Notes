\thispagestyle{empty}


\tableofcontents
\section{Introduction}
Let $f_i(x_1,\cdots ,x_n),i\in \left\{1,\cdots ,k\right\} $ be polynomials with coefficient in $\R$ or $\C$. An \textit{affine algebraic variety} is the common zero of $X=X(f_1,\cdots ,f_k)=\left\{x:f_i(x)=0\forall i\right\} $. We incorporate the field into the notation 
\[
  X\left( \C \right) =\left\{x\in \C^{n}:f_i(x)=0\forall i\right\} ,
\] 
\[
  X(\R)=\left\{x\in \R^{n}:f_i(x)=0 \forall i\right\} \left( f_i\text{have real coefficients} \right). 
\] 
These can be thought of naturally as topological spaces with the topology they inherit from $\C^{n}$ or $\R^{n}$(Alternately, you can use the ``Zariski topology'' induced by declaring that zero sets of polynomials are closed). 

It is frequently in convenient that $X\left( \C \right) $ is essentially never compact. To remedy this, we shift our attention to projective space:
\begin{align*}
  \CP^{n}= &\frac{\C^{n+1}\backslash \left\{0\right\} }{\text{dilations}}\\
  =&\sfrac{\left\{\left( z_0,z_1,\cdots ,z_{n} \right) \in \C^{n+1}\backslash \left\{0\right\} \right\} }{(z_0,\cdots ,z_n)\sim (\lambda z_0,\cdots ,\lambda z_n), \forall \lambda \neq 0}\\
  = & \sfrac{\Sph^{2n+1}}{u(1)},  
\end{align*}
hence compact.

Given homogeneous polynomials $F_i(x_0,\cdots ,x_n)$ we obtain a ``projective variety'' $X=X\left( F_1,\cdots ,F_k \right) =\left\{x\in \CP^{n}:F_i(x)=0\forall i\right\} $. As before, if the polynomial have real coefficients, we have $X(\R)=\left\{x\in \mathbb{RP}^{n}:F_i(x)=0\forall i\right\} $. 

We can ask about the relation between the topology and geometry of $X\left( \R \right) ,X(\C)$ and the algebraic properties of $X$. For example, say $X$ is the zero set of a single homogeneous polynomial of degree $d$ called $F$, can we recover $d$ from looking at $X(\C),X(\R)$?(Only a sensible question for $F$ irreducible.) It turns out that $X_F(\C)$ determines a homology class $\left[ X_F(\C) \right] \in H_{2n-2}\left( \CP^{n};\Z \right) $, and this group is cyclic with generator $\left[ H \right] $ induced by $\CP^{n-1}\hookrightarrow \CP^{n}$, and $\left[ X_F(\C) \right] =d\cdot \left[ H \right] $.

We can recover $d$ from the intrinsic geometry of $X_F\left( \C \right) $ using the ``Chern classes'' in $H^{*}\left( X_F(\C) \right) $. Over the real numbers, $H_{n-1}\left( \mathbb{RP}^{n};\Z_2 \right) $ is cyclic with generator $\left[ H \right] $ and $\left[ X_F\left( \R \right)  \right] =d\cdot \left[ H \right] $, so we recover $d \mod 2$. 
It is possible to show that $X_F(\R)$ does not provide an upper bound for $d$.

From a different point of view, the Nash embedding theorem shows that any smooth manifold over $\R$ is diffeomorphic to $X(R)$ for some real smooth projective variety.
For complex manifolds the analogue statement is false. To be diffeomorphic to a complex projective variety, a manifold must be complex, K\"{a}hler, Hodge, and then it will have an embedding into some $\CP^{n}$ and Chow's theorem guarantees that it's algebraic.

\section{Holomorphic functions}
This section we want to talk about holomorphic functions.

Recall some concepts / theorems from one complex variable. Let   $\mathcal{U}\subset \C, f:\mathcal{U}\to \C, \mathcal{U}$ open,
\begin{enumerate}
  \item $f$ is holomophic $\Leftrightarrow$ f is analytic: $\forall  z_0 \in  u, \exists \varepsilon >0$ s.t. $f(z)=\sum_{n=0}^{\infty} a_n (z-z_0)^n,\forall z\in B_{\varepsilon }(z_0)$.
  \item  $f$ is holomorphic $\Leftrightarrow$ $f$ satisfies Cauchy integral formula: $f\in \C^{1}$ and $\forall z_0 \in  \mathcal{U}$, $\varepsilon $ small enough 
    \[
      f(z_0)=\frac{1}{2\pi i}\int_{\partial B_\varepsilon (z_0)}\frac{ f(z)}{z-z_0}\,\mathrm{d}z.
    \] 
  \item $f$ is holomorphic $\Leftrightarrow$ $f$ satisfies Cauchy-Riemann equations. Writing $z=x+iy,x,y\in \R$, $f(x,y)=u(x,y)+iv(x,y), u,v$ real values, $u,v$ are continuous differentiable and satisfy
    \[
    \frac{\partial u}{\partial x} =\frac{\partial v}{\partial y},  \frac{\partial u}{\partial y} =-\frac{\partial v}{\partial x} . 
    \] 
\end{enumerate}

Let's introduce the differential operators:
\[
  \partial_z=\frac{\partial ~}{\partial z} =\frac{1}{2}\left( \partial_x-i\partial_y \right) ,\partial_{\overline{z}}=\frac{\partial ~}{\partial \overline{z}} =\frac{1}{2}\left( \partial_x+i\partial_y \right) .
\] 	
Then $\partial_z z=1=\partial_{\overline{z}}\overline{z},\partial_z\overline{z}=0=\partial_{\overline{z}}z$. In these terms Cauchy-Riemann equations are $\partial_{\overline{z}}f=0$.

Geometrically, $f:\mathcal{U}\subset \C=\R^{2}\to \C=\R^{2}, f\left( \begin{bmatrix} x\\y \end{bmatrix}  \right)=\begin{bmatrix} u\\v \end{bmatrix}  $ induces $D_{z_0}f:T_{z_0}\R^2\to T_{f(z_0)}\R^2$ with respect to the standard basis, this is the real Jacobian of $f$ :
\[
  J_{\R}(f)=\begin{bmatrix} \frac{\partial u}{\partial x} &\frac{\partial u}{\partial y} \\ \frac{\partial v}{\partial x} &\frac{\partial v}{\partial y}  \end{bmatrix} .
\] 
\begin{example}
  \[
    \left( D_{(x_0,y_0)}f \right) \left( \partial_x \right) =\partial_t |_{t=0}f(x_0+t,y_0)=\partial_t|_{t=0}\begin{bmatrix} u(x_0+t,y)\\v(x_0+t,y) \end{bmatrix} =\begin{bmatrix} \frac{\partial u}{\partial x} &\frac{\partial v}{\partial x}  \end{bmatrix} 
  .\] 
\end{example}
After we complexify, $D_{z_0}^{\C}f:T_{z_0}\R^{2}\otimes \C\to T_{f(z)}\R^{2}\otimes \C$. We can write this matrix in the basis $\partial_z,\partial_{\overline{z}}$:
\[
  \begin{bmatrix} \partial_z(u+iv) &\partial_{\overline{z}}(u+iv)\\ 
  \partial_z(u-iv) & \partial_{\overline{z}}(u-iv)\end{bmatrix}
  = \begin{bmatrix} 
  \partial_z f & \partial_{\overline{z}}f\\
\partial_z \overline{f}&\partial_{\overline{z}}\overline{f}\end{bmatrix} 
.\] 
(note: $\partial_{\overline{z}}\overline{f}=\overline{\partial_z f}, \overline{\partial_z \overline{f}}=\partial_{\overline{z}}f$, etc.)
The funciton $f$ is holomorphic if and only if this matrix is diagnal. The complex Jocabian for $f$ holomorphic is $J_{\C}f=\begin{bmatrix} \partial_z f & 0\\ 0 & \partial_{\overline{z}}\overline{f} \end{bmatrix} $.

Holomorphic functions of one variable satisfy some important theorems:
\begin{enumerate}
  \item Maximum principle: $\mathcal{U}\subset \C$ open, connected, $f:\mathcal{U}\to \C$ is holomorphic, nonconstant, then $|f|$ has no local maximum in $\mathcal{U}$. If $\mathcal{U}$ is bounded and $f$ extends to a continuous $\overline{\mathcal{U}}\to \C$, then $\max |f|$ occurs on $\partial \overline{\mathcal{U}}$.
  \item Identity theorem: $f,g:\mathcal{U}\to \C$ holomoprhic and $\mathcal{U}$ connected. If $$\left\{z\in \mathcal{U}:f(z)=g(z)\right\} $$ contains an open set then it is all of $\mathcal{U}$.
  \item Extension theorem: $f:B_\varepsilon (z_0) \backslash  \left\{z_0\right\} \to \C$ is holomorphic and bounded, then it extends to a holomorphic function $B_{\varepsilon }(z_0)\to \C$.
  \item Liouville's theorem: $f:\C\to \C$ holomorphic and bounded implies $f$ is constant.
  \item Riemann mapping theorem: If $\mathcal{U}\subset \C$is a simply connected, proper open set, then $\mathcal{U}$ is biholomorphic to the unit ball $B_1(0)$.
  \item Residue theorem: If $f:B_\varepsilon (0) \backslash  \left\{0\right\} \to \C$ is holomorphic and $f(z)=\sum_{j\in \Z}^{} a_jz^{j}$ is its Laurent series. Then
    \[
      a_{-1}=\frac{1}{2\pi i}\int_{\partial B_{\varepsilon  /2}(0)}f(z)\,\mathrm{d}z.
    \] 
\end{enumerate}


\begin{definition}
  $\mathcal{U}\subset \C^{n}$ open, $f:\mathcal{U}\to \C$ continuously differentiable,  then $f$ is holomorphic at $a\in \mathcal{U}$ if for all $j\in \left\{1,\cdots ,n\right\} $ the function of one variable 
  \[
    z_j\mapsto f(a_1,\cdots ,a_{j-1}, z_j, a_{j+1},\cdots ,a_n)
  \] is holomorphic. i.e., $\partial_{\overline{z_j}}f=0,\forall j \in \left\{1,\cdots ,n\right\} $.
\end{definition}
If we write 
\[
\mathrm{d}f=\sum \frac{\partial f}{\partial z_j} \mathrm{d}z_j+\sum \frac{\partial f}{\partial \overline{z}_j} \mathrm{d}\overline{z}_j.
\] 
Denote the first term $\partial f$ and the second $\overline{\partial}f$, then $f$ is holomorphic iff $\overline{\partial}f=0$.

\begin{definition}
  For $a\in \C^{n}, R\in \left( \R^{+} \right) ^{n}$, the polydisc around $a$ with multindices $R$ is the set $D(a,R)=\left\{z\in \C^{n}:|z_j-a_j|<R_j,\\forall j\in \left\{1,\cdots ,n\right\} \right\} $. If $R=(1,\cdots ,1)$ and $a=0$ we abbreviate $D(0,1)$ by $\mathbb{D}$ and refer to it as the unit disc in $\C^{n}$. 
\end{definition}
Repeatedly applying the Cauchy formula in $1$-variable, we obtain the following theorem:
\begin{theorem}
  $f:D(w,\varepsilon )\to \C$ holomorphic and $z\in D(w,\varepsilon )$ then 
  \[
    f(z)=\frac{1}{\left( 2\pi i \right) ^{n}}\int_{\partial D(w,\varepsilon )}\frac{f(\zeta_1,\cdots ,\zeta_n)}{(\zeta-z_1)\cdots (\zeta_n-z_n)}\,\mathrm{d}\zeta_1\cdots \mathrm{d}\zeta_n.
  \] 
\end{theorem}
Using this we can show that for any point $w\in U$, $\exists $ multi disc $D(w,\varepsilon )$ inside $U$ such that for all $z\in D(w,\varepsilon )$, $f(z)=\sum_{|\alpha|=0}^{\infty} \frac{\partial^{\alpha}_z f}{\alpha!}(z-w)^{\alpha}$. Here $\alpha$ is a multi-index $\alpha \in \left( \N_0 \right) ^{n}$, and 
\begin{align*}
  (z-w)^{\alpha}&= (z_1-w_1)^{\alpha_1}\cdots (z_n-w_n)^{\alpha_n} \\
  \alpha! &= \alpha_1!\cdots \alpha_n! \\
  \partial^{\alpha}_{z}f=\partial_{z_1}^{\alpha_1}\cdots \partial_{z_n}^{\alpha_n}f
.\end{align*}
From the list above, the maximum principle, the identity theorem and Liouville's theorem generalize easily.
Riemann extension theorem holds but is harder to prove and Riemann mapping definitely fails. There are also phenomenon that do not have analogues in one complex variable. 
\begin{theorem}[Hartogs Extension Theorem]
  $n\ge 2$, $f:U \backslash  K\to \C$ holomorphic with $U \subset  \C^{n}$ open, $K\subset U$ compact. If $U \backslash  K$ is connected, then $\exists ! F:U\to \C$ holomorphic extending $f$.
\end{theorem}

\section{Solving $\partial u = f$}
This section, we will talk about how severable complex variable holomorphic funcitons are NOT like one complex variable holomorphic functions.

\begin{example}
  Consider $H=\left\{(z,w)\in \C^2:|z|<1, \frac{1}{2}<|w|<1\right\}\cup \left\{|z|<\frac{1}{2},|w|<1\right\}  $

\begin{figure}[ht]
    \centering
    \incfig{fig1}
    \caption{}
    \label{fig:fig1}
\end{figure}
Let $f$ be holomorphic function on $H$, then we claim: $\exists F$ holomorphic  on $\mathbb{D}=\left\{|z|<1,|w|<1\right\} $ such that $F|_{H}=f$.
In fact 
\[
  F(z,w)=\frac{1}{2\pi i }\int_{|\zeta|=r} \frac{f(z,\zeta)}{\zeta-w}\,\mathrm{d}\zeta.\quad  \frac{1}{2}<r<1
\] 
is holomorphic. Indeed, 
\[
  \partial_{\overline{z}}\left( \frac{f(z,\zeta)}{\zeta-w} \right) =0. 
\] 
For any fixed $z$ with $|z|<\frac{1}{2}$, $w\mapsto f(z,w)$ is holomorhic on all of $\left\{|w|<1\right\} $ and so we have $F(z,w)=f(z,w)$ for any $|z|<\frac{1}{2},|w|<r$ by the Cauchy integral formula. Hence $F=f$ on all $H$.

The region $H=\mathbb{D}\backslash \text{some closed subset of }\mathbb{D}$ may not use Hartogs extension theorem directly since $\mathbb{D}\backslash H$ is not compact. We provide it here just to give some intuition of the difference betweend several variables and one variable condition.
\end{example}
Before the proof of Hartogs extension theorem, we do some preparations.
\begin{lemma}
  Let $f\in C^1(\overline{\Omega})$, then 
  \[
  \int_{\partial \Omega}f\,\mathrm{d}z=\int_{\Omega}\frac{\partial f}{\partial \overline{z}} \,\mathrm{d}\overline{z}\mathrm{d}z=2 i \int_{\Omega}\frac{\partial f}{\partial \overline{z}} \,\mathrm{d}x\mathrm{d}y.
  \] 
\end{lemma}
Note: When we write $\mathrm{d}\overline{z}\mathrm{d}z$, it means $\mathrm{d}\overline{z}\wedge \mathrm{d}z=2 i \mathrm{d}x\wedge \mathrm{d}y$, the same for $\mathrm{d}x\mathrm{d}y$, hence not the usual one.
\begin{proposition}
  Let $u\in C^{1}(\overline{\Omega})$, then 
  \[
    u(\zeta)=\frac{1}{2\pi i}\int_{\partial \Omega} \frac{u(z)}{z-\zeta}\,\mathrm{d}z-\frac{1}{\pi}\int_{\Omega}\frac{\frac{\partial u}{\partial \overline{z}} (z)}{z-\zeta}\,\mathrm{d}x\mathrm{d}y.
  \] 
\end{proposition}
\begin{proof}
  Fix $\zeta$, let $\varepsilon <d\left( \zeta,\partial \overline{\Omega} \right) $ and let $\Omega_\varepsilon =\left\{z\in  \Omega:|z-\zeta|\ge\varepsilon \right\} $. Applying Lemma to $f(z)=\frac{u(z)}{z-\zeta}$, we obtain
  \[
    \int_{\partial \Omega_\varepsilon } \frac{u(z)}{z-\zeta}\,\mathrm{d}z=2i \int_{\Omega_\varepsilon }\frac{\partial_{\overline{z}}u}{z-\zeta}\,\mathrm{d}x\mathrm{d}y.
  \] 
  As $\varepsilon \to 0$, the LHS converges to $\int_{\partial \Omega} \frac{u(z)}{z-\zeta}\,\mathrm{d}z-2\pi i u(\zeta)$.
\end{proof}
\begin{theorem}
  Let $\phi \in C^{\infty}_c\left( \C \right) $. let $u(\zeta)=\frac{1}{\pi}\int_{\C}\frac{\phi(z)}{z-\zeta}\,\mathrm{d}x\mathrm{d}y$. Then $u$ is an analytic function outside of the $\mathrm{supp}\phi$ and $u$ is smooth on $\C$ and $\partial_{\overline{z}}u=\phi$.
\end{theorem}
\begin{proof}
  Interchanging derivatives and the integral, we see that $u\in C^{\infty}(\C)$ and by a change variables we have
  \[
    u(\zeta)=-\frac{1}{\pi}\int_{\C}\frac{\phi(\zeta-z)}{z}\,\mathrm{d}x\mathrm{d}y.
  \] 
  So \[
    \partial_{\overline{\zeta}}u(\zeta)=-\frac{1}{\pi}\int_{\C}\frac{\partial_{\overline{\zeta}}\phi(\zeta-z)}{z}\,\mathrm{d}x\mathrm{d}y=\frac{1}{\pi}\int_{\C}\frac{\partial_{\overline{z}}u(z)}{\zeta-z}\,\mathrm{d}x\mathrm{d}y.
  \] 
  Applyng the proposition to any disc containing $\mathrm{supp}\phi$, we get that this equals $\phi(\zeta)$, i.e., $\partial_{\overline{\zeta}}u=\phi$.
\end{proof}
\begin{remark}
  Even though $\phi$ has compact support, no solution of $\partial_{\overline{z}}u=\phi$ can have compact support if $\int_{\C}\phi(z)\,\mathrm{d}x\mathrm{d}y\neq 0$. Indeed, if $u(\zeta)=0\forall|\zeta|>R$ for $R$ sufficiently large, then 
  \[
    0=\int_{|z|=R} u(z)\,\mathrm{d}z=\int_{|z|<R}\partial_{\overline{z}}u\,\mathrm{d}x\mathrm{d}y=2i \int_{|z|<R}\phi \, \mathrm{d}x\mathrm{d}y.
  \] 
\end{remark}
\begin{theorem}
  Suppose $f_j\in C_C^{\infty}(\C^{n})$, $ j\in \left\{1,\cdots ,n\right\} ,n>1$ satisfy $\partial_{\overline{z}_j}f_k=\partial_{\overline{z}_k}f_j,\forall j,k \in \left\{1,\cdots ,n\right\} $, then there exists a $u\in C_C^{\infty}(\C^{n})$ such that $\partial_{\overline{z}_j}u=f_j,\forall j$.
\end{theorem}
\begin{proof}
  Define $$u(z)=\frac{1}{2\pi i}\int_{\C} \frac{f_1(\zeta,z_2,\cdots ,z_n)}{\zeta-z_1}\,\mathrm{d} \overline{\zeta} \mathrm{d}\zeta=\frac{-1}{2\pi i}\int_{\C}\frac{f_1\left( z_1-\zeta,z_2,\cdots ,z_n \right) }{\zeta}\, \mathrm{d}\overline{\zeta}\mathrm{d}\zeta.$$
  We note that $u\in C^{\infty}(\C)$ and since $f_1$ has compact suport, $u$ vanishes if $|z_2|+\cdots +|z_n|$ large enough.
  By the previous theorem, $\partial_{\overline{z}_1}u=f_1$. Also differentiating, 
  \[
    \partial_{\overline{z}_j}u=\frac{1}{2\pi i}\int_{\C}\frac{\partial_{\overline{z}_1}f_j\left(\zeta,z_2,\cdots ,z_n  \right) }{\zeta-z_1}\,\mathrm{d}\overline{\zeta}\mathrm{d}\zeta=f_j.
  \]
  Hence $u$ solve the system of equations. Let $K=\bigcup_{j=} ^{n}\mathrm{supp}f_j$, $u$ is holomorphic on $\C^{n}\backslash  K$ and we know that $u$ is zero if $|z_2|+\cdots +|z_n|$ sufficiently large. So by the identity theorem $u$ must vanish on the component of $\C^{n}\backslash  K$ $\Rightarrow$ $u\in C_c^{\infty}(\C^{n})$.
\end{proof}

\section{Proof of Hartogs theorem}
\begin{theorem}
  Let $\mathcal{U}$ be a domain in $\C^{n}$,$n>1$ (domain is nonempty connected open set). Let $K$ be a compact subset of $\mathcal{U}$ s.t. $\mathcal{U}\backslash K$ is connected. Then every holomorophic function on $\mathcal{U}\backslash  K$ extends to a holomorphic function on $\mathcal{U}$.
\end{theorem}
\begin{proof}
  Given $f$ analytic on $\mathcal{U}\backslash  K$, choose $\theta \in  C_c^{\infty}(\mathcal{U})$ s.t. $\theta|_K\equiv 1$. Define $f_0\in C^{\infty}\left( \mathcal{U} \right) $ by setting 
  \[
    f_0(z)=\begin{cases}
      0,&z\in K\\
      (1-\theta)f, z \in U.
    \end{cases}
  \]
  We shall construct $v \in C^{\infty}(\C^{n})$ s.t.  $f_0+v$ is the required holomorphic extension of $f$. In order for $f_0+v$ to be holomorphic we need $\partial_{\overline{z}_j}(f_0+v)=0=\partial_{\overline{z}_j}(1-\theta)f+\partial_{\overline{z}_j}v=-(\partial_{\overline{z}_j}\theta )f+\partial_{\overline{z}_j}v$. That is, we need 
  \[
    \partial_{\overline{z}_j}v=\left( \partial_{\overline{z}_j}\theta \right) f,\quad \forall j\in \left\{1,\cdots ,n\right\} .
  \]

  We know from last section that we can find $v\in C_c^{\infty}(\mathcal{U})$ solving this system of equations.
  Since $v$ has compact support and is holomorphic outside of the support of $\theta$, it must vanish on the unbounded component of $\C^{n}\backslash \mathrm{supp}\theta$. Since $\mathrm{supp}\theta\subset \mathcal{U}$, there is an open set $W$ in $\mathcal{U}\backslash  K$ where $v\equiv 0$ and so $f_0+v= f_0=f$ in $W$. But $\mathcal{U}\backslash K$ is connected and $f,f_0+v$ are holomorphic, so they must coincide on all of $\mathcal{U}\backslash K$. Thus $f_0+v$ is the desired holomorphic extension of $f$.
\end{proof}
\begin{corollary}
  Let $\mathcal{U}\subset \C^{n}$ domain, $n>1$, $f$ holomorphic function in $\mathcal{U}$. The zero set $f^{-1}(0)$ of $f$ is never a compact subset of $\mathcal{U}$.
\end{corollary}
\begin{proof}
  First assume $\mathcal{U}\backslash K$ connected. If $f^{-1}(0)=K$ were a compact subset of $\mathcal{U}$, then we could extend $\frac{1}{f}$ from $\mathcal{U} \backslash K$ to $\mathcal{U}$ holomorphically. In particular, we would get some nonzero value at  $K$ for $\frac{1}{f}$, which is a contradiction. 

  Without knowing that $\mathcal{U}\backslash K$ is connected, the previous proof gives us $g_0+v$ holomorphic with $v\equiv 0$ on a connected component of $\mathcal{U}\backslash K$. It may not agree with $g$ on all of $\mathcal{U}\backslash K$, but this is enough to say that $\frac{1}{f}$ is bounded on some sequence approaching $f^{-1}(0)$ and that's a contradiction.
\end{proof}

Heuristically, we might expect that the zero set of nontrivial holomorphic function  $f:\mathcal{U}\subset \C^{n}\to \C$ will have complex codimension $1$, i.e., complex dimension $n-1$. If $f$ is a polynomial of degree  $1$, then its zero set is an affine subspace of complex dimension $n-1$. If $D_p f:\C^{n}\to \C$ is nonzero at each point $p\in f^{-1}(0)$, then $f$ is ``well-approciamted'' by its linear approximation and $f^{-1}(0)$ should be locally modelled by open subset of $\C^{n-1}$. Indeed, the complex version of the implicit function theorem holds and shows that $f^{-1}(0)$ in this case is a smooth submanifold of complex dimension $n-1$.(Just like in the $\R$ setting)

Things are more complicated if the derivative vanishes. In one complex variable, we know that if  $f(z)=0$ and $f\not\equiv 0$, then $0$ is a root of a finite order, say $p$, meaning that there is a holomorphic function $g$ s.t. $g(0)\neq 0$ and $f(z)=z^{p}g(z)$.

In several complex variables, after a change of coordinates, assume that $F(z_n)=f(0',z_n),0' \in \C^{n-1}$ is non-trivial. So it has a zero of finite order, say $p$, $F(z_n)=z_n^{p}g_{0'}(z_n)$. Using the continuity of $f$ we can apply Rouche's theorem from one complex variable and conclude that there is a polydisc  $D(0',\varepsilon ') \subset \C^{n-1}$ such that for all $z' \in D(0',\varepsilon ')$ the function $z\mapsto f(z',z)$ has exactly $p$ zeros in $D(0,\varepsilon _n)\subset \C$. In particular, we see again that the zeros of a holomorphic function of several variables are not isolated.

\section{Weierstrass preparation theorem}
What does a holomorphic function look like near a zero? In one variable case, we know that if $f(z)=0$, then $f(z)=z^{p}g(z),p \in  \N$, $g$ holomorphic $g(z)\neq 0$ or $g\equiv 0$.

Suppose we're given a function $f(z_1,\cdots ,z_{n-1},w)$ holomorphic near $0\in \C^{n}$, $f(0,\cdots ,0)=0$ and $w$-axis not contained in $f^{-1}(0)$. That is to say, if $f_{z'}(w)=f(z',w),z' \in  \C^{n-1}$, then $f_{0'}(w)$ is not identically zero. We know that $f_{0'}(w)=w^{p}g(w)$, $g(0)\neq 0,p \in \N$. Hence there exists $r>0$ s.t. $|f_{0'}(w)|>\delta>0$ whenever $|w|=r$, so $\exists \varepsilon >0$ s.t. $|z'|<\varepsilon $, $|w|=r$ then $|f_{z'}(w)|>\frac{\delta}{2}$. 

Recall: $h(w)=\widetilde{h}(w)\prod_i (w-a_i)$, then $\frac{h'(w)}{h(w)}=\frac{\widetilde{h}'(w)}{\widetilde{h}(w)}+\sum_{i}^{} \frac{1}{w-a_i}$, so 
\[
  \frac{1}{2\pi i}\int_{\gamma}b(w) \frac{h'(w)}{h(w)}\,\mathrm{d}w=\sum_{i}b(a_i).
\]
Writing $f_{z'}(w)=\widetilde{f}_{z'}(w)\prod_{i=1}^{p}\left( w-a_i(z') \right) $, we see that
\[
  \sum a_i(z')^{q}=\frac{1}{2\pi i}\int_{|w|=r}w^{q}\frac{f'_{z'}(w)}{f_{z'}(w)}\,\mathrm{d}w.
\]
This shows that the LHS is a holomorphic functin of $z'$. Hence the elementary symmetric functions of the  $a_i(z')$ are holomorphic functions of  $z'$. Hence $g_{z'}(w)=w^{p}-\sigma_1(z')w^{p-1}+\cdots +(-1)^{d}\sigma_{d}(z')=\prod_{i}(w-a_i(z'))$ is also a holomorphic function of $z'$. That is, $g(z',w)=g_{z'}(w)$ is holomorphic on $\left\{|z'|<\varepsilon ,|w|<r\right\} $ and on this set it has the same zeros as $f(z',w)$. So lets define 
\[
  h(z',w)=\frac{f(z',w)}{g(z',w)}.
\] 
Clearly it is well-defined and holomorphic off the zero set. Also, for fixed $z'$, $h_{z'}(w)$ has only removable singularities, so it extends to a function on $D(0',\varepsilon ')\times D(0,r)$ which is holomorphic in $w$ for each $z'$ and holomorphic off the zero set. Writing $h(z',w)=\frac{1}{2\pi i}\int_{|u|=r}\frac{h(z',u)}{u-w}\,\mathrm{d}u$, we see that $h$ is holomorphic in $z'$.

A Weierstrass polynomial in $w$ is a polynomial of the form
\[
  w^{p}+\alpha_1(z')w^{p-1}+\cdots +\alpha_p(z'),\quad \alpha_j(0)=0,  \alpha_j \text{ holomorphic},j=1,\cdots ,p.
\]
\begin{theorem}[Weierstrass Preparation Theorem]
  If  $f$ is holomorphic near the origin in $\C^{n}$ such that $f(0)=0$ and not identically zero on the  $w$-axis, then there is a neighborhood of  zero in which $f$ can be written uniquely as 
  \[
  f=g\cdot h
  \] 
  where $g$ is a Weierstrass polynomial of degree $p$ and $h$ doesn't vanish at the origin.
\end{theorem}

\begin{theorem}[Riemann Extension Theorem]
  Suppose $f(z',w)$ not indentically zero is holomorphic in a ball $\mathbb{B}\subset \C^{n}$, and $g(z',w)$ is holomorphic on $\overline{\mathbb{B}} \backslash  f^{-1}(0)$ and bounded. Then $g$ extends to a holomorphic function on $\mathbb{B}$.
\end{theorem}

\begin{proof}
  WLOG, assume that $w$-axis is not contained in  $f^{-1}(0)$. As before, there are $r,\varepsilon $ s.t. $|f(z',w)|>\delta>0$ whenever $|z'|<\varepsilon ,|w|<r$. The $1$-variable Riemann extension theorem applies to each $g_{z'}$ and the extension $\widetilde{g}_{z'}(w)=\frac{1}{2\pi i}\int_{|w|=r}\frac{g_z'(\theta)}{w-\theta}\,\mathrm{d}\theta$. Hence $\widetilde{g}(z',w)=\widetilde{g}_{z'}(w)$ is holomorphic in $(z',w)$ for all $|z'|<\varepsilon ,|w|<r$. 
\end{proof}

 The theorem says A locally bounded holomorphic function on the complement of a ``thin'' set can be uniquely extended.

 Finally, let's discuss the failure of the Riemann mapping theorem in several complex variables.
\begin{example}
  Consider $H=\left\{z\in \C^{n}: \Im z_1> 0\right\} $ and $\mathbb{B}^{n}=\left\{z\in \C^{n}:|z|<1\right\} $. If $\varphi :H\to \mathbb{B}^{n}$ is holomorphic, then for each $z_1$ with $\Im z_1>0$ the function $(z_2,\cdots ,z_n)\mapsto \varphi(z_1,z_2,\cdots ,z_n)$ is holomorphic and bounded on $\C^{n-1}$, hence by Liouville's theorem it's constant.
\end{example}

\begin{theorem}[Poincar\'{e}]
 For $n>1$, the unit polydisc $\mathbb{D}^{n}$ and the unit ball $\mathbb{B}^{n}$ are not biholomorphic. 
\end{theorem}
\begin{proof}
  (Simha, Indian mathematician) Assume $\varphi:\mathbb{D}^{n}\to \mathbb{B}^{n}$ is a biholomorphic with $\varphi(0)=0$. Let $\psi:\mathbb{B}^{n}\to \mathbb{D}^{n}$ be the inverse map. 

  Claim: We must have $D_0\varphi(\mathbb{D}^{n})\subset \mathbb{B}^{n}$ and $D_0 \psi(\mathbb{B}^{n})\subset \mathbb{D}^{n}$. 
  
  Given the claim we have $D_0\varphi(\mathbb{D}^{n})=\mathbb{B}^{n}$. In particular $D_0\varphi(\partial \mathbb{D}^{n})=\partial \mathbb{B}^{n}$, but this is impossible since $D_0\varphi$ is linear, $\partial \mathbb{D}^{n}$ contains linear pieces of positive dimension and $\partial B=\mathbb{B}^{n}$ does not.

  Proof of the claim: 
  \begin{enumerate}
    \item $D_0\varphi(\mathbb{D}^{n})\subset  \mathbb{B}^{n}$. Write $\varphi=\left( \varphi_1,\cdots ,\varphi_n \right) $, $v\in \mathbb{D}^{n}$, $u=(u_1,\cdots ,u_n)\in  \mathbb{B}^{n}$. Applying Schwarz's Lemma to the function
      \[
	t\mapsto u_j\varphi_{j}(tv).
      \] 
      We see that $|\ipd{\overline{u}}{\left( D_0\varphi \right)(v) }|\le 1$. As this holds for all $u \in \mathbb{B}^{n}$, we must have $|D_0 \varphi|\le 1$.
    \item $D_0 \psi(\mathbb{B}^{n})\subset \mathbb{D}^{n}$. Write $\psi=\left( \psi_1,\cdots ,\psi_n \right) $, $u=\left( u_1,\cdots ,u_n \right) \in \mathbb{B}^{n}$. Applying Schwarz's Lemma to the function 
      \[
	t\mapsto \psi_j\left( tu_1,\cdots ,tu_n \right) .
      \] 
      We see that $|\sum_{}^{} u_k \partial_{z_k}\psi_j(0)|\le 1$ for $1\le j\le n$, hence $D_0\psi(\mathbb{B}^{n})\subset \mathbb{D}^{n}$.
  \end{enumerate}
\end{proof}

\section{Complex manifolds}
\begin{definition}
  Let $M$ be a metrizable topological space. A real (complex) coordinate chart is a homeomorphism $\varphi:\mathcal{U}\to \mathcal{V}$ between an open subset $\mathcal{U}\subset M$ and an open subset $\mathcal{V}\subset \R^{n}$ (or $\C^{n}$). A smooth (holomorphic) atlas is a collection of charts $\left\{\varphi_\alpha:\mathcal{U}\to \mathcal{V}_\alpha\right\} $ s.t. $M=\bigcup_{\alpha} \mathcal{U}_\alpha$ and whenever $\mathcal{U}_\alpha \cap \mathcal{U}_{\alpha'}\neq \O$ the 'transition map' $\varphi_\alpha\circ \varphi_{\alpha'}^{-1}:\varphi_{\alpha'}\left( \mathcal{U}_\alpha \cap \mathcal{U}_{\alpha'} \right) \to \varphi_\alpha \left( \mathcal{U}_\alpha \cap \mathcal{U}_{\alpha'} \right) $ between open subsets of $\R^{n}$ (or $\C^{n}$) is smooth (or holomorphic). 
\end{definition}


Two atlases are equivalent if their union is again an atlas. A smooth (or complex) manifold is a metrizable space $M$ together with an equivalence class of atlases.  

We'll assume that all charts use the same dimensional Euclidean space and refer to $n$ as the real dimension of $M$ (or $m$ as the complex dimension of $M$).

\begin{example}
  \begin{enumerate}
    \item Any open subset of $\R^{n}$ is an $n$-dimensional manifold with a single chart and sinimilarly any open subset of $\C^{m}$ is an  $m$-dimensional complex manifold with a single chart.
    \item The sphere $\Sph^2$ is a $2$-dimensional real manifold and a $1$-dimensional complex manifold.

\begin{figure}[ht]
    \centering
    \incfig{sphere}
    \caption{$\Sph^2$ and its chart $\phi_N,\phi_S$ }
    \label{fig:sphere}
\end{figure}
The real charts $\phi_N$ and $\phi_S$ are (See Figure \ref{fig:sphere})
      \begin{align*}
        \phi_N: \Sph^2 \backslash \left\{N\right\}  &\longrightarrow \R^2 \\
	(x,y,z) &\longmapsto \phi_N(	(x,y,z)) = \left( \frac{x}{1-z},\frac{y}{1-z} \right) 
      ,\end{align*}
      \begin{align*}
        \phi_S: \Sph^2 \backslash \left\{S\right\}  &\longrightarrow \R^2 \\
	(x,y,z) &\longmapsto \phi_S(	(x,y,z)) = \left( \frac{x}{1+z},\frac{y}{1+z} \right) 
      .\end{align*}
  What is $\phi_N^{-1}$?
  
  $(u,v)=\left( \frac{x}{1-z},\frac{y}{1-z} \right) $, $u^2+v^2=\frac{x^2}{(1-z)^2}+\frac{y^2}{(1-z)^2}=\frac{1-z^2}{(1-z)^2}$, so $z= \frac{u^2+v^2-1}{u^2+v^2+1}$. Then 
  \[
    \phi_{N}^{-1}(u,v)=\left( \frac{2u}{1+u^2+v^2}, \frac{2v}{1+u^2+v^2}, \frac{u^2+v^2-1}{u^2+v^2+1} \right) .
  \] 
  So 
  \[
    \phi_S\circ \phi_{N}^{-1}(u,v)=\left( \frac{u}{u^2+v^2},\frac{v}{u^2+v^2} \right) .
  \]
  This is a smooth atlas.

  Note that if we use $\phi_N:\Sph^2 \backslash  \left\{N\right\} \to \C,(x,y,z)\mapsto \frac{x+iy}{1-z}$ and $\phi_S:\Sph^2\backslash \left\{S\right\} \to \C,(x,y,z)\mapsto \frac{x+iy}{1+z}$, then
  \[
    \phi_S\circ \phi_N^{-1}(u+iv)= \frac{u+iv}{u^2+v^2}=\frac{1}{u-iv},
  \] 
  i.e., $w\mapsto \frac{1}{\overline{w} }$ and this map is not holomorphic. Simple fix: replace $\phi_S$ with $\widetilde{\phi}_S:\Sph^2 \backslash \left\{S\right\} \to \C, (x,y,z)\mapsto \frac{x-iy}{1+z}$. Then $\widetilde{\phi}_S\circ \phi^{-1}_N(w)=\frac{1}{w}:\C \backslash \left\{0\right\} \to \C\backslash \left\{0\right\} $ and this map is holomorphic. $\Sph^2$ together with the equivalence class of the atlas $\left\{\phi_N:\Sph^2\backslash \left\{N\right\} \to \C,\phi_S:\Sph^2\backslash \left\{S\right\} \to \C\right\} $ is a complex manifold called the Riemann sphere.
\end{enumerate}
\end{example}

\begin{definition}
  If $M$ is a smooth (complex) manifold, a function $M\to \R$ (or $M\to \C$)is smooth at $\zeta \in M$ (holomorphic at $\zeta \in M$) if there is a chart $\varphi:\mathcal{U}\to \mathcal{V}$ with $\mathcal{U}$ a neighborhood s.t. $f\circ \varphi^{-1}:\mathcal{V}\to \R$ is smooth ($f\circ \varphi^{-1}:\mathcal{V}\to \C$ is holomorphic). 

  If $N$ is another smooth (complex) manifold, then a function $f:M\to N$ is smooth (or holomorphic) at $\zeta \in M$ if there are charts $\varphi:\mathcal{U}\to \mathcal{V},\varphi':\mathcal{U}'\to \mathcal{V}'$ with $\mathcal{U}$ a neighborhood of $\zeta$ in $M$ and $\mathcal{U'}$ a neighborhood of $f(\zeta)$ in $M'$ and $\varphi'\circ f\circ \varphi^{-1}:\mathcal{V}\to \mathcal{V}'$is smooth at $\varphi(\zeta)$ (or holomorphic at $\varphi(\zeta)$).
\end{definition}


\begin{example}
  $\mathbb{CP}^{n}=$ the space of all complex lines in $\C^{n+1}$ through the origin = $\left( \C^{n+1}\backslash \left\{0\right\}  \right) / \C^{\times }$. There is a natrual surjective map $\pi:\C^{n+1}\backslash  \left\{0\right\} \to \mathbb{CP}^{n}, z\mapsto [z]$. Here $[z]$ is the equivalence class of $\left\{\lambda z:\lambda \in \C^{\times }\right\} $. Topologize $\mathbb{CP}^{n}$ through $\pi$, i.e., we declare $\pi(\mathcal{U})\subset \mathbb{CP}^{n}$ open for all $\mathcal{U}\subset \C^{n+1}\backslash \left\{0\right\} $ open. For any point $z=\left( z_0,z_1,\cdots ,z_n \right) $, we denote $\pi(z)$ by $(z_0:z_1:\cdots :z_n)$, and call these ``homogeneous coordinates''. Note $(z_0:z_1:\cdots :z_n)=\left( \lambda z_0:\lambda z_1:\cdots :\lambda z_n \right) \forall \lambda \in  \C^{\times }$.

  For $0\le j\le n$, denote $\mathcal{U}_j=\left\{\left( z_0:z_1:\cdots :z_n \right) \in \mathbb{CP}^{n}:z_j\neq 0\right\} $. These form the ``standard open cover of $\mathbb{CP}^{n}$ ''. Define coordinate charts $\varphi_j:\mathcal{U}_j\to \C^{n},(z_0:z_1:\cdots :z_n)\mapsto \frac{1}{z_j}(z_0,\cdots ,\hat{z}_j,\cdots ,z_n).$ These are homeomorphisms with $\varphi_j^{-1}:\C^{n}\to \mathcal{U}_j, (w_1,\cdots ,w_n)\mapsto (w_1:\cdots :1:\cdots :w_n)$ by putting  $1$ in the $j$ spot.
  $\varphi_j\left( \mathcal{U}_j\cap \mathcal{U}_k \right) =\left\{(w_1,\cdots ,w_n) \in \C^{n}: w_k\neq 0 \text{ if }j<k \text{ or } w_{k+1}\neq 0 \text{ if }j>k\right\} $. Assume $j<k$, then 
  \begin{align*}
    \varphi_k\circ \varphi_j^{-1}: \mathcal{U}_j\cap \mathcal{U}_k &\longrightarrow \varphi_k(\mathcal{U}_j\cap \mathcal{U}_k) \\
    (w_1,\cdots ,w_n) &\longmapsto  \frac{1}{w_k}\left( w_1,\cdots ,w_{j-1},1,w_j,\cdots ,\widehat{w}_k,\cdots ,w_n \right) 			
  .\end{align*}
  This is holomorphic, so the atlas of $\left\{\varphi_j:\mathcal{U}_j\to \C^{n}\right\} $ gives $\mathbb{CP}^{n}$ the structure of a complex manifold of complex dimension $n$.

  As smooth manifolds, $\mathbb{CP}^{n}\simeq \Sph^{2n+1} /\Sph^1$, $\Sph^{2n+1}\subset \C^{n+1},\Sph^{1}\subset \C$.
  Hence $\mathbb{CP}^{n+1}$ is compact.
\end{example}

\section{More examples of complex manifods}
Recall the implicit function theorem: Let $\mathcal{U}\subset \R^{n}_\zeta\times \R^{n}_\eta$ be an opens set containing the origin. Suppose $\Phi:\mathcal{U}\to \R^{n}$ is a smooth map, $\Phi(0,0)=0$. If the $k\times k$ matrix $\left( \frac{\partial \Phi_i}{\partial \eta_j} (0,0)\right) $ is invertible, then there exists a neighborhood $V_0 \subset  \R^{n}$ of $0$ and $W_0 \subset  \R^{n}$ of $0$ and a smooth map $F:V_0\to W_0$ s.t. $\mathrm{graph}(F)=\Phi^{-1}(0) \cap  V_0 \times W_0$, i.e., $\Phi(\zeta,\eta)=0$ for $(\zeta,\eta)\in  V_0\times W_0$ iff $\eta=F(\zeta)$.

Usse the implicit function, it is easy to see that there's an implicit function theorem where we replace $\R$ with $\C$ and smooth with holomorphic.
The inverse function theorem gives a coordinate chart on $V(\Phi)=\left\{(\zeta,\eta):\Phi(\zeta,\eta)=0\right\} $. If the rank of $J(\Phi)$ is always $k$ on $\Phi^{-1}(0)$, thn these charts give $\Phi^{-1}(0)$ the structure of a complex manifold of complex dimension $n$.

Since the inverse function theorem is local, it applies to maps between complex maniflds.

\begin{example}
  \begin{enumerate}
    \item $\mathcal{M}_{m\times n}(\C)\simeq \C^{nm}$ is a complex manifold.
    \item $\mathrm{GL}_n(\C)$ is an open subset of $\C^{n^2}$ so a complex manifold.
    \item $\SL_n(\C)=\left\{A \in \mathcal{M}_n(\C):\mathrm{det}A=1\right\}=\mathrm{det}^{-1}(1) $.
      Claim: $1$ is a regular value of $\det:\mathcal{M}_n(\C)\to \C$, i.e., $D_A \det $ is not zero at all $A \in \SL_n(\C)$. Indeet, $D_A \det(A)=\partial_t|_{t=0}\det(A+tA)=\partial_t|_{t=0}(1+t)^{n}\det A=n\neq 0$. Hence by inverse function theorem, $\SL_n(\C)$ is a complex manifold.
    \item Suppose we're given a bilinear form on $\C^{n}$ : $(v,w)\mapsto w^{T}Qv$, $Q=Q^{T}$ or $Q=-Q^{T}$, $Q\in \GL_n(\C)$. Let 
      \begin{align*}
	\mathrm{Or}(Q)&=\left\{A\in \mathcal{M}_n(\C):w^{T}Qv =(Aw)^{T}Q(Av),\forall v,w \in \C^{n}\right\}\\
	&=\left\{A\in \mathcal{M}_n(\C):A^{T}QA=Q\right\}. 
      \end{align*}
      Let $F(A)=A^{T}QA$, $\mathrm{Or}(Q)=F^{-1}(Q)$.
      \begin{align*}
        F: \C^{n^2} &\longrightarrow \text{Symmetric /anti-symmetric} \\
        A &\longmapsto A^{T}QA
      .\end{align*}
	$D_A F(B)=\partial_t|_{t=0}F(A+tB)=B^{T}QA+A^{T}QB$. Because $Q$ is invertible, this maps onto the symmetric or anti-symmetric matrices. Thus $\mathrm{Or}(Q)$ is a complex manifold. If $Q=\begin{bmatrix} 0 & -I_n\\ I_n & 0 \end{bmatrix} $ then $\mathrm{Or}(Q)=\text{Sp}(2n,\C)$. 
  \end{enumerate} 
  These are examples of complex Lie groups. A (complex) Lie group is a (complex) manifold $G$ that is also   a group s.t. the group operations are smooth (holomorphic).

  If in the last example we had used sesquilinear forms so $(v,w)\mapsto w^*Qv$ then the map $F(A)=A^*QA$ is no longer holomorphic. For example, $\mathcal{U}_n=\left\{A\in \mathcal{M}_n(\C):A^*A=I_n\right\} $. $\mathcal{U}\simeq \Sph^{1}$, it is not even even dimensional over $\R$.
\end{example}

\begin{example}[Complex tori]
 Let $X$ be the quotient $\C^{n} / \Z^{2n}$, where $\Z^{2n}\subset \R^{2n}\simeq \C^{n}$ is the natural inclusion. Endow $X$ with the quotient topology $\pi:\C^{n}\to \C^{n} /\Z^{2n}$. If $\mathcal{U}\subset C^{n}$ is a small open subset s.t. 
 \[
   (\mathcal{U}+\gamma)\cap \mathcal{U}=\O,\forall  \gamma \in \Z^{2n},
 \] 
 then $\pi|_{\mathcal{U}}:\mathcal{U}\to \pi (\mathcal{U})$ is a bijection and covering $X$ with these provides a holomorphic atlas. The transition maps are translations by vectors in $\Z^{2n}$. More generally if $V$ is a complex vector space, and $\gamma_1,\cdots ,\gamma_n$ is a real basis over $\R$, then let $\Gamma=\left\{a_1\gamma_1+\cdots a_n\gamma_n:a_i \in \Z\right\} $ $= \text{free abelian group }(\gamma_1,\cdots ,\gamma_n)$, then $X=V / \Gamma$ is a complex manifold. As smooth manifolds, these are diffeomorphic to $(\Sph^{1})^{n}$ but in general they wil not be equivalen as complex manifolds.

 If $n=2$. $\Gamma \subset \C$ is of the form $\gamma_1\Z+\gamma_2\Z$. By multiplying $\C$ by $\pm \frac{1}{\gamma_1}$ we can arrange $\Gamma =\Z+\tau \Z$ with $\Im(\tau )> 0$. It turns out that $\Gamma_1=\Z+\tau_1 \Z$, $\Gamma_2=\Z+\tau _2\Z$ define the same complex manifold if and only if $\exists A\in \SL_2(\Z)=\left\{A\in \mathcal{M}_2(\Z):\det(A)=1\right\} $ s.t. $\tau_1=A(\tau_2)$.  
\end{example}

\begin{definition}
  When $X$ is a topological space with an action of a group $G$, we define the orbit space $X / G$ to be the equivalent classes of the relation $x\sim y\Leftrightarrow \exists g\in G$ s.t. $x=g\cdot y$ and we endow it with the quotient topology from $\pi :X\to X /G$ (need not be ``nice'' topology, e.g., $\C^\times $ acting on $\C$).
  
  Say that the action is ``free'' if $g\cdot x=x \Rightarrow g=I$ and that the action is ``proper'' if $G\times  X\to X \times X,(g,x)\mapsto (g\cdot x,x)$ is a proper map.
\end{definition}
\begin{theorem}
  Given a smooth (holomorphic) free, proper action of a (complex) Lie group  $G$ on a smooth (complex) manifold $X$, the orbit space $X /G$ is naturally a (complex) manifold.
\end{theorem}

\begin{example}
  $\mathbb{CP}^{n}$ and complex tori are examples of this theorem.
\end{example}

\begin{example}[Grassmannian manifolds]
  Let $V$ be a complex vector space of complex dimension  $n+1$. Define  
  \[
    \mathrm{Gr}_k(V)=\left\{W\subset V \text{ subspaces }:\mathrm{dim}W=k\right\}.
  \] 
  So $\mathrm{Gr}_k(\C^{n+1})=\mathbb{CP}^{n}$.

  To show that $\mathrm{Gr}_k(\C^{n+1})$ is a complex manifold, note that $W\in \mathrm{Gr}_k(\C^{n+1})$ is generated by the rows of a $k\times (n+1)$ matrix $A$. Denote these matrices by $\mathcal{A}_{k\times (n+1)}$. This is an open subset of $\mathcal{M}_{k\times (n+1)}$, hence a complex manifold. We obtain a natural surjection $\pi:\mathcal{A}_{k\times (n+1)}\to \mathrm{Gr}_k(\C^{n+1})$ which is the quotient by the natural action of $\GL_k(\C)$ on $\mathcal{A}_{k\times (n+1)}$. 
\end{example}

\section{Vector bundles}

\begin{definition}
  A vector bundle over a manifold is a collection of vector spaces parametrized by the manifold.
\end{definition}
\begin{example}  
Over the circle, we have two rank $1$ real vector bundles.
\begin{enumerate}
  \item The trivial bundle, $E=\Sph^{1}\times \R\to \Sph^{1}$.
  \item The infinite M\"{o}bius band $E=\text{M\"{o}bius}\to \Sph^{1}$.
\end{enumerate}
Both of them are ``locally trivial'' in that every point has a neighborhood $\mathcal{U}$ where the bundle looks like $\mathcal{U}\times \R\to \mathcal{U}$. But while the first bundle is globally trivial, the second one is not.

Similarly, $\Sph^2\subset \R^3$ has a tangent bundle.
\end{example}

\begin{definition}
  Let $M$ be a smooth manifold, $\mathbb{F}\in \left\{\R,\C\right\} $. A rank $k$ $\mathbb{F}$-vector bundle over $M$ consists of 
  \begin{enumerate}
    \item A smooth manifold $E$ together with surjective map $E\xrightarrow{\pi}M$.
    \item For each $\zeta\in M$, the fiber $E_{\zeta}=\pi ^{-1}(\zeta)$ is an $\mathbb{F}$-vector space of dimension $k$.
    \item Every $\zeta \in M$ has a neighborhood $\mathcal{U}$ and a differmorphism $h:\pi^{-1}(\mathcal{U})\to \mathcal{U}\times  \mathbb{F}^{k}$ such that the diagram 
      \[
	\begin{tikzcd}[column sep=tiny]
	\pi^{-1}(\mathcal{U})\arrow[rr,"h"] \arrow[rd,"\pi"']  & & \mathcal{U}\times \mathbb{F}^{k}\arrow[ld, "p"] \\
	& \mathcal{U} &  
      \end{tikzcd}
      \] 
      commute and $h|_{\pi^{-1}(\zeta)}:E_{\zeta }\to \mathbb{F}^{k}$ is an isomorphism. 
  \end{enumerate}
  $M$ is called the base, $E$ the total space, $\mathbb{F}^{k}$ the fiber of the vector bundle; often denote this by $\mathbb{F}^{k}- E\to M$. The data in (c) is called a local trivialization. The trivial rank $k$ $\mathbb{F}$-vector bundle over $M$ is $M\times \mathbb{F}^{k}\xrightarrow{p} M$. I'll denote this the total space of this bundle by $\underline{\mathbb{F}}^{k}$. 
  
\end{definition}
Another way of thinking about a vector bundle comes from looking at the transitions between different local trivializations. If $\left( \mathcal{U}_{\alpha},h_{\alpha} \right) $ and $\left( \mathcal{U}_{\beta},h_{\beta } \right) $ are local trivializations such that $\mathcal{U}_{\alpha}\cap \mathcal{U}_\beta \neq \O$, then we have the map  
\begin{align*}
  h_{\alpha}\circ h^{-1}_{\beta }: \mathcal{U}_\alpha \cap \mathcal{U}_\beta \times \mathbb{F}^{k} &\longrightarrow \mathcal{U}_{\alpha}\cap \mathcal{U}_\beta \times \mathbb{F}^{k} \\
  (\zeta,v) &\longmapsto (\zeta,g_{\alpha \beta }(\zeta)v)
\end{align*}
and so is equivalent to a map $g_{\alpha \beta }:\mathcal{U}_\alpha \cap \mathcal{U}_\beta \to \GL_n(\mathbb{F})$. We call these maps the transition maps. The transition maps $\left( \mathcal{U}_\alpha, g_{\alpha \beta }:\mathcal{U}_{\alpha}\cap \mathcal{U}_{\beta }\to \GL_k(\mathbb{F}) \right) $ make up a (\v{C}ech) cocycle meaning:
$g_{\alpha \alpha}=\mathrm{Id}\forall \alpha$ and $g_{\alpha \beta }\circ g_{\beta \gamma}\circ g_{\gamma \alpha}=\mathrm{Id}$ whenever $\mathcal{U}\cap \mathcal{U}_\beta \cap \mathcal{U}_\gamma\neq \O$. This data determines a rank $k$ $\mathbb{F}$-vector bundle over $M$ and is determined from a rank $k$ $\mathbb{F}$-vector bundle over $M$. 
\begin{definition}
  On a complex manifold $M$,
a $\C$-vector bundle over $M$ is holomorphic if its transition maps are holomorphic maps $\mathcal{U}_\alpha \cap  \mathcal{U}_\beta \to \GL_k(\C)$.
\end{definition}

An $\mathbb{F}$-line bundle is a rank $1$ $\mathbb{F}$-vector bundle. A verctor bundle morphism between $E\to M$ and $E'\to M'$ is a smooth map $E\xrightarrow{\Phi}E'$ s.t. 
\[
\begin{tikzcd}
  E\arrow[r, "\Phi"]\arrow[d] &  E'\arrow[d] \\
  M\arrow[r,"F"]  & M'
\end{tikzcd}
\] 
commute for some smooth map $F:M\to M'$ and the restrictions $\Phi|_{\pi^{-1}(\zeta)}:E_{\zeta }\to E'_{F(\zeta)}$ are linear. A vector bundle isomorphism is a bijective vector bundle morphism. 

A section of a vector bundle $E\xrightarrow{\pi}M$ is a map $M\xrightarrow{s}E$ such that $\pi\left( s(\zeta) \right) =\zeta \forall \zeta \in M$, i.e.,
\[
  \begin{tikzcd}[column sep=tiny]
  & E\arrow[rd,"\pi"]  & \\
  M \arrow[ur,"s"]\arrow[rr,"\mathrm{Id}"]& & M  
\end{tikzcd}
\]
We denote the set of section by $C^{\infty}(M;E)$ ($C^{\infty}(M;E)\subsetneq C^{\infty}(M,E)$).

\begin{example}[Tangent bundle, first definition]
  Let $M$ be an $\mathbb{F}$-manifold of dimension $n$ and let $\left( \mathcal{U}_\alpha,\phi_\alpha:\mathcal{U}_\alpha\to \mathcal{V}_\alpha \right) $ be an atlas, so $\mathcal{U}\subset M$, $\mathcal{V}_{\alpha}\subset \mathbb{F}^{n}$ and $\mathcal{U}_\alpha$ forms a cover of $M$.

  The tansition map on overlaps 
  \[
    \phi_{\alpha}\circ \phi_{\beta }^{-1}:\phi_{\beta }\left( \mathcal{U}_\alpha \cap \mathcal{U}_\beta  \right) \to \phi_\alpha\left( \mathcal{U}_\alpha \cap  \mathcal{U}_\beta  \right) 
  \] 
  are diffeomorphisms (biholomorphisms). If we take derivatives we get
  \[
    D\left( \phi_\alpha\circ \phi_{\beta }^{-1} \right) :\phi_\beta\left( \mathcal{U}\cap \mathcal{U}_\beta  \right) \times \mathbb{F}^{n}\to \phi_\alpha \left( \mathcal{U}_\alpha \cap  \mathcal{U}_\beta  \right) \times \mathbb{F}^{n}.
  \] 
Define 
\[
  TM= \sfrac{\bigsqcup \mathcal{U}_\alpha \times \mathbb{F}^{n}}{(u,v)\sim (u',v') \begin{matrix}\text{ if } u,u' \in \mathcal{U}_\alpha \cap \mathcal{U}_{\beta }& \\ \text{ and } D\left( \phi_{\alpha}\circ \phi_{\beta}^{-1} \right)\left( \phi_\beta (u),v \right) =\left( \phi_\alpha (u'),v' \right)&\end{matrix}  } .
\]
That is, we take tangent vectors in each coordinate charts and identify them if they correspond to each other under the chain rule.
\end{example}

Meta-Theorem:  Any canonical construction in linear algebra gives rise to a geometric version for smooth / holomorphic vector bundles.
\begin{example}
  Let $E$ and $F$ be vector bundles over a manifold $M$, then
  \begin{enumerate}
    \item $E\oplus F\to M$ is the vector bundle whose fiber $(E\oplus F)_{\zeta }$ is canonical isomorphic to $E_\zeta \oplus E_\zeta $.
    \item $E\otimes F\to M$ is the vector bundle whose fiber $(E\otimes F)_{\zeta}$ is canonical isomorphic to $E_{\zeta }\otimes E_{\zeta}$.
    \item Taking alternating /symmetric parts of $\mathop{\otimes}\limits_{j=1}^{m} E\to M$ produces $\wedge ^{m}E\to M$( $k$-th exterior power) or  $S^{k}E\to M$ ($k$-th symmetric power).
    \item  $\mathrm{Hom}(E,F)\to M$ with fiber cannonically isomorphic to $\mathrm{Hom} (E_\zeta,F_\zeta)$.
    \item $E^* \to M$ is the vector bundle $\mathrm{Hom}(E,\underline{\mathbb{F}})$.
  \end{enumerate}
  For example, if $E$ has transition maps $g_{\alpha \beta }$ and $F$ has transition maps $h_{\alpha \beta}$, then $E\oplus F$ is the vector bundle with transition maps $\begin{bmatrix} g_{\alpha\beta }&0\\0&h_{\alpha\beta } \end{bmatrix} $ and $E^{*}$ is the vector bundle with transition maps $f_{\alpha\beta }=\left( g_{\beta \alpha}^{-1} \right)^t $.
\end{example}

\section{Geometric structures}
Another way of thinking about manifolds, real or complex, in line with modern algebraic geometry is through ``geometric structures''.
\begin{definition}
  Let $\mathbb{F}\in \left\{\R,\C\right\} $, $X$ a topological space. For any open set $\mathcal{U}\subset X$ open, let $C(\mathcal{U})=C^{0}(\mathcal{U})$ denote the continuous functions $\mathcal{U}\to \mathbb{F}$. A geometric structure $\mathcal{A}$ on $X$ is a collection of subrings $\mathcal{A}(\mathcal{U})\subset C(\mathcal{U})$,$\forall \mathcal{U}\subset X$ open, satisfying:
  \begin{enumerate}
    \item The constant functions are in $\mathcal{A}(\mathcal{U})$;
    \item If $f \in \mathcal{A}\left( \mathcal{U} \right) $ and $\mathcal{V}\subset \mathcal{U}$, then $f|_{\mathcal{V}}\in \mathcal{A}\left( \mathcal{V} \right) $ ;
    \item If $\left\{\mathcal{U}_i\right\} $ is a collection of open subsets of $X$, $\mathcal{U}=\bigcup_{i} \mathcal{U}_i$ and we are given $f_i \in \mathcal{A}\left( \mathcal{U}_i \right) $ s.t. $f_i|_{\mathcal{U}_j}=f_j|_{\mathcal{U}_i}$ whenever $\mathcal{U}_i\cap \mathcal{U}_j\neq \O$,then $\exists ! f \in \mathcal{A}\left( \mathcal{U} \right) $ s.t. $f|_{\mathcal{U}_i}=f_i\forall i$.
  \end{enumerate}
  The pair $(X,\mathcal{A})$ is called a geometric space and functions in $\mathcal{A}\left( \mathcal{U} \right) $ will sometimes be called distinguished. (b) and (c) tells us that being distinguished is an open property. Typical examples are differentiability and analyticity.
\end{definition}
In the language of sheaves (which we will discuss later): $\mathcal{A}$ is a subsheaf of the sheaf of continuous functions.
\begin{example}
  \begin{enumerate}
    \item If $\mathcal{U}\subset \R^{n}$ open and let  $C^{\infty}$ be the geometric structure $\mathcal{V}\mapsto C^{\infty}\left( \mathcal{V} \right) $ then $\left( \mathcal{U},C^{\infty} \right) $ is a geometric space.
    \item If $\mathcal{U}\subset \C^{n}$ open and let $\theta $ be the geometric structure $\mathcal{V}\mapsto\theta (\mathcal{V})=$ holomorphic functions from $\mathcal{V}$ to $\C$, then $(\mathcal{V},\theta )$ is a geometric space.
  \end{enumerate}
\end{example}
\begin{definition}  
A morphism of geometric spaces $f:(X,\mathcal{A}_X)\to \left( Y,\mathcal{A}_Y \right) $is a continuous map $f:X\to Y$ with the property that if $g\in \mathcal{A}_Y(\mathcal{U})$ then $g\circ f\in \mathcal{A}_X\left( f^{-1}\left( \mathcal{U} \right)  \right) $(we write this map as $f^{*}:\mathcal{A}_Y(\mathcal{U})\to \mathcal{A}_X\left( f^{-1}(\mathcal{U}) \right) $). $f$ is an isomorphism if there is an inverse morphism.
\end{definition}

\begin{example}
\begin{enumerate}
  \item Morphism $f:\left( \mathcal{U},C_{\mathcal{U}}^{\infty} \right) \to \left( \mathcal{V},C_{\mathcal{V}}^{\infty} \right) $ is the same as $f \in C^{\infty}\left( \mathcal{U},\mathcal{V} \right) $.
  \item Morphism $f:\left( \mathcal{U},\theta _{\mathcal{U}} \right) \to \left( \mathcal{V},\theta _{\mathcal{V}} \right) $ is the same as $f:\mathcal{U}\to \mathcal{V}$ holomorphic.
  \item If $\mathcal{U}\subset X$ open, then $\left( \mathcal{U}, \mathcal{A}|_{\mathcal{U}} \right)\to \left( X,\mathcal{A} \right)  $ is a morphism.
  \item If  $X$ is a real (complex) manifold  with atlas $\left\{\varphi_{\alpha}:\mathcal{U}_{\alpha}\to \mathcal{V}_{\alpha}\right\},\mathcal{U}_\alpha \subset X,\mathcal{V}_{\alpha}\subset \mathbb{F}^{n}$. We define a geometric structure $C_{X}^{\alpha}$ ($\theta _{X}$ resp.) as follows: For $\mathcal{U}\subset X$ open, define
    \begin{align*}
      C_X^{\infty}\left( \mathcal{U} \right) &:=\left\{f\in C(\mathcal{U}):(f|_{\mathcal{U}\cap \mathcal{U}_\alpha})\circ \varphi_\alpha ^{-1}:\varphi_\alpha\left( \mathcal{U}\cap \mathcal{U}_\alpha  \right) \to \R \text{ smooth }\forall \mathcal{U}\cap \mathcal{U}_\alpha\neq \O\right\}\\
      \theta _X\left( \mathcal{U} \right) &:=\left\{\cdots \C \text{ holomorphic }\cdots \right\}
    .\end{align*}
\end{enumerate}  
\end{example}
\begin{definition}[Equivalent definition]
  A smooth (complex) manifold of $\mathbb{F}$-dimension $n$ is a geometric structure $\left( X,\mathcal{A} \right) $ in which every point $\zeta\in X$ has a neighborhood $\mathcal{U}$ s.t. $\left( \mathcal{U},\mathcal{A}|_{\mathcal{U}} \right) \cong \left( \Omega,C^{\infty}_{\Omega} \right) $, $\Omega\subset \R^{n}$ open (or s.t.  $\left( \mathcal{U},\mathcal{A}|_{\mathcal{U}} \right) \cong \left( \Omega, \theta _{\mathcal{U}} \right) $, $\Omega\subset \C^{n}$ open).
\end{definition}
Given a geometric space $\left( X,\mathcal{A} \right) $ and a point $\zeta \in X$ we localize $\mathcal{A}$ to $\zeta $ :
\begin{enumerate}
  \item We define $\mathcal{A}_\zeta $ as a ring of equivalence classes of functions. Each $[f]=\left[ f \right] _\zeta \in \mathcal{A}_\zeta $ is represented by $f\in \mathcal{A}(\mathcal{U})$, $\zeta\in \mathcal{U}$ open and two representatives $f_1 \in \mathcal{A}\left( \mathcal{U}_1 \right) $, $f_2 \in  \mathcal{A}(\mathcal{U}_2)$ are equivalent if there is an open $\mathcal{W}$, $\zeta \in \mathcal{W}$ s.t. $f_1|_{\mathcal{W}}=f_2|_{\mathcal{W}}$. 
  \item We call $\mathcal{A}_\zeta $ the stalk of $\mathcal{A}$ at $\zeta $ and $\left[ f \right] _{\zeta }$ the germ of $f$ at $\zeta $.
\end{enumerate}
A derivation of the algebra $\mathcal{A}_{\zeta }$ is an $\mathbb{F}$-linear map 
$D:\mathcal{A}_\zeta \to \mathbb{F}$ that satisfies the Leibnize rule at $\zeta$:
\[
  D\left( [f]\left[ g \right]  \right) =f(\zeta)D\left( \left[ g \right]  \right) +D\left( \left[ f \right]  \right) g(\zeta) \quad \forall  \left[ f \right],\left[ g \right] \in \mathcal{A}_{\zeta}. 
\]
The real tangent sapce of $X$ at $\zeta $ is the $\R$-vector spaces of derivations of $C_{X,\zeta }^{\infty}$ and the complex tangent space of $X$ at $\zeta $ is the $\C$-vector spaces of derivations of $\theta _{X,\zeta }$.

\begin{example}
  \begin{enumerate}
    \item For $\mathcal{U}\subset \R^{n}$ open, $D_j:\C_{\zeta }^{n}\to \mathbb{F},D_j\left( \left[ f \right]  \right) =\frac{\partial f}{\partial x_j} (\zeta)$ is a derivation.
    \item For $\mathcal{U}\subset \C^{n}$ open, $D_j:\theta _\zeta \to \mathbb{F}, D_j\left( \left[ f \right]  \right) =\frac{\partial f}{\partial z_j} (\zeta)$ is a derivation.
  \end{enumerate}
\end{example}
Notice that for a manifold, the stalk of $\mathcal{A}$ is the same as the stalk of $\mathcal{A}|_{\mathcal{U}}$, $\mathcal{U}$ coordinate chart containing $\zeta$.

Let $\mathcal{J}_{\zeta }\subset \mathcal{A}$ be the ideal of germs that vanish at $\zeta$ and then define $\mathcal{J}_{\zeta}^2$ an ideal generated by $\left\{[f]\in \mathcal{A}_\zeta:\left[ f \right] =\left[ g \right] \left[ h \right] \text{ for some }\left[ g \right] ,\left[ h \right] \in \mathcal{J}_{\zeta }\right\} $.

The map
\begin{align*}
   \mathcal{A}_\zeta  &\longrightarrow \sfrac{\mathcal{A}_{\zeta }}{\mathcal{J}_{\zeta }} \\
   \left[ f \right]  &\longmapsto f(\zeta)
\end{align*}is evaluation at $\zeta$.
The map 
$\mathcal{J}_{\zeta } \longrightarrow  \sfrac{\mathcal{J}_{\zeta }}{\mathcal{J}_{\zeta }^2}$, for a manifold, is the ``total derivative'' (exterior derivative).
Given a function $f\in \mathcal{A}(\mathcal{U})$, $\mathcal{U}\subset \mathbb{F}^{n}$, we can write 
\[
  f(w)=f(\zeta)+\left( \partial_{w_1}f \right) (\zeta)(w_1-\zeta_1)+\cdots \left( \partial_{w_n }f \right) (\zeta)(w_n-\zeta_n)+\text{ things that vanish to 2nd order at }\zeta.
\] 
If $\left[ f \right] \in \mathcal{J}_{\zeta }$ then $f(\zeta)=0$ while the remainder is in $\mathcal{J}_\zeta^2$ so the class of $\left[ f \right] $ in $\sfrac{\mathcal{J}_\zeta }{\mathcal{J}_{\zeta }^2}$ is the class of $\left[ \left( \partial_{w_1 }f \right) (\zeta)(w_1-\zeta_1)+\cdots +\left( \partial_{w_n }f \right)\left( w_n-\zeta_n \right)   \right]=: \sum_{j=1}^{n} \left( \partial_{w_j}f \right) (\zeta) \,\mathrm{d} w_j$. Thus $\sfrac{\mathcal{J}_{\zeta }}{\mathcal{J}_{\zeta }^2}\cong \mathbb{F}^{n}$ with each choice of coordinates inducing a basis o f $\sfrac{\mathcal{J}_{\zeta }}{\mathcal{J}_{\zeta }^2}$.

\begin{claim} $\mathrm{Der}\left( \mathcal{A}_\zeta  \right) \cong \left( \sfrac{\mathcal{J}_{\zeta }}{\mathcal{J}^2_{\zeta }} \right) ^{*}= \text{ annihilator of }$ $\mathcal{J}_{\zeta }^2$ in $\mathcal{J}_\zeta ^{*}$.
\end{claim}
\begin{proof}
  There is an obvious map $\mathrm{Der}\left( \mathcal{A}_\zeta  \right) \to \mathcal{J}^{*}_{\zeta }$ by restriction, by the Leibnitz rule elements in the image vanish on $\mathcal{J}^2_{\zeta }$. Conversely, given $\phi \in J_{\zeta }^{*}$ s.t. $\phi|_{\mathcal{J}^2_{\zeta }}\equiv 0$, define $D:\mathcal{A}_{\zeta }\to \mathbb{F}$ by $D\left( \left[ f \right]  \right) =\phi \left( \left[ f \right] -\left[ f(\zeta) \right]  \right) $. Note 
  \begin{align*}
    D\left( \left[ f \right] \left[ g \right]  \right) &=\phi\left( \left[ fg \right] -\left[ fg(\zeta) \right]  \right) \\
						       &= \phi \left( \left[ f-f(\zeta) \right] \left[ g-g(\zeta) \right] +\left[ f(\zeta)(g-g(\zeta)) \right] +\left[ g(\zeta)\left( f-f(\zeta) \right)  \right]  \right) \\
						       &= f(\zeta) \phi\left( \left[ g-g(\zeta) \right]  \right) +g(\zeta)\phi\left( \left[ f-f(\zeta) \right]  \right)\\
						       &= f(\zeta)D\left( \left[ g \right]  \right) +g(\zeta) D\left( \left[ f \right]  \right)
  .\end{align*}
\end{proof}

\section{Paths and Tangents}

Last section we defined $t_pM$ at $p \in M$ to be the derivations of $\mathcal{A}_p$. These fit together to form $tM=\bigcup t_pM $. Over any coordinate chart $\varphi:\mathcal{U}\to \mathcal{V}$ we can form $t_p\mathcal{U}=t_pM$ for $p \in \mathcal{U}$, and then use what we established last time to see that there is an induced map $D\varphi:t\mathcal{U}\to t\mathcal{V}=\mathcal{V}\times \R^{n}$ and see that it coincides with the usual $D\varphi$. Hence we see that $tM$ is naturally identified with $TM$. 

An alternative way of thinking about a tangent vector to $M$ at $p \in M$ is as an equivalence class of paths through $p$. We can do this both in the settings of the tangent bundle of a smooth manifold or the holomorphic tangent bundle of a complex manifold.

For $\mathbb{F}\in \left\{\R,\C\right\} $ apath through $p \in M$ will refer to a pair $\left( \mathcal{U},\gamma \right) $ with $\mathcal{U}\subset \mathbb{F}$ a neighborhood of the origin and $\gamma:\mathcal{U}\to \mathcal{M}$ smooth if $\mathbb{F}=\R$ (holomorphic if $\mathbb{F}=\C$) s.t. $\gamma(0)=p$.

Two paths through $p$ are equivalent if there is a coordinate chart $\varphi:\mathcal{W}\to \mathcal{V}$ with $p \in W$ such that 
\[
\partial_t|_{t=0}\varphi\circ \gamma_1=\partial_t|_{t=0}\varphi circ \gamma_2,\quad  \mathbb{F}=\R
\] 
or 
\[
\partial_t|_{z=0}\varphi\circ \gamma_1=\partial_t|_{z=0}\varphi\circ \gamma_2,\quad \mathbb{F}=\C.
\] 
It's easy to see that these tangent vectors can be identified with the vector space of derivatives in the coordinate directions in any local coordinates at $p$. This allows us to identify this notion of tangent vector and the corresponding vector bundle with the tangent bundle.

Given a smooth (holomorphic) map between smooth (complex) manifolds $F:M\to N$, its derivative is a smooth (holomorphic) map $DF:TM\to TN$:
 \begin{align*}
   D_p F: T_p M &\longrightarrow T_{F(p)}N \\
  \left[ \gamma \right]  &\longmapsto \left[ F\circ \gamma \right] 
.\end{align*}
The chain rule says that the tangent space is ``functorial'':
$
\begin{tikzcd}[column sep= tiny]
  L\arrow[rr,"F"] \arrow[rd,"G\circ F"]  & & M \arrow[ld,"G"] \\
  & N &
\end{tikzcd}
$ 
imples $
\begin{tikzcd}[column sep= tiny]
  TL\arrow[rr,"DF"] \arrow[rd,"DG\circ DF"'] & & TM \arrow[ld,"DG"] \\
			  &TN&
\end{tikzcd}$, that is, $D\left( G\circ F \right) =DG\circ DF$.

A section of the tangent bundle is a map $M\to TM$ that assigns to each point a tangent vector at that point. Sctions of the tangent bundle are called vector fields.
\begin{example}
  $\mathcal{U}\subset M$ open, given $\mathbf{V}\in C^{\infty}\left( \mathcal{U};TM|_{\mathcal{U}} \right) $ we can think of $\mathbf{V}$ as assigning to each $p\in \mathcal{U}$ a derivation of $C^{\infty}_{M,p}$ or $\theta _{M,p}$ and we can put these together to define the Lie derivative of the vector field:
  \begin{align*}
    \mathcal{L}_{\mathbf{V}}: \mathcal{A}(\mathcal{U}) &\longrightarrow \mathcal{A}(\mathcal{U}) \\
    f &\longmapsto \mathcal{L}_{\mathbf{V}}(f)(\zeta )=\mathbf{V}(\zeta )\left[ f \right] _{\zeta }
  .\end{align*}
  This defines a derivation of $\mathcal{A}(\mathcal{U})$ with Leibnitz rule 
  \[
    \mathcal{L}_{\mathbf{V}}\left( fg \right) =f\mathcal{L}_{\mathbf{V}}(g)+g\mathcal{L}_{\mathbf{V}}(f)
  \] 
  and every derivation of $\mathcal{A}(\mathcal{U})$ arises in this way. Often $\mathcal{L}_{\mathbf{V}}(f)$ is denoted $\mathbf{V}(f)$. 
  
  We can use this to define the Lie bracket of vector fields. Given $\mathbf{V}_1,\mathbf{V}_2 \in C^{\infty}\left( \mathcal{U};TM|_{\mathcal{U}} \right) $, let 
  \begin{align*}
    \left[ \mathbf{V}_1,\mathbf{V}_2 \right] : \mathcal{A}\left( \mathcal{U} \right)  &\longrightarrow \mathcal{A}\left( \mathcal{U} \right)  \\
    f &\longmapsto \left[ \mathbf{V}_1,\mathbf{V}_2 \right] (f)=\mathcal{L}_{\mathbf{V}_1}\left( \mathcal{L}_{\mathbf{V}_2}(f) \right) -\mathcal{L}_{\mathbf{V}_2}\left( \mathcal{L}_{\mathbf{V}_1}(f) \right) 
  .\end{align*}
  This is a derivation:
  \begin{align*}
    \left[ \mathbf{V}_1,\mathbf{V}_{2} \right] (fg)&= \mathcal{L}_{\mathbf{V}_1}\left( \mathcal{L}_{\mathbf{V}_2}(fg) \right) -\mathcal{L}_{\mathbf{V}_2}\left( \mathcal{L}_{\mathbf{V}_1}(fg) \right) \\
    &= \mathcal{L}_{\mathbf{V}_1}\left( f\mathcal{L}_{\mathbf{V}_2}(g)+g\mathcal{L}_{\mathbf{V}_2}(f) \right) -\mathcal{L}_{\mathbf{V}_2}\left( f\mathcal{L}_{\mathbf{V}_1}(g)+g\mathcal{L}_{\mathbf{V}_1}(f) \right) \\
    &=\mathcal{L}_{\mathbf{V}_1}(f)\mathcal{L}_{\mathbf{V}_2}(g)+f\mathcal{L}_{\mathbf{V}_1}\mathcal{L}_{\mathbf{V}_2}(g)+\mathcal{L}_{\mathbf{V}_1}(g)\mathcal{L}_{\mathbf{V}_2}(f)+g\mathcal{L}_{\mathbf{V}_1}\mathcal{L}_{\mathbf{V}_2}(f)\\
    &-\mathcal{L}_{\mathbf{V}_2}(f)\mathcal{L}_{\mathbf{V}_1}(g)-f\mathcal{L}_{\mathbf{V}_2}\mathcal{L}_{\mathbf{V}_1}(g)-\mathcal{L}_{\mathbf{V}_2}(g)\mathcal{L}_{\mathbf{V}_1}(f)-g\mathcal{L}_{\mathbf{V}_2}\mathcal{L}_{\mathbf{V}_1}(f)\\
    &=f\left[ \mathbf{V}_1,\mathbf{V}_2 \right](g) +g\left[ \mathbf{V}_1,\mathbf{V}_2 \right](f) 
  .\end{align*}
  Hence it is the Lie derivative of a vector field which we denote $\left[ \mathbf{V}_1,\mathbf{V}_2 \right] $. So this map could better be denoted $\mathcal{L}_{\left[ \mathbf{V}_1,\mathbf{V}_2 \right] }$. In local coordinates, if $\mathbf{V}_1=\sum_{}^{} a_j\partial_{\zeta _j}$, $\mathbf{V}_2=\sum b_j \partial_{\zeta _j}$, then 
  \[
    \left[ \mathbf{V}_1,\mathbf{V}_2 \right] =\sum_{j}^{k} \left( a_k\partial_{\zeta _k}b_j-b_k\partial_{\zeta _k}a_j \right) \partial_{\zeta _j}.
  \] 
\end{example}

\begin{definition}
  Given a smooth (holomorphic) map $F:M\to N$ between smooth (complex) manifolds and a smooth (holomorphic) vector bundles $E\overset{\pi}{\to}N$, we get a smooth (holomorphic) vector bundle $F^{*}E\to M$ known as the \textit{pull-back} vector bundle together with a vector bundle morphism $F^{*}E\to M$, i.e., we have a commutative diagram 
  \[
  \begin{tikzcd}
    F^{*}E\arrow[r] \arrow[d] &E\arrow[d,"\pi"]\\
    M \arrow[r,"F"]& N
  \end{tikzcd}.
\]
In fact, $F^{*}E=\left\{(m,\mathbf{v})\in M\times E:F(m)=\pi(\mathbf{v})\right\} $ and the maps are $
  \begin{tikzcd}
    (m,\mathbf{v})\arrow[d,mapsto] \arrow[r,mapsto]&\mathbf{v}\\
    m &
  \end{tikzcd}$.
\end{definition}

This satisfies a ``universal property'': It is the smallest vector bundle over $M$ with a vector bundle map to $E$ covering $F$ in that, given
 \[
\begin{tikzcd}
  \mathscr{E}\arrow[r] \arrow[d] &E\arrow[d,"\pi"] \\
  M\arrow[r,"F"] &N
\end{tikzcd}
\] 
there exists a unique vector bundle morphism $\mathscr{E}\to F^{*}E$ such that 
\[
\begin{tikzcd}
  \mathscr{E}\arrow[rdd, bend right]\arrow[rd, dashed]  \arrow[rrd, bend left] & &\\
   & F^{*}E \arrow[r]\arrow[d]  &E\arrow[d,"\pi"] \\
   & M\arrow[r,"F"]&N 
\end{tikzcd}
\] 
commutes.

Notice that the pull-back of a trivial bundle $N\times \mathbb{F}^{k}$ is the trivial bundle $M\times \mathbb{F}^{k}$. We obtain local trivializations of $F^{*}E$ by pulling-back local trivializations of $E$.

\section{Covariant derivatives}

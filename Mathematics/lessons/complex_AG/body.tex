\section{Introduction}
Let $f_i(x_1,\cdots ,x_n),i\in \left\{1,\cdots ,k\right\} $ be polynomials with coefficient in $\R$ or $\C$. An \textit{affine algebraic variety} is the common zero of $X=X(f_1,\cdots ,f_k)=\left\{x:f_i(x)=0\forall i\right\} $. We incorporate the field into the notation 
\[
  X\left( \C \right) =\left\{x\in \C^{n}:f_i(x)=0\forall i\right\} ,
\] 
\[
  X(\R)=\left\{x\in \R^{n}:f_i(x)=0 \forall i\right\} \left( f_i\text{have real coefficients} \right). 
\] 
These can be thought of naturally as topological spaces with the topology they inherit from $\C^{n}$ or $\R^{n}$(Alternately, you can use the ``Zariski topology'' induced by declaring that zero sets of polynomials are closed). 

It is frequently in convenient that $X\left( \C \right) $ is essentially never compact. To remedy this, we shift our attention to projective space:
\begin{align*}
  \CP^{n}= &\frac{\C^{n+1}\backslash \left\{0\right\} }{\text{dilations}}\\
  =&\sfrac{\left\{\left( z_0,z_1,\cdots ,z_{n} \right) \in \C^{n+1}\backslash \left\{0\right\} \right\} }{(z_0,\cdots ,z_n)\sim (\lambda z_0,\cdots ,\lambda z_n), \forall \lambda \neq 0}\\
  = & \sfrac{\Sph^{2n+1}}{u(1)},  
\end{align*}
hence compact.

Given homogeneous polynomials $F_i(x_0,\cdots ,x_n)$ we obtain a ``projective variety'' $X=X\left( F_1,\cdots ,F_k \right) =\left\{x\in \CP^{n}:F_i(x)=0\forall i\right\} $. As before, if the polynomial have real coefficients, we have $X(\R)=\left\{x\in \mathbb{RP}^{n}:F_i(x)=0\forall i\right\} $. 

We can ask about the relation between the topology and geometry of $X\left( \R \right) ,X(\C)$ and the algebraic properties of $X$. For example, say $X$ is the zero set of a single homogeneous polynomial of degree $d$ called $F$, can we recover $d$ from looking at $X(\C),X(\R)$?(Only a sensible question for $F$ irreducible.) It turns out that $X_F(\C)$ determines a homology class $\left[ X_F(\C) \right] \in H_{2n-2}\left( \CP^{n};\Z \right) $, and this group is cyclic with generator $\left[ H \right] $ induced by $\CP^{n-1}\hookrightarrow \CP^{n}$, and $\left[ X_F(\C) \right] =d\cdot \left[ H \right] $.

We can recover $d$ from the intrinsic geometry of $X_F\left( \C \right) $ using the ``Chern classes'' in $H^{*}\left( ? \right) $. Over the real numbers, $H_{n-1}\left( \mathrm{RP}^{n;\Z_2} \right) $ is cyclic with generator $\left[ H \right] $ and $\left[ X_F\left( \R \right)  \right] =d\cdot \left[ H \right] $, so we recover $d \mod 2$. 
It is possible to show that $X_F(\R)$ does not provide an upper bound for $d$.

From a different point of view, the Nash embedding theorem shows that any smooth manifold over $\R$ is diffeomorphic to $X(R)$ for some real smooth projective variety.
For complex manifolds the analogue statement is false. To be diffeomorphic to a complex projective variety, a manifold must be complex, K\"{a}hler, Hodge, and then it will have an embedding into some $\CP^{n}$ and Chow's theorem guarantees that it's algebraic.

\section{Holomorphic functions}
This section we want to talk about holomorphic functions.

Recall some concepts / theorems from one complex variable. Let   $\mathcal{U}\subset \C, f:\mathcal{U}\to \C, \mathcal{U}$ open,
\begin{enumerate}
  \item $f$ is holomophic $\Leftrightarrow$ f is analytic: $\forall  z_0 \in  u, \exists \varepsilon >0$ s.t. $f(z)=\sum_{n=0}^{\infty} a_n (z-z_0)^n,\forall z\in B_{\varepsilon }(z_0)$.
  \item  $f$ is holomorphic $\Leftrightarrow$ $f$ satisfies Cauchy integral formula: $f\in \C^{1}$ and $\forall z_0 \in  \mathcal{U}$, $\varepsilon $ small enough 
    \[
      f(z_0)=\frac{1}{2\pi i}\int_{\partial B_\varepsilon (z_0)}\frac{ f(z)}{z-z_0}\,\mathrm{d}z.
    \] 
  \item $f$ is holomorphic $\Leftrightarrow$ $f$ satisfies Cauchy-Riemann equations. Writing $z=x+iy,x,y\in \R$, $f(x,y)=u(x,y)+iv(x,y), u,v$ real values, $u,v$ are continuous differentiable and satisfy
    \[
    \frac{\partial u}{\partial x} =\frac{\partial v}{\partial y},  \frac{\partial u}{\partial y} =-\frac{\partial v}{\partial x} . 
    \] 
\end{enumerate}

Let's introduce the differential operators:
\[
  \partial_z=\frac{\partial ~}{\partial z} =\frac{1}{2}=\left( \partial_x-i\partial_y \right) ,\partial_{\overline{z}}=\frac{\partial ~}{\partial \overline{z}} =\frac{1}{2}\left( \partial_x+i\partial_y \right) .
\] 	
Then $\partial_z z=1=\partial_{\overline{z}}\overline{z},\partial_z\overline{z}=0=\partial_{\overline{z}}z$. In these terms Cauchy-Riemann equations are $\partial_{\overline{z}}f=0$.

Geometrically, $f:\mathcal{U}\subset \C=\R^{2}\to \C=\R^{2}, f\left( \begin{bmatrix} x\\y \end{bmatrix}  \right)=\begin{bmatrix} u\\v \end{bmatrix}  $ induces $D_{z_0}f:T_{z_0}\R^2\to T_{f(z_0)}\R^2$ with respect to the standard basis, this is the real Jacobian of $f$ :
\[
  J_{\R}(f)=\begin{bmatrix} \frac{\partial u}{\partial x} &\frac{\partial u}{\partial y} \\ \frac{\partial v}{\partial x} &\frac{\partial v}{\partial y}  \end{bmatrix} .
\] 
\begin{example}
  \[
    \left( D_{(x_0,y_0)}f \right) \left( \partial_x \right) =\partial_t |_{t=0}f(x_0+t,y_0)=\partial_t|_{t=0}\begin{bmatrix} u(x_0+t,y)\\v(x_0+t,y) \end{bmatrix} =\begin{bmatrix} \frac{\partial u}{\partial x} &\frac{\partial v}{\partial x}  \end{bmatrix} 
  .\] 
\end{example}
After we complexify, $D_{z_0}^{\C}f:T_{z_0}\R^{2}\otimes \C\to T_{f(z)}\R^{2}\otimes \C$. We can write this matrix in the basis $\partial_z,\partial_{\overline{z}}$:
\[
  \begin{bmatrix} \partial_z(u+iv) &\partial_{\overline{z}}(u+iv)\\ 
  \partial_z(u-iv) & \partial_{\overline{z}}(u-iv)\end{bmatrix}
  = \begin{bmatrix} 
  \partial_z f & \partial_{\overline{z}}f\\
\partial_z \overline{f}&\partial_{\overline{z}}\overline{f}\end{bmatrix} 
.\] 
(note: $\partial_{\overline{z}}\overline{f}=\overline{\partial_z f}, \overline{\partial_z \overline{f}}=\partial_{\overline{z}}f$, etc.)
The funciton $f$ is holomorphic if and only if this matrix is diagnal. The complex Jocabian for $f$ holomorphic is $J_{\C}f=\begin{bmatrix} \partial_z f & 0\\ 0 & \partial_{\overline{z}}\overline{f} \end{bmatrix} $.

Holomorphic functions of one variable satisfy some important theorems:
\begin{enumerate}
  \item Maximum principle: $\mathcal{U}\subset \C$ open, connected, $f:\mathcal{U}\to \C$ is holomorphic, nonconstant, then $|f|$ has no local maximum in $\mathcal{U}$. If $\mathcal{U}$ is bounded and $f$ extends to a continuous $\overline{\mathcal{U}}\to \C$, then $\max |f|$ occurs on $\partial \overline{\mathcal{U}}$.
  \item Identity theorem: $f,g:\mathcal{U}\to \C$ holomoprhic and $\mathcal{U}$ connected. If $$\left\{z\in \mathcal{U}:f(z)=g(z)\right\} $$ contains an open set then it is all of $\mathcal{U}$.
  \item Extension theorem: $f:B_\varepsilon (z_0) \backslash  \left\{z_0\right\} \to \C$ is holomorphic and bounded, then it extends to a holomorphic function $B_{\varepsilon }(z_0)\to \C$.
  \item Liouville's theorem: $f:\C\to \C$ holomorphic and bounded implies $f$ is constant.
  \item Riemann mapping theorem: If $\mathcal{U}\subset \C$is a simply connected, proper open set, then $\mathcal{U}$ is biholomorphic to the unit ball $B_1(0)$.
  \item Residue theorem: If $f:B_\varepsilon (0) \backslash  \left\{0\right\} \to \C$ is holomorphic and $f(z)=\sum_{j\in \Z}^{} a_jz^{j}$ is its Laurent series. Then
    \[
      a_{-1}=\frac{1}{2\pi i}\int_{\partial B_{\varepsilon  /2}(0)}f(z)\,\mathrm{d}z.
    \] 
\end{enumerate}


\begin{definition}
  $U\subset \C^{n}$ open, $f:U\to \C$ continuously differentiable,  then $f$ is holomorphic at $a\in U$ if for all $j\in \left\{1,\cdots ,n\right\} $ the function of one variable 
  \[
    z_j\mapsto f(a_1,\cdots ,a_{j-1}, z_j, a_{j+1},\cdots ,a_n)
  \] is holomorphic. i.e., $\partial_{\overline{z_j}}f=0,\forall j \in \left\{1,\cdots ,n\right\} $.
\end{definition}
If we write 
\[
\mathrm{d}f=\sum \frac{\partial f}{\partial z_j} \mathrm{d}z_j+\sum \frac{\partial f}{\partial \overline{z}_j} \mathrm{d}\overline{z}_j.
\] 
Denote the first term $\partial f$ and the second $\overline{\partial}f$, then $f$ is holomorphic iff $\overline{\partial}f=0$.

\begin{definition}
  For $a\in \C^{n}, R\in \left( \R^{+} \right) ^{n}$, the polydisc around $a$ with multindices $R$ is the set $D(a,R)=\left\{z\in \C^{n}:|z_j-a_j|<R_j,\\forall j\in \left\{1,\cdots ,n\right\} \right\} $. If $R=(1,\cdots ,1)$ and $a=0$ we abbreviate $D(0,1)$ by $\mathbb{D}$ and refer to it as the unit disc in $\C^{n}$. 
\end{definition}
Repeatedly applying the Cauchy formula in $1$-variable, we obtain the following theorem:
\begin{theorem}
  $f:D(w,\varepsilon )\to \C$ hilomorphic and $z\in D(w,\varepsilon )$ then 
  \[
    f(z)=\frac{1}{\left( 2\pi i \right) ^{n}}\int_{\partial D(w,\varepsilon )}\frac{f(\xi_1,\cdots ,\xi_n)}{(\xi-z_1)\cdots (\xi_n-z_n)}\,\mathrm{d}\xi_1\cdots \mathrm{d}\xi_n.
  \] 
\end{theorem}
Using this we can show that for any point $w\in U$, $\exists $ multi disc $D(w,\varepsilon )$ inside $U$ such that for all $z\in D(w,\varepsilon )$, $f(z)=\sum_{|\alpha|=0}^{\infty} \frac{\partial^{\alpha}_z f}{\alpha!}(z-w)^{\alpha}$. Here $\alpha$ is a multi-index $\alpha \in \left( \N_0 \right) ^{n}$, and 
\begin{align*}
  (z-w)^{\alpha}&= (z_1-w_1)^{\alpha_1}\cdots (z_n-w_n)^{\alpha_n} \\
  \alpha! &= \alpha_1!\cdots \alpha_n! \\
  \partial^{\alpha}_{z}f=\partial_{z_1}^{\alpha_1}\cdots \partial_{z_n}^{\alpha_n}f
.\end{align*}
From the list above, the maximum principle, the identity theorem and Liouville's theorem generaliz easily.
Riemann extension theorem holds but is harder to prove and Riemann mapping definitely fails. There are also phenomenon that do not have analogues in one complex variable. 
\begin{theorem}[Hartog Extension Theorem]
  $n\ge 2$, $f:U \backslash  K\to \C$ holomorphic with $U \subset  \C^{n}$ open, $K\subset U$ compact. If $U \backslash  K$ is connected, then $\exists ! F:U\to \C$ holomorphic extending $f$.
\end{theorem}

\section{Solving $\partial u = f$}


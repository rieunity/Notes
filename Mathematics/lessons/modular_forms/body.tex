
\section{Modular Group, Congruence Subgroup and Modular Forms}
\begin{definition}
  The \textit{modular group} is the group of 2-by-2 matrices with integer entries and determinant 1:
  \begin{equation*}
    \SL:=\left\{ \begin{bmatrix} a & b\\
    c & d\end{bmatrix}:a,b,c,d\in \Z,ad-bc=1  \right\} .
  \end{equation*}
  The \textit{principal congruence subgroup of level $N$} is 
  \begin{equation*}
    \Gamma(N):=\left\{ \begin{bmatrix} a&b\\
    c&d\end{bmatrix} \in \SL:a\equiv d\equiv 1\pmod{N},  b\equiv c\equiv 0 \pmod{N}\right\} 
 . \end{equation*}
 $\Gamma$ is a congruence subgroup if 
 \[
   \Gamma(N)\subset \Gamma\subset \SL \text{ for some }N.
 \] 
\end{definition}
\begin{example}
  $\forall N\in \N$, 
  \[
    \Gamma_0(N)=\left\{ \begin{bmatrix} a&b\\c&d \end{bmatrix} \in \SL:c\equiv 0 \pmod{N} \right\} 
  \]
  and 
  \[
    \Gamma_1(N)=\left\{ \begin{bmatrix} a&b\\c,d \end{bmatrix} \in \SL:a\equiv d\equiv 1\pmod{N}, c\equiv 0\pmod{N} \right\} 
  \] 
  are congruence subgroups. Their relations are
  \[
    \Gamma(N)\subset \Gamma_1(N)\subset \Gamma_2(N)\subset \SL.
  \] 		

\end{example}
\begin{definition}
   $\mathcal{H}$ is the \textit{upper half plane} defined by 
   \[
     \mathcal{H}:=\left\{ \tau \in \C:\Im(\tau)>0 \right\} 
   .\]
   \textit{Action of $\SL$ on $\mathcal{H}$} is defined by
   \[
     \gamma(\tau)= \frac{a\tau+b}{c\tau+d}
   \] 
  for arbitrary $\gamma \in \SL$ and $\tau \in \mathcal{H}$.
   
\end{definition}

\[
\gamma(\tau)= \frac{a\tau+b}{c\tau+d}=\frac{(a\tau+b)(c \overline{\tau}+d}{|c\tau+d|^2}=\frac{ac|\tau|^2+bd+ad\tau+bc \overline{\tau}}{|c\tau+d|^2},
\]
then
\[
  \Im \gamma(\tau)= \frac{\Im(ad\tau+bc \overline{\tau})}{|c\tau+d|^2}= \frac{(ad-bc)\Im\tau}{|c\tau+d|^2}>0.
\]

Hence $\gamma(\tau)\in \mathcal{H}$ if $\tau \in \mathcal{H}$.


Let $\gamma =\begin{bmatrix} a&b\\c&d \end{bmatrix} $ and $\gamma'=\begin{bmatrix} a'&b'\\c'&d' \end{bmatrix} $, then
\[
  \gamma\gamma'=\begin{bmatrix} aa'+bc'&ab'+bd'\\ca'+dc'&c b'+d d' \end{bmatrix} .
\] 
It's easy to verify 
\[
  \gamma(\gamma'(\tau))=\gamma\gamma'(\tau).
\]
Now we consider actions of $\SL$ on functions $f: \mathcal{H}\to \C$. Write \[
  \gamma =\begin{bmatrix} a&b\\c&d \end{bmatrix} \in \SL ,\quad \tau \in \mathcal{H}.
\] 
\[
  j(\gamma,\tau):=c\tau+d.
\] 
For $k\in \Z$, define $[\gamma]_k:=$ $\text{ the weight-}k \text{ operator acting on functions } \mathcal{H}\to \C \text{ such that }$
\[
  \left( f[\gamma_k] \right)(\tau)=j(\gamma,\tau)^{-k}f(\gamma(\tau))=(c\tau+d)^{-k}f\left( \frac{a\tau+b}{c\tau+d} \right) . 
\]
\begin{lemma}
   $\forall \gamma,\gamma' \in \SL, \forall \tau \in \mathcal{H}$, we have 
   \begin{enumerate}
     \item $j(\gamma\gamma',\tau)=j(\gamma,\gamma'(\tau))j(\gamma',\tau)$;
     \item $[\gamma\gamma']_k=[\gamma]_k[\gamma']_k$;
     \item  $\frac{\mathrm{d}\gamma(\tau)}{\mathrm{d}\tau}=\frac{1}{j(\gamma,\tau)^2}$.
   \end{enumerate}
\end{lemma}
\begin{proof}
  \begin{enumerate}
    \item 
      \begin{align*}
	\gamma  \begin{bmatrix} \tau\\1 \end{bmatrix} =&\begin{bmatrix} \gamma(\tau)\\1 \end{bmatrix} j(\gamma,\tau)\\
	\gamma\gamma'\begin{bmatrix} \tau \\1 \end{bmatrix} =&\begin{bmatrix} \gamma\gamma'(\tau)\\
      1\end{bmatrix}j(\gamma\gamma',\tau)\\
      .\end{align*}
      Also
      \begin{align*}
	\gamma\gamma'\begin{bmatrix} \tau\\1 \end{bmatrix} =&\gamma\begin{bmatrix} \gamma'(\tau)\\1 \end{bmatrix} j(\gamma',\tau)\\
	=& \begin{bmatrix} \gamma\gamma'(\tau)\\1 \end{bmatrix} j(\gamma,\gamma'(\tau))j(\gamma',\tau)
      .\end{align*}
      \[
	\Rightarrow j(\gamma\gamma',\tau)=j(\gamma,\gamma'(\tau))j(\gamma',\tau).
      \] 
    \item 
      \begin{align*}
	&(f[\gamma\gamma']_k)(\tau)\\
	=& j(\gamma\gamma',\tau)^{-k}f(\gamma\gamma'(\tau))\\
	=& j(\gamma,\gamma'(\tau))^{-k}j(\gamma',\tau)^{-k}f(\gamma\gamma'(\tau))\\
	=& j(\gamma',\tau)f([\gamma]_k)(\gamma'(\tau))\\
	=&(f[\gamma]_k[\gamma']_k)(\tau)
      .\end{align*}
    \item 
      \begin{align*}
	\frac{\mathrm{d}\gamma(\tau)}{\mathrm{d}\tau}=& \frac{a(c\tau+d)-(a\tau+b)c}{(c\tau+d)^2}\\
	=& \frac{1}{(c\tau+d)^2}\\
	=& \frac{1}{j(\gamma,\tau)^2}
      .\end{align*}
  \end{enumerate}
\end{proof}
\begin{definition}
  Let $\Gamma= \text{ congruence subgroup}$ and $k\in \Z$ $f:\mathcal{H}\to \C$ is a weakly modular form function of weight $k$ with repect to $\Gamma$ if $f$ is meromorphic on $\mathcal{H}$ and 
  \[
    f[\gamma]_k=f,
  \] 
  i.e.,
  \[
    (c\tau+d)^{-k}f\left( \frac{a\tau+b}{c\tau+d} \right) = f.
  \] 
\end{definition}
Suppose $f$ is a weakly modular function of weight $k$ with respect to $\Gamma$. 
 \[
   \Gamma \supset \Gamma(N)\Rightarrow \begin{bmatrix} 1&N\\0&1 \end{bmatrix} \in \Gamma 
\] 
\[
  f\begin{bmatrix} 1&N\\0&1 \end{bmatrix} _{k}=f(\tau+N)=f(\tau)
\]
\[
  \Rightarrow \exists \text{ minimal } h\in \N \text{ such that } f(\tau+h)=f(\tau)
\]
\[
  \Rightarrow f(\tau)=g( e^{2\pi i\tau /h}) \text{ for some }g.  
\] 
\[
\tau \in \mathcal{H}\Leftrightarrow  |e^{2\pi i \tau/h|}<1
\] 
\[
  \Im(\tau)\to \infty \Rightarrow e^{2\pi i\tau /h}\to 0.
\] 
$g(z)$ is meromorphic on $0<|z|<1$.

$z\to 0$ : $g(z)=\sum_{n=-\infty}^{\infty} a_n z^n$.\\
$z=e^{2\pi i \tau /h}\Rightarrow \Im(\tau)\to \infty$
\begin{equation}\label{1-1}
  f(\tau)=g(e^{2\pi i\tau/ h})=\sum_{n=-\infty}^{\infty} a_n e^{2\pi in\tau /h}.
\end{equation}
We say $f(\tau)$ is holomorphic at $\infty$ if $a_n=0$ for all $n<0$ in (\ref{1-1}). In this case we write $f(\infty)=a_0$.

$\forall \sigma \in \SL$, $f[\sigma]_k$ is a weakly modular function of weight $k$ with respect to $\sigma^{-1}\Gamma\sigma$. Indeed, let $\gamma \in \Gamma$,
\[
  f[\gamma]_k(\tau)=f(\tau)
\] 
\[
  j(\gamma,\tau)^{-k}f(\gamma(\tau))=f(\tau)
\] 
\[
  j(\gamma,\sigma(\tau))^{-k}f(\gamma\sigma(\tau))=f(\sigma(\tau))
\] 
\begin{align*}
  ((f[\sigma]_k)[\sigma^{-1}\gamma\sigma]_k)(\tau)=&j(\sigma^{-1}\gamma\sigma,\tau)^{-k}(f[\sigma]_k)(\sigma^{-1}\gamma\sigma(\tau))\\
  =& j(\sigma^{-1}\gamma\sigma,\tau)^{-k}j(\sigma,\sigma^{-1}\gamma\sigma(\tau))^{-k}f\left(\gamma\sigma(\tau)  \right) \\
  =& j(\sigma^{-1}\gamma\sigma,\tau)^{-k}j(\sigma,\sigma^{-1}\gamma\sigma(\tau))^{-k}j(\gamma,\sigma(\tau))^{k}f(\sigma(\tau))\\
  = & j(\sigma,\tau)^{-k}f(\sigma(\tau))\\
  =& f[\sigma]_k
.\end{align*}

\begin{definition}
  \begin{enumerate}
    \item []
    \item $f:\mathcal{H}\to \C$ is a modular formof weight $k$ with respect to $\Gamma$ if $f$ is weakly modular function and $f\sigma_k$ is holomorphic at $\infty \forall \sigma \in \SL$.
    \item $f$ is a cusp form if $f$ is a modular form and $f\sigma_k$ vanishes at $\infty$ $ \forall \sigma \in \SL$.
  \end{enumerate}
\end{definition}
\begin{proposition}
  Assume $k=\text{odd}$, Then any modular form $f\equiv 0$.
\end{proposition}
\begin{proof}
  $\gamma =\begin{bmatrix} -1&0\\0&-1 \end{bmatrix} \in \Gamma$, for arbitrary $\tau \in \mathcal{H}$,
  \[
    j(\gamma,\tau)^{-k}f(\gamma(\tau))=(-1)^{-k}f( \frac{-\tau+0}{0+(-1)})=(-1)^{-k}f(\tau)=-f(\tau).
  \] 
\end{proof}
\section{Case $\Gamma=\SL$, Eisenstein Series}
In this section assume $\Gamma=\SL$. $\Gamma$ is generated by $\begin{bmatrix} 1&1\\0&1 \end{bmatrix} $ and $\begin{bmatrix} 0 &-1\\1&0 \end{bmatrix} $.
\[
  \left( f\begin{bmatrix} 1&1\\0&1 \end{bmatrix} _k \right) (\tau)=f(\tau+1)
  \] since $j\left( \begin{bmatrix} 1&1\\0&1 \end{bmatrix},\tau  \right)=1 $.
  \[
    \left( f\begin{bmatrix} 0&-1\\1&0 \end{bmatrix}_k  \right)= \frac{1}{\tau^{k}}f\left( -\frac{1}{\tau} \right)  
    \] since $j\left( \begin{bmatrix} 0&-1\\1&0 \end{bmatrix} ,\tau \right) =\tau$.
    Let  $f$ be a modular form of weight $k$, then
    \begin{align*}
      f(\tau+1)=&f(\tau)\\
      f\left( -\frac{1}{\tau} \right) =\tau^{k}f(\tau)
    .\end{align*}
    (to find a nontrivial modular form, $k$ must be even.)
    \begin{definition}[Eisenstein Series]
      Assume $k=\text{even}>2$. Define 
      \[
	G_k(\tau)=\sum_{(c,d)\in \Z^2\backslash (0,0)}^{} \frac{1}{(c\tau+d)^k}.
      \] 
      $G_k(\tau)$ is called Eisenstein series.
    \end{definition}
    $\tau \in \mathcal{H}\Rightarrow G_k(\tau)$ is absolutely convergent and holomorphic.
    \begin{align*}
      \begin{bmatrix} a&b\\c&d \end{bmatrix} \in \SL\Rightarrow&G_k\left( \frac{a\tau+b}{c\tau+d} \right)\\
      =&\sum_{(c',d')\neq (0,0)}^{} \frac{1}{\left( c' \frac{a\tau+b}{c\tau+d}+d' \right)^{k} }\\
      =&\left( c\tau+d \right) ^{k}\sum_{(c',d')\neq(0,0)}^{} \frac{1}{((c'a+cd')\tau+(c'b+d d'))^{k}}
    .\end{align*}
    As $(c',d')$ walks through all $\neq (0,0)$, so does $(c'a+cd',c'b+d d')$. Hence
    \[
      G_k\left( \frac{a\tau+b}{c\tau+d} \right) =\left( c\tau+d \right) ^{k}G_k(\tau).
    \]
    When $\Im(\tau)\to \infty$,
  \begin{equation*}
    \frac{1}{(c\tau+d)^{k}}\to\left\{\begin{aligned}
	&0 & \text{ if }c\neq 0,\\
	&d^{-k}& \text{ if }c=0.
    \end{aligned}\right.
  \end{equation*}
  \begin{align*}
    \Rightarrow G_k(\infty)=& \lim_{\Im(\tau) \to \infty} G_k(\tau)\\
    =& \sum_{d=-\infty,d\neq 0}^{\infty} \frac{1}{d^{k}}=2\zeta(k)
  .\end{align*}
  Assume $f$ is a modular form of weight  $k$, $k=\text{even}>2$. Define
  \begin{align*}
    D=&\left\{ q\in \C:|q|<1 \right\}\\
    D'=&D-\{0\} 
  .\end{align*}
 Construct the mapping
 \begin{align*}
   \mathcal{H}\to&D'\\
   \tau\mapsto&e^{2\pi i\tau}=q
 \end{align*}
 and
 \begin{align*}
   g:D'\to&\C\\
   q\mapsto & f\left( \log(q)/(2\pi i) \right). 
 \end{align*}
$g$ is well defined even though the logarithm is only determined up to $2\pi i\Z$. Then 
\[
  f(\tau)=g(e^{2\pi i\tau}).
\]

At $\tau=\infty$, $G_k(\tau)=\sum_{n=0}^{\infty} a_ne^{2\pi in\tau}=2\zeta(k)$  $\Rightarrow a_0=2\zeta(k)$. $a_n=?$ 
\begin{proposition}
  Let $\tau \in \mathcal{H}$, we have
  \[
    \frac{1}{\tau}+\sum_{d=1}^{\infty} \left(\frac{1}{\tau-d}+\frac{1}{\tau+d}\right)=\pi\cot \pi\tau.
  \] 
\end{proposition}
\begin{proof}
  \[
    \sin \pi\tau=\pi\tau \prod_{n=1}^{\infty}\left( 1- \frac{\tau^2}{n^2} \right). 
  \] 
  \[
    \log \sin \pi\tau =\log \pi+\log\tau+\sum_{n=1}^{\infty} \log\left( 1- \frac{\tau^2}{n^2} \right) .
  \] 
  Taking the derivative, we obtain
  \[
    \pi\cot\pi\tau=\frac{1}{\tau}+\sum_{n=1}^{\infty} \frac{-2\tau /n^2}{1-\tau^2 /n^2}=\frac{1}{\tau}+\sum_{n=1}^{\infty} \left( \frac{1}{\tau-n}+\frac{1}{\tau+n} \right). 
  \] 
\end{proof}

\begin{align*}
  \frac{1}{\tau}+\sum_{d=1}^{\infty} \left( \frac{1}{\tau-d}+\frac{1}{\tau+d} \right) =& \pi i \frac{e^{i\pi\tau}+e^{-i\pi\tau}}{e^{i\pi\tau}-e^{-i\pi\tau}}\\
  =& -\pi i \frac{1+e^{2\pi i\tau}}{1-e^{2\pi i\tau}}\\
  =& -\pi i-2\pi i \sum_{m=0}^{\infty} e^{2\pi im\tau}
.\end{align*}
Differentiating $(k-1)$ times we get
\[
  (-1)^{k-1}(k-1)!\sum_{d=-\infty}^{\infty} \frac{1}{(\tau+d)^{k}}=-2\pi i \sum_{m=0}^{\infty} (2\pi im)^{k-1}e^{2\pi im\tau}.
\] 
\[
  \Rightarrow \sum_{d=-\infty}^{\infty} \frac{1}{(\tau+d)^{k}}-\frac{(2\pi i)^{k}}{(k-1)!}\sum_{m=1}^{\infty} m^{k-1}e^{2\pi im\tau}=0.
\] 
\begin{align*}
  \Rightarrow G_k(\tau)=&\sum_{c\neq 0}^{} \sum_{d=-\infty}^{\infty} \frac{1}{(c\tau+d)^{k}}+\sum_{d\neq 0}^{\infty} \frac{1}{d^{k}}\\
  =& 2 \sum_{c=1}^{\infty} \sum_{d=-\infty}^{\infty} \frac{1}{(c\tau+d)^{k}}+2\zeta(k)\\
  =& 2 \sum_{c=1}^{\infty} \frac{(2\pi i)^{k}}{(k-1)!}\sum_{m=1}^{\infty} m^{k-1}e^{2\pi imc\tau}+2\zeta(k)\\
  =& 2 \sum_{n=1}^{\infty} \frac{(2\pi i)^{k}}{(k-1)!}\left( \sum_{m|n}^{} m^{k-1} \right) e^{2\pi in\tau}+2\zeta(k)
.\end{align*}
Then we get the following conclusion:
\begin{proposition}
  Let $k= \text{ even }>2$,$G_k(\tau)=\sum_{n=0}^{\infty} a_ne^{2\pi in\tau}$, then
  \[
    a_0=2\zeta(k)
  \] 
  \[
    a_n=\frac{2(2\pi i)^{k}}{(k-1)!}\sigma_{k-1}(n)
  \] 
  where $n>0$ and $\sigma_{k-1}(n)=\sum_{m|n} m^{k-1} $.
\end{proposition}

A group $G$ acting on a set $S$ gives rise to an equivalent relation $\sim$ on $S$:
 \[
s_1\sim s_2 \Leftrightarrow \exists g\in G \text{ such that }s_2=gs_1.
\]
The quotient space $\sfrac{S}{G}=$ the set of the equivalent class.
\begin{question}
  $\SL$ acts on $\mathcal{H}$, what is the quotient space?
\end{question}
\begin{theorem}
  Let $D=\left\{ \tau \in \mathcal{H}:|\Re\tau|\le \frac{1}{2},|\tau|\ge 1 \right\} $. Then $D$ is a fundamental domain for $\SL$ in the sense such that 
  \begin{enumerate}
    \item $\forall \tau \in \mathcal{H}$, $\exists \gamma \in \SL$ such that $\gamma(\tau)\in D$;
    \item If $\tau_1,\tau_2 \in D$, $\tau_1\neq\tau_2$ and $\tau_2=\gamma(\tau_1)$ for some $\gamma \in \SL$, then either $\Re(\tau_1)=\pm \frac{1}{2},\tau_2=\tau_1\mp 1 $ or $|\tau_1|=1$,$\tau_2=-\frac{1}{\tau_1}$.
  \end{enumerate}
\end{theorem}
\begin{figure}[ht]
    \centering
    \incfig{domain-d}
    \caption{Domain D}
    \label{fig:domain-d}
\end{figure}
\begin{proof}
  Write $\gamma=\begin{bmatrix} a&b\\c&d \end{bmatrix} \in \SL$, $\gamma_1=\begin{bmatrix} 1&1\\0&1 \end{bmatrix} \in \SL$, $\gamma_2=\begin{bmatrix} 0&-1\\1&0 \end{bmatrix} \in \SL$. Then
  \[
    \gamma_1^{n}(\tau)=\tau+n
  \] 
  \[
    \gamma_2(\tau)=-\frac{1}{\tau}.
  \] 
  \begin{enumerate}
    \item $\forall \tau \in \mathcal{H}$, $\Im\gamma(\tau)= \frac{\Im(\tau)}{|c\tau+d|^2}$. There exists only finite $(c,d)$ such that $|c\tau+d|<$ a given volume. $\Rightarrow$ ($\exists \gamma $ such that $|c\tau+d|=\min$ $\Rightarrow \Im\gamma(\tau)=\max$ and $\exists n\in \Z$ such that $|\Re\gamma_1^{n}(\gamma(\tau))|\le \frac{1}{2}$).

      Write $\tau_1=\gamma_1^{n}(\gamma(\tau))$,$\tau_2=\gamma_2(\tau_1)$.
      \begin{align*}
	\Im\tau_2=\frac{\Im\tau_1}{|\tau_1|^2} \text{ and }&\Im\tau_2\le \Im\gamma(\tau)= \Im\tau_1\\
	\Rightarrow & |\tau_1|\ge 1\Rightarrow \tau_1\in D
      .\end{align*}
    \item Suppose $\tau_1,\tau_2 \in D$,$\tau_1\neq\tau_2$,$\gamma(\tau_1)=\tau_2$ for some $\gamma\in \SL$. We assume $\Im\tau_1\le \Im\tau_2$, then
      \begin{align*}
	\Im\tau_1\le & \frac{\Im \tau_1}{|c\tau_1+d|^2}\\
	\Rightarrow &|c\tau_1+d|\le 1
      .\end{align*}
      Since $\Im\tau_1\ge \frac{\sqrt{3} }{2}$, then $c=0,\pm 1$. 
      \begin{itemize}
	\item  [Case 1:] $c=0,a=d=\pm 1$
	  \begin{align*}
	    \Rightarrow & \tau_2=\tau_1\pm b\\
	    \Rightarrow &\left\{ 
	    \begin{aligned}
	      &\Re \tau_1=\pm \frac{1}{2}\\
	      &\tau_2=\tau_1\mp 1
	    \end{aligned}\right.
	  .\end{align*}
	\item [Case 2:] $c=1,|\tau_1+d|\le 1$
\begin{align*}
  \Rightarrow & \left\{
    \begin{aligned}
      & d=0\Rightarrow |\tau_1|=1,\tau_2=-\frac{1}{\tau_1}\\
      &  \text{ or } \tau_1=\rho, d=1\\
      & \text{ or } \tau_1=-\overline{\rho},d=-1
    \end{aligned}\right.
.\end{align*}
	\item [Case 3:] $c=-1$, similar to Case 2. 
      \end{itemize}
  \end{enumerate}
  Here $\rho := e^{2\pi i/3}=-\frac{1}{2}+\frac{\sqrt{3} }{2}i$.
\end{proof}

The quotient space $\sfrac{\mathcal{H}}{\SL}$ is obtained by identifying the left side and right side of $D$ and identifying the left and right parts of the bottom circle of $D$.

Assume  $f$ is holomorphic on $\mathcal{H}$ and $\infty(\Im\tau\to \infty)$. 

At $p\in \mathcal{H}$, $m=$ order of $f$ at $p$ means
\[
  \lim_{\tau\to p}\frac{f(\tau)}{(\tau-p)^{m}}
\] 
exists and not equals $0$. We use $v_{p}(f)=m$ to represent this meanning.

At $\infty$, if $a_m\neq 0$ in $f(\tau)=\sum_{n=m}^{\infty} a_ne^{2\pi in\tau}$, write $v_{\infty}(f)=m$.

If $p_1,p_2$ are equivalent ( $p_2=\gamma p_1$ for some $\gamma \in \SL$), then $v_{p_1}(f)=v_{p_2}(f)$.

\begin{theorem}\label{thm2-5}
  Suppose $f$ is a (non-zero) modular form of weight $k$ ($k$ even),  $\rho = e^{ \frac{2\pi i}{3}}$. Then 
  \begin{equation}\label{2-2}
    v_{\infty}(f)+\frac{1}{2}v_i(f)+\frac{1}{3}v_\rho (f)+\sum_{p \in \sfrac{\mathcal{H}}{\SL}}\nolimits^{*} v_p(f)=\frac{k}{12}
  \end{equation} 
  where $\sum^{*} $ means the sum is over $p \in \sfrac{H}{\SL}$ and $p\not\sim i,\rho$, i.e., $\sum_{p\in \sfrac{\mathcal{H}}{\SL}}^{*}:=\sum_{p\in D\backslash \{i,\rho\} }$.
\end{theorem}
 $k=2$: right side $=\frac{1}{6}$, left side either $=0$ or $\ge \frac{1}{3}$. Hence modular forms of weight $2$ don't exist.\\
 $k=4$: right side  $=\frac{1}{3}$, $f=G_4$,$v_\rho f=1$,  $v_{\infty}(f)=v_i(f)=0$, $\sum_{p \in \sfrac{\mathcal{H}}{\SL}}^{*} v_p(f)=0$.\\
 $k=12$:  $G_4^3$, $G_6^2$, $\exists $ linear combination
 \[
 \Delta=c_1G_4^3+c_2G_6^2, \Delta:\text{ weight }=12
 \] 
 such that $\Delta(\infty)=0$. Since 
 \[
   v_{\infty}(\Delta)+\frac{1}{2}v_i(\Delta)+\frac{1}{3}v_\rho(\Delta)+\sum\nolimits^{*}=\frac{12}{12}=1
 \]
 and $v_{\infty}(\Delta)=1$, we have $\Delta(\tau)\neq 0$ $\forall \tau \in \mathcal{H}$.

 \begin{definition}
   Define
   \begin{align*}
     \mathcal{M}_k:=&\text{ space of modular forms of weight }k\\
     \mathcal{S}_k:=&\text{ space of cusp}\left( f(\infty)=0 \right) \text{ forms of weight }k
   .\end{align*}
 \end{definition}

 If $\mathrm{dim}\mathcal{M}_k>0$, then
\[
\mathrm{dim}\mathcal{S}_k=\mathrm{dim}\mathcal{M}_k-1.
\]
In fact, $\mathcal{M}_k=\mathcal{S}_k\oplus \C$

If $f\in \mathcal{M}_k$, then $\Delta f\in S_{k+12}$, then we establish an isomorphism 
\begin{align*}
  \mathcal{M}_k\to &S_{k+12}\\
  f\mapsto&\Delta f
.\end{align*}
If we know $\mathrm{dim}\mathcal{M}_k$ for all $k\le 12$, then we know all the conditions.

Before tge proof of Theorem \ref{thm2-5}, we introduce the following lemma in complex analysis:
\begin{lemma}
  \begin{equation}
    \frac{1}{2\pi i}\int_C \frac{\mathrm{d}\tau}{\tau-\rho}=- \frac{\theta}{2 \pi}
  \end{equation}
  where  $C$ is given by
\begin{figure}[ht]
    \centering
    \incfig{argument-principle}
    \caption{Argument principle}
    \label{fig:argument-principle}
\end{figure}
\end{lemma}
\textit{Proof of Theorem \ref{thm2-5}}. Assume $f\neq 0$ on the boundary of $D$, except at $\tau=\rho,i-\overline{\rho}$, consider the countour $L$ in Figure \ref{fig:countour}.
\begin{figure}[ht]
    \centering
    \incfig{countour-L}
    \caption{Countour $L$}
    \label{fig:countour}
\end{figure}
Residue Theorem $\Rightarrow$ 
\begin{equation}\label{2-3}
  \frac{1}{2\pi i}\int_{L} \frac{\mathrm{d}f}{f}=\sum\nolimits_{p}^{*}v_p(f).
\end{equation}
Recall $f(\tau)=g(e^{2\pi i\tau})$ and let $z=e^{2\pi i\tau}$. Set $\Im A'=T$. Then  $z=-e^{-2\pi T}\to -e^{-2\pi T}$ along a circle $\omega$:
\begin{equation}\label{2-4}
   \frac{1}{2\pi i}\int_{A'}^{A}\frac{\mathrm{d}f}{f}=\frac{1}{2\pi i}\int_{\omega} \frac{\mathrm{d}g}{g}=-v_{\infty}(f).
\end{equation}
Since 
\[
  \frac{1}{2\pi i}\int_A^{B} \frac{\mathrm{d}f}{f}=\frac{1}{2\pi i}\int_A^{B} \frac{\mathrm{d}f(\tau+1)}{f(\tau+1)}
\] 
and $\tau+1:A'\to E'$, we get 
   \begin{equation}
     \frac{1}{2\pi i}\left(\int_A^{B}+\int_{E'}^{A'}\right) \frac{\mathrm{d}f(\tau)}{f(\tau)}=0.
   \end{equation}
$\tau:B\to B'$:
 \[
   \frac{\mathrm{d}f}{f}\sim \frac{v_\rho(f)}{\tau-\rho},
\] 
hence 
\begin{equation}
  \frac{1}{2\pi i}\int_B^{B'}\frac{\mathrm{d}f}{f}\to -\frac{1}{6}v_\rho(f).
\end{equation}
$\tau:B'\to C\Rightarrow -\frac{1}{\tau}:E\to C'$:
\begin{align*}
  \frac{\mathrm{d}f(\tau)}{f(\tau)}=&\frac{\mathrm{d}\tau^{-k}}{\tau^{-k}}+\frac{\mathrm{d}f(-\frac{1}{\tau})}{f(-\frac{1}{\tau})}\\
  =& -k \frac{\mathrm{d}\tau}{\tau}+ \frac{\mathrm{d}f(-\frac{1}{\tau})}{f(-\frac{1}{\tau})}.
\end{align*}
\begin{equation}
  \frac{1}{2\pi i}\int_{B'}^{C}\frac{\mathrm{d}f}{f}=\frac{1}{2\pi i}\int_{B'}^{C}\frac{k}{\tau}\mathrm{d}\tau+\frac{1}{2\pi i}\int_{E}^{C'} \frac{\mathrm{d}f(\tau)}{f(\tau)}=\frac{k}{12}-\frac{1}{2\pi i}\int_{C'}^{E} \frac{\mathrm{d}f(\tau)}{f(\tau)}.
\end{equation}
$\tau:C\to C'$:
 \[
   \frac{\mathrm{d}f}{f}\sim \frac{v_i(f)}{\tau-i}
\] 
\begin{equation}
  \frac{1}{2\pi i}\int_{C}^{C'}\frac{\mathrm{d}f}{f}\to -\frac{1}{2}v_i(f).
\end{equation}
$\tau:E\to E'$: Similarly 
\begin{equation}\label{2-9}
  \frac{1}{2\pi i}\int_{E}^{E'} \frac{\mathrm{d}f}{f}\to -\frac{1}{6}v_\rho(f).
\end{equation}
By (\ref{2-4})$\to$ (\ref{2-9}) we obtain the conclusion.\hfill$\square$\par
If $f$ has a zero ($\neq \rho,-\overline{\rho},i$) on the boundary of $D$, modify $L$ as Figure \ref{fig:modified}
\begin{figure}[ht]
    \centering
    \incfig{modified-conuntour}
    \caption{Modified conuntour}
    \label{fig:modified}
\end{figure}

As vector space over $\C$, let 
\[
\mathcal{M}_k= \text{ the space of modular form of weight }k
\] 
\[
\mathcal{S}_k=\text{ the space of cusp form of weight }k.
\] 
In case $\mathrm{dim}\mathcal{M}_k>1$ say $d=\mathrm{dim}\mathcal{M}_k$, $f_1,f_2,\cdots,f_d$ a basis of $\mathcal{M}_k$. i.e.,
\[
\mathcal{M}_k=\{c_1f_1+c_2f_2+\cdots+c_df_d:c_1,\cdots,c_d \in \C\}. 
\] 
\begin{align*}
  c_1f_1+\cdots c_df_d\in S_k \Leftrightarrow & c_1f_1(\infty)+c_df_d(\infty)=0\\
  \Rightarrow & \mathrm{dim}\mathcal{S}_k=\mathrm{dim}\mathcal{M}_k-1
.\end{align*}
In general $\mathrm{dim}\mathcal{M}_k\le \mathrm{dim}\mathcal{S}_k+1$.

Write
\begin{align*}
  g_2=&60G_4\\
  g_3=&140 G_6
.\end{align*}
Define $\Delta=g_2^{3}-27g_3^2$, notice that $g_2^3$ and $g_3^2$ are both modular forms of weight $12$. Then $\Delta\in \mathcal{M}_k$. 
\[
  \Delta(\infty)=0\Rightarrow \Delta\in \mathcal{S}_k.
\] 
\begin{align*}
  (\ref{2-2})\overset{f=\Delta}{\Longrightarrow}&v_{\infty}(\Delta)+\frac{1}{2}v_i(\Delta)+\frac{1}{3}v_{\rho}(\Delta)+\sum\nolimits^{*}v_p(\Delta)=1\\
  \Delta(\infty)=0\Longrightarrow & v_{oo}(\Delta)\ge 1\\
  \overset{\text{ combine with the above }}{\Longrightarrow} & v_{\infty}(\Delta)=1,\quad v_i(\Delta)=v_\rho(\Delta)=v_p(\Delta)=0\\
  \Longrightarrow & \Delta(\tau)\neq 0, \forall \tau \in \mathcal{H}
.\end{align*} 
\begin{theorem}
  \begin{enumerate}
    \item []
    \item When $k<0$ and $k=2$, $\mathcal{M}_k=0$.
    \item The map: $\begin{matrix} \mathcal{M}_k \to &\mathcal{S}_{k+12}\\f\mapsto& \Delta f \end{matrix} $ is an isomorphism.
    \item When $k=0,4,6,8,10$,$\mathrm{dim}\mathcal{M}_k=1$, $\mathrm{dim}S_k=0$, Their basis are $1,G_4,G_6,G_8,G_{10}$ respectively.
  \end{enumerate}
\end{theorem}
\begin{proof}
  \begin{enumerate}
    \item []
    \item Suppose $\exists$ non-zero modular $f\in \mathcal{M}_k$. Left side of $(\ref{2-2})\ge 0 \Rightarrow k\ge 0$.

      For $k=2$,  $v_{\infty}(f),v_i(f),v_\rho(f),v_p(f)$ are non-negative integers. If one of them $\ge 1$, the left side of $(\ref{2-2})\ge  \frac{1}{3}$. But the right side $=\frac{1}{6}$.
    \item It suffices to prove that $f\mapsto \Delta f$ is surjective. $\forall g\in S_{k+12}$, consider $h=\frac{g}{\Delta}$. $\left\{\begin{matrix} v_{\infty}(g)\ge &1\\v_{\infty}(\Delta)=1\end{matrix}\right. \Rightarrow v_{\infty}(h)\ge 1$ $\Rightarrow$ $h$ is holomorphic at $\infty$. 
      \[
      \left\{ 
	\begin{aligned}
	  g &\text{ is holomorphic on }\mathcal{H}\\
	\Delta^{-1}& \text{ is holomorphic and } \Delta\neq 0 \text{ on }\mathcal{H}
	\end{aligned}\right.\Rightarrow  h  \text{ is holomorphic on }\mathcal{H}.
      \]
      \[
	h\left( \frac{a\tau+b}{c\tau+d} \right) =\frac{g\left( \frac{a\tau+b}{c\tau+d} \right) }{\Delta\left( \frac{a\tau+b}{c\tau+d} \right) }=\frac{(c\tau+d)^{k+12}g(\tau)}{(c\tau+d)^{12}\Delta(\tau)}=(c\tau+d)^{k}h(\tau)
      \] 
      \[
      \Rightarrow h\in \mathcal{M}_k.
      \]
    \item For $k=0,4,6,8,10$,  $\exists $ non-zero $f\in \mathcal{M}_k$ 
      \[
     \left. \begin{aligned}
	f\equiv 1 \text{ for }&k=0\\
	f=G_{k} \text{ for }&k\ge 4
    \end{aligned}\right\}\Rightarrow \mathrm{dim}\mathcal{M}_k\ge 1.
      \] 
      By b we have $S_{k}\simeq \mathcal{M}_{k-12}$, but $k-12\le 0$ $\Rightarrow \mathcal{M}_{k-12}=0$ $\Rightarrow$ $S_k=0$. 
  \end{enumerate}
\end{proof}

\begin{corollary}\label{crc2-9}
  For $k\ge 0$, we have
  \[
  \mathrm{dim}\mathcal{M}_k=\left\{
    \begin{aligned}
      &\left[ \frac{k}{12} \right] &  \text{ if }k\equiv 2 \pmod{12},\\
      &\left[ \frac{k}{12} \right]+1 & \text{ if }k\not\equiv 2\pmod{12}. 
    \end{aligned}\right.
  \]
  \begin{proof}
    $0\le k\le 10$ can be directly verified.

    For $k=12$,  $\mathcal{S}_{12}\simeq \mathcal{M}_0$ $\Rightarrow$ $\mathrm{dim}\mathcal{S}_12=1$$\Rightarrow$ $\mathrm{dim}\mathcal{M}_{12}=2$. By $\mathcal{S}_{k+12}\simeq \mathcal{M}_{k}$ we get 
    \[
    \mathrm{dim}\mathcal{M}_{k+12}=\mathrm{dim}\mathcal{S}_{k+12}+1=\mathrm{dim}\mathcal{M}_k+1,
    \] 
    then use induction.
  \end{proof}
\end{corollary}

\begin{corollary}
  For $k\ge 12$, the set $\left\{G_4^{\alpha}G_6^{\beta}: 4\alpha+6\beta=k,\alpha,\beta\ge 0\right\}$ is a basis of $\mathcal{M}_k$.
\end{corollary}
\begin{proof}
  The elements of $i\left\{G_4^{\alpha}G_6^{\beta}:4\alpha+6\beta=k,\alpha,\beta\ge 0\right\} $ are linearly independent and the number of $G_4^{\alpha}G_6^{\beta}$ is $\mathrm{dim}\mathcal{M}_k$ by Corollary \ref{crc2-9}.
\end{proof}
\section{Complex Tori}
A \textit{Riemann surface} is an 1-dimensional connected complex manifold.
\begin{proposition}
  If $f:S_1\to S_2$ is a holomorphic map of compact Riemann surfaces, then either the image of $f$ is a point, or $f$ is surjective.
\end{proposition}
\begin{proof}
  Suppose $X$ and $Y$ are compact Riemann surfaces and $f:X\to Y$ is holomorphic. Since $f$ is continuous and $X$ is compact and connected, so is the image $f(X)$, making $f(X)$ closed. Unless $f$ is constant $f$ is open by the Open Mapping Theorem of complex analysis, applicable to Riemann surfaces since it is a local result, making $f(X)$ open as well. So $f(X)$ is either a single point or a connected, open, closed subset of the connected set $Y$, i.e., all of $Y$.
\end{proof}

Assume $\omega_1,\omega_2\in \C$, linearly independent over $\R$ (for normalization, set $\Im \frac{\omega_1}{\omega_2}>0$).
\begin{figure}[ht]
    \centering
    \incfig{relation-between-two-numbers}
    \caption{Relation between two numbers}
    \label{fig:relation-between-two-numbers}
\end{figure}

Let $\Lambda=\omega_1\Z\oplus\omega_2\Z=\{\omega_1n_1+\omega_2n_2:n_1,n_2\in \Z\} $, $\Lambda$ is a lattice and a discrete subgroup fo $\C$.
\begin{lemma}
  Let 
  \begin{align*}
    \Lambda=&\omega_1\Z\oplus \omega_2\Z,\\
    \Lambda'=&\omega'_1\Z\oplus\omega'_2\Z
  .\end{align*}
Then $\Lambda=\lambda'$ $\Leftrightarrow$ $\exists \begin{bmatrix} a&b\\c&d \end{bmatrix} \in \SL$ such that 
\[
  \begin{bmatrix} \omega'_1\\\omega'_2 \end{bmatrix} =\begin{bmatrix} a&b\\c&d \end{bmatrix} \begin{bmatrix} \omega_1\\ \omega_2 \end{bmatrix}.
\]

\end{lemma}
\begin{definition}[Complex Tori]
  A \textit{complex tori} is a quotien of $\C$ by $\Lambda$ :
  \[
    \sfrac{\C}{\Lambda}=\{z+\Lambda:z\in \C\}. 
  \] 
\end{definition}
In algebra: $\sfrac{\C}{\Lambda}=$ an abelian group,
\[
  \left( z_1+\Lambda \right) +\left( z_2+\Lambda \right) =(z_1+z_2)+\Lambda.
\] 
In topology: $\sfrac{\C}{\Lambda}=$ the parallelogram on identifying the opposite side (see Figure \ref{fig:parallelogram}).
\begin{figure}[ht]
    \centering
    \incfig{parallelogram}
    \caption{The parallelogram $\sfrac{\C}{\Lambda}$}
    \label{fig:parallelogram}
\end{figure}

\[
z_1+\Lambda=z_2+\Lambda\Leftrightarrow z_1-z_2 \in \Lambda.
\] 
\begin{proposition}
  Suppose $\varphi:\sfrac{\C}{\Lambda}\to \sfrac{\C}{\Lambda'}$ is holomorphic. $\exists m,b\in \C$ such that 
  \[
    m\Lambda\subset \Lambda',\quad \left( m\Lambda =\{mz:z\in \Lambda\}  \right) 
  \] 
  and 
  \[
    \varphi(z+\Lambda)=mz+b+\Lambda'.
  \] 
  $\varphi$ is invertable $\Leftrightarrow$ $m\Lambda=\Lambda'$.
\end{proposition}

\begin{proof}
  Let 
  \[
    \begin{matrix} p:\C&\to& \sfrac{\C}{\Lambda}\\ z&\mapsto &z+\Lambda\end{matrix} 
  \]
  and 
  \[
    \begin{matrix} p':\C&\to & \sfrac{\C}{\Lambda'}\\
    z&\mapsto & z+\Lambda'\end{matrix} .
  \]
  By universal cover lifting: $\exists \widetilde{\varphi}:\C\to \C$ holomorphic such that the diagram
  \[
    \begin{tikzcd}
      \C \arrow{r}{\widetilde{\varphi}} \arrow[swap]{d}{p} & \C \arrow{d}{p'}\\
      \sfrac{\C}{\Lambda} \arrow{r}{\varphi} & \sfrac{\C}{\Lambda'}
    \end{tikzcd}
  \] 
is commutative: $p'\circ \widetilde{\varphi}=\varphi\circ p$.
$\forall \lambda \in \Lambda$, 
\begin{align*}
 & p'\left( \widetilde{\varphi}(z+\lambda)-\widetilde{\varphi}(z) \right)\\
 = & \varphi\left( p(z+\lambda) \right) -\varphi\left( p(z) \right) \\
 =& \varphi\left( p(z) \right) -\varphi\left( p(z) \right) \\
 = & 0+\Lambda'\\
 \Rightarrow & \widetilde{\varphi}\left( z+\lambda \right)-\tilde{\varphi}(z)\in \Lambda' 
.\end{align*}
Left side $=$ holomorphic function in $z$ taking values in discrete $\Lambda'$ $\Rightarrow$ it is constant
\begin{align*}
  \Rightarrow & \widetilde{\varphi}'(z+\lambda)-\tilde{\varphi}'(z)=0 \text{ (here the prime symbol means taking derivative) }\\
  & \widetilde{\varphi}' = \text{ entire function and period }=\lambda \in \Lambda\\
  \Rightarrow & \widetilde{\varphi}' \text{ is bounded}\\
  \Rightarrow & \widetilde{\varphi}'= \text{ constant by Liouville's Theorem}\\
  \Rightarrow & \widetilde{\varphi}=mz+b \text{ for some }m,b \in \C \\
  \Rightarrow & \varphi(z+\Lambda)=mz+b+\Lambda'
.\end{align*}
To prove $m\Lambda\subset \Lambda'$, $\forall z\in \Lambda$,
\[
  \left. \begin{matrix} \varphi(z+\Lambda)=&mz+b+\Lambda'&\\
  \varphi(0+\Lambda)=&b+\Lambda'&\end{matrix} \right\}\Rightarrow mz \in \Lambda' \text{ since }\varphi(z+\Lambda)=\varphi(0+\Lambda).
\]
$\varphi$ is invertable $\Leftrightarrow$ directly verified.
\end{proof}

\begin{corollary}
  Suppose $\varphi:\sfrac{\C}{\Lambda}\to \sfrac{\C}{\Lambda'}$ is holomorphic:
  \[
    \varphi(z+\Lambda)=mz+b+\Lambda', \quad m\Lambda\subset \Lambda'.
  \] 
  Then $\varphi$ is a group homomorphism $\Leftrightarrow$ $b\in \Lambda'$.
\end{corollary}

\begin{definition}
  A nonzero holomorphic homomorphism:
  \[
    \sfrac{\C}{\Lambda}\to \sfrac{\C}{\Lambda'}
  \] 
  is called an \textit{isogeny} .
\end{definition}
Isogeny is surjective, its kernel is finite (it is discrete, otherwise the map is zero).

  A  \textit{curve} $C$ (in $\R^2$) means $\exists$ polynomial $F(x,y)$ such that 
  \[
    (x,y)\in C\Leftrightarrow F(x,y)=0.
  \] 
  Assume $\Lambda=\omega_1\Z\oplus\omega_2\Z$, then $\sfrac{\C}{\Lambda}=$ a complex curve, why?

  Define (Weierstrass-$p$ function)
  \[
  p=p_{\Lambda}:\C\to \C \cup \{\infty\} 
  \] by 
  \[
    p(z)=\frac{1}{z^2}+\sum_{\begin{smallmatrix} &\omega \in \Lambda\\ &\omega\neq 0 \end{smallmatrix} }^{} \left( \frac{1}{\left( z+\omega \right) ^2}-\frac{1}{\omega^2} \right) .
  \] 
  At each $z\in \Lambda$, $p(z)$ has a double pole, otherwise $p(z)$ is holomorphic. 
  
  $\forall \lambda \in \Lambda$, $\lambda\neq 0$, 
  \begin{align*}
    p(z-\lambda)= & \frac{1}{(z-\lambda)^2}+\sum_{\begin{smallmatrix} & \omega \in \Lambda\\ &\omega\neq 0\\&\omega\neq -\lambda  \end{smallmatrix} }\left( \frac{1}{(z-\lambda-\omega)^2}-\frac{1}{\omega^2} \right) +\left( \frac{1}{z^2}-\frac{1}{\lambda^2} \right)
  .\end{align*}
 By the virtue of 
 \[
   \lim_{z\to \infty}\left( \sum_{\begin{smallmatrix} &\omega \in \Lambda\\&|\omega|<z\\&\omega\neq 0,-\lambda  \end{smallmatrix} }\frac{1}{\omega^2}- \sum_{\begin{smallmatrix}&\omega \in \Lambda\\&|\omega|<z\\&\omega\neq 0,-\lambda  \end{smallmatrix} }\frac{1}{(\omega+\lambda)^2} \right) =0
 \] 
 \[
   \Rightarrow p(z) \text{ is } \Lambda\text{-periodical},
 \] 
 i.e., $p(z-\lambda)=p(z),\forall \lambda \in \Lambda$.
\[
  p'(z)=-2\sum_{\omega \in \Lambda}\frac{1}{(z-\omega)^3}, p'(z-\lambda)=p'(z),\forall \lambda \in \Lambda.
\] 
Identify $\sfrac{\C}{\Lambda}$ with the parallelogram (see Figure \ref{fig:parallelogram}), consider
\begin{align*}
  \left\{
    \begin{matrix} \sfrac{\C}{\Lambda}\\z\neq 0 \end{matrix} 
  \right\}&\to  \C^2\\
  z&\mapsto  (p(z),p'(z))
.\end{align*}
For $k=\text{ even }>2$,
\[
  G_k(\Lambda)= \sum_{\begin{smallmatrix}&\omega \in \Lambda\\&\omega\neq 0  \end{smallmatrix} }\frac{1}{\omega^{k}}=\sum_{(c,d)\neq(0,0)}\frac{1}{\left(c\omega_1+d\omega_2  \right) ^{k}}.
\]
Laurent expansion of $p(z)$ and $p'(z)$ at  $z=0,\omega\neq 0,\omega \in \Lambda,|z|<|\omega|$ $\Rightarrow$  $\frac{1}{(z-\omega)^2}-\frac{1}{\omega^2}=\frac{1}{\omega^2}\left( \frac{1}{\left( 1- \frac{z}{\omega} \right) ^2}-1 \right) =\cdots$ $\Rightarrow$ when $z\to 0$:
\begin{equation}\label{3-11}
   p(z)= \frac{1}{z^2}+3G_4(\Lambda)z^2+5G_6(\Lambda)z^{4}+\mathcal{O}(|z|^{6})
\end{equation}
\begin{equation}\label{3-12}
  p'(z)=-\frac{2}{z^3}+6G_4(\Lambda)z+20 G_6(\Lambda)z^3+\mathcal{O}(|z|^{5}).
\end{equation}
Write 
\begin{align*}
  g_2(\Lambda)=&60G_4(\Lambda)\\
  g_3(\Lambda)=&140 G_6(\Lambda)
.\end{align*}
(\ref{3-11}),(\ref{3-12}) $\Rightarrow$ the function
\[
  F(z)=p'^{2}(z)-\left[ 4p^3(z)-g_2(\Lambda)p(z)-g_3(\Lambda) \right] 
\] 
is holomorphic and $F(0)=0$  $\Rightarrow$ $F(z)\equiv 0$ (bounded entire function) $\Rightarrow$ the point $\left(p(z),p'(z)\right)$ lies in the curve
\begin{equation}
  E: y^2=4x^3-g_2(\Lambda)x-g_3(\Lambda).
\end{equation}

\begin{proposition}
  The map $ \begin{matrix} \varphi:\sfrac{\C}{\Lambda}\setminus \{0\} &\to & E\\ z &\mapsto &(p(z),p'(z)) \end{matrix}  $ is bijective.
\end{proposition}
\begin{proof}
  (i) We prove $\forall s\in \C$, the function $p(z)-s$ has exactly two roots on $\sfrac{\C}{\Lambda}$. First assume $\varphi(z)-s\neq 0$ on boundary of $\sfrac{\C}{\Lambda}$.
  \[
    \text{ \# of roots }=\frac{1}{2\pi i}\int_{L} \frac{p'(z)}{p(z)-s}\mathrm{d}s
  \] 
  where $L$ is the countour in Figure \ref{fig:countour-l-on-boundary}.
\begin{figure}[ht]
    \centering
    \incfig{countour-l-on-boundary}
    \caption{Countour L on boundary of $\sfrac{\C}{\Lambda}$}
    \label{fig:countour-l-on-boundary}
\end{figure}
\begin{align*}
  p(z+\omega_2)=p(z)\Rightarrow & \int_{L_1}+\int_{L_2}=0,\\
  p(z+\omega_1)=p(z)\Rightarrow & \int_{L_3}+\int_{L_4}=0
.\end{align*}
$z\to o$, $\begin{matrix} p(z)-s&\sim &\frac{1}{z^2}\\p'(z)&\sim&-\frac{2}{z^3} \end{matrix}$ $\Rightarrow$ $\frac{p'(z)}{p(z)-s}\sim-\frac{2}{z}$ 
\[
\Rightarrow \frac{1}{2\pi i}\int_{C_1}\to-2 \frac{-\theta}{2\pi}=\frac{\theta}{\pi}.
\] 
Similarly,
\[
\frac{1}{2\pi i}\int_{C_3}\to \frac{\pi-\theta}{\pi}.
\] 
\[
  \Rightarrow \frac{1}{2\pi i}\left( \int_{C_1}+\int_{C_3} \right) \to 1.
\] 
Similarly,
\[
  \frac{1}{2\pi i}\left( \int_{C_2}+\int_{C_4} \right) \to 1.
\]
Hence 
\[
  \frac{1}{2\pi i}\int_{L} \frac{p'(z)}{p(z)-s}\mathrm{d}s=2.
\]

\begin{remark}
  If 
  \begin{align*}
   & p(z)-s=0\\
   & z\neq -z+\Lambda \in \sfrac{\C}{\Lambda}
  \end{align*}
  then $p(z)-s=0$ has two distinct roots(if $z$ is the root of $p(z)-s=0$, then $-z+\lambda,\lambda\in \Lambda$ is also the root).

  If $z= \frac{\omega_1}{2},\frac{\omega_2}{2},\frac{\omega_1+\omega_2}{2}$, then
  \begin{align*}
    &-z+\Lambda=z\\
    \Rightarrow& p'(z)=p'(-z)=-p'(z)\\
    \Rightarrow& p'(z)=0\\
    \Rightarrow & z  \text{ is a double root}
  .\end{align*} 
\end{remark}
If $p(z)=0$ ont the boundary, then modify $L$.\\
\\
(ii) $\varphi$ is surjective.\\
$\forall (x,y)\in E$, $(i)\Rightarrow \exists z$ such that $p(z)=x$. Let $y'=p'(z)$, then
 \[
   (x,y) \in E,(x,y')\in E\Rightarrow y'^2=y^2\Rightarrow y'=\pm y.
\] 
If $y'=-y\neq 0$, then:\\
$\exists \lambda \in \Lambda$ such that $-z+\lambda\neq z $ $\Rightarrow$ and :
\[
\left\{\begin{aligned}
   & p(-z+\lambda)=x,\\
   & p'(-z+\lambda)=-y'=y\quad (\text{ recall }p'(-z+ \lambda )=-p'(z)).
\end{aligned}\right.
\]
(iii) $\varphi$ is injective. \\
Suppose $\varphi(z_1)=\varphi(z_2)=(x,y)$ 
 \[
   \Rightarrow p(z_1)=p(z_2)=x\Rightarrow z_2=z_1 \text{ or } z_2=-z_1+\lambda.
\] 
In case $z_2=-z_1+\lambda$,
\begin{align*}
  \Rightarrow & y=p'(z_2)=p'(-z_1+\lambda)=-p'(z_1)=-y\\
  \Rightarrow & p'(z_1)=0
\end{align*}
$\Rightarrow p'(z_1)=0$, $z_1$ is a double root of $p(z)-x=0$ $\Rightarrow$ $z_2=z_1$.
\end{proof}
Recall 
\[
  G_k(\Lambda)=\sum_{(c,d)\neq(0,0)} \frac{1}{(c\omega_1+d\omega_2)^{k}}
\] 
and 
\[
  G_k(\tau)= \sum_{(c,d)\neq (0,0)}\frac{1}{c\tau+d)^{k}}
\] 
\[
  \Rightarrow G_k(\Lambda)=\frac{1}{\omega_2^{k}}G_k\left( \frac{\omega_1}{\omega_2} \right). 
\] 
\begin{align*}
 & \delta(\tau)=g_2^3(\tau)-27g_3^2(\tau)\\
  \Rightarrow & g_2^3(\Lambda)-27g_3^2(\Lambda)=\delta \left( \frac{\omega_1}{\omega_2} \right) \frac{1}{\omega_2^{12}} \neq 0. \left( \text{ Recall } \delta(\infty)=0, \delta(\tau)\neq 0 \forall\tau \in \mathcal{H} \right) 
\end{align*}
\begin{definition}
Suppose $C:F(x,y)$ is a curve. If $\forall (x_0,y_0)\in C$ we have
\[
  \frac{\partial F}{\partial x} \lvert_{(x_0,y_0)}\neq 0 \text{ or } \frac{\partial F}{\partial y} \lvert_{(x_0,y_0)}\neq 0,
\] 
then $C$ is a non-singular curve.
\end{definition}

Let $E:y^2-(4x^3-g_2(\Lambda)x-g_3(\Lambda))=0$, then
\[
  E \text{ is non-singular}\Leftrightarrow 4x^3-g_2(\Lambda)x-g_3(\Lambda)=0 \text{ has no multiple roots}
\] 
\[
  \Leftrightarrow g_2^3(\tau)-27g_3^2(\tau)\neq 0.
\] 
Given $E:y=4x^3-C_2x-C_3$, $\Delta=C_2^3-C_3^2$. If $\exists \Lambda$ such that 
\[
\begin{cases}
  C_2=g_2(\Lambda)&\\
  C_3=g_3(\Lambda)&
\end{cases}
\] 
then $\Delta\neq 0$, $E$ is a non-singular curve.
Let 
\begin{equation}
  j(\tau):= \frac{1728 g_2^3(\tau)}{\Delta(\tau)}
.\end{equation} 
It is easy to verify that
\[
  j\left( \frac{a\tau+b}{c\tau+d} \right) =j(\tau),\forall \begin{bmatrix} a&b\\c&d \end{bmatrix} \in \SL,
\] 
$j(\tau)$ is holomorphic on $\mathcal{H}$ ($\Delta(\tau)\neq 0$ on $\mathcal{H}$) and $j(\tau)$ has a simple pole at $\infty$.

\begin{lemma}\label{lma3-9}
  The map $\begin{matrix} \mathcal{H}&\to & \C \\ \tau & \mapsto & j(\tau) \end{matrix} $ is surjective.
\end{lemma}
\begin{proof}
  $\forall s \in \C$, let $f=f_s=1728g_2^3-s\Delta$, i.e.,
  \[
    f(\tau)=1728g_2^3(\tau)-s\Delta(\tau).
  \] 
  $f$ is of weight $12$ modular form.
  \begin{align*}
    &f(\infty)\neq 0\\
    \Rightarrow & \frac{1}{2}v_i(f)+\frac{1}{3}v_\rho(f)+\sum\nolimits^{*}v_p(f)=1\\
    \Rightarrow & \text{ one of }v_i(f),v_\rho(f),v_p(f)>0
  .\end{align*}
\end{proof}

\begin{proposition}
  If $a_2^3-27a_3^2\neq 0$ then $\exists \Lambda$ such that 
  \[
  \begin{cases}
    g_2(\Lambda)=a_2&\\
    g_3(\Lambda)=a_3.&
  \end{cases}
  \] 
\end{proposition}
\begin{proof}
  Lemma \ref{lma3-9} $\Rightarrow$ $\exists \tau \in \mathcal{H}$ such that 
  \begin{align*}
    &j(\tau)=\frac{1728g_2^3(\tau)}{\Delta(\tau)}\\
    \Rightarrow & \frac{g_2(\tau)^3}{g_2^3(\tau)-27g_3^2(\tau)}=\frac{a_2^3}{a_2^3-27a_3^2}\\
    \Rightarrow & \frac{a_2^3}{g_2^3(\tau)}=\frac{a_3^2}{g_3^2(\tau)}
  .\end{align*}
  Choose $\omega_2\neq 0$ such that $\frac{a_2}{g_2(\tau)}=\omega_2^{4}$, let $\omega_1=\tau \omega_2$ $\Rightarrow$ $\frac{a_2}{g_2(\Lambda)}= \frac{a_2}{g_2(\tau)\omega_2^{4}}=1$ $\Rightarrow$ $\frac{a_3^2}{g_3^2(\Lambda)}= \frac{a_2^3}{g_2^3(\tau)\omega_2^{6}}=1$ $\Rightarrow$ $ \frac{a_3}{g_3(\Lambda)}=\pm 1$. Replace $\omega_2$ by $i\omega_2$ if necessary $\Rightarrow$ $ \frac{a_3}{g_2(\Lambda)}=1$.
\end{proof}

\begin{remark}
  $\exists$ a surjection between $\{ \text{ complex tori }\} $ and $\{\text{ curves }E:y^2=4x^3-a_2x-a_3,a_2^3-27a_3^2\neq 0\} $.
  Write 
   \begin{align*}
     \Lambda=\omega_1\Z\oplus \omega_2\Z & \quad \tau= \frac{\omega_1}{\omega_2}\in \mathcal{H},\\
     \Lambda'=\omega_1'\Z\oplus \omega_2'\Z &\quad \tau'=\frac{\omega_1'}{\omega_2'}\in \mathcal{H}.
  \end{align*}
  Recall that\\
  $\varphi:\sfrac{\C}{\Lambda}\to \sfrac{\C}{\Lambda'}$ is holomorphically group-homomorphism $\Leftrightarrow$ $\exists$ $m\in \C$ such that $\varphi(z+\Lambda)=mz+\Lambda',m \Lambda\subset \Lambda'$.\\
  $\varphi$ is isomorphic $\Leftrightarrow$ $m\Lambda=\Lambda'$.\\
  $\sfrac{\C}{\Lambda}\simeq \sfrac{\C}{\Lambda'}$ $\Leftrightarrow$ $m\in \C$ such that $m\Lambda=\Lambda'$ $\Leftrightarrow$ $\begin{bmatrix} m\omega_1\\m\omega_2 \end{bmatrix} =\begin{bmatrix} a&b\\c&d \end{bmatrix} \begin{bmatrix} \omega'_1\\\omega'_2 \end{bmatrix} $ for some $\begin{bmatrix} a&b\\c&d \end{bmatrix} \in \SL$
  \begin{align*}
    & \frac{m\omega_1}{m\omega_2}=\frac{a\omega_1'+b\omega_2'}{c\omega_1'+d\omega_2'}\\
    \Leftrightarrow & \tau = \frac{a\tau'+b}{c\tau'+d}\\
    \Leftrightarrow &\tau,\tau' \text{ are } \SL\text{-equivalent}\\
  .\end{align*}
$\Rightarrow$ $\exists$ a bijection between
$\{\text{ isomorphism class of }\sfrac{\C}{\Lambda}\} $ and $\{\SL\text{-equivalence class of }\mathcal{H}\} $.
\[
\sfrac{\C}{\Lambda}\to \tau.
\] 
\end{remark}

\section{The Congruence Subgroup Case: Basic Results}
Let
\begin{gather*}
  \Gamma := \text{ a congruence subgroup},\\
  s,s'\in \Q \cup \{\infty\}, \infty := \lim_{\Im \tau\to \infty}\tau,\tau \in \mathcal{H},\\
  \gamma =  \begin{bmatrix} a&b\\c&d \end{bmatrix} \in \Gamma\\
 \gamma(s):= \frac{as+b}{cs+d}.
\end{gather*}
Then
\[
  \gamma(\infty)= \begin{cases}
    \frac{a}{c} & \text{ if }c\neq 0\\
    \infty & \text{ if }c=0
  \end{cases}\quad \text{ and }\quad \gamma(s)=\infty  \text{ if } cs+d=0,s \in \mathcal{H}.
\]
If $s'=\gamma(s),\gamma\in \Gamma$, $s'\neq s$ are $\Gamma$-equivalent, denoted by $s'\sim s$. 
 \begin{definition}
  A \textit{cusp} of $\Gamma$ is a $\Gamma$-equivalence class of points in $\Q\cup \{\infty\} $.
\end{definition}
\begin{exercise}
  \begin{enumerate}
    \item []
    \item Let $\Gamma=\SL$, only one cusp $=\{\infty\} $.
    \item Let $p$ be prime, $\Gamma_0(p)=\left\{\begin{bmatrix} a&b\\c&d \end{bmatrix} \in \SL:c\equiv 0 \pmod{p}\right\} $. How many cusps?
  \end{enumerate}
\end{exercise}
\begin{solution}
  The first is obvious. Consider $\frac{m}{n}\in \Q$, $(m,n)=1,n>0$.
  \begin{align*}
    &\frac{m}{n}\sim \infty\\
    \Leftrightarrow & c \frac{m}{n}+d=0\Leftrightarrow n\equiv 0 \pmod{p}
  .\end{align*}
  Hence one cusp $=\left\{ \frac{m}{n},n\equiv 0 \pmod{p} \right\} \cup \left\{ \infty \right\} $  and another cusp $=\left\{ \frac{m}{n}:(m,n)=(n,p)=1 \right\} $.
\end{solution}

\begin{proposition}
  Let $\Gamma$ be a congruent subgroup, $\sfrac{\C}{\Lambda}$, $\bsfrac{\Gamma}{\mathcal{H}}$ be the quotient space of $\Gamma$ acting on $\mathcal{H}$. Let 
  \[
    X(\Gamma)= \bsfrac{\Gamma}{\mathcal{H}}\cup \left[ \text{ cusps of }\Gamma \right] .
  \] 
  Then $X(\Gamma)$ has a natural structure as a compact Riemann surface.
\end{proposition}

\begin{definition}[Elliptic points w.r.t $\Gamma$ ]
  $\tau \in \mathcal{H}$ is an \textit{elliptic point} of $\Gamma$ if $\exists \gamma \in \Gamma$, $\gamma\neq \pm I$ such that $\gamma(\tau)=\tau$. 
  \[
   \gamma(\tau)=\tau\Leftrightarrow \sigma\gamma\sigma^{-1}\left( \sigma(\tau) \right)=\sigma(\tau) .
  \]
\end{definition}
$\gamma$ is an elliptic point of $\Gamma$ $\Rightarrow$ $\tau $ is an elliptic point of $\SL$.

\begin{lemma}\label{lma4-5}
  Let $\gamma \in \SL$.
  \begin{enumerate}
    \item 
  If $\gamma^2=-I$, then $\gamma$ is conjugate to 
  $\begin{bmatrix} 0 & -1\\1&0 \end{bmatrix} ^{\pm 1}$, i.e., 
  \[
    \sigma \gamma\sigma^{-1} = \begin{bmatrix} 0&-1\\1&0 \end{bmatrix} ^{\pm 1}.
  \]
    \item 
    If $\gamma^2+\gamma+I=0$, then $\gamma$ is conjugate to $\begin{bmatrix} 0&1\\-1&-1 \end{bmatrix} ^{\pm 1}$.
    \item
      If $\gamma^2-\gamma+I=0$, then $\gamma$ is conjugate to $\begin{bmatrix} 0&-1\\1&1 \end{bmatrix}^{\pm 1} $.
  \end{enumerate}
\end{lemma}
\begin{lemma}
  Any elliptic point of $\SL$ is equivalent to $i$ or $\rho=e^{ \frac{2\pi i}{3}}$.
\end{lemma}
\begin{proof}
  For $\gamma=\begin{bmatrix} a&b\\c&d \end{bmatrix} \neq \pm I$, 
  \begin{align*}
    \gamma(\tau )=\tau \Leftrightarrow & a\tau +b=c\tau^2 +d \tau \\
    \Leftrightarrow & c\tau ^2+(d-a)\tau -b=0
  .\end{align*}
  If $c=0$, then $a=d$ $\Rightarrow$ $\gamma=\pm I$. Assume $c\neq 0$, $\tau \notin \R$ $\Rightarrow$ $(d-a)^2+4bc<0$
  \begin{align*}
    \Rightarrow & (d+a)^2+4(bc-ad)<0\\
    \Rightarrow&(d+a)^2-4<0\\
    \Rightarrow & |d+a|<2\\
    \Rightarrow & \left| \begin{matrix} a-x&b\\c&d-x \end{matrix} \right|=x^2+1 \text{ or } x^2\pm x+1\\
    \Rightarrow & \gamma^2+I=0 \text{ or } \gamma^2\pm \gamma+I=0
  .\end{align*}
  By Lemma \ref{lma4-5}: 
  \begin{align*}
    \gamma^2+I=0 \Rightarrow & \gamma=\sigma \begin{bmatrix} 0&-1\\1&0 \end{bmatrix} ^{\pm 1}\sigma^{-1}\\
    \gamma(\tau )=\tau \Rightarrow&\tau \sim i
  .\end{align*}
  Similarly $\gamma^2\pm \gamma+I=0 \Rightarrow\tau \sim\rho$.
\end{proof}
Let 
\begin{align*}
  \mathcal{M}_k(\Gamma)=&\text{ space of weight }k  \text{ modular forms w.r.t. }\Gamma,\\
  \mathcal{S}_k(\Gamma)=&\text{ space of weight }k \text{ cusp forms w.r.t. }\Gamma,\\
  g=&\text{ genus of }X(\Gamma)
.\end{align*}
\begin{theorem}
  Suppose $k$ is even,
  \begin{align*}
    \varepsilon _2=& \text{ number of elliptic points of }\Gamma \text{ which are }\SL\sim i,\\
    \varepsilon_3=& \text{ number of elliptic points of }\Gamma \text{ which are }\SL\sim\rho,\\
    \varepsilon _{\infty}=& \text{ number of cusps of }\Gamma
  .\end{align*}
  Then

  \begin{align*}
    \mathrm{dim}\mathcal{M}_k(\Gamma)=&\begin{cases}
      (k-1)(g-1)+\left[ \frac{k}{4} \right] \varepsilon_2+\left[ \frac{k}{3} \right] \varepsilon _3+\frac{k}{2}\varepsilon _{\infty} & k>=2,\\
      1&k=0,\\
      0&k<0.
    \end{cases}\\
      \mathrm{dim}\mathcal{S}_k(\Gamma)=& \begin{cases}
	(k-1)(g-1)+\left[ \frac{k}{4} \right] \varepsilon _2+\left[ \frac{k}{3} \right] \varepsilon _3+\left( \frac{k}{2}-1 \right) \varepsilon _{\infty}&k>=4,\\
	g&k=2,\\
	0&k<=0.
      \end{cases}
  \end{align*}
\end{theorem}

$X(\Gamma)\overset{\text{called}}{=}$ a \textit{modular curve}.\\
\textbf{Modularity Theorem(Version $X_{\C}$ )}. Suppose $\sfrac{\C}{\Lambda}$ is a complex elliptic curve with $j(\Lambda)\in \Q$. Then for some $N \in \N$, there exists a surjective holomorphic function $X(\Gamma_0(N))\to \sfrac{\C}{\Lambda}$.

\section{Hecke Operators}
Let $\Gamma_1$ and $\Gamma_2$ be congruence subgroups,
\[
  \GL_2^{+}(\Q):=\left\{\begin{bmatrix} a&b\\c&d \end{bmatrix} :a,b,c,d\in \Q,ad-bc>0\right\}.	
\] 
Let $\alpha \in \GL_2^{+}(\Q)$, write 
\[
\Gamma_1\alpha\Gamma_2:=\left\{\gamma_1\alpha\gamma_2:\gamma_1\in \Gamma_1,\gamma_2\in \Gamma_2\right\}. 
\] 
$\Gamma_1\alpha\Gamma_2$ is called a double coset in $\GL_2^{+}(\Q)$.

\begin{remark}
  Let $G=$ group, $S=$ set, then $G$ acts on $S$ is denoted by $\bsfrac{G}{S}=\left\{ \text{ orbits of }G \text{ on }S\right\} $ $=\left\{Gs:s \in S\right\} $. Indeed, if $S=$ group $G\vartriangleleft S$ $\Rightarrow$ $\bsfrac{G}{S}=\left\{Gs:s \in S\right\} =\sfrac{S}{G}$.
\end{remark}
\textbf{Fact}: $\bsfrac{\Gamma_1}{\Gamma_1\alpha \Gamma_2}=\left\{\Gamma_1\alpha_2\gamma_2:\gamma_2 \in \Gamma_2\right\} $ is finite.\\
$\Gamma=$ congruence subgroup $\Rightarrow$ $\left[ \SL_2(\Z):\Gamma \right] <\infty$.

\begin{lemma}
  Let $\Gamma$ be a congruence subgroup and $\alpha \in \GL_2^{+}(\Q)$, then $\alpha^{-1}\Gamma\alpha \cap \GL_2^{+}(\Q)$ is a congruence subgroup.
\end{lemma}
\begin{proof}
  $\exists \widetilde{N}\in \N$ such that $\Gamma(\widetilde{N})\subset \Gamma$ and
  \[
    \widetilde{N}\alpha\in M_2(\Z)=\left\{\begin{bmatrix} a&b\\c&d \end{bmatrix} :a,b,c,d \in \Z\right\}, \widetilde{N}\alpha^{-1}\in  M_2(\Z).
  \]
  Let $N=\widetilde{N}^3$,
  \begin{align*}
    \alpha \Gamma(N)\alpha^{-1}\subset & \alpha(I+NM_2(\Z))\alpha^{-1}\\
    =& I+ \widetilde{N}\widetilde{N}\alpha M_2(\Z)\widetilde{N}\alpha^{-1}\\
    \subset & I+\widetilde{N}M_2(\Z)
  \end{align*}
  \begin{align*}
    \Rightarrow&\alpha\Gamma(N)\alpha^{-1}\subset \SL\cap \left( I+\widetilde{N}M_2(\Z) \right)=\Gamma(\widetilde{N})\\
    \Leftrightarrow & \alpha \Gamma(N)\alpha^{-1}\subset \Gamma(\widetilde{N})\\
    \Rightarrow& \Gamma(N)\subset \alpha^{-1}\Gamma(\widetilde{N})\alpha\subset \alpha^{-1}\Gamma\alpha\\
    \Rightarrow& \Gamma(N)\subset \alpha^{-1}\Gamma\alpha \cap \SL\\
    \Rightarrow& \alpha^{-1}\Gamma\alpha \cap \SL \text{ is a congruence sub group}
  .\end{align*}
\end{proof}
\begin{lemma}
  Write $\Gamma_3=\alpha^{-1}\Gamma_1\alpha \cap \Gamma_2 \left( \Gamma_3\subset \Gamma_2 \right) $,
  \[
  \bsfrac{\Gamma_1}{\Gamma_1\alpha\Gamma_2}=\left\{\Gamma_1\alpha\gamma_2:\gamma \in \Gamma_2\right\}. 
  \] 
  The map  $\begin{matrix}\varphi:& \Gamma_2 & \to & \bsfrac{\Gamma_1}{\Gamma_1\alpha\Gamma_2}\\&\gamma_2 & \mapsto &\Gamma_1\alpha\gamma_2 \end{matrix} $ induces a bijection
  \[
  \bsfrac{\Gamma_3}{\Gamma_2}\to \bsfrac{\Gamma_1}{\Gamma_1\alpha \Gamma_2}.
  \] 
\end{lemma}
\begin{proof}
  It is obvious that $\varphi$ is surjective. Suppose $\varphi(\gamma_2)=\varphi(\gamma_2')$,
  \begin{align*}
    & \Gamma_1\alpha \gamma_2=\Gamma_1\alpha \gamma_2'\\
    \Leftrightarrow&\Gamma_1\alpha\gamma_2\gamma_2'^{-1}=\Gamma_1\alpha\\
    \Leftrightarrow& \alpha(\gamma_2\gamma_2'^{-1})\alpha^{-1}\in \Gamma_1\\
    \Leftrightarrow&\gamma_2\gamma_2'^{-1}\in \alpha^{-1}\Gamma_1\alpha\\
    \Leftrightarrow&\gamma_2\gamma_2'^{-1}\in \Gamma_3
  .\end{align*}
\end{proof}

Zeta function is defined by
\[
  \zeta(s)=\sum_{n=1}^{\infty} \frac{1}{n^{s}}=\prod_{p}\left( 1-\frac{1}{p^{s}} \right) ^{-1}, \Re s>1.
\] 
Let $f=$ cusp form at $\infty:f(\tau )=\sum_{n=1}^{\infty} a_n(f)e^{2\pi n\tau }$. Define
\[
  L(f,s)=\sum_{n=1}^{\infty} a_n(f) n^{-s}
\] 
for some $f$. Then 
\[
  L(f,s)=\prod_{p}\left( 1- \frac{a_p(f)}{p^{s}}+\frac{1}{p^{2s}} \right) ^{-1}.
\] 
\\
\subsection{How to draw the fundamental domain for some types of congruence subgroup $\Gamma$ (by a student in class)}
 
\textbf{1. Principle}\\
$\Gamma=$ congruence subgroup, $\exists \min h>0$ s.t. $\begin{bmatrix} 1&h\\0&1 \end{bmatrix} \in \Gamma$. Recall 
    \[
      \Im \gamma(\tau )=\frac{\Im(\tau) }{|c\tau +d|^2}
    \]
    \begin{align*}
      D:=&\left\{\tau \in \mathcal{H}:0\le \Re(\tau) \le h, \Im(\tau)  \text{ max on }\Gamma\tau  \right\} \\
      =&\left\{\tau  \in \mathcal{H}:0\le \Re(\tau )\le h, \forall \begin{bmatrix} a&b\\c&d \end{bmatrix} \in \Gamma,|c\tau +d|\ge 1\right\} .
    \end{align*} 

\begin{proposition}
  $D$ is a fundamental domain for $\bsfrac{\Gamma}{\mathcal{H}}$.
\end{proposition}
\begin{proof}
  \begin{enumerate}
       \item $\forall \tau \in \mathcal{H},\exists \gamma\in \Gamma$ s.t. $\gamma(\tau )\in D$
       \item If $\tau \in D$, $\Gamma\tau \cap D=\left\{\tau \right\} $.
  \end{enumerate}
\end{proof}

\noindent\textbf{2. An example}\\
$\Gamma=\Gamma_0(13)$
\begin{align*}
  \begin{bmatrix} a'&b'\\c&d \end{bmatrix} \begin{bmatrix} a&b\\c&d \end{bmatrix}^{-1} = \begin{bmatrix} a'&b'\\c&d \end{bmatrix} \begin{bmatrix} d&-b\\-c&a \end{bmatrix} =\begin{bmatrix} 1&\alpha\\0&1 \end{bmatrix} 
.\end{align*}
$\gamma \in \SL$ maintain
\[
\mathrm{d}s^2= \frac{\mathrm{d}x^2+\mathrm{d}y^2}{y^2}.
\] 
Circles in this metric are geodesics, hence $\gamma$ maps circles to circles or vertical lines(See Figure \ref{fig:transform-of-circles}).
\begin{figure}[ht]
    \centering
    \incfig{transform-of-circles}
    \caption{transform of circles}
    \label{fig:transform-of-circles}
\end{figure}

Then we can draw the fundamental domain $D$(See Figure \ref{fig:fundamental-domain-of-13}).
\begin{figure}[ht]
    \centering
    \incfig{fundamental-domain-of-13}
    \caption{fundamental domain of $\Gamma_0(13)$}
    \label{fig:fundamental-domain-of-13}
\end{figure}
\begin{enumerate}
  \item [\textcircled{1}] $\gamma=\begin{bmatrix} 1&0\\-13&1 \end{bmatrix}: \begin{matrix} 0&\to&0&\to & 1\\\frac{2}{13}&\to&-\frac{2}{13}&\to&\frac{11}{13} \end{matrix}  $
  \item [\textcircled{2}] $\gamma=\begin{bmatrix} -6 &1\\-13&2 \end{bmatrix}:\begin{matrix} \frac{1}{13}&\to &\frac{7}{13}\\\frac{3}{13}&\to&\frac{5}{13} \end{matrix}  $
  \item [\ldots]
\end{enumerate}

\noindent\textbf{3. What can we see from the fundamental domain}\\
\[
d:=\left[ \SL: \left\{\pm I\right\} \Gamma \right] .
\] 
Let $D_0=\bsfrac{\Gamma}{\mathcal{H}}$, then $d= \frac{|D|}{|D_0|}$.
\begin{figure}[ht]
    \centering
    \incfig{area-of-the-domain}
    \caption{area of the domain}
    \label{fig:area-of-the-domain}
\end{figure}
\begin{align*}
  S=&\iint_S \frac{\mathrm{d}x\mathrm{d}y}{y^2}\\
  =&\int_{r\cos\alpha}^{r\cos\beta}\int ^{\infty}_{\sqrt{r^2-x^2} } \frac{\mathrm{d}y}{y^2}\mathrm{d}x\\
  =& \int_{r\cos\alpha}^{r\cos\beta}\frac{1}{\sqrt{r^2-x^2} }\mathrm{d}x\\
  =&\beta-\alpha
.\end{align*}
We can also calculate it by Gauss-Bonet Theorem.
\[
  d=\left( \Gamma_0(13) \right) =14.
\] 

We can easily calculate $\varepsilon _2=2,\varepsilon _3=2,\varepsilon _\infty=2$. Since
\[
2-2g=V-E+F,
\] 
and $V=8,E=7,F=1$, we get  $g=0$.
\begin{proposition}[Genus formula]
  \[
  g=1+\frac{d}{12}-\frac{\varepsilon _\infty}{2}-\frac{\varepsilon _3}{3}-\frac{\varepsilon _{2}}{4}
  \]
\end{proposition}
\begin{proof}
Cusp points: $n_1,n_2,\cdots,n_t$,
\[
  |D|=\pi\left( n_1+\cdots n_t \right) +\frac{\pi}{3}\left( \varepsilon _3+3\left( V-(\varepsilon _\infty-1)  -\varepsilon _2-\varepsilon _3\right) \right) 
\] 
On the other hand,
\[
|D|=d \frac{\pi}{3}.
\] 
\[
\Rightarrow d=3V-3\varepsilon _2-2\varepsilon _3+3\delta
\] 
\[
V=\frac{d}{3}+\varepsilon _2+\frac{2}{3}\varepsilon _3-\delta
\] 
where $\delta=(n_1+\cdots+n_t)-(\varepsilon _\infty-1)$.
\begin{align*}
  2E=&3\left( V-(\varepsilon _\infty-1)-\varepsilon _2-\varepsilon _3 \right) +(n_1+\cdots+n_t)+\varepsilon _2+\varepsilon _3\\
  \Rightarrow E=&\frac{3V}{2}-\left( \varepsilon _\infty-1 \right) -\varepsilon _2-\varepsilon _3+\frac{\delta}{2}\\
  =&\frac{d}{2}-(\varepsilon _\infty-1)+\frac{\varepsilon _2}{2}-\delta
.\end{align*}
\[
F=1.
\] 
\end{proof}
\begin{proposition}
  Let $0\neq f \in \mathcal{M}_k(\Gamma)$, then
  \[
    \left( \sum_{p \in  \text{cusp} }^{} v_p(f)+\frac{1}{2}\left( \sum_{p \in \# 2}^{} v_p(f) \right) +\frac{1}{3}\left( \sum_{p \in \# 3}^{} v_p(f) \right)  \right) +\sum\nolimits^{*}v_p(f)=\frac{dk}{12}
  \] 
\end{proposition}

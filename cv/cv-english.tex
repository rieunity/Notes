%%%%%%%%%%%%%%%%%%%%%%%%%%%%%%%%%%%%%%%%%
% Compact Academic CV
% LaTeX Template
% Version 2.0 (6/7/2019)
%
% This template originates from:
% https://www.LaTeXTemplates.com
%
% Authors:
% Dario Taraborelli (http://nitens.org/taraborelli/home)
% Vel (vel@LaTeXTemplates.com) 
%
% License:
% CC BY-NC-SA 3.0 (http://creativecommons.org/licenses/by-nc-sa/3.0/)
%
%%%%%%%%%%%%%%%%%%%%%%%%%%%%%%%%%%%%%%%%%

%----------------------------------------------------------------------------------------
%	PACKAGES AND OTHER DOCUMENT CONFIGURATIONS
%----------------------------------------------------------------------------------------

\documentclass[11pt]{article} % Default document font size

%%%%%%%%%%%%%%%%%%%%%%%%%%%%%%%%%%%%%%%%%
% Compact Academic CV
% Structural Definitions
% Version 1.0 (6/7/2019)
%
% This template originates from:
% https://www.LaTeXTemplates.com
%
% Authors:
% Dario Taraborelli (http://nitens.org/taraborelli/home)
% Vel (vel@LaTeXTemplates.com)
%
% License:
% CC BY-NC-SA 3.0 (http://creativecommons.org/licenses/by-nc-sa/3.0/)
%
%%%%%%%%%%%%%%%%%%%%%%%%%%%%%%%%%%%%%%%%%

%----------------------------------------------------------------------------------------
%	REQUIRED PACKAGES AND MISC CONFIGURATIONS
%----------------------------------------------------------------------------------------

\usepackage{graphicx} % Required for including images

\setlength{\parindent}{0pt} % Stop paragraph indentation

%----------------------------------------------------------------------------------------
%	MARGINS
%----------------------------------------------------------------------------------------

\usepackage{geometry} % Required for adjusting page dimensions and margins

\geometry{
	paper=a4paper, % Paper size, change to letterpaper for US letter size
	top=3.25cm, % Top margin
	bottom=4cm, % Bottom margin
	left=3.5cm, % Left margin
	right=3.5cm, % Right margin
	headheight=0.75cm, % Header height
	footskip=1cm, % Space from the bottom margin to the baseline of the footer
	headsep=0.75cm, % Space from the top margin to the baseline of the header
	%showframe, % Uncomment to show how the type block is set on the page
}

%----------------------------------------------------------------------------------------
%	FONTS
%----------------------------------------------------------------------------------------

\usepackage[utf8]{inputenc} % Required for inputting international characters
\usepackage[T1]{fontenc} % Output font encoding for international characters

%\usepackage[semibold]{ebgaramond} % Use the EB Garamond font with a reduced bold weight

%----------------------------------------------------------------------------------------
%	SECTION STYLING
%----------------------------------------------------------------------------------------

%\usepackage{sectsty} % Allows changing the font options for sections in a document

%\sectionfont{\fontsize{13.5pt}{18pt}\selectfont} % Set font options for sections
%\subsectionfont{\mdseries\scshape\normalsize} % Set font options for subsections
%\subsubsectionfont{\mdseries\upshape\bfseries\normalsize} % Set font options for subsubsections

%----------------------------------------------------------------------------------------
%	MARGIN YEARS
%----------------------------------------------------------------------------------------

\usepackage{marginnote} % Required to output text in the margin

\newcommand{\years}[1]{\marginnote{\scriptsize #1}} % New command for adding years to the margin
\renewcommand*{\raggedleftmarginnote}{} % Left-align the years in the margin
\setlength{\marginparsep}{-10pt} % Move the margin content closer to the text
\reversemarginpar % Margin text to be output into the left margin instead of the default right margin

%----------------------------------------------------------------------------------------
%	COLOURS
%----------------------------------------------------------------------------------------

\usepackage[usenames, dvipsnames]{xcolor} % Required for specifying colours by name

%----------------------------------------------------------------------------------------
%	LINKS
%----------------------------------------------------------------------------------------

\usepackage[bookmarks, colorlinks, breaklinks]{hyperref} % Required for links

% Set link colours
\hypersetup{
	linkcolor=blue,
	citecolor=blue,
	filecolor=black,
	urlcolor=MidnightBlue
}

\renewcommand\labelenumi{[\theenumi]}
 % Include the file specifying the document structure and styling

% Set PDF meta-information
\hypersetup{
	pdftitle={Yunlei Wang - Curriculum vitae},
	pdfauthor={Yunlei Wang}
}

%----------------------------------------------------------------------------------------
\begin{document}

%----------------------------------------------------------------------------------------
%	CONTACT AND GENERAL INFORMATION
%----------------------------------------------------------------------------------------

{\LARGE\bfseries Yunlei Wang} % Name
\bigskip\bigskip\medskip % Whitespace

China University of Geosciences Wuhan (CUG)\\ % Address
Office 1101\\ New Office Building of CUG
%\medskip % Whitespace

\vspace{0.01\textheight} 

Mobile: +86 13936256225\\ % Mobile number
Email: \href{wcghdpwyl@gmail.com}{wcghdpwyl@gmail.com}\\ % Email address
Webpage: \href{http://rieunity.github.io}{http://rieunity.github.io}% Academic/personal website

\vspace{0.01\textheight} % Whitespace between contact information and specific CV information

%------------------------------------------------

Born: April 23, 1993---Lu'an, Anhui province, China\\ % Date of birth
Nationality: China % Nationality

%----------------------------------------------------------------------------------------
%	EDUCATION
%----------------------------------------------------------------------------------------

\section*{Education}

\years{2019-}\textsc{MSc} in Mathematics, School of Mathematics and Physics, China University of Geosciences Wuhan, expected by July 2022\\
(advisor: Prof. Ming Wang)\\
GPA: 3.87/4 \\
\years{2015, fall} Exchange students in Department of Physics,  Fudan University\\
\years{2012-2016}\textsc{BSc} in Applied Physics, School of Science, Harbin Institute of Technology (HIT)\\
Thesis: \emph{Calculations of Two Decay Processes $J /\psi \to \mu^+\mu^-\gamma$ and $J /\psi \to (\mu^+\mu^-)_{BS}\gamma$} \\
(advisor: Prof. Tianpeng Wang) \\
GPA: 3.74/4 

%----------------------------------------------------------------------------------------
%	GRANTS, HONOURS AND AWARDS
%----------------------------------------------------------------------------------------

\section*{Research interests \& experiences}
My primary interests lie in harmonic analysis, especially the uncertainty principle and its applications in quantitative unique continuation. 

Now I have transformed interests into algebraic geometry. Due to the blank of courses relevant to algebra and geometry in school, I learned some basic algebraic geometry and its prerequisities through books and video lectures online, most of them were taken notes and some were in latex form. More details about my learning road may be seen in \url{https://rieunity.github.io/about/}.

\section*{Activities \& competitions}
\years{2015.7} Summer Camp of Theoretical Physics, Peiking University\\
\years{2014.3} The Chinese Mathematics Competitions, the third prize  in the final\\
\years{2012.7} Physics Enlightment Summer School of National Top Students, Nanjing University

%\section*{Conference attended}

\section*{Teaching assistance}
2019 Fall, Real Analysis, China University of Geosciences. \\
2020 Spring, Functional Analysis, China University of Geosciences.
%----------------------------------------------------------------------------------------
%	PUBLICATIONS AND TALKS
%----------------------------------------------------------------------------------------

\section*{Publications}
\begin{enumerate}
\leftskip-0.13in
\item Yunlei Wang, Ming Wang, Observability inequality at two time points for the KdV equation from measurable sets, \emph{J. Math. Anal.Appl.} 505(2) (2022).
\end{enumerate}
\iffalse
\subsection*{Journal articles}


\years{1905a}Einstein, Albert (1905), “On a Heuristic Viewpoint Concerning the Production and Transformation of Light", \emph{Annalen der Physik} 17: 132–148.\\
\years{1905b}Einstein, Albert (1905), A new determination of molecular dimensions. \emph{PhD dissertation}.\\
\years{1905c}Einstein, Albert (1905), “On the Motion—Required by the Molecular Kinetic Theory of Heat—of Small Particles Suspended in a Stationary Liquid", \emph{Annalen der Physik} 17: 549–560.\\
\years{1905d}Einstein, Albert (1905), “On the Electrodynamics of Moving Bodies", \emph{Annalen der Physik} 17: 891–921.\\
\years{1905e}Einstein, Albert (1905), “Does the Inertia of a Body Depend Upon Its Energy Content?", \emph{Annalen der Physik} 18: 639–641.\\
\years{1915}Einstein, Albert (1915), “Die Feldgleichungen der Gravitation (The Field Equations of Gravitation)", \emph{Koniglich Preussische Akademie der Wissenschaften}: 844–847\\
\years{1917a}Einstein, Albert (1917), “Kosmologische Betrachtungen zur allgemeinen Relativitätstheorie (Cosmological Considerations in the General Theory of Relativity)", \emph{Koniglich Preussische Akademie der Wissenschaften}\\
\years{1917b}Einstein, Albert (1917), “Zur Quantentheorie der Strahlung (On the Quantum Mechanics of Radiation)", \emph{Physikalische Zeitschrift} 18: 121–128

%------------------------------------------------

\subsection*{Books}

\years{1954}Einstein, Albert (1954), \emph{Ideas and Opinions}, New York: Random House, ISBN 0-517-00393-7

%------------------------------------------------

\subsection*{Newspaper articles}

\years{1940}Einstein, Albert, et al. (December 4, 1948), “To the editors", \emph{New York Times}\\
\years{1949}Einstein, Albert (May 1949), “Why Socialism?", \emph{Monthly Review}.

%----------------------------------------------------------------------------------------
%	TEACHING
%----------------------------------------------------------------------------------------

\section*{Teaching}

\ldots

%------------------------------------------------

\section*{Service to the profession}

\ldots

\vfill % Whitespace before final footer

%----------------------------------------------------------------------------------------
%	FINAL FOOTER
%----------------------------------------------------------------------------------------

% Any final footer text such as a URL to the latest version of this CV, last updated date, compiled in XeTeX, etc
\fi
\begin{center}
	\scriptsize
	Last updated: \today
\end{center}

%----------------------------------------------------------------------------------------

\end{document}
